\documentclass[11pt]{article}
\usepackage[scaled=0.92]{helvet}
\usepackage{geometry}
\geometry{letterpaper,tmargin=1in,bmargin=1in,lmargin=1in,rmargin=1in}
\usepackage[parfill]{parskip} % Activate to begin paragraphs with an empty line rather than an indent %\usepackage{graphicx}
\usepackage{amsmath,amssymb, amscd}
\usepackage{mathrsfs, dsfont}
\usepackage[all,cmtip]{xy}
%\diagramstyle[labelstyle=\scriptstyle]
\usepackage{tabularx}
\usepackage[font=footnotesize,labelfont=bf]{caption}
\usepackage{graphicx}
\usepackage{xcolor}
%\usepackage[linkbordercolor ={1 1 1} ]{hyperref}
%\usepackage[sf]{titlesec}
\usepackage{natbib}
\usepackage{../../Tianpei_Report}

%\usepackage{appendix}
%\usepackage{algorithm}
%\usepackage{algorithmic}

%\renewcommand{\algorithmicrequire}{\textbf{Input:}}
%\renewcommand{\algorithmicensure}{\textbf{Output:}}



\begin{document}
\title{Lecture 4: Gaussian measure}
\author{ Tianpei Xie}
\date{ Jul. 14th., 2015 }
\maketitle
\tableofcontents
\newpage
\section{Definitions}
\subsection{Probability measure on infinite dimensional function space}
\begin{itemize}
\item Denote $\cB^{1}\equiv \cB(\bR)$ as the Borel $\sigma$-algebra on $\bR$ and  $\cB^{n}\equiv \cB(\bR^{n})$ as the Borel $\sigma$-algebra on $\bR^{n}, n\ge 1$.  Consider the sample space $\Omega$ as a locally convex Hausdorff topological space, where the algebra of cylinders $\srC_{0}$ and cylindrical $\sigma$-algebra $\srC$ is defined. Note that $\srC$ is the $\sigma$-algebra generated by $\srC_{0}$ and $\srC \subset \srB\equiv \cB(\Omega)$, where $\cB(\Omega)$ is the Borel $\sigma$-algebra on $\Omega$.  Let $\Omega^{*}$ be the dual space of continuous linear functionals on $\Omega$ consists of the random variable $\xi: \Omega\rightarrow \bR$, which is $(\srB, \cB^{1})$ measureable.  Here 

Note that the cylinder set 
\begin{align*}
C_{\xi}[A; t_{1},\ldots t_{n}] &\equiv \set{\omega\;|\; (\xi_{t_{1}}(\omega), \cdots, \xi_{t_{n}}(\omega)) \in A } \in \srC; \quad A\in \cB^{n}, \forall n\ge 1
\end{align*}

\item Here consider the random function as $\xi_{\cdot}: T\times \Omega  \rightarrow \bR$, and for each subset $N = \set{t_{1},\ldots t_{n}}$, $\xi_{N}: (\Omega, \srB ) \rightarrow (\bR^{n}, \cB^{n})$, $n\ge 1$.  On the other hand, for fixed $\omega$,  the whole sample-path $\xi_{t}(\omega)$ is seen as a function in $\bR^{T}$  (usually smooth, or integrable functions s.t. structure of $T$). Therefore, the random function $\xi_{\cdot}$ is a mapping $(\Omega, \srB) \rightarrow (\bR^{T},  \cB^{T})$. Here $\cB^{T}\equiv \cB(\bR^{T})$ is the Borel $\sigma$-algebra on function space $\bR^{T}$, with respect to which each coordinate functional (evaluation functional) $\pi_{t}: \bR^{T}\rightarrow \bR$, $\pi_{t}(x) = x(t)$, are $(\cB^{T}, \cB^{1})$ measureable. 

\item We can define the probability measure $\cP$ on the cylindrical $\sigma$-algebra $\srC$ and for locally convex Hausdorff topological space $\Omega$, $\cP$ can be uniquely extended to the Borel $\sigma$-algebra $\srB$. Therefore, the probability measure $\bP$ on Borel set of $\bR$ can be induced from $\cP$ via the measureable function $\xi\in \Omega^{*}$; i.e.,  
\begin{align}
  \bP(A) &\equiv \cP\set{ \omega\in \Omega\; |\; \xi(\omega) \in A  } = \cP\circ\xi^{-1}(A), \quad A\in \cB^{1}, \label{eqn: measurable_xi}
\end{align}

\item Similarly, the probability measure $\bP$ on function space $\bR^{T}$ can be induced from the measureable random function $\xi_{\cdot}: (\Omega, \srB) \rightarrow (\bR^{T},  \cB^{T})$; i.e., 
\begin{align*}
 \bP(A) &\equiv \cP\set{ \omega\in \Omega\; |\; \xi_{\cdot}\equiv \xi(\cdot,\omega) \in A  } = \cP\circ\xi_{\cdot}^{-1}(A), \quad A\in \cB^{T}.
\end{align*} The above $\bP$ is said to be the \emph{distribution of random function} $\xi_{\cdot}\equiv \xi(\cdot,\omega)$.  

\item In particular, for any $n\ge 1$, any $A\in \cB^{n} \subset \cB^{T}$, $\bP$ can be defined via each $n$-dimensional cylinder set $C_{\xi}[A; t_{1},\ldots t_{n}] \in \srC$. 
\begin{align}
 \bP(A) &\equiv  \cP\paren{C_{\xi}[A_{n}; t_{1},\ldots t_{n}]} \nonumber \\
 &= \cP\set{ \omega\in \Omega\; |\;  (\xi_{t_{1}}(\omega), \cdots, \xi_{t_{n}}(\omega) )  \in A_{n}  } \nonumber\\
 &= \cP\circ \xi_{T_{n}}^{-1}(A), \quad A_{n}\in \cB^{n} \label{eqn: measureable_finite_dim_fun}\\
 &= \cP\brac{(\xi_{t_{1}}, \cdots, \xi_{t_{n}} )  \in A_{n}}, \nonumber
\end{align} where $\xi_{T_{n}}= \xi_{\cdot}\circ \pi_{T_{n}}$, $\xi_{\cdot}: (\Omega, \srB) \rightarrow (\bR^{T},  \cB^{T})$ and $\pi_{N}: (\bR^{T}, \cB^{T} )\rightarrow (\bR^{n},\cB^{n})$ as $\pi_{T_{n}}(f)= (f_{t_{1}}, \cdots, f_{t_{n}})$, $T_{n}= \set{t_{1},\ldots t_{n}} \subset T$, $n\ge 1$ is the evaluation map. The above formula holds for all $n\ge 1$, $(t_{1},\ldots t_{n})\subset T$.

This is called a \emph{finite $n$-dimensional joint distribution} of random function $\xi_{\cdot}$. 


Note that each cylinder set can be obtained from the evaluation map . Then 
\begin{align*}
C_{\xi}[A; N] &= (\hat{\pi}_{T_{n}}(\xi_{\cdot}))^{-1}(A); A\in \cB^{n}
\end{align*}
where $\hat{\pi}_{t}: \bR^{T\times \Omega}\rightarrow \bR^{\Omega}: \xi_{\cdot} \mapsto \xi_{t}$ is considered as the coordinate operator (evaluation operator).

\item For a nondecreasing sequence of sets $A_{n}\uparrow A$, $A_{n}=  \pi_{T_{n}}A_{n+1} \subset A_{n+1}$, $n\ge 1$ and $T_{n} = [t_{1}, \ldots, t_{n}] \uparrow T$, where $T_{n+1} = T_{n}\cup\set{t_{n+1}}$, the cylinder sets $C_{\xi}[A_{n};T_{n}] \downarrow  C_{\xi}[A; T] \equiv \bigcap_{n=1}^{\infty}C_{\xi}[A_{n};T_{n}]$. 

Therefore the \emph{distribution of random function} $\xi_{\cdot}$, $\bP$, is completely determined by all its finite $n$-dimensional joint distribution; i.e., 
\begin{align*}
\bP(A) &= \cP\paren{C_{\xi}[A; T]}\\
&=\cP\paren{\lim\limits_{n\rightarrow \infty}C_{\xi}[A_{n}; T_{n}] }\\
&=\sum_{n\rightarrow \infty}\cP\paren{C_{\xi}[A_{n}; T_{n}] }\\
&= \sum_{n\rightarrow \infty}\bP(A_{n}),
\end{align*}
where $A_{n}=  \pi_{T_{n}}A$ for any $n\ge 1$, any $T_{n}\subset T$, any $A\in \cB^{T}$.
\end{itemize}
\subsection{Important functional, operators}
\begin{itemize}
\item Define the \emph{barycenter} $\omega_{a}\in \Omega$ of a measure $\cP$ if for any continuous linear functional (random variable) $\xi\in \Omega^{*}$, 
\begin{align}
\xi(\omega_{a}) &= \int_{\Omega}\xi(\omega)\cP(d\omega) \equiv m(\xi)  , \label{expr: mean_operator}
\end{align}
where $m\in \Omega^{**}$ is a linear functional on random variable $\xi \in \Omega^{*}$, called \emph{mean} functional.

\item The linear operator  $K: \Omega^{*} \rightarrow \Omega$ is called the \emph{covariance operator} of a measure $\cP$ if  for any $\xi, \eta \in \Omega^{*}$,  the following equality holds,
\begin{align}
\xi\paren{ K(\eta ) } &= \int_{\Omega}\xi(\omega - \omega_{a})\eta(\omega - \omega_{a})\cP(d\omega) \nonumber\\
&= \int_{\Omega}\paren{\xi(\omega) - \xi(\omega_{a})}\paren{\eta(\omega) - \eta(\omega_{a})}\cP(d\omega)\nonumber\\
&= \int_{\Omega}\paren{\xi(\omega) - m(\xi)}\paren{\eta(\omega) - m(\eta)}\cP(d\omega) \label{expr: covariance_operator}
\end{align}

\item For $\xi$ and $\eta\in \Omega^{*}$, so $\xi(\omega) = \inn{\omega}{\xi}$ and $\eta(\omega) = \inn{\omega}{\eta}$, where $\inn{\cdot}{\cdot}: \Omega\times \Omega^{*}  \rightarrow \bR$ is the \emph{duality bilinear products}, so the covariance operator $K$ corresponds to 
\begin{align}
 \widehat{K}(\xi, \eta) &= \int_{\Omega}\xi(\omega - \omega_{a})\eta(\omega - \omega_{a})\cP(d\omega) \nonumber\\
&= \int_{\Omega}\inn{\omega - \omega_{a}}{\xi} \inn{\omega - \omega_{a}}{\eta}\cP(d\omega) \nonumber\\
&= \inn{K\eta}{\xi}\equiv  \xi\paren{ K(\eta ) }
\end{align}
where $\widehat{K}: \Omega^{*} \times \Omega^{*} \rightarrow \bR$ is a functional on $\Omega^{*}\times \Omega^{*}$.

\item Note that $K$ is self-adjoint, i.e. $\xi\paren{ K\eta } = \eta\paren{K \xi} $, or $\inn{K\eta}{\xi} = \inn{K\xi}{\eta}$.

\item Define the \emph{characteristic functional} of measure $\cP$ as a complex-valued functional on $\Omega^{*}$ given by the formula, 
\begin{align}
\phi_{\cP}(\xi) &= \int_{\Omega} \exp\paren{j\; \xi(\omega)}\cP(d\omega). \label{expr: char_functional}
\end{align}
\end{itemize}
\subsection{Gaussian measure}
\begin{itemize}
\item A measure $\cP$ defined on some algebra that contains $\srC$ is called \emph{Gaussian} if the distribution for any (continuous) \emph{linear} functional $\xi \in X^{*}$ with respect to the measure $\cP$ is a Gaussian distribution in $\bR$; i.e., 
\begin{align}
\cP\circ \xi^{-1} &= \cN(\omega_{a}, \sigma^{2}) \label{expr: Gaussian_measure}
\end{align}
for some $m, \sigma$.

\item For any \emph{finite-dimensional joint distribution}, we see that 
\begin{align}
\cP\circ \xi_{T}^{-1} &= \cN(m, K),
\end{align}
where $m$ is the mean functional and $K$ is covariance operator. 

\item Denote the class of all Radon Gaussian measures on the Borel $\sigma$-algebra $\srB$ of $\Omega$ as $\cG(\Omega)$, and the subclass of all centered Radon Gaussian measures as $\cG_{0}(\Omega)$.

\end{itemize}
\newpage
\section{Theorem}

\newpage
\section{Examples}
\newpage
\bibliographystyle{plainnat}
\bibliography{reference.bib}
\end{document}