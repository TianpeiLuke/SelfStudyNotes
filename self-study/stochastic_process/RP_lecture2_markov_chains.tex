\documentclass[11pt]{article}
\usepackage[scaled=0.92]{helvet}
\usepackage{geometry}
\geometry{letterpaper,tmargin=1in,bmargin=1in,lmargin=1in,rmargin=1in}
\usepackage[parfill]{parskip} % Activate to begin paragraphs with an empty line rather than an indent %\usepackage{graphicx}
\usepackage{amsmath,amssymb, mathrsfs,  mathtools, dsfont}
\usepackage{tabularx}
\usepackage{tikz-cd}
\usepackage[font=footnotesize,labelfont=bf]{caption}
\usepackage{graphicx}
\usepackage{xcolor}
%\usepackage[linkbordercolor ={1 1 1} ]{hyperref}
%\usepackage[sf]{titlesec}
\usepackage{natbib}
\usepackage{../../Tianpei_Report}

%\usepackage{appendix}
%\usepackage{algorithm}
%\usepackage{algorithmic}

%\renewcommand{\algorithmicrequire}{\textbf{Input:}}
%\renewcommand{\algorithmicensure}{\textbf{Output:}}



\begin{document}
\title{Lecture 2: Markov Chains}
\author{ Tianpei Xie}
\date{ Feb. 1st., 2023 }
\maketitle
\tableofcontents
\newpage
\section{Markov Chain}
\subsection{Basic Concepts}
\begin{itemize}
\item \emph{\textbf{Markov Chain}} $(X_t)_t$ is a \emph{\textbf{probabilistic graphical model}} over a chain graph $\cG=(\cV, \cE_{C})$, where each random variable $X_t$ only has exactly one children $X_{t+1}$ and one parent $X_{t-1}$. Denote the index of variable $t$ as the time. Markov chain  $(X_t)_t$ is also a \emph{\textbf{stochastic process}}. 

\item By Markov property, 
\begin{align*}
P(X_{t+1} | X_t, X_{t-1}, \ldots, X_1) &= P(X_{t+1} | X_t) .
\end{align*} It is seen that the transition probability does not depend on the time $t$, i.e. Markov chain is \textbf{time-invariant}.

\item We can see that the joint distribution on $\mb{X}_{0:t} = [X_0, \ldots, X_t]$ can be factorized by transition probabilities
\begin{align*}
P(\mb{X}_{1:t} ) &= P(X_0)\prod_{s=1}^{t}P(X_s| X_{s-1})
\end{align*} by Markov property. Denote $\pi_{0}(i) := P(X_0 = i)$ as the \emph{\textbf{initial probability}}.


\item Define the \emph{\textbf{transition kernel}} of Markov Chain as the \emph{time-invariant} \emph{\textbf{transition probability}}
\begin{align}
K(x, y)  = p(x, y) &:= P(X_{t+1} = y | X_t = x).  \label{eqn: mc_kernel}
\end{align} Then the \emph{\textbf{$m$-step transition probability}} is defined as
\begin{align}
K^{m}(x, y) &= P(X_{t+m} = y | X_t = x). \label{eqn: mc_kernel_m}
\end{align}

\item In \emph{general setting} \citep{robert1999monte}, $K: \cX \times \cB(\cX) \rightarrow \bR$ is a function so that $K(x, \cdot)$ is a \emph{\textbf{probability measure}} for all $x\in \cX$ and $K(\cdot, A)$ is a \emph{\textbf{measurable function}} for all $A\in \cB(\cX)$. $K$ can also be considered as a \emph{\textbf{functional}} such that
\begin{align*}
K\,(h(x)) &= \int h(y) K(x, dy), \quad h \in L_1(\lambda).
\end{align*} where $\lambda$ is the dominated measure.


\item For $X_t \in \cX:= \set{1, \ldots, n}$ as discrete random variable with $\abs{\cX} = n$, we can define the \textbf{\emph{transition matrix}}
\begin{align}
\mb{K} &= \brac{K(i,j)}_{n\times n} \label{eqn: mc_kernel_mat}
\end{align}

\item We can see that $\mb{X}_{(t+1):(t+m-1)} := [X_{t+1}, \ldots, X_{t+m-1}]$, the $m$-step transition can be computed using transition kernel
\begin{align}
P(X_{t+m}| X_t) &= \sum_{\mb{X}_{(t+1):(t+m-1)}}\,P(X_{t+m} |\mb{X}_{(t+1):(t+m-1)} , X_t)P(\mb{X}_{(t+1):(t+m-1)}) \nonumber\\
&= \sum_{\mb{X}_{(t+1):(t+m-1)}}\,P(X_{t+m} |X_{t+m-1})P(\mb{X}_{(t+1):(t+m-1)}) \nonumber\\
&= \ldots \nonumber\\
&= \sum_{\mb{X}_{(t+1):(t+m-1)}}\prod_{i=0}^{m-1}P(X_{t+i+1}|X_{t+i}) \label{eqn: mc_kernel_m_kernel_prod}\\
\Rightarrow \brac{K^{m}(i, j)} &= \brac{K(l, m)}^{m} = \mb{K}^{m}  \label{eqn: mc_kernel_m_kernel_prod_mat}
\end{align}

\item We have the \emph{\textbf{Chapman-Kolmogorov equation}} \citep{ross2014introduction}:
\begin{align}
K^{m+n}(x, y)  &= \sum_{z}K^{m}(x, z)K^{n}(z, y), \quad \forall x,y \in \cX \label{eqn: mc_ck_eqn}\\
\Rightarrow \mb{K}^{m+n} &= \mb{K}^{m}\mb{K}^{n} \nonumber
\end{align} That is, we split the  $(m+n)$-step path from $x\rightarrow y$ into all possible combination of a $m$-step path from $x\rightarrow z$ and a $n$-step path from $z \rightarrow y$ for some intermediate state $z$.

\item And the marginal distribution on state $X_t$ can be computed as
\begin{align}
P(X_t) &= \sum_{X_{t-1}\in \cX}P(X_t|X_{t-1})P(X_{t-1}) \nonumber\\
\Rightarrow \mb{\pi}_{t} &= \mb{K}\mb{\pi}_{t-1} \label{eqn: mc_marginal_kernel}
\end{align} where $\mb{\pi}_{t}:= [P(X_t = i)]$
\end{itemize}
\subsection{Hitting time}
\begin{itemize}
\item \begin{definition}
Define $T_{j} = \min\set{t\ge 1: X_{t} = j}$ as the time steps for Markov Chain $(X_t)_t$ to \emph{hit} state $j$ for the \emph{\textbf{first time}}. $T_j$ is called the state $j$'s \underline{\textbf{\emph{first hitting time}}}.


Denote $f_{i,j}$ be the \emph{\textbf{probability of ever hitting state}} $j$ (\emph{\textbf{within finite time}}) \emph{\textbf{starting from state}} $i$. That is
\begin{align}
f_{i,j} &:= P(T_{j} < \infty | X_{0} = i)\label{eqn: mc_hit_time_prob}
\end{align} Denote $f_{i,j}^{(m)}$ be the \emph{probability of hitting at state $j$ at time $m$} starting from state $i$
\begin{align}
f_{i,j}^{(m)} &:= P(T_{j} = m | X_{0} = i)\label{eqn: mc_hit_time_prob_m}
\end{align}

We can generalize the hitting time for a set of states $T_{A} := \min\set{t\ge 1: X_{t} \in A}$.
\end{definition}

\item We can the following the \emph{\textbf{kernel recursion formula}}
\begin{align}
K^{m}(x, y)&= \sum_{n=1}^{m} P(T_{y} = n | X_{0} = x)K^{m-n}(y, y) \label{eqn: mc_hitting_time_kernel}\\
&:= \sum_{n=1}^{m}f_{x,y}^{(n)}\,K^{m-n}(y, y). \nonumber 
\end{align} That is, we \emph{categorize} all possible $m$-step path from $x\rightarrow y$ according to the first time the path visiting $y$. (This is called the \underline{\emph{\textbf{First-Step analysis}}})

\item Similarly, we have the \textbf{\emph{hitting time recursion formula}}:
\begin{align}
f_{x,y}^{(m)} := P(T_{y} = m | X_{0} = x)&= \sum_{z \neq i}\,P(T_{y} = m - 1 | X_{0} = z)\,K(x, z) \label{eqn: mc_hitting_time_recursion}. \\
&:= \sum_{z \neq i}\,f_{z,y}^{(m-1)}\,K(x, z).\nonumber
\end{align} This formula break down the $m$-step path from $x\rightarrow y$ into two parts: a path from $x\rightarrow z$ and a $(m-1)$-step path from intermediate state $z\rightarrow y$ (This is also the \emph{First-Step analysis}).

\item Define $N(y):= \sum_{t=0}^{\infty}\ind{X_{t} = y}$ is the \emph{\textbf{total number of times hitting the state}} $y$.
\begin{align}
P(N(y) \ge 1 |\, X_{0} = x) &= P(T_{y} < \infty | X_{0} = x) = f_{x,y} \label{eqn: mc_hit_times} \\
P(N(y) \ge m |\, X_{0} = x) &= P(N(y) \ge 1 |\, X_{0} = x)P^{m-1}(N(y) \ge 1 |\, X_{0} = y) \label{eqn: mc_hit_times_hitting_time_recusion} \\
&=f_{x,y}\, f^{m-1}_{y,y}\nonumber
\end{align} Note that in order to visit $y$ at least $m$ times, we need to visit $y$ first time and stating from $y$ recurrently visit $y$ $(m-1)$ times.

The random variable $N(x)|X_0 = x$ follows a \underline{\textbf{\emph{geometric distribution}}} with mean $1/(1- f_{x,x})$.
\begin{align}
P(N(x) = m |\, X_{0} = x) &= (1- f_{x,x})f^{m-1}_{x,x} \label{eqn: mc_hit_times_m_times_recurrent}
\end{align}

\item Define $G(x, y) := \E{}{N(y) | X_{0} = x}$ as the \emph{\textbf{expected number of \textbf{total visits}}} of $y$ starting from $x$.
\begin{align}
G(x, y) &:= \E{}{N(y) | X_{0} = x} \nonumber\\
&=\E{}{\sum_{t=0}^{\infty}\ind{X_{t} = y} | X_{0} = x}  \nonumber\\
&= \sum_{t=0}^{\infty}\E{}{\ind{X_{t} = y}| X_{0} = x} =\sum_{t=0}^{\infty}K^{t}(x, y) \label{eqn: mc_expect_hit_times}
\end{align} Note that $\E{}{\sum_{t=0}^{\infty}\ind{X_{t} = y} | X_{0} = x} = \sum_{t=0}^{\infty}\E{}{\ind{X_{t} = y}| X_{0} = x}$ is true since $Z_{t} := \ind{X_{t} = y}$ is non-negative random variable.

Since $N(y)$ is geometric distributed, we can compute $G(x, y)$ via 
\begin{align}
G(x, y) &= \frac{f_{x,y}}{1- f_{y,y}} \label{eqn: mc_expect_hit_times_2}
\end{align}

\item Define $G(x,x) = \E{}{N(x) | X_{0} = x}$ as the \emph{expected number of \textbf{total returns}} starting from state $x$.
\end{itemize}


\section{Classification of States}
\subsection{Equivalence class by communication}
\begin{itemize}
\item \begin{definition}
For any pair $x, y \in \cX$, if there exists $n \in \bN_{+}$ so that $K^{n}(x, y) > 0$, then the state $y$ is \emph{\textbf{accessible}} from state $x$. This is equivalent to say that the probability of hitting time of $y$ being finite starting from $x$ is above zero, i.e. \underline{$f_{x,y} >0$}.
\end{definition}

\item If $x$ is accessible from $y$, and $y$ is accessible from $x$, then we say that $x$ and $y$ \underline{\emph{\textbf{coummunicate}}},
$x \leftrightarrow y$. It is easy to check that this is an \underline{\emph{\textbf{equivalence relation}}}:
\begin{enumerate}
\item $x \leftrightarrow x$;
\item If $x \leftrightarrow y$, then $y \leftrightarrow x$;
\item If $x \leftrightarrow z$ and $z \leftrightarrow y$, then $x \leftrightarrow y$ 
\end{enumerate}

\item Thus we can partition the state space $\cX$ into several \emph{\textbf{equivalence classes}} $\cX= \bigcup_{k}\cX^{k}$ and within each class, all states communicate to each other. 

Equivalently, it means that the kernel $\mb{K}$ can be \emph{rearranged} into a \textbf{\emph{block-diagonal matrix}}.


\item 
 \begin{definition}
A Markov Chain is \underline{\emph{\textbf{irreducible}}} if \underline{it has \textbf{only one equivalence class}}, i.e. all states in $\cX$ communicate to each other.
\end{definition}


\item Based on hitting time, we can categorize states into two groups:
 
\begin{itemize}
\item \begin{definition} A state $i$ is \underline{\emph{\textbf{recurrent}}} if and only if \underline{$f_{i,i} =  P(T_{i} < \infty | X_{0} = i) = 1$}, i.e. the Markov Chain will definitely revisit the state $i$ after stating from $i$.
\end{definition}

Note that it follows from \eqref{eqn: mc_expect_hit_times} that 
\begin{proposition} (Characterization of recurrence via \textbf{$n$-step return probabilities})\\
A state $i$ is recurrent if and only if $\sum_{t=0}^{\infty}K^{t}(i, i) = \infty$.
\end{proposition}

\item \begin{definition} A recurrent state $i$ is \emph{\textbf{positive recurrent}} if the \emph{expected returning time} $\E{}{T_{i} | X_0 = i} < \infty$; otherwise we say it is \emph{\textbf{null recurrent}}.
\end{definition}

\item \begin{definition}  A state $i$ is called \emph{\textbf{transient}} if $f_{i,i} < 1$.
\end{definition}
\end{itemize}


\item 
\begin{proposition}
The following conditions are \emph{\textbf{equivalent}}:
\begin{enumerate}
\item state $i$ is recurrent state;
\item The ever returning probability $f_{i,i} = 1$;
\item The probability of total number of visiting is $P(N(i) = \infty |\, X_{0} = j) = f_{j,i}$ and $P(N(i) = \infty |\, X_{0} = i) = 1$;
\item The expected total number of returning is infinite $G(i, i) = \infty$;
\item The sum of all $n$-step return probabilities $\sum_{t=0}^{\infty}K^{t}(i, i) = \infty$.
\end{enumerate}
\end{proposition}


\item 
\begin{proposition}
If $i$ is recurrent, and $i \rightarrow j$, then also $j \rightarrow i$.
\end{proposition}

\item 
\begin{proposition}
If $i$ is positive recurrent, and $i \leftrightarrow j$, then $j$ is also positive recurrent.
\end{proposition}

\item
\begin{proposition}
If $i$ is recurrent, and $i \rightarrow j$, then $j$ is also recurrent. Therefore, in any \textbf{equivalent class}, \textbf{either all states are recurrent} or \textbf{all are transient}. In particular, if the chain is \emph{\textbf{irreducible}}, then either all states are recurrent or all are transient.
\end{proposition}
Based above proposition, we can classify \emph{\textbf{each class}, and \textbf{an irreducible Markov Chain} as recurrent or transient}. 

\item 
\begin{proposition}
If a closed subset $S_0 \subset \cX$ only has finitely many states, then there must be at least one recurrent state. In particular any finite Markov chain must contain at least one positive recurrent state. 
\end{proposition}

\begin{proposition}
An \textbf{irreducible finite state Markov chain} must be \textbf{positive recurrent}. 
\end{proposition}


\item 
\begin{proposition}
Any recurrent class is a \textbf{closed} subset of states.
\end{proposition}

\item Let $S_{T}$ be a set of \emph{\textbf{transient} states} and $C$ be a closed set of \emph{\textbf{irreducible}}, \emph{\textbf{recurrent} state}, the \emph{\textbf{absorption probability}} is defined as
\begin{align}
p_{C}(x) &= P(T_{C} < \infty | X_{0} = x), \quad \forall x\in S_{T}.
\end{align} It is the probability of hitting recurrent state set starting from a transient state.

\item \begin{theorem}
Suppose  $S_{T}$ is a set of \textbf{transient} states and $C$ is a closed irreducible set of \textbf{recurrent} state, then the following system of equations has \textbf{unique} solution,
\begin{align}
f(x) &= \sum_{y \in C}K(x, y) + \sum_{y \in S_{T}}K(x, y)f(y), \quad \forall\, x \in S_{T} \label{eqn: trans_recur_eqn_unique_sol}
\end{align} and the unique solution is the absorption probability $f(x) = p_{C}(x)$.
\end{theorem}

\item The recurrence definition ("with \emph{infinite number of visits}") can be generalized as the \emph{\textbf{Harris recurrence}} in general theory \citep{robert1999monte}.
 \begin{definition}
A set $A$ is \emph{\textbf{Harris recurrent}} if $P(N_{A} = \infty | X_0 = x) = 1$ for all $x \in A$, where $N_A := \sum_{t=0}^{\infty}\ind{X_{t} \in A}$. The chain $(X_t)_t$ is \emph{\textbf{Harris recurrent}} if there exists a measure $p$ such that $(X_t)_t$ is $p$-\emph{irreducible} and for every set $A$ with $p(A) > 0$, $A$ is \emph{Harris recurrent}.
\end{definition}

\end{itemize}

\subsection{Foster's theorem and Poke's lemma}
\begin{itemize}
\item \begin{theorem}(\textbf{Foster's theorem})\\
Consider an irreducible Markov chain $(X_t)_t$ with state space $\cX=\set{0,1,\ldots}$ and transition matrix $\mb{K}$ and suppose there exists a function $h: \cX \rightarrow \bR$ such that
\begin{flalign*}
(1)\quad & \inf_{x\in \cX}h(x) > -\infty  &\\
(2)\quad & \sum_{y\in \cX}K(x, y)\,h(y) < \infty  \quad \forall x \in \cS &\\
(3)\quad & \sum_{y\in \cX}K(x, y)\,h(y) < h(x)  - \epsilon  \quad \forall x \not\in \cS &
\end{flalign*} for some finite set $\cS \subset \cX$ and some $\epsilon >0$, then the Markov chain $(X_t)_t$ is \textbf{positive recurrent}.
\end{theorem}  

\item \begin{lemma} (\textbf{Poke's lemma})\\
Consider an irreducible Markov chain $(X_t)_t$ with state space $\cX=\set{0,1,\ldots}$ and transition matrix $\mb{K}$. Assume that for all $x\in \cX$ and all $t\ge 0$, $\E{}{X_{t+1} | X_t = x} < \infty$ and $\lim_{i\rightarrow \infty}\sup_{j \ge i}\E{}{ X_{t+1} - X_{t} | X_t = j} < 0$. Then the Markov chain $(X_t)_t$ is \textbf{positive recurrent}.
\end{lemma}
\end{itemize}

\section{Limiting and stationary distribution}
\subsection{Property of limiting distributions}
\begin{itemize}
\item \begin{definition}
The probability of states $\set{\pi(x), \forall x\in \cX}$ is a \underline{\textbf{\emph{stationary distribution}}} if and only if 
\begin{align}
\pi(y) &= \sum_{x \in X}K(x, y)\pi(x), \forall y \in \cX \label{eqn: mc_stationary}\\
\mb{\pi}^{T} &= \mb{\pi}^{T}\,\mb{K}  \label{eqn: mc_stationary_mat}
\end{align} That is, $\mb{\pi}$ is the eigenvector of stochastic matrix $\mb{K}$ corresponding to eigenvalue $\lambda_0 = 1$.
\end{definition}


\item A stationary distribution is also called an \emph{\textbf{invariant distribution}} \citep{robert1999monte, liu2001monte}, \emph{\textbf{steady-state distribution}} \citep{ross2014introduction} or \emph{\textbf{equilibrium distribution}} \citep{brooks2011handbook, ross2014introduction}. This is due to its \emph{time invariant property} or \emph{the global balance equation} \eqref{eqn: mc_global_balance_eqn}.

\item Let the initial distribution be the stationary distribution $P(X_0 = x)= \pi(x)$. Note that 
\begin{align}
\pi_1(y) = P(X_1 = y) &= \sum_{x}K(x, y)\pi(x) = \pi(y), \forall y \in \cX.
\end{align} In other word, \emph{\textbf{the stationary distribution does not change over time}}. 

In measure theory, the invariant measure $\pi$ satisfies:
\begin{align*}
\pi(B) &= \int_{\cX} K(x, B)\pi(dx), \quad \forall B\in \cB(\cX).
\end{align*}


\item \begin{proposition}\label{thm: limit_stationary}
Suppose that the \textbf{limiting distribution} $\lim_{t\rightarrow \infty}P(X_t = y)$ exists, and 
\begin{align*}
\lim_{t\rightarrow \infty}K^{t}(x, y) &= \pi(y), \quad \forall x, y\in \cX
\end{align*} which is independent of where it starts from, then the Markov Chain has a \textbf{unique stationary distribution} and 
\begin{align}
\lim_{t\rightarrow \infty}P(X_t = y) &= \pi(y), \quad \forall y \in \cX
\end{align} i.e. the limit distribution is stationary distribution.
\end{proposition} Note that $P(X_t = y) = \sum_{x\in \cX}K^{t}(x, y)\pi_0(x)$.

\item \begin{proposition} (\textbf{Global Balance Equation})\\
The stationary distribution $\set{\pi(x), \forall x\in \cX}$ satisfies the following \textbf{global balance equation}:
\begin{align}
\sum_{j \in \cX}\pi(i)K(i, j) &=  \sum_{j \in \cX}\pi(j)K(j, i).  \label{eqn: mc_global_balance_eqn}
\end{align} This means the total flow out of $i$ (LHS) is equal to the total flow into $i$ (RHS) in steady state.
\end{proposition}

\item \begin{proposition} (\textbf{Detailed Balance Equation})\\
For distribution $\set{\pi(x), \forall x\in \cX}$, if the following \textbf{detailed balance equation} is satisfied
\begin{align}
\pi(i)K(i, j) &=  \pi(j)K(j, i),\quad \forall i, j\in \cX  \label{eqn: mc_detailed_balance_eqn}
\end{align} then $\set{\pi(x), \forall x\in \cX}$ is a stationary distribution.
\end{proposition}

\item \begin{definition}
Define $\mu_{i} := \E{}{T_{i} | X_{0} = i}$ as the \underline{\emph{\textbf{expected first return time}}}, i.e. the number of transition that it takes for Markov chain  when starting from state $i$ and returning to that state.
\end{definition}


\item Let $G^{(n)}(x, y) = \E{}{N^{(n)}(y) | X_{0} = x}$ where $N^{(n)}(y) = \sum_{t=0}^{n}\ind{X_{t} = y}$. $N^{(n)}(y)$ is the total amount of time staying at state $y$ within $n$ transitions. Then
\begin{itemize}
\item \begin{theorem}
For \textbf{transient state} $y$
\begin{align*}
&\lim_{n\rightarrow \infty}N^{(n)} < \infty, \quad (\text{w.p.1})\\
 \Rightarrow &\lim_{n\rightarrow \infty}\frac{N^{(n)}}{n} = 0, \quad (\text{w.p.1})\\
&\lim_{n\rightarrow \infty}G^{(n)}(x, y) = \frac{f_{x,y}}{1 - f_{y,y}} < \infty\\
\Rightarrow &\lim_{n\rightarrow \infty}\frac{G^{(n)}(x, y)}{n} = 0, \quad \forall x\in \cX
\end{align*} That is, the frequency of visiting transient state $y$ goes to $0$ as $n\rightarrow \infty$. 
\end{theorem}

\item \begin{theorem}
For \textbf{recurrent state} $y$
\begin{align*}
&\lim_{n\rightarrow \infty}\frac{N^{(n)}}{n} = \frac{\ind{T_{y} < \infty}}{\mu_{y}}, \quad (\text{w.p.1})\\
&\lim_{n\rightarrow \infty}\frac{G^{(n)}(x, y)}{n} = \frac{f_{x,y}}{\mu_{y}}, \quad \forall x\in \cX
\end{align*} where $\mu_{y} := \E{}{T_{y} | X_{0} = y}$  is the \underline{\textbf{expected first return time}} of state $y$.
That is, the frequency of visiting \textbf{positive recurrent} state $y$ converge to $\frac{1}{\mu_{y}}$ as $n\rightarrow \infty$; otherwise for \textbf{null recurrent state} $y$, it converges to zero.
\end{theorem}
\end{itemize}

\item 
\begin{theorem} (Stationary distribution for transient and null recurrent states)\\
Let $\set{\pi(x), \forall x\in \cX}$  be stationary distribution. If $x \in \cX$ is \textbf{transient} or \textbf{null recurrent} state, then 
\begin{align*}
\pi(x) = 0.
\end{align*} 
\end{theorem}

\item \begin{theorem} (\textbf{Kac's Theorem})\citep{ross2014introduction}\\
An \underline{\textbf{irreducible recurrent}} Markov Chain has a \textbf{unique} \textbf{stationary distribution} $\set{\pi(x)}$, given 
\begin{align}
\pi(x) &= \frac{1}{\mu_x}, \quad \forall x\in \cX
\end{align} where $\mu_{x} := \E{}{T_{x} | X_{0} = x}$  is the \textbf{expected first return time} of state $x$.
\end{theorem} It implies that as $n\rightarrow \infty$, for any state $x\in \cX$, the fraction of time that Markov Chain stays at $x$ is unchanged and is the reciprocal of the expected first return time.
\end{itemize}

\subsection{Ergodicity}
\begin{itemize}
\item Under what condition we have $\forall y \in \cX$,
\begin{align}
\lim_{t\rightarrow \infty}P(X_t = y) &= \pi(y) ?
\end{align} This is the question that ergodicity property tries to answer.

\item \begin{definition}
The \underline{\textbf{\emph{periodicity}}} of a state $x \in \cX$ is defined as 
\begin{align}
d(x) &= \text{g.c.d.}\set{t\ge 0: K^{t}(x, x) > 0}
\end{align} where $\text{g.c.d.}$ is the \emph{\textbf{greatest common divisor}}. 
\end{definition}

\item \begin{theorem}
If $x\leftrightarrow y$ (i.e. $f_{x,y} > 0$ and $f_{y,x} > 0$), then $d(x) = d(y)$.
\end{theorem}

\item \begin{definition}
If $d(x) \ge 2$, then state $x$ is \emph{\textbf{periodic}}. If $d(x) = 1$, then state $x$ is \underline{\emph{\textbf{aperiodic}}}
\end{definition}
Based on above theorem, the periodicity property is \emph{closed} under the equivalence class $C$.

\item \begin{definition}
A Markov Chain is  \underline{\emph{\textbf{irreducible}}, \emph{\textbf{positive recurrent}} and \emph{\textbf{aperiodic}}}, then it is called \underline{\emph{\textbf{ergodic}}}.
\end{definition}

\item \begin{theorem}\label{thm: ergodic}
A Markov Chain is \textbf{irreducible and positive recurrent} having stationary distribution $\mb{\pi}$. 
\begin{itemize}
\item If the Markov Chain is also \textbf{aperiodic}, then 
\begin{align}
\lim_{t\rightarrow \infty}K^{t}(x, y) &= \pi(y), \quad \forall x, y\in \cX \label{eqn: mc_ergodic}
\end{align} 

\item If the Markov chain is \textbf{periodic} with period $d$, then there exists $r \in \bN_{+}$,  $0\le r\le d-1$ such that
\begin{align}
K^{t}(x, y) &= 0, \quad \forall x, y\in \cX \label{eqn: mc_periodic}
\end{align} \textbf{unless} $t = m\,d + r$ for some $m\in \bN_{+}$ and
\begin{align}
\lim_{m\rightarrow \infty}K^{m\,d + r}(x, y) &= d\,\pi(y), \quad \forall x, y\in \cX \label{eqn: mc_periodic_2}
\end{align} Note that periodicity only appears on discrete time Markov chain.
\end{itemize}
\end{theorem}
Based on the Theorem \ref{thm: ergodic} and Proposition \ref{thm: limit_stationary}, when a Markov chain is ergodic, its marginal state distribution will converge to the stationary distribution.
\end{itemize}

\subsection{Mean hitting time formula}
\begin{itemize}
\item \begin{definition}
Let $(X_t)_t$ be a stochastic process and let $\set{\cF_{t}, t\ge 0}$ be an increasing family of $\sigma$-field. 

A random variable $T: (\Omega, \cF) \rightarrow (\bN_{+}\cup\set{+\infty}, 2^{\bN_{+}\cup\set{+\infty}})$ is called a \underline{\textbf{\emph{stopping time}}} with respect to $\set{\cF_{t}, t\ge 0}$, if $\forall k \ge 0$, $\ind{T = k}$ is \textbf{$\cF_{t}$-measurable} (i.e. $\set{T = k} \in \cF_{t}$ and $\cF = \bigcup_{t\ge 0} \cF_t$.)
\end{definition}

For Markov chain $(X_t)_t$, the \emph{\textbf{first hitting time}} is defined as $T_{A}^{(1)} = \min\set{t > 0: X_t \in A}$. Also we can define n-th hitting time as $T_{A}^{(n)} = \min\set{t > T_{A}^{(n-1)}: X_t \in A}$. All of these $\set{T_{A}^{(1)}, \ldots, T_{A}^{(n)}, \ldots }$ are all \emph{\textbf{stopping time}}. 

\item \begin{theorem}(\textbf{Strong Markov property}) \citep{robert1999monte}\\
For every initial distribution $\mb{\pi}$ and every stopping time $\tau$ which is almost surely finite,
\begin{align}
\E{}{h(X_{\tau+1},X_{\tau+2},\ldots)\,|\,x_{1}\ldots,x_{\tau}} &= \E{}{h(X_1 ,X_2, \ldots )}, \label{eqn: mc_strong_markov}
\end{align} provided the expectations exist.
\end{theorem} We can thus condition on a random number of instants while keeping the fundamental properties of a Markov chain.
We can proof that for the intervals $\tau_1 = T_{x}^{(1)}$, $\tau_{i} := \tau_{x}^{(i)} - \tau_{x}^{(i-1)}$, $i=2,\ldots,$, then $\set{\tau_1,\ldots, \tau_n,\ldots}$ are i.i.d.
\item 
\begin{theorem}
Let $(X_t)_t$ be a \textbf{positive recurrent} Markov chain with state space $\cX$ and stationary distribution $\mb{\pi}$. Let $T$ be any \textbf{stopping time} of $(X_t)_t$ such that for arbitrary $x \in \cX$, $X_{T} = x$. Then for all $y \in \cX$, 
\begin{align}
\E{T}{\sum_{t=0}^{T-1}K^{t}(x,y)\,|\,x} = \E{}{\sum_{t=0}^{T-1}\ind{X_{t} = y}\,|\,X_0 = x} &= \pi(y)\,\E{}{T | X_{0} = x}. \label{eqn: stopping_time_pos_recur}
\end{align}
\end{theorem}
%That is, the expected number of times visiting $y$ before stopping time $T$ is equal to $\pi(y)$ times the expected stopping time $T$. Therefore, 
%\begin{align*}
%\mu_x = \E{}{T | X_{0} = x}&= \frac{\E{T}{\sum_{t=0}^{T-1}K^{t}(x,y)\,|\,X_0 = x} }{\pi(y) }
%\end{align*}

\item \begin{theorem}
Let $i \neq j$ and $T$ be the first time returning $i$ after visiting $j$, $T = \min\{t > \tau_{j}, X_t = i| X_0 = i\}$, $\tau_j = \min\set{t > 0: X_t = j}$ and $\tau_i = \min\set{t > 0: X_t = i}$. Then
\begin{flalign*}
(1)\quad &\E{}{T | X_0 = i} = \E{}{\tau_i | X_0 = i} + \E{}{\tau_j | X_0 = i}&\\
(2)\quad &\E{}{\sum_{t=0}^{T-1}\ind{X_t = j}\,|\,X_0 = i} = \E{}{\sum_{t=\tau_j}^{T-1}\ind{X_t = j}\,|\,X_0 = i} + \E{}{\sum_{t=0}^{\tau_i-1}\ind{X_t = j}\,|\,X_0 = j}&\\
(3)\quad &\E{}{\sum_{t=0}^{\tau_i-1}\ind{X_t = j}\,|\,X_0 = j} = \pi(j)\paren{\E{}{\tau_j\,|\,X_0 = i} + \E{}{\tau_i\,|\,X_0 = j} }&
\end{flalign*}
\end{theorem} The "\emph{number of visits to $j$ before first returning to $i$}" is geometric distributed with mean $p:=P(\tau_j > \tau_i | X_0 = i)$, thus (3) can be computed as 
\begin{align*}
\E{}{\sum_{t=0}^{\tau_i-1}\ind{X_t = j}\,|\,X_0 = j}&= \frac{1}{P(\tau_j > \tau_i | X_0 = i)}.
\end{align*}  



\item \begin{definition}
For finite state, ergodic Markov chain $(X_t)_t$ with stationary distribution $\mb{\pi}$, define the \underline{\textbf{\emph{fundamental matrix}}} as
\begin{align}
\mb{Z} &:= \paren{\mb{I} - \paren{\mb{K} - \mb{1}\mb{\pi}^{T}}}^{-1}  \label{eqn: mc_fundamental_mat}\\
&= \mb{I} + \sum_{t\ge 0}\paren{\mb{K}^t - \mb{1}\mb{\pi}^{T}}\nonumber
\end{align} and its $(i,j)$ element is 
\begin{align}
Z_{i,j} &= \sum_{t=0}^{\infty}(K^{t}(i, j) - \pi_{j}) \label{eqn: mc_fundamental_mat_elem}
\end{align}
\end{definition} Note that $\mb{Z} = \paren{\mb{I} - \mb{Q}}^{-1} = \sum_{t=0}^{\infty}\mb{Q}^{t}$, where $\mb{Q}:= \paren{\mb{K} - \mb{1}\mb{\pi}^{T}}$.

\item 
\begin{theorem} (\textbf{Mean hitting time formula})\\
For finite state, ergodic Markov chain $(X_t)_t$ with stationary distribution $\mb{\pi}$, and $Z_{i,j}$ is defined as \eqref{eqn: mc_fundamental_mat_elem}, then 
\begin{align}
Z_{i,i} &= \pi(i)\,\E{}{\tau_i | X_0\sim \mb{\pi}}, \quad i\in \cX, \label{eqn: mc_mean_hitting_time}\\
Z_{j,j} - Z_{i,j} &= \pi(j)\,\E{}{\tau_j | X_0 = i}, \quad i, j \in \cX, \label{eqn: mc_mean_hitting_time_2}
\end{align} where $\tau_i:= \min\set{t\ge 0: X_{t} = i}$ is the stopping time/first hitting time.
Thus 
\begin{align*}
\E{}{\tau_i | X_0\sim \mb{\pi}} &= \frac{Z_{i,i}}{\pi(i)} \\
\E{}{\tau_j | X_0 = i} &= \frac{\paren{Z_{j,j} - Z_{i,j}}}{\pi(j)}
\end{align*}
\end{theorem}
\end{itemize}

\section{Time-reversible Markov Chain}
\begin{itemize}
\item \begin{definition}
A Markov chain $(X_t)_t$ is called \underline{\emph{\textbf{time-reversible}}}, if it has stationary distribution $\mb{\pi}$ and the detailed balance equation is satisfied:
\begin{align}
\pi(i)K(i, j) &=  \pi(j)K(j, i),\quad \forall i, j\in \cX.  \label{eqn: mc_detailed_balance_eqn_time_rev}
\end{align}
\end{definition}

\item From this definition, we can see that reversibility implies that the stationary distribution exists, but not \emph{vice versa}. 

\item The reversed process $(Y_{k})_k := (X_{t-k})_{k}$ is a Markov chain and its transition probability 
\begin{align}
Q(i,j) &= \frac{\pi(j)K(j,i)}{\pi(i)} \label{eqn: mc_rev_kernel}
\end{align} Note that $(Y_{k})_k$ and $(X_t)_t$ are statistically equivalent since $Q(i,j) = K(i,j)$.

\item 
\begin{theorem}
An ergodic Markov chain $(X_t)_t$ for which $K(i,j) =0$ whenever $K(j, i) = 0$ is \textbf{time-reversible} if and only if starting from any state $i$, any path back to $i$ has the \textbf{same probability} as its reverse path. That is, for path $i \rightarrow i_1\rightarrow i_2 \rightarrow \ldots \rightarrow i_k \rightarrow i$ and its reverse path $i \leftarrow i_1 \leftarrow i_2 \leftarrow \ldots  \leftarrow i_k \leftarrow i$
\begin{align}
K(i, i_1)\,K(i_1, i_2)\,\ldots\,K(i_k, i) &= K(i, i_k)\,\ldots\,K(i_2, i_{1})\,K(i_1, i), \quad \forall i,i_1,\ldots, i_k\in \cX \label{eqn: mc_path_rev}
\end{align} 
\end{theorem}

\item 
\begin{theorem} (\textbf{Reversal Test})\\
Let $\mb{K}$ be a stochastic matrix indexed by a countable set $\cX$ and let $\mb{\pi}$ be a probability distribution on $\cX$. Let $\mb{Q}$ be a stochastic matrix indexed by $\cX$ such that
\begin{align}
\pi(i)Q(i, j) &=  \pi(j)K(j, i),\quad \forall i, j\in \cX.  \label{eqn: mc_time_rev_test}
\end{align}  Then $\mb{\pi}$ is a stationary distribution of $\mb{K}$
\end{theorem}

\item 
\begin{proposition}
For finite state, ergodic Markov chain $(X_t)_t$ with stationary distribution $\mb{\pi}$, and $Z_{i,j}$ is defined as \eqref{eqn: mc_fundamental_mat_elem}, then $(X_t)_t$ is time-reversible if and only if
\begin{align}
\pi(i)Z_{i,j} &=  \pi(j)Z_{j,i},\quad \forall i, j\in \cX.  \label{eqn: mc_rev_fundamental_mat}
\end{align}
\end{proposition}
Note that $\pi(i)\E{}{\tau_{j} | X_0 = i}  \neq  \pi(j)\E{}{\tau_{i} | X_0 = j}$.

\item \begin{theorem} (\textbf{Cycle-tour property})\\
For states $(i_0,i_1,\ldots, i_{m}) \subset \cX$ of a time-reversible Markov chain, 
\begin{align}
&\E{}{\tau_{i_1} | X_0 = i_0} + \E{}{\tau_{i_2} | X_0 = i_1} + \ldots + \E{}{\tau_{i_0} | X_0 = i_{m}} \nonumber \\
&= \E{}{\tau_{i_m} | X_0 = i_{0}} + \E{}{\tau_{i_{m-1}} | X_0 = i_m} + \ldots +  \E{}{\tau_{i_0} | X_0 = i_1}\label{eqn: mc_cycle_tour}
\end{align}
\end{theorem}
\end{itemize}

\section{Ergodic Theorem and Central Limit Theorem}
\begin{itemize}
\item Consider the empirical mean of samples generated by Markov Chain
\begin{align}
S_T(h) &= \frac{1}{T}\sum_{t=1}^{T}h(X_t).  \label{eqn: mc_sample_mean}
\end{align} We are considering the limit behavior of \eqref{eqn: mc_sample_mean}. 

\item 
\begin{theorem} (\textbf{Ergodic Theorem})  \citep{robert1999monte} \\
If $(X_t)_t$ is Harris recurrent with a $\sigma$-finite invariant measure $\pi$, then for any $f, g \in L_1(\pi)$ with $\E{\pi}{g} \neq 0$, 
\begin{align}
\lim_{T\rightarrow \infty}\frac{S_T(f)}{S_T(g)} &= \frac{\E{\pi}{f}}{\E{\pi}{g}} = \frac{\int f(x)d\pi(x)}{\int g(x)d\pi(x)} \label{eqn: ergodic theorem}
\end{align}
\end{theorem}
It can be shown that if $(X_t)_t$  is \emph{\textbf{Harris positive}} with \textbf{stationary distribution} $\pi$ and if $S_T(h)$ \emph{converges} $\mu_0$-almost surely ($\mu_0$ a.s.) to $\E{\pi}{h}$, for an initial distribution $\mu_0$, this convergence occurs for \emph{\textbf{every initial distribution}} $\mu$·

\begin{corollary}\citep{liu2001monte}\\
If a \textbf{finite state-space} Markov chain $(X_t)_t$ is irreducible and aperiodic with stationary distribution $\pi$, then $S_T(h)$ converges to $\E{\pi}{h}$ \textbf{almost surely} for any initial distribution $\mu$.
\end{corollary}


\item \begin{theorem}(\textbf{Central Limit Theorem for discrete atoms})\\
If $(X_t)_t$ is Harris positive recurrent with an atom $\alpha$ such that
\begin{align*}
\E{}{T^2_{\alpha} | X_0 \in \alpha} < \infty, \qquad \E{}{\paren{\sum_{t=1}^{T_{\alpha}}\abs{h(X_t)}}^2 \,|\, X_0 \in \alpha} < \infty, \\
\text{and the variance }\sigma_{h}^2:= \pi(\alpha)\;\E{}{\paren{\sum_{t=1}^{T_{\alpha}}\set{h(X_t) - \E{\mb{\pi}}{h(X)}}}^2 \,|\, X_0\in \alpha} > 0,
\end{align*}
then the \textbf{Central Limit Theorem} applies:
\begin{align}
\sqrt{T}\paren{S_T(h) - \E{\pi}{h}} \stackrel{\cL}{\sim} \cN(0, \sigma_{h}^2) \label{eqn: mc_clt_discrete}
\end{align} 
\end{theorem}

\begin{corollary}\citep{liu2001monte}\\
For \textbf{finite state-space}, irreducible and aperiodic Markov chain $(X_t)_t$, the Central Limit Theorem holds, i.e. $\sqrt{T}\paren{S_T(h) - \E{\pi}{h}} \stackrel{\cL}{\sim} \cN(0, \sigma_{h}^2)$ for any initial distribution $\mu$.
\end{corollary}

\item \begin{theorem} (\textbf{Central Limit Theorem for reversible chains})\\
If $(X_t)_t$ is \underline{\textbf{aperiodic, irreducible, and reversible}} with stationary distribution $\mb{\pi}$, the \textbf{Central Limit Theorem} applies when
\begin{align}
0< \sigma_{h}^2 &= \E{\mb{\pi}}{h^2(X_t)} + 2\sum_{s=1}^{\infty}\E{\mb{\pi}}{h(X_t)\, h(X_{t+s})} < \infty. \label{eqn: mc_asym_variance}
\end{align}
\end{theorem}

%
%\item \begin{theorem} \citep{robert1999monte}\\
%If $(X_t)_t$ is \underline{\textbf{aperiodic, irreducible, positive Harris recurrent}} with invariant distribution $\mb{\pi}$ and \underline{\textbf{geometrically ergodic}}, and if, in addition,
%\begin{align*}
%\E{\mb{\pi}}{\abs{h(X)}^{2+\epsilon}} < \infty
%\end{align*} for some $\epsilon >0$, then 
%\begin{align*}
%\sqrt{n}\paren{S_{n}(h)/n - \E{\mb{\pi}}{h(X)}} \stackrel{\cL}{\sim} \cN(0, \sigma^2)
%\end{align*} where $\sigma$ is given in \eqref{eqn: mc_asym_variance}.
%\end{theorem}
\end{itemize}

\newpage
\bibliographystyle{plainnat}
\bibliography{reference.bib}
\end{document}