\documentclass[11pt]{article}
\usepackage[scaled=0.92]{helvet}
\usepackage{geometry}
\geometry{letterpaper,tmargin=1in,bmargin=1in,lmargin=1in,rmargin=1in}
\usepackage[parfill]{parskip} % Activate to begin paragraphs with an empty line rather than an indent %\usepackage{graphicx}
\usepackage{amsmath,amssymb, mathrsfs,  mathtools, dsfont}
\usepackage{tabularx}
\usepackage{tikz-cd}
\usepackage[font=footnotesize,labelfont=bf]{caption}
\usepackage{graphicx}
\usepackage{xcolor}
%\usepackage[linkbordercolor ={1 1 1} ]{hyperref}
%\usepackage[sf]{titlesec}
\usepackage{natbib}
%\usepackage{tikz-cd}

\usepackage{../../Tianpei_Report}

%\usepackage{appendix}
%\usepackage{algorithm}
%\usepackage{algorithmic}

%\renewcommand{\algorithmicrequire}{\textbf{Input:}}
%\renewcommand{\algorithmicensure}{\textbf{Output:}}



\begin{document}
\title{Summary Part 1: Probabilistic Methods for Non-Asymptotic Analysis}
\author{ Tianpei Xie}
\date{Jan. 26th., 2023 }
\maketitle
\tableofcontents
\newpage
\section{Basic Inequalities}
\subsection{Arithmetic, Calculus and Algebra}
\begin{itemize}
\item \begin{remark} (\textbf{\emph{Basic Inequalities}}) 
\begin{enumerate}
\item \textbf{\emph{Arithmetic Mean-Geometric Mean Inequality}}:
\begin{align*}
\sum_{i=1}^n \alpha_i x_i &\ge \prod_{i=1}^{n}x_i^{\alpha_i}
\end{align*} where $\sum_{i=1}^{n}\alpha_i = 1$ and $\alpha_i \ge 0$. Simple case, $\frac{a + b}{2} \ge \sqrt{ab}$

\item $a^2 + b^2 \ge \pm 2ab$; Also $a + b \ge 2\sqrt{ab}$
\item  $(a + b)^2 \le 2(a^2 + b^2)$; Also $(\sqrt{a} + \sqrt{b})^2 \le 2(a + b)$
\item $x + 1 \le e^x$ 
\item $\log x \le x - 1$ for $x > 0$ 
\item $e^s - e^t \le e^t(s - t)$  for $s \ge t$
\item \begin{align*}
h(x) := (x + 1)\log(x + 1) - x &\ge \frac{x^2}{2(1 + x/3)}, \quad \text{ for }x >0 
\end{align*} 
\item \begin{align*}
-\log(1- x) - x &\le \frac{x^2}{2(1 - x)}, \quad \text{ for }x \in (0, 1) 
\end{align*}
\item \begin{align*}
h_1(x) := 1 + x - \sqrt{1 + 2x} &\ge \frac{x^2}{2(1 + x)}, \quad \text{ for }x > 0.
\end{align*}
\item \textbf{\emph{Log-Sum Inequality}}:\\
 For non-negative numbers $a_1 \xdotx{,} a_n$ and $b_1 \xdotx{,} b_n$,
\begin{align*}
\sum_{i=1}^{n}a_i \log\frac{a_i}{b_i} \ge \paren{\sum_{i=1}^{n}a_i} \log\frac{\sum_{i=1}^{n}a_i}{\sum_{i=1}^{n}b_i} 
\end{align*} with equality if and only if $\frac{a_i}{b_i}$ is constant.
\item Taylor series: 
\begin{align*}
e^x &:= \sum_{n=0}^{\infty} \frac{x^n}{n!} = 1+ x + \frac{x^2}{2} + \ldots, \quad \forall x \\
\log(1 + x) &:= \sum_{n=1}^{\infty}\frac{(-1)^{n+1}}{n}x^n = x - \frac{x^2}{2} + \ldots, \quad \abs{x} < 1\\
\frac{1}{1 - x}&:=  \sum_{n=0}^{\infty} x^n = 1 + x + x^2 + \ldots, \quad \abs{x} < 1\\
(1 + x)^{\alpha} &:= \sum_{n=0}^{\infty}{\alpha \choose n} x^n
\end{align*}
\end{enumerate}
\end{remark}
\end{itemize}
\subsection{Function Space, Convexity and Duality}
\begin{itemize}
\item \begin{proposition} (\textbf{Jensen's inequality}) \citep{vershynin2018high}\\
Let $(\Omega, \srF, \bP)$ be a probability space. Let $f:\Omega \to \bR$ be a $\bP$-measurable function and  $\varphi: \bR \to \bR$ be \textbf{convex function}. Then
\begin{align}
\varphi\paren{\E{}{X}} := \varphi\paren{\int X d\bP} &\le \int \varphi \circ X d\bP := \E{}{\varphi\paren{X}}. \label{ineqn: jensen}
\end{align}
\end{proposition}

\item \begin{remark}
As a simple consequence of Jensen's inequality, $\norm{X}{L^p}$ is an \emph{\textbf{increasing function in $p$}}, that is
\begin{align}
\norm{X}{L^p} \le \norm{X}{L^q}\,\quad \text{for any } 1 \le p \le q \le \infty  \label{ineqn: lp_norm}
\end{align}
This inequality follows since $\varphi(x) = x^{q/p}$ is a \emph{convex function} if $q/p \ge 1$.
\end{remark}

\item \begin{proposition} (\textbf{Minkowski's inequality}) \citep{vershynin2018high}\\
For any $p\in [1, \infty]$, $X, Y \in L^p(\Omega, \bP)$, 
\begin{align}
\norm{X+ Y}{L^p} \le \norm{X}{L^p} + \norm{Y}{L^p},  \label{ineqn: norm_triangle_inequality}
\end{align} which implies that $\norm{\cdot}{L^p}$ is a norm.
\end{proposition}

\item \begin{proposition} (\textbf{Cauchy-Schwarz inequality}) \citep{vershynin2018high}\\
For any random variables $X, Y \in L^2(\Omega, \bP)$, the following inequality is satisfied:
\begin{align}
\abs{\inn{X}{Y}_{L^2}} := \abs{\E{}{XY}} \le \norm{X}{L^2} \, \norm{Y}{L^2}. \label{ineqn: cauchy_schwarz_inequality}
\end{align}
\end{proposition}

This inequalities can be extended to \emph{conjugate spaces} $L^p$ and $L^q$ 
 \begin{proposition} (\textbf{H\"older's inequality}) \citep{vershynin2018high}\\
For $p,q \in (1, \infty)$, $1/p + 1/q = 1$, then the random variables $X \in L^p(\Omega, \bP)$, $Y \in L^q(\Omega, \bP)$ satisfy
\begin{align}
\abs{\inn{X}{Y}_{L^2}} := \abs{\E{}{XY}}  \le \norm{X}{L^p} \, \norm{Y}{L^q}. \label{ineqn: holder_inequality}
\end{align}
\end{proposition}

\item \begin{definition}(\textbf{\emph{Legendre Transform}}) \\
Let $\cX \subset \bR^n$ be a \emph{\textbf{convex set}}, and $f: \cX \to \bR$ a \textbf{\emph{convex}} function; then its \underline{\emph{\textbf{Legendre transform}}} is the function  $f^{*}: \cX^{*}\to \bR$ defined by
\begin{align*}
f^{*}(x^{*}) &= \sup _{x\in \cX}(\inn{x^{*}}{x} - f(x)), \quad x^{*}\in \cX^{*}
\end{align*}
where $\sup$ denotes the supremum, and the domain $\cX^{*}$  is
\begin{align*}
\cX^{*} &= \set{x^{*}\in \bR^n: \sup _{x\in \cX}(\inn{x^{*}}{x}-f(x))<\infty }.
\end{align*} The function $f^{*}$ is called \underline{\emph{the \textbf{convex conjugate function}}} of $f$.
\end{definition}

\item \begin{theorem} (\textbf{Fenchel's Inequality / Fenchel-Young Inequality})\\
Suppose $f^{*}: \cX^{*} \to \bR$ is the  convex conjugate of function $f: \cX \to \bR$. For every $x \in \cX$ and $p \in \cX^{*}$, i.e., independent $(x, p)$ pairs,
\begin{align}
\inn{p}{x} &\le f(x) + f^{*}(p). \label{ineqn: fenchel_inequality}
\end{align}
\end{theorem}

\item \begin{theorem} (\textbf{Young's Convolution Inequality})\\
Suppose  $f$ is in the Lebesgue space $L^{p}(\bR^{d})$ and $g$ is in $L^{q}(\bR^{d})$ and $1/p + 1/q = 1/r + 1$ with $ 1 \leq p,q,r \leq \infty$. Then
\begin{align}
\norm{f * g}{r} \le \norm{f}{p} \norm{g}{q}. \label{ineqn: fenchel_convolution_inequality}
\end{align}
Here the star denotes \textbf{convolution}:
\begin{align*}
(f*g)(t) &= \int_{\bR^{d}} f(t - \tau)g(\tau) d\tau
\end{align*}
\end{theorem}
\end{itemize}

\subsection{Probability Theory}
\begin{itemize}
\item Assume a probability space $(\Omega, \srF, \bP)$ and a random variable $X: \Omega \to \bR$ is a real-valued measurable function on $\Omega$.

\item For a random variable $X$, the \emph{\textbf{expectation}} and \emph{\textbf{variance}} are denoted as
\begin{align*}
\E{}{X} &= \int X d\bP \\
Var(X) &= \E{}{\paren{X - \E{}{X}}^2}
\end{align*}

\item The \emph{\textbf{moment generating function}} of $X$ and its \emph{\textbf{logarithm}} are denoted as
\begin{align*}
M_{X}(\lambda) &:= \E{}{e^{\lambda X}} \\
\psi_{X}(\lambda) &:= \log   \E{}{e^{\lambda X}}
\end{align*}

\item For $p > 0$, \emph{\textbf{the $p$-th moment} of $X$} is defined as $\E{}{X^p}$, and the \emph{\textbf{$p$-th absolute moment}} is $\E{}{\abs{X}^p}$.

\item The \emph{\textbf{$L^p$ norm}} of $X$ is
\begin{align*}
\norm{X}{L^p} &:= \E{}{\abs{X}^p}^{1/p}
\end{align*} where $1 \le p  < \infty$. Note that the $L^p$ space is a \emph{Banach space}, which is defined as
\begin{align*}
L^{p}(\Omega, \bP) := \set{X: \norm{X}{L^p} < \infty }.
\end{align*}

\item \emph{The \textbf{essential supremum}} of $\abs{X}$ is the \emph{\textbf{$L^\infty$ norm}} of $X$
\begin{align*}
\norm{X}{L^\infty} &:= \text{ess sup}\abs{X}
\end{align*} Similarly, $L^{\infty}$ is a Banach space as well
\begin{align*}
L^{\infty}(\Omega, \bP) := \set{X: \norm{X}{L^\infty} < \infty }.
\end{align*}

\item For $p=2$, $L^2$ space is a \emph{Hilbert space} with inner product between random variables $X, Y \in L^2(\Omega, \bP)$
\begin{align*}
\inn{X}{Y}_{L^2} &:= \E{}{XY} = \int X Y d\bP
\end{align*} The \emph{\textbf{standard deviation}} is 
\begin{align*}
\sigma(X) &= \paren{Var(X) }^{1/2} = \norm{X - \E{}{X}}{L^2}.
\end{align*} The \emph{\textbf{covariance}} is defined as 
\begin{align*}
cov(X, Y) &:= \inn{X- \E{}{X}}{Y - \E{}{Y}} \\
&= \E{}{\paren{X- \E{}{X}}\paren{Y - \E{}{Y}}}
\end{align*} When we consider random variables as vectors in the Hilbert space $L^2$, the identity above gives a \emph{\textbf{geometric interpretation} of the notion of covariance}. The more the vectors $X - \E{}{X}$ and $Y - \E{}{Y}$ are aligned with each other, the bigger their inner product and covariance are.

\item The \emph{\textbf{cumulative distribution function (CDF)}} is defined as
\begin{align*}
F_{X}(t) &:= \bP\brac{X \le t}, \quad t\in \bR.
\end{align*}

The following result is important 
\begin{lemma} (\textbf{Integral Identity}).  \citep{vershynin2018high}\\
Let $X$ be a \textbf{non-negative} random variable.
Then
\begin{align}
\E{}{X} &= \int_{0}^{\infty}\bP\brac{X > t} dt. \label{eqn: integral_identity}
\end{align}
The two sides of this identity are either finite or infinite simultaneously.
\end{lemma}

\item \begin{theorem} (\textbf{Central Limit Theorem, Linderberg-Le{\'v}y})\\
Let $X_1 \xdotx{,} X_n$ be \textbf{independent identically distributed} random variables with mean $\E{}{X_i} = 0$ and variance $\text{Var}(X_i) = 1$. Then 
\begin{align}
\frac{1}{\sqrt{n}}\sum_{i=1}^{n}X_i  \stackrel{d}{\rightarrow} N(0, 1) \label{eqn: central_limit_theorem} \\
\text{i.e. } \lim\limits_{n \to \infty}\sup_{t \in \bR}\abs{\bP\set{\frac{1}{\sqrt{n}}\sum_{i=1}^{n}X_i \le t} - \Phi(t)} &= 0 \nonumber
\end{align} where $\Phi(t) = \int_{-\infty}^{t}\frac{1}{\sqrt{2\pi}}e^{-u^2/2}du = \bP\set{g \le t}$ for some Gaussian variable $g$.
\end{theorem}

\item \begin{theorem} (\textbf{Central Limit Theorem, Nonasymptotic, Berry-Esseen}) \citep{vershynin2018high} \\
Let $X_1 \xdotx{,} X_n$ be \textbf{independent identically distributed} random variables with mean $\E{}{X_i} = 0$, variance $\text{Var}(X_i) = \sigma^2$ and $\rho := \E{}{\abs{X_i}^3} <\infty$. Then with some constant $C > 0$, 
\begin{align}
\sup_{t \in \bR}\abs{\bP\set{\frac{1}{\sigma \sqrt{n}}\sum_{i=1}^{n}X_i \le t} - \Phi(t)} &\le  \frac{C}{ \sigma^3\sqrt{n}}\rho \label{eqn: central_limit_theorem_nonasym} 
\end{align} where $\Phi(t) = \int_{-\infty}^{t}\frac{1}{\sqrt{2\pi}}e^{-u^2/2}du = \bP\set{g \le t}$ for some Gaussian variable $g$.
\end{theorem}

\item \begin{remark} The \emph{Berry-Esseen} version of \emph{central limit theorem} is \emph{\textbf{non-asymptotic}} and it has a bound
\begin{align*}
\bP\set{\frac{1}{\sqrt{n}}\sum_{i=1}^{n}X_i \le t} &\le \bP\set{g \le t} + \frac{C}{\sqrt{n}}\rho = \int_{-\infty}^{t}\frac{1}{\sqrt{2\pi}}e^{-u^2/2}du +  \frac{C}{\sqrt{n}}\rho 
\end{align*} This bound is \emph{\textbf{sharp}}, i.e. the equality is attained when $X_i \sim \text{Bernoulli}(1/2)$.
\end{remark}

\item \begin{theorem} (\textbf{Poisson Limit Theorem}).   \citep{vershynin2018high} \\
Let $X_{N,i}$, $1 \le i \le N$, be independent random variables $X_{N,i} \sim \text{Ber}(p_{N,i})$, and let $S_N = \sum^{N}_{i=1}X_{N,i}$. Assume that, as $N \to \infty$
\begin{align*}
\max_{i\le N}p_{N,i} \rightarrow 0\quad  \text{ and }\quad \E{}{S_N} = \sum^{N}_{i=1}p_{N,i} \rightarrow \lambda < \infty,
\end{align*}
Then, as $N \to \infty$,
\begin{align*}
S_N =  \sum^{N}_{i=1}X_{N,i} \stackrel{d}{\rightarrow} \text{Pois}(\lambda)
\end{align*}
\end{theorem}
\end{itemize}


\subsection{Information Theory}
\begin{itemize}
\item \begin{definition} (\textbf{\emph{Shannon Entropy}}) \citep{thomas2006elements}\\
Let $(\Omega, \srF, \bP)$ be a probability space and $X: \bR \to \cX$ be a random variable. Define $p(x)$ as \emph{the probability density function} of $X$ with respect to a base measure $\mu$ on $\cX$. \underline{\emph{\textbf{The Shannon Entropy}}} is defined as 
\begin{align*}
H(X) &:= \E{p}{-\log p(X)} \\
&= \int_{\Omega} -\log p(X(\omega)) d\bP(\omega) \\
&= - \int_{\cX} p(x)  \log p(x) d\mu(x)
\end{align*}
\end{definition}

\item \begin{definition} (\textbf{\emph{Conditional Entropy}}) \citep{thomas2006elements}\\
If a pair of random variables $(X, Y)$ follows the joint probability density function $p(x, y)$ with respect to a base product measure $\mu$ on $\cX \times \cY$. Then \emph{\textbf{the joint entropy}} of $(X, Y)$, denoted as $H(X, Y)$, is defined as
\begin{align*}
H(X, Y) &:=  \E{X, Y}{-\log p(X, Y)} = - \int_{\cX \times \cY} p(x, y)  \log p(x, y) d\mu(x, y)
\end{align*} Then \emph{\textbf{the conditional entropy}} $H(Y | X)$ is defined as
\begin{align*}
H(Y | X) &:= \E{X, Y}{-\log p(Y|X)}  = -\int_{\cX \times \cY} p(x, y)  \log p(y | x) d\mu(x, y) \\
&=  \E{X}{\E{Y}{-\log p(Y|X)}} = \int_{\cX}p(x) \paren{-\int_{\cY}p(y|x) \log p(y|x) d\mu(y)}d\mu(x)
\end{align*}
\end{definition}

\item \begin{proposition}(\textbf{Properties of Shannon Entropy})  \citep{thomas2006elements}\\
Let $X, Y, Z$ be random variables. 
\begin{enumerate}
\item (\textbf{Non-negativity}) $H(X) \ge 0$;
\item (\textbf{Concavity}) $H(p) := \E{p}{-\log p(X)}$ is a concave function in terms of p.d.f. $p$, i.e.
\begin{align*}
H(\lambda p_1 + (1- \lambda) p_2) \ge \lambda H(p_1) + (1- \lambda) H(p_2)
\end{align*} for any two p.d.fs $p_1, p_2$ on $\cX$ and any $\lambda \in [0,1]$.
\end{enumerate}
\end{proposition}

\item \begin{definition} (\textbf{\emph{Relative Entropy / Kullback-Leibler Divergence}}) \citep{thomas2006elements}\\
Suppose that $P$ and $Q$ are \emph{probability measures} on a measurable space $\cX$, and $P$ is \emph{absolutely continuous} with respect to $Q$, then \underline{\emph{\textbf{the relative entropy}}} or \underline{\emph{\textbf{the Kullback-Leibler divergence}}} is defined as
\begin{align*}
\kl{P}{Q} &:=\E{P}{\log\paren{\frac{dP}{dQ}}} = \int_{\cX} \log\paren{\frac{dP(x)}{dQ(x)}} dP(x)
\end{align*} where $\frac{dP}{dQ}$ is \emph{the Radon-Nikodym derivative} of $P$ with respect to $Q$. Equivalently, the KL-divergence can be written as
\begin{align*}
\kl{P}{Q} &= \int_{\cX} \paren{\frac{dP(x)}{dQ(x)}} \log\paren{\frac{dP(x)}{dQ(x)}} dQ(x) 
\end{align*} which is \emph{the entropy of $P$ relative to $Q$}. Furthermore, if $\mu$ is a base measure on $\cX$ for which densities $p$ and $q$ with $dP = p(x)d\mu$ and $dQ = q(x) d\mu$ exist, then 
\begin{align*}
\kl{P}{Q} &= \int_{\cX} p(x)\log\paren{\frac{p(x)}{q(x)}} d\mu(x)
\end{align*}
\end{definition}

\item \begin{definition}(\textbf{\emph{Mutual Information}}) \citep{thomas2006elements}\\
Consider two random variables $X, Y$ on $\cX \times \cY$ with joint probability distribution $P_{(X, Y)}$ and marginal distribution $P_{X}$ and $P_{Y}$. \underline{\emph{\textbf{The mutual information $I(X; Y)$}}} is \emph{the relative entropy} between \emph{the joint distribution} $P_{(X, Y)}$ and \emph{the product distribution} $P_{X}\otimes P_{Y}$:
\begin{align*}
I(X; Y) &= \kl{P_{(X, Y)}}{P_{X}\otimes P_{Y}} = \E{P_{(X, Y)}}{\log \frac{dP_{(X,Y)}}{dP_{X} \otimes dP_{Y}}}
\end{align*} If $P_{(X, Y)}$ has a probability density function $p(x,y)$ with respect to a base measure $\mu$ on $\cX \times \cY$, then 
\begin{align*}
I(X; Y) &=\int_{\cX \times \cY} p(x, y)\log\paren{\frac{p(x, y)}{p_{X}(x)p_{Y}(y)}} d\mu(x, y)
\end{align*}
\end{definition}

\item \begin{proposition} (\textbf{Properties of Relative Entropy and Mutual Information})  \citep{thomas2006elements}\\
Let $X, Y$ be random variables.
\begin{enumerate}
\item (\textbf{Non-negativity})  Let $p(x), q(x)$ be probability density function of $P ,Q$.
\begin{align*}
\kl{P}{Q} \ge 0
\end{align*} with equality if and only if $p(x) = q(x)$ almost surely. Therefore, the mutual information is non-negative as well:
\begin{align*}
I(X; Y) \ge 0
\end{align*} with equality if and only if $X$ and $Y$ are independent.
\item (\textbf{Symmetry})  $I(X; Y) = I(Y; X)$
\item (\textbf{Information Gain via Conditioning}) The mutual information $I(X; Y)$ is the reduction in the uncertainty of $X$ due to the knowledge of $Y$ (and vice versa)
\begin{align}
I(X; Y) &= H(X) - H(X | Y) \label{eqn: mutual_information_gain}\\
&= H(Y) - H(Y | X) \nonumber\\
&= H(X) + H(Y) - H(X, Y) \nonumber
\end{align}
\item (\textbf{Shannon Entropy as Self-Information})  $I(X; X) = H(X)$
\item (\textbf{Joint Convexity of Relative Entropy}) The relative entropy $\kl{p}{q}$ is \textbf{convex} in the pair $(p, q)$; that is, if $(p_1, q_1)$ and $(p_2, q_2)$ are two pairs of probability density functions, then for $\lambda \in [0, 1]$,  
\begin{align}
\kl{\lambda p_1 + (1- \lambda) p_2}{\lambda q_1 + (1- \lambda) q_2} &\le \lambda \kl{p_1}{q_1} + (1- \lambda) \kl{p_2}{q_2} \label{ineqn: kl_divergence_joint_convex}
\end{align}
\end{enumerate}
\end{proposition}



\item \begin{proposition} (\textbf{Conditioning Reduces Entropy}) \citep{thomas2006elements}\\
From non-negativity of mutual information, we see that the entropy of $X$ is non-increasing when conditioning on $Y$
\begin{align}
H(X | Y) \le  H(X)  \label{ineqn: conditional_entropy}
\end{align} where equality holds if and only if $X$ and $Y$ are independent.
\end{proposition}


\item \begin{proposition} (\textbf{Chain Rule for Entropy}) \citep{thomas2006elements}\\
Let $X_1, X_2 \xdotx{,} X_n$ be drawn according to $p(x_1, x_2 \xdotx{,} x_n)$. Then
\begin{align}
H(X_1, X_2 \xdotx{,} X_n) &= \sum_{i=1}^{n}H(X_i | X_{i-1} \xdotx{,} X_1) \label{eqn: chain_rule_entropy}
\end{align}
\end{proposition}

\item \begin{proposition} (\textbf{Sub-Additivity of Entropy}) \citep{thomas2006elements}\\
Let $X_1, X_2 \xdotx{,} X_n$ be drawn according to $p(x_1, x_2 \xdotx{,} x_n)$. Then
\begin{align}
H(X_1, X_2 \xdotx{,} X_n) &\le \sum_{i=1}^{n}H(X_i)  \label{ineqn: sub_additivity_entropy}
\end{align} with equality if and only if the $X_i$ are independent.
\end{proposition}


\item \begin{proposition} (\textbf{Chain Rule for Relative Entropy}) \citep{thomas2006elements}\\
Let $P_{(X, Y)}$ and $Q_{(X, Y)}$ be two probability measures on product space $\cX \times \cY$ and $P \ll Q$. Denote the marginal distributions $P_X, Q_X$ and $P_Y$, $Q_Y$ on $\cX$ and $\cY$, respectively. $P_{Y|X}$ and $Q_{Y|X}$ are conditional distributions (Note that $P_{Y|X} \ll Q_{Y|X}$).  Define \textbf{the conditional relative entropy} as
\begin{align*}
\E{X}{\kl{P_{Y | X}}{Q_{Y | X}}} := \E{X}{\E{P_{Y|X}}{\log \paren{\frac{dP_{Y| X}}{dQ_{Y | X}}}}}. 
\end{align*} Then the relative entropy of joint distribution $P_{(X, Y)}$ with respect to $Q_{(X, Y)}$ is 
\begin{align}
\kl{P_{(X, Y)}}{Q_{(X, Y)}} &= \kl{P_{X}}{Q_{X}} + \E{X}{\kl{P_{Y | X}}{Q_{Y | X}}} \label{eqn: chain_rule_kl}
\end{align} In addition, let $P$ and $Q$ denote two joint distributions for $X_1, X_2 \xdotx{,} X_n$, let $P_{1:i}$ and $Q_{1:i}$ denote the marginal distributions of $X_1, X_2 \xdotx{,} X_i$ under $P$ and $Q$, respectively. Let $P_{X_i | 1 \ldots i-1}$ and $Q_{X_i | 1 \ldots i-1}$ denote the conditional distribution of $X_i$ with respect to $X_1, X_2 \xdotx{,} X_{i-1}$ under $P$ and under $Q$.
\begin{align}
\kl{P}{Q} &= \sum_{i=1}^{n}\E{P_{1:i-1}}{\kl{P_{X_i | 1 \ldots i-1}}{Q_{X_i | 1 \ldots i-1}}} \label{eqn: chain_rule_kl_k}
\end{align} 
\end{proposition}

\item \begin{proposition} (\textbf{Han's Inequality}) \citep{thomas2006elements, boucheron2013concentration}\\
Let $X_1, X_2 \xdotx{,} X_n$ be random variables. Then
\begin{align}
H(X_1, X_2 \xdotx{,} X_n) &\le \frac{1}{n-1}\sum_{i=1}^{n}H(X_1 \xdotx{,} X_{i-1}, X_{i+1} \xdotx{,} X_n) \label{ineqn: han_inequality} \\
\Leftrightarrow H(X) &\le \frac{1}{n-1}\sum_{i=1}^{n}H(X_{(-i)}) \nonumber
\end{align}
\end{proposition}
\end{itemize}

\newpage
\section{Summary: General Proof Stratgy for Concentration Problem}
There are many proof techniques introduced. We can summarize them as follows:
\begin{enumerate}
\item \emph{\textbf{The Cram{\'e}r-Chernoff Method}}:\\
This class of methods essentially apply \emph{\textbf{the Markov inequality}} on \emph{\textbf{exponential transform}} $e^{\lambda X}$ with parameter $\lambda$. The key is to \emph{\textbf{bound}} the \emph{\textbf{log-moment generating function}} from above and then use \emph{\textbf{the Legendre transform}} to find the concentration bound. 

Specifically, for a real-valued random variable $X$, any $\lambda \ge 0$, the following inequality holds
\begin{align*}
\bP\set{X \ge t} = \bP\set{e^{\lambda X} \ge e^{\lambda t}} &\le e^{-\lambda t} \E{}{e^{\lambda X}} = \exp\paren{-\lambda t + \psi_{X}(\lambda) }
\end{align*} where $\psi_{X}(\lambda) := \log   \E{}{e^{\lambda X}}$. One can choose optimal $\lambda^{*}$ that \emph{\textbf{minimizes} the upper bound above}.
Since $\psi_{X}(\lambda)$ is a \emph{\textbf{convex function}}, we can define its \underline{\emph{\textbf{Legendre transform}}}
\begin{align*}
\psi^{*}_{X}(t) &:= \sup_{\lambda \in \bR}\set{\lambda\,t - \psi_{X}(\lambda)}.
\end{align*} The expression of the right-hand side is known as the \textbf{\emph{convex conjugate}} of $\psi_{X}$. The Legendre transform of log-moment generating function is also its \emph{convex conjugate}.  Thus we have
\begin{align*}
\bP\set{X \ge t}  &\le \exp\set{- \psi^{*}_{X}(t)}
\end{align*}
The lower bound can be found by applying above formula  to $-X$. % Since $e^t$ is monotone convex function, we can obtain the following inequality

In other word, in order to prove concentration around mean 
\begin{align*}
\bP\set{f(X) \ge  \E{}{f(X)} + t} \text{ or } \bP\set{f(X) \le  \E{}{f(X)} - t}
\end{align*}
using \underline{\textbf{\emph{the Cram\'er-Chernoff Method}}}, we just need to find \underline{\emph{\textbf{the upper bound}}} $\phi(\lambda)$ of \emph{\textbf{the logarithmic moment generating function}} $\psi(\lambda) $
\begin{align*}
\psi(\lambda) := \log \E{}{e^{\lambda (f(X) - \E{}{f(X)})}} &\le \phi(\lambda).
\end{align*}

\begin{remark} (\textbf{\emph{Advantages and Disadvantages of  Cram{\'e}r-Chernoff Method}})\\
There are several advantages for this method:
\begin{enumerate}
\item The derivation is \textbf{\emph{distribution-free}}, since \emph{\textbf{Markov inequality}} is based on fundamental properties of \emph{measure and integration theory}. Moreover, \emph{the bounds on logarithmic moment generating function} $\psi(\lambda)$ can be used to \emph{\textbf{characterize different distributions}} in terms of \emph{their concentration behavior}.

\item This method is \emph{\textbf{widely applicable}}. Most of techniques we learned here is to compute \emph{the upper bound} for $\psi(\lambda)$ and then apply the \emph{Cram{\'e}r-Chernoff method}. 

\item The formula is \emph{\textbf{easy to compute}} if the \emph{\textbf{simple bounds}} on \emph{logarithmic moment generating function} is \emph{computed}. Then it will compute the rate via \emph{\textbf{Legendre transform} of upper bound of $\psi(\lambda)$}.

\item The function $\psi(\lambda)$ easily handles \emph{product measures} $\bP = \otimes_{k=1}^{n}\bP_k$ (i.e. \emph{\textbf{independent variables}}).
\begin{align*}
\psi_Z(\lambda) &= \log \E{}{e^{\lambda\sum_{i=1}^{n}X_i}} = \log \prod_{i=1}^{n}\E{}{e^{\lambda X_i}}  = n\, \psi_{X}(\lambda)
\end{align*} and consequently,
\begin{align*}
\psi_Z^{*}(t) &= n\,\psi_{X}^{*}\paren{\frac{t}{n}}.
\end{align*} For \textbf{\emph{martingale difference sequence}}, we see that by conditioning on previous input
\begin{align*}
\E{}{\exp\set{\lambda \paren{\sum_{k=1}^{n}D_k }}} &= \E{}{\E{}{\exp\set{\lambda \paren{\sum_{k=1}^{n}D_k }} \;\Big|\; \srB_{n-1} }} \\
&= \E{}{\exp\set{\lambda \paren{\sum_{k=1}^{n-1}D_k }} \E{}{\exp\set{\lambda D_n }  \;\Big|\; \srB_{n-1}}}
\end{align*} If we can control each martingale difference by 
\begin{align*}
\log \E{}{\exp\set{\lambda D_n }  \;\Big|\; \srB_{n-1}} \le  \phi(\lambda)
\end{align*} then we have 
\begin{align*}
\psi_Z(\lambda) &\le \log \E{}{\exp\set{\lambda \paren{\sum_{k=1}^{n-1}D_k }}} + \phi(\lambda) \\
&\le \ldots \\
&\le n \phi(\lambda).
\end{align*}
\end{enumerate}

\emph{\textbf{The main disadvantage}} is that the Chernoff bound is \emph{\textbf{not necessarily sharp}}, since the Markov inequality is not necessarily sharp. 
\end{remark}

\item \emph{\textbf{Entropy Method}}:\\
\emph{The entropy method} focus on \emph{the \underline{\textbf{tensorization property}} of the \textbf{entropy functional}} $\text{Ent}(X)$
\begin{align*}
\text{Ent}(X)&:=  \E{}{X\log X} - \E{}{X}\log\paren{\E{}{X}}.
\end{align*}  Specifically,  let $Z_1, Z_2 \xdotx{,} Z_n$ be \emph{independent random variables} taking values in $\cX$, and let $f: \cX^n \to [0, \infty)$ be a measurable function. Letting $X = f(Z_1, Z_2 \xdotx{,} Z_n)$ such that $\E{}{X\log X} < \infty$, we have 
\begin{align*}
\text{Ent}(X)  &\le \E{}{\sum_{i=1}^{n} \text{Ent}_{(-i)}(X)}. 
\end{align*} where  $\E{(-i)}{\cdot}$ is the conditional expectation operator conditioning on $Z_{(-i)}$, which is equal to $Z$ after dropping $i$-component. In other word, the \emph{\textbf{key strategy}} for proving concentration using entropy method is to \emph{find \underline{\textbf{the upper bound}} for each \textbf{\underline{single variable} entropy functional}}
\begin{align*}
\text{Ent}_{(-i)}(X)&:=  \E{(-i)}{X\log X} - \E{(-i)}{X}\log\paren{\E{(-i)}{X}} \equiv H_{\Phi}(\bP_i).
\end{align*} Note that for independent random variables $Z$, this term \emph{\textbf{depends only on distribution of $Z_i$}}, since the rest $Z_{(-i)}$ are \emph{\textbf{controlled}} by the conditioning. For distributions such as \emph{Gaussian, Bernoulli and Poisson}, one can use \underline{\emph{\textbf{the logarithmic Sobolev inequalities}}} to derive \emph{the upper bound of the entropy functional} via \emph{\textbf{norm of gradients}}.

To obtain \emph{the concentraion bound}, we use \underline{\emph{\textbf{the Herbst's argument}}}; that is, the find the bound 
\begin{align*}
\text{Ent}(e^{\lambda X})& \le \E{}{e^{\lambda X}} \phi(\lambda) 
\end{align*} and using the differential equation for the log-moment generating function $\psi$
\begin{align*}
\frac{\text{Ent}(e^{\lambda Z})}{\E{}{e^{\lambda Z}}} &= \lambda\; \psi'(\lambda) - \psi(\lambda) = \lambda^2 \paren{\frac{ \psi(\lambda)}{\lambda}}',
\end{align*} we can obtain the upper bound for $\psi(\lambda)$:
\begin{align*}
 \paren{\frac{ \psi(\lambda)}{\lambda}}' &\le  \lambda^{-2}\phi(\lambda) \\
 \paren{\frac{ \psi(\lambda)}{\lambda}} &\le \lim\limits_{\lambda \to 0}\paren{\frac{ \psi(\lambda)}{\lambda}} + \int_{0}^{\lambda}s^{-2}\phi(s) ds \\
 \psi(\lambda) &\le \lambda\paren{\E{}{X} + \int_{0}^{\lambda}s^{-2}\phi(s) ds}. 
\end{align*} Finally, we apply \emph{the Chernoff bound}.
 
In general, \emph{\textbf{the key advantage}} of the \emph{\textbf{entropy method}} is that the tensorization property allows us to \emph{\underline{\textbf{generalize}} the concentration result from \textbf{$1$-dimensional distribution} to \textbf{$n$-dimensional product distribution}}. 

The main effort is to find a concentration inequality for \emph{\textbf{entropy of single variable distribution}}. One way to find such concentration is to use  \emph{\textbf{the logarithmic Sobelev inequalities}}.


\item \emph{\textbf{Transportation Method}}:\\
\emph{The transportation method} is closed related to various \emph{statistical divergence} esp. \emph{\textbf{the Kullback-Leibler divergence}} and \emph{\textbf{the information inequality}}. The centrial part of the proof is to show that for \emph{given distribution} $\bP$ of concern, \underline{\emph{\textbf{the transportation cost inequality}}} holds:
\begin{align*}
\cW_{1}^{d}(\bQ, \bP) := \min_{\gamma \in \Pi(\bQ, \bP)}\E{\gamma}{d(Y, X)} &\le \phi^{*-1}\paren{\kl{\bQ}{\bP}} \quad \forall \text{ distribution } \bQ
\end{align*} where $\Pi(\bQ, \bP)= \set{ \gamma \in \cP(\cX \times \cX):  Y_{\#}\gamma = \bQ, \; X_{\#}\gamma = \bP }$ i.e. $\gamma$ is a \emph{\textbf{coupling}} of marginal distribution  $\bQ$ and $\bP$. And, for every $s \ge 0$, 
\begin{align*}
\phi^{*-1}(s) = \inf\{t \in \text{dom}(\phi^{*}): \phi^{*}(t) > s\}
\end{align*}  is defined as the \textbf{\emph{the generalized inverse}} of \emph{the Legendre transform} $\phi^{*}= \sup_{\lambda \in (0,b)}(\lambda x - \phi(\lambda))$. 

There are \emph{two ways to proceed}:
\begin{enumerate}
\item Based on \emph{the duality of $1$-Wasserstein distance}, this \emph{transportation cost inequality} implies that for any \emph{$1$-Lipschitz function} $f: \cX \to \bR$ with respect to metric $d$
\begin{align*}
\E{\bQ}{f(Y)} - \E{\bP}{f(X)} = \E{\gamma}{f(Y) - f(X)} &\le \cW_{1}^{d}(\bQ, \bP)  \le \phi^{*-1}\paren{\kl{\bQ}{\bP}}.
\end{align*} 

\item Or, we use \emph{the Cauchy-Schwartz inequality}
\begin{align*}
\E{\bQ}{f(Y)} - \E{\bP}{f(X)} = \E{\gamma}{f(Y) - f(X)} &\le \sum_{i=1}^{n}\alpha_i \E{\gamma}{d(Y_i, X_i)}\\
&\le \paren{\sum_{i=1}^{n}\alpha_i^2}^{1/2}\paren{ \sum_{i=1}^{n} (\E{\gamma}{d(Y_i, X_i)})^2}^{1/2}
\end{align*} If we can show that the quadratic of transportation cost
\begin{align*}
 \min_{\gamma \in \Pi(\bQ, \bP)} \sum_{i=1}^{n}(\E{\gamma}{d(Y_i, X_i)})^2 \le \varphi\paren{\kl{\bQ}{\bP}}
\end{align*} Then
\begin{align*}
\E{\bQ}{f(Y)} - \E{\bP}{f(X)} \le \paren{ \paren{\sum_{i=1}^{n}\alpha_i^2}\varphi\paren{\kl{\bQ}{\bP}}}^{1/2}
\end{align*}
\end{enumerate}

Finally by the \emph{\textbf{transportation lemma}}, we can show that 
\begin{align*}
\psi_{f(X)}(\lambda) := \E{\bP}{e^{\lambda (f(X) - \E{}{f(X)})}} &\le \phi(\lambda).
\end{align*} The concentration follows from \emph{Chernoff bound} with rate function $\phi^{*}(t)$.

Note that the transportation cost inequality has \emph{\textbf{the tensorization property}} as well. This allows us to generalize the the inequality from $1$-dimension distribution to product distributions.

\begin{remark} (\textbf{\emph{Advantages and Disadvantages of  Transportation Method}})\\
There are several advantages for this method:
\begin{enumerate}
\item \emph{The optimal transport problem} and \emph{the Wasserstein distance} is closely related to \emph{\textbf{the information geometry}} of probability space $\cP(\cX)$. In particular, the transportation cost inequality relates the optimal transport cost to the relative entropy:
\begin{align*}
\cW_{p}^{d}(\bQ, \bP) &\le \varphi\paren{\kl{\bQ}{\bP}}.
\end{align*} This provides an alternative \emph{\textbf{information theoretical interpretation}} of the concentration behavior of independent random variables.

\item  \emph{The low optimal transportation cost}  is closely associated with \emph{\textbf{the concentration of measure}} in $\bP \in \cP(\cX)$. In fact, we can bound the concentration function $\alpha_{\bP, (\cX, d)}(t)$ from above by the upper bound of optimal transport cost.

\item \emph{\textbf{The dual formulation}} naturally leads to \emph{the concentration of Lipschitz function} or other \emph{strong uniform continuous functions}.

\item The concept of \emph{\textbf{coupling}} $\gamma \in \Pi(\bQ, \bP)$ allows us to extend the concentration results to \emph{\textbf{dependent variables}}, such as \emph{Markov chains, Markov random field} etc. In those cases, we can separate the conditional distribution $\bP(X_{i} | X_{1:i-1})$  and the marginal distributions $\bP(X_{1:i-1})$.
\end{enumerate}
\end{remark}

\item \emph{\textbf{Concentration of Measure and Isoperimetric Inequalities}}:\\
The applicablity of \emph{\textbf{isoperimetric Inequalities}} are \emph{limited} to a few cases such as \emph{\textbf{Gaussian measure}}, \emph{\textbf{Bernoulli measure}} (or \emph{the uniform distribution on \textbf{binary hypercube}}, \emph{compact manifolds}, \emph{Lebesgue measure} on $\bR^n$, \emph{graph vertex and edge boundaries} etc.). 

The key is to derive the upper bound for \emph{\textbf{the concentration function}}: 
\begin{align*}
\alpha_{\bP, (\cX, d)} := \sup\set{ \bP\set{A_t^c}:  A \subset \cX, \bP(A) \ge \frac{1}{2}  }.
\end{align*} Note that for $d(x, A) := \inf_{y\in A}d(x, y)$,  the \emph{$t$-blowup of $A$} is defined as  
\begin{align*}
\bP\set{A_t^c} := \bP\set{d(X, A) \ge t}.
\end{align*} Then the goal is to find \emph{\textbf{the  isopermetric inequality}}
\begin{align*}
\alpha_{\bP, (\cX, d)} \le \exp\paren{- \phi(t)} \quad \Leftrightarrow  \quad \bP(A)\bP\paren{A_t^c} := \bP(A)\bP\set{d(X, A) \ge t} \le \exp\paren{- \phi(t)}.
\end{align*} By Levy's inequality,  for \emph{Lipschitz function} $f: \cX \to \bR$,   let $A:=\set{x: f(x) \le \text{Med}(f(X))}$, so that $\bP(A) \ge 1/2$, and the complement of $t$-blowup of $A$ becomes
\begin{align*}
\bP\paren{A_t^c}  = \bP\set{f(X) \ge \text{Med}(f(X)) + t}
\end{align*} Then the above isopermetric inequality is equivalent to 
\begin{align*}
 \bP\set{f(X) \ge \text{Med}(f(X)) + t} \le 2\exp\paren{- \phi(t)}\\
  \bP\set{f(X) \le \text{Med}(f(X)) - t} \le 2\exp\paren{- \phi(t)}.
\end{align*}
Besides the existing result for \emph{Gaussian and Bernoulli random variables}, the Talagrand's \emph{\textbf{convex distance inequality}} is very useful to derive such isopermetric inequality based on \emph{weighted Hamming distance}.
\begin{align*}
\bP(A)\bP\set{\sup_{\alpha \in \bR_{+}^{n}: \norm{\alpha}{2} = 1}\inf_{y\in A} \sum_{i=1}^{n}\alpha_i\ind{x_i \neq y_i} \ge t} \le  \exp\paren{-\frac{t^2}{4}}. \nonumber
\end{align*} Note that if $A$ is a convex set,
\begin{align*}
d(x, A):= \inf_{y \in A}\norm{x - y}{2} \le d_{T}(x, A) := \sup_{\alpha \in \bR_{+}^{n}: \norm{\alpha}{2} = 1}\inf_{y\in A} \sum_{i: x_i \neq y_i}\alpha_i
\end{align*}

If \emph{\textbf{isoperimetric theorem}} exists \emph{\textbf{for given distribution or space}}, then the derived concentration bound is known to be \emph{\textbf{sharp}} due to \emph{\textbf{the concentration of measure phenomenon}}. This is the main \emph{\textbf{advantage}} of using isoperimetric inequalities. However, proving \emph{\textbf{isoperimetric theorem}} is extremely hard and thus is not widely available. 

\emph{Transportation methods} and \emph{logarithmic Sobolev inequalities} can also use to show \emph{isoperimetric inequalities}.

\end{enumerate}


\newpage
\section{Summary: Distribution-Free Concentration Inequality}
\begin{itemize}
\item \begin{remark} (\textbf{\emph{Distribution-Free Concentration Inequality}}) \\
Some concentration results are based on \emph{\textbf{assumption} on specific underling distributions} such as \emph{Gaussian, Bernoulli, Poisson, sub-Gaussian, sub-Gamma} etc. On the other hand, some concentration results are based on assumption on specific function class such as \emph{bounded (actually is sub-Gaussian), Lipschitz function, bounded difference, convex function} etc. The latter results do not rely on specific distribution assumption, so it is called \emph{\textbf{the distribution-free concentration inequality}}. 

We list out several important inequalities:
\begin{enumerate}
\item \begin{theorem} (\textbf{Markov's Inequality}). \citep{vershynin2018high}\\
For any \textbf{non-negative} random variable $X$ and $t > 0$, we have
\begin{align*}
\bP\set{X \ge t} &\le \frac{\E{}{X}}{t} 
\end{align*}
\end{theorem}

\item \begin{theorem} (\textbf{Chebyshev's Inequality}). \citep{vershynin2018high}\\
Let $X$ be a random variable with mean $\mu$ and variance $\sigma^2$. Then, for any $t > 0$, we have
\begin{align*}
\bP\set{\abs{X - \mu} \ge t} &\le \frac{\sigma^2}{t^2}. 
\end{align*}
\end{theorem}

\item \begin{theorem} (\textbf{Chernoff's inequality}) \citep{boucheron2013concentration}\\
Let $X$ be a real-valued random variable. For $\lambda \ge 0$,  $\psi_{X}(\lambda)$ is the \textbf{the logarithm of moment generating function} of $X$ and $\psi^{*}_{X}(t)$ is its \textbf{Legendre (Cram{\'e}r) transform}. Then 
\begin{align*}
\bP\set{X \ge t} &\le \exp\paren{-\psi^{*}_{X}(t)}. 
\end{align*}
\end{theorem}

\item  \begin{theorem} (\textbf{Hoeffding's inequality}) \citep{boucheron2013concentration} \\
Let $X_1 \xdotx{,} X_n$ be independent random variables such that $X_i$ takes its values in $[a_i, b_i]$ \textbf{almost surely} for all $i \le n$. 
Then for every $t > 0$,
\begin{align*}
\bP\set{\sum_{i=1}^{n}\paren{X_i - \E{}{X_i}} \ge  t} \le \exp\paren{- \frac{2t^2}{\sum_{i=1}^{n}(b_i - a_i)^2}}. 
\end{align*}
\end{theorem}

\item \begin{corollary} (\textbf{Azuma-Hoeffding Inequality})\citep{wainwright2019high}\\
Let $\set{(D_k, \srB_k), k \ge 1}$ be a \textbf{martingale difference sequence} for which there are constants $\set{(a_k, b_k)}^{n}_{k=1}$ such that $D_k \in [a_k, b_k]$ almost surely for all $k = 1 \xdotx{,} n$. Then, for all $t \ge 0$,
\begin{align*}
\bP\set{\abs{\sum_{k=1}^{n}D_k} \ge t } &\le  2 \exp\paren{- \frac{2 t^2}{ \sum_{k=1}^{n}(b_k - a_k)^2}}
\end{align*}
\end{corollary}

\item \begin{theorem} (\textbf{McDiarmid's Inequality / Bounded Differences Inequality})\citep{boucheron2013concentration, wainwright2019high}\\
Suppose that $f$ satisfies \textbf{the bounded difference property} \eqref{eqn: bounded_difference_property} with parameters $(L_1 \xdotx{,} L_n)$ i.e. for each index $k = 1, 2 \xdotx{,} n$,
\begin{align*}
\abs{f(x_1 \xdotx{,} x_n) - f(x_1 \xdotx{,} x_{i-1},x'_{i}, x_{i+1} \xdotx{,} x_n)} \le L_k, \quad\text{ for all }x, x' \in \cX^n. 
\end{align*} Assume that the random vector $X = (X_1, X_2 \xdotx{,} X_n)$ has \textbf{independent} components. Then
\begin{align*}
\bP\set{\abs{f(X) - \E{}{f(X)}} \ge t } &\le  2 \exp\paren{- \frac{2 t^2}{ \sum_{k=1}^{n}L_k^2}}. \
\end{align*}
\end{theorem} Note that  \emph{functions with bounded difference property} are \emph{\textbf{Lipschitz function} with respect to \textbf{Hamming distance}}.

\item \begin{theorem} (\textbf{Concentration of Separately Convex Lipschitz Functions}) \citep{boucheron2013concentration}\\
Let  $Z:= (Z_1 \xdotx{,} Z_{n})$ be independent random variables, each taking values in the interval $[a_i, b_i]$ and let $f : \bR^n \to \bR$ be a \textbf{separately convex function} (i.e. $f$ is \textbf{convex in each coordinate} while the \textbf{others} are \textbf{fixed}) such that
\begin{align*}
\abs{f(x) - f(y)} &\le L\,\norm{x - y}{} \quad \text{for all }x, y \in [0, 1]^n.
\end{align*}
Then $X = f(Z_1 \xdotx{,} Z_{n})$ satisfies, for all $t > 0$,
\begin{align*}
\bP\set{f(Z) - \E{}{f(Z)} \ge t } &\le \exp\paren{- \frac{t^2}{2L^2 \sum_{k=1}^{n}(b_k - a_k)^2}}. 
\end{align*}
\end{theorem} Convex Lipschitz assumption is \emph{\textbf{stronger}} than bounded difference assumption.

\item \begin{theorem} (\textbf{Concentration of Quasi-Convex Lipschitz Functions}) \citep{boucheron2013concentration}\\
Let  $Z:= (Z_1 \xdotx{,} Z_{n})$ be independent random variables taking values in the interval $[0, 1]$ and let $f : [0, 1]^n \to \bR$ be a \textbf{quasi-convex function}; that is
\begin{align*}
\set{z: f(z) \le s} \text{ is convex set for all }s \in \bR. 
\end{align*} Moreover, $f$ is Lipschitz function satisfying
\begin{align*}
\abs{f(x) - f(y)} &\le \norm{x - y}{} \quad \text{for all }x, y \in [0, 1]^n.
\end{align*}
Then $X = f(Z_1 \xdotx{,} Z_{n})$ satisfies, for all $t > 0$,
\begin{align*}
\bP\set{f(Z)  \ge  \text{Med}(f(Z)) + t } &\le 2\exp\paren{- \frac{t^2}{4}}, \\
\bP\set{f(Z)  \le \text{Med}(f(Z)) -t } &\le 2\exp\paren{- \frac{t^2}{4}}. \nonumber
\end{align*}
\end{theorem}
\end{enumerate}
\end{remark}
\end{itemize}
\newpage
\section{Comparison: Gaussian Tail Bound vs. Poisson Tail Bound}
\begin{itemize}
\item \begin{remark} (\emph{\textbf{Gaussian Tail Bound} vs. \textbf{Poisson Tail Bound}})\\
Based on Chernoff bound, we can derive the tail bound for two important class of distributions: 
\begin{enumerate}
\item \emph{\textbf{Gaussian tail bound}}: for any $t > 0$, 
\begin{align*}
\bP\set{X \ge t} &\le \exp\paren{-\frac{t^2}{2 \nu}},
\end{align*} where $\nu >0$.
\item \emph{\textbf{Poisson tail bound}}:  for any $t > \E{}{X} = \nu$,
\begin{align*}
\bP\set{X \ge t} &\le \exp\paren{-  \nu h\paren{\frac{t}{\nu}}} = e^{-\nu}\paren{\frac{e \nu}{t + \nu}}^{t+\nu}
\end{align*} where $h(x) = (1 + x) \log(1 + x) - x$ for all  $x \ge -1$ and $\nu >0$.

Note that for \emph{\textbf{small deviation} around the mean}, the \emph{tail of Poisson distribution} $\text{Pois}(\nu)$ behaves like \emph{Gaussian} $\cN(\nu, \nu)$:
\begin{align*}
\bP\set{\abs{X - \nu} \ge t}& \le 2\exp\paren{- \frac{c t^2}{\nu}}
\end{align*}
But in \emph{the \textbf{large deviation} regime}, the Poisson tail is \emph{\textbf{heavier} than Gaussian}. Such distribution is a \emph{\textbf{sub-Gamma distribution}}. 
\end{enumerate}
\end{remark}

\item \begin{remark}
\emph{\textbf{The Bennet inequality}} captures \emph{\textbf{the Poisson tail bevarior}}:  for sum of  $n$ independent random variables $X_i$ such that $X_i \le b$ almost surely with zero mean, finite variance $\nu = \sum_{i=1}^n \E{}{X_i^2}$.  The Bennet inequality provides a tail bound as
\begin{align*}
\bP\set{\sum_{i=1}^n X_i \ge t} &\le \exp\paren{-  \frac{\nu}{b^2} h\paren{\frac{b t}{\nu}}}
\end{align*} In small deviation regime, where $u:= bt/\nu \ll 1$, $h(u) \approx u^2$, the Bennet inequality gives a Gaussian tail bound $\approx \exp\paren{- t^2/\nu}$.

In large deviation regime, $u \gg bt/\nu \ge 2$, $h(u) \ge \frac{1}{2}u \log(u)$, thus the Bennet inequality gives a Poisson tail bound $\approx (\nu/b^2t)^{t /(2 b^2)}$
\end{remark}
\end{itemize}

\newpage
\section{The Cram{\'e}r-Chernoff Method}
\subsection{From Markov Inequality to Cram\'er-Chernoff Method}
\begin{itemize}
\item 
\begin{proposition} (\textbf{Markov's Inequality}). \citep{vershynin2018high}\\
For any \textbf{non-negative} random variable $X$ and $t > 0$, we have
\begin{align}
\bP\set{X \ge t} &\le \frac{\E{}{X}}{t} \label{ineqn: markov_inequality}
\end{align}
\end{proposition}

\item \begin{proposition} (\textbf{Chebyshev's Inequality}). \citep{vershynin2018high}\\
Let $X$ be a random variable with mean $\mu$ and variance $\sigma^2$. Then, for any $t > 0$, we have
\begin{align}
\bP\set{\abs{X - \mu} \ge t} &\le \frac{\sigma^2}{t^2}. \label{ineqn: chebyshev_inequality}
\end{align}
\end{proposition}

\item \begin{remark}(\textbf{\emph{Cram\'er-Chernoff Method}})\\
In this section we describe and formalize the Cram{\'e}r-Chernoff bounding method. This method determines \emph{the best possible bound} for a \emph{\textbf{tail probability}} that one can possibly obtain using \emph{Markov's inequality} with an exponential function $\phi(t) = e^{\lambda t}$.

Recall that for a real-valued random variable $X$, any $\lambda \ge 0$, the following inequality holds
\begin{align*}
\bP\set{X \ge t} &\le e^{-\lambda t} \E{}{e^{\lambda X}} = \exp\paren{-\lambda t + \psi_{X}(\lambda) }
\end{align*} where $\psi_{X}(\lambda) := \log   \E{}{e^{\lambda X}}$. One can choose optimal $\lambda^{*}$ that \emph{\textbf{minimizes} the upper bound above}.
Since $\psi_{X}(\lambda)$ is a \emph{\textbf{convex function}}, we can define its \underline{\emph{\textbf{Legendre transform}}}
\begin{align*}
\psi^{*}_{X}(t) &:= \sup_{\lambda \in \bR}\set{\lambda\,t - \psi_{X}(\lambda)}.
\end{align*} The expression of the right-hand side is known as the \underline{\emph{\textbf{Fenchel-Legendre dual function}}} (or the \textbf{\emph{convex conjugate}}) of $\psi_{X}$. The Legendre transform of log-moment generating function is also its convex conjugate. % Since $e^t$ is monotone convex function, we can obtain the following inequality

In other word, in order to prove concentration around mean 
\begin{align*}
\bP\set{f(X) \ge  \E{}{f(X)} + t} \text{ or } \bP\set{f(X) \le  \E{}{f(X)} - t}
\end{align*}
using \underline{\textbf{\emph{the Cram\'er-Chernoff Method}}}, we just need to find \underline{\emph{the upper bound}} of \emph{the logarithmic moment generating function}
\begin{align*}
\psi(\lambda) := \log \E{}{e^{\lambda (f(X) - \E{}{f(X)})}} &\le \phi(\lambda)
\end{align*}
\end{remark}

\item \begin{proposition} (\textbf{Chernoff's inequality}) \citep{boucheron2013concentration}\\
Let $X$ be a real-valued random variable. For $\lambda \ge 0$,  $\psi_{X}(\lambda)$ is the \textbf{the logarithm of moment generating function} of $X$ and $\psi^{*}_{X}(t)$ is its \textbf{Legendre (Cram{\'e}r) transform}. Then 
\begin{align}
\bP\set{X \ge t} &\le \exp\paren{-\psi^{*}_{X}(t)}. \label{ineqn: chernoff_inequality}
\end{align}
\end{proposition}

\item \begin{remark}
The \textbf{\emph{Legendre transform}} is also called \emph{\textbf{the Cram{\'e}r transform}} \citep{boucheron2013concentration}.

Since $\psi_{X}(0) = 0$, its \emph{Legendre transform} $\psi^{*}_{X}(t)$ is \emph{\textbf{nonnegative}}.
\end{remark}

\item \begin{definition} (\textbf{\emph{The Rate Function}})\\
\underline{\emph{\textbf{The rate function}}} is defined as \emph{\textbf{the Legendre transformation}} of \emph{the logarithm of the moment generating function} of a random variable. That is, 
\begin{align}
\psi^{*}_{X}(t) &:= \sup_{\lambda \in \bR}\set{\lambda\,t - \psi_{X}(\lambda)}, \label{eqn: rate_function}
\end{align} where $\psi_{X}(\lambda) := \log   \E{}{e^{\lambda X}}$. Thus, by \emph{Chernoff's inequality}, we can bound \emph{the tail probabilities} of random variables via \emph{its rate function}.
\end{definition}

\item \begin{remark}(\emph{\textbf{Sums of independent random variables}})\\
The reason why Chernoff's inequality became popular is that it is very simple to use when applied to a sum of independent random
variables. As an illustration, assume that $Z := X_1 \xdotx{+} X_n$ where $X_1 \xdotx{,} X_n$ are \emph{\textbf{independent} and \textbf{identically distributed} real-valued random variables}.  Denote the logarithm of the moment-generating function of the $X_i$ by $\psi_X(\lambda) = \log \E{}{e^{\lambda X_i}}$, and the corresponding \emph{Legendre transform} by $\psi_X^{*}(t)$. Then, by independence, for all $\lambda$ for which $\psi_X(\lambda) < \infty$,
\begin{align*}
\psi_Z(\lambda) &= \log \E{}{e^{\lambda\sum_{i=1}^{n}X_i}} = \log \prod_{i=1}^{n}\E{}{e^{\lambda X_i}}  = n\, \psi_{X}(\lambda)
\end{align*} and consequently,
\begin{align*}
\psi_Z^{*}(t) &= n\,\psi_{X}^{*}\paren{\frac{t}{n}}.
\end{align*} Thus \emph{the Chernoff's inequality} states that 
\begin{align*}
\bP\set{Z \ge t} &\le \exp\paren{-\psi^{*}_{Z}(t)} = \exp\paren{-n\,\psi_{X}^{*}\paren{\frac{t}{n}}}.
\end{align*}
\end{remark}

\item \begin{example} (\emph{\textbf{Normal Distribution}})\\
Let $X$ be a \emph{\textbf{centered normal random variable}} with variance $\sigma^2$. Then
\begin{align*}
\psi_X(\lambda) = \frac{\lambda^2 \sigma^2}{2},\, \quad \lambda_t = \frac{t}{\sigma^2}
\end{align*} and, therefore for every $t > 0$, 
\begin{align*}
\psi_{X}^{*}(t) &= \frac{t^2}{2\sigma^2}.
\end{align*} Hence, \emph{Chernoff's inequality} implies, for all $t > 0$,
\begin{align*}
\bP\set{X \ge t} &\le \exp\paren{-\frac{t^2}{2\sigma^2}}.
\end{align*} \emph{Chernoff's inequality} appears to be quite sharp in this case. In fact, one can show that it cannot be improved uniformly by more than a factor of $1/2$. \qed
\end{example}


\item \begin{example} (\emph{\textbf{Poisson Distribution}})\\
Let $X$ be a \emph{\textbf{Poisson random variable}} with parameter $\nu$, that is, $\bP\set{X = k} = \frac{1}{k!}e^{–\nu}\nu^k$ for all $k = 0, 1, 2, \ldots$ Let $Z = X - \nu$ be the \emph{corresponding centered variable}. Then by direct calculation,
\begin{align*}
\psi_Z(\lambda) = \nu\paren{e^{\lambda} - \lambda - 1 },\, \quad \lambda_t = \log\paren{1 + \frac{t}{\nu}}
\end{align*} Therefore \emph{the Legendre transform} equals, for every $t > 0$,
\begin{align*}
\psi_{Z}^{*}(t) &= \nu h\paren{\frac{t}{\nu}}.
\end{align*} where the function $h$ is defined, for all  $x \ge -1$, by $h(x) = (1 + x) \log(1 + x) - x$. Similarly,
for every $t \le \nu$,
\begin{align*}
\psi_{-Z}^{*}(t) &= \nu h\paren{-\frac{t}{\nu}}.
\end{align*} 
\end{example}


\item \begin{example} (\emph{\textbf{Bernoulli Distribution}})\\
Let $X$ be  a \emph{\textbf{Bernoulli random variable}} with probability of success $p$, that is, $\bP\set{X = 1} = 1- \bP\set{X =0} = p$. Let $Z = X -  p$ be the \emph{corresponding centered variable}. If $0 < t < 1 - p$, we have
\begin{align*}
\psi_Z(\lambda) = \log\paren{p e^{\lambda} + 1 - p } - p \,\lambda ,\, \quad \lambda_t = \log \frac{(1-p)(p + t)}{p(1 - p - t)}
\end{align*} and therefore, for every $t \in (0, 1 - p)$,
\begin{align*}
\psi_{Z}^{*}(t) &= (1-p-t)\log\frac{1 - p - t}{1 - p} + (p + t)\log \frac{p + t}{p}.
\end{align*} Equivalently, setting $a = t + p$ for every $a \in (p, 1)$,
\begin{align*}
\psi_{Z}^{*}(t) = h_p(a)&= (1- a)\log\frac{1- a}{1 - p} + a \log\frac{a}{p}.
\end{align*}  We note here that $h_p(a)$ is just the \emph{\textbf{Kullback-Leibler divergence}} $\kl{\bP_a}{\bP_p}$ between a Bernoulli distribution $\bP_a$ of parameter $a$ and a Bernoulli distribution $\bP_p$ of parameter $p$. 
\begin{align*}
\bP\set{X \ge t} \le \exp\paren{- \kl{\bP_{p+t}}{\bP_p}}
\end{align*}
\end{example}
\end{itemize}

\subsection{Sub-Gaussian Random Variables}
\begin{itemize}
\item \begin{definition}(\textbf{\emph{Sub-Gaussian Random Variable}})\\
A \emph{\textbf{centered}} random variable $X$ is said to be \underline{\emph{\textbf{sub-Gaussian} with \textbf{variance factor} $\nu$}} if
\begin{align}
\psi_X(\lambda) &\le  \frac{\lambda^2 \nu}{2}, \quad \text{ for every }\lambda \in \bR. \label{eqn: sub_gaussian_def_1}
\end{align} We denote the collection of such random variables by $\cG(\nu)$.
\end{definition}

\item \begin{proposition} (\textbf{Moment Characterization of Sub-Gaussian Random Variables})  \citep{boucheron2013concentration}\\
Let $X$ be a random variable with $\E{}{X} = 0$. If for some $\nu > 0$
\begin{align}
\bP\set{X > t} \lor  \bP\set{-X > t} \le \exp\paren{-\frac{t^2}{2\nu}}, \quad \text{ for all } t > 0   \label{eqn: sub_gaussian_def_2}
\end{align}
then for every integer $q \ge 1$,
\begin{align}
\E{}{X^{2q}} \le 2q! (2\nu)^q \le q! (4\nu)^q. \label{eqn: sub_gaussian_even_power_moment}
\end{align}
\textbf{Conversely}, if for some positive constant $C$
\begin{align*}
\E{}{X^{2q}} \le  q! C^q,
\end{align*} then $X \in \cG(4C)$ (and therefore \eqref{eqn: sub_gaussian_even_power_moment} holds with $\nu = 4C$).
\end{proposition}

\item \begin{proposition} (\textbf{Equivalent Definitions for Sub-Gaussian Random Variables}).  \citep{vershynin2018high}\\
Let $X$ be a random variable. Then the following properties are \textbf{equivalent}; the parameters $K_i > 0$ appearing in these
properties differ from each other by at most an absolute constant factor.
\begin{enumerate}
\item The \textbf{tails} of $X$ satisfy
\begin{align*}
\bP\set{\abs{X} \ge t} \le 2 \exp\paren{-t^2/K_1^2}\quad\text{ for all }t \ge 0.
\end{align*}

\item The \textbf{moments} of $X$ satisfy
\begin{align*}
\norm{X}{L^p} = \paren{\E{}{\abs{X}^p}}^{1/p} \le K_2 \sqrt{p}\quad \text{ for all }p \ge 1.
\end{align*}

\item The \textbf{moment-generating function (MGF)} of $X^2$ satisfies
\begin{align*}
\E{}{\exp(\lambda^2 X^2)} \le \exp(K_3^2 \;\lambda^2) \quad \text{ for all $\lambda$ such that $\abs{\lambda} \le \frac{1}{K_3}$}
\end{align*}

\item The \textbf{MGF} of $X^2$ is \textbf{bounded} at some point, namely
\begin{align*}
\E{}{\exp(X^2 / K_4^2)} \le 2.
\end{align*}
Moreover, if $\E{}{X} = 0$ then properties $(1)$-$(4)$ are also \textbf{equivalent} to the following one.

\item The \textbf{MGF} of $X$ satisfies
\begin{align*}
\E{}{\exp(\lambda X)} \le  \exp(K_5^2\,\lambda^2)\quad\text{ for all }\lambda \in \bR.
\end{align*}
\end{enumerate}
\end{proposition}

\item \begin{definition} (\textbf{\emph{Sub-Gaussian Norm}}) \\
The \underline{\emph{\textbf{sub-gaussian norm}}} of $X$, denoted $\norm{X}{\psi_2}$, is defined
to be the \emph{\textbf{smallest}} $K_4$ that satisfies 
\begin{align*}
\E{}{\exp(X^2 / K_4^2)} \le 2.
\end{align*} In other words, we define
\begin{align}
\norm{X}{\psi_2}= \inf\set{t > 0: \E{}{\exp(X^2 / t^2)} \le 2}.  \label{eqn: sub_gaussian_norm}
\end{align}
\end{definition}

\item \begin{remark} (\emph{\textbf{Sub-Gaussian  Characterizations via Sub-Gaussian Norm}})\\
We can restate the properties of sub-gaussian random variables in terms of sub-gaussian norm:
\begin{align*}
\bP\set{\abs{X} \ge t} &\le 2 \exp\paren{-c t^2/\norm{X}{\psi_2}^2}\quad\text{ for all }t \ge 0; \\
\norm{X}{L^p} &\le C \norm{X}{\psi_2} \sqrt{p}\quad \text{ for all }p \ge 1; \\
\E{}{\exp(X^2 / \norm{X}{\psi_2}^2)} &\le 2; \\
\text{ if }\E{}{X} = 0, \;\;\text{ then } \E{}{\exp(\lambda X)} &\le  \exp(C \lambda^2 \norm{X}{\psi_2}^2)\quad\text{ for all }\lambda \in \bR.
\end{align*}
\end{remark}

\item \begin{example}
Here are some classical examples of sub-gaussian distributions.
\begin{enumerate}
\item  (\textbf{\emph{Gaussian}}): As we already noted, $X \sim N(0, 1)$ is a sub-gaussian random
variable with $\norm{X}{\psi_2} \le C$, where $C$ is an absolute constant. More generally, if $X \sim N(0, \sigma^2)$ then $X$ is sub-gaussian with
\begin{align}
\norm{X}{\psi_2} \le C\sigma \label{eqn: gaussian_sub_guassian_norm}
\end{align}
\item  (\emph{\textbf{Bernoulli}}): Let $X$ be a random variable with \emph{\textbf{symmetric Bernoulli distribution}}. Since $\abs{X} = 1$, it follows that X is a
sub-gaussian random variable with
\begin{align}
\norm{X}{\psi_2} \le \frac{1}{\sqrt{\log 2}} \label{eqn: sym_bernoulli_sub_guassian_norm}
\end{align}
\item (\emph{\textbf{Bounded}}): More generally, any \emph{\textbf{bounded random variable}} $X$ is sub-gaussian with
\begin{align}
\norm{X}{\psi_2} \le C\norm{X}{\infty} \label{eqn: bounded_sub_guassian_norm}
\end{align}
where $C = 1/\sqrt{\log 2}$.
\end{enumerate}
\end{example}
\end{itemize}

\subsection{Sub-Exponential and Sub-Gamma Random Variables}
\begin{itemize}
\item \begin{remark}
For \emph{exponential distribution} $X \sim \exp(a)$ with \emph{rate} $a$ (\emph{inverse} of \emph{scale parameter}), the p.d.f. and moment generating function
\begin{align*}
f_{X}(x) &= a e^{-ax},\quad  x > 0\\
M_{X}(\lambda) &= \frac{1}{1  - \lambda/a},\quad  0 < \lambda < a
\end{align*}

For \emph{Gamma distribution} $X \sim \Gamma(a, 1/b)$ with \emph{shape parameter} $a$ and \emph{scale parameter} $b$, the p.d.f. and the moment generating function
\begin{align*}
f_{X}(x) &= \frac{1}{\Gamma(a)\,b^a} x^{a-1} e^{-x/b},\quad  x > 0\\
M_{X}(\lambda) &= \paren{\frac{1}{1  - b \lambda }}^{a},\quad  0 < \lambda < 1/b
\end{align*}Also $\E{}{X} = ab$ and $\text{Var}(X) = ab^2$.
\end{remark}



\item \begin{definition}(\emph{\textbf{Sub-Exponential Random Variables}})\\
A \emph{\textbf{nonnegative}} random variable $X$ has a \emph{\textbf{sub-exponential distribution}} if there exists a constant $a > 0$ such that
\begin{align*}
\E{}{e^{\lambda X}} &\le \frac{1}{1 - \lambda/a} \quad \text{for every $\lambda$ such that $0 < \lambda < a$} \\
\text{or }\quad \psi_X(\lambda) &\le \log\paren{ \frac{1}{1 - \lambda/a}}
\end{align*}
\end{definition}

\item \begin{definition}(\emph{\textbf{Sub-Gamma Random Variables}})\\
A real-valued \emph{\textbf{centered} random variable} $X$ is said to be \emph{\underline{\textbf{sub-gamma} on \textbf{the right tail}} with \textbf{variance factor} $\nu$ and \textbf{scale parameter} $c$} if
\begin{align*}
\psi_X(\lambda)  \le \frac{\lambda^2 \nu}{2(1 - c\lambda)} \quad \text{for every $\lambda$ such that $0 < \lambda < 1/c$} 
\end{align*} We denote the collection of such random variables by $\Gamma_{+}(\nu, c)$.

Similarlly, a real-valued centered random variable $X$ is said to be \emph{\underline{\textbf{sub-gamma} on \textbf{the left tail}} with \textbf{variance factor} $\nu$ and \textbf{scale parameter} $c$} if $-X$ is \emph{\textbf{sub-gamma on the right tail} with variance factor $\nu$ and tail parameter $c$}. We denote the collection of such random variables by $\Gamma_{-}(\nu, c)$. 

Finally, $X$ is simply said to be \emph{\underline{\textbf{sub-gamma}} with \textbf{variance factor} $\nu$ and \textbf{scale parameter} $c$} if $X$ is \emph{sub-gamma} \emph{\textbf{both}} \emph{on the right and left tails} with \emph{\textbf{the same}} \emph{variance factor} $\nu$ and \emph{scale parameter} $c$. The collection of such random variables is denoted by  $\Gamma(\nu, c)$.

Observe that $\Gamma(\nu, 0)= \cG(\nu)$.
\end{definition}

\item \begin{remark}
To derive the definition fo sub-gamma distribution, we see that \emph{the variance factor} $\nu:= ab^2$ and $c:= b$. Also $\E{}{X} = ab$. The logarithmic moment generating function of Gamma distribution $\Gamma(a, 1/b) = \Gamma(\nu/c^2, 1/c)$ is 
\begin{align*}
\psi_{X - \E{}{X}}(\lambda) = a\log\paren{\frac{1}{1 - b \lambda}} - \lambda a b  \le  \frac{\lambda^2 b^2 a}{2(1 - b\lambda)} \equiv  \frac{\lambda^2 \nu}{2(1 - c\lambda)}
\end{align*}
The last inequality is due to 
\begin{align*}
\log \paren{\frac{1}{1 - u}} - u \le \frac{u^2}{2(1 - u)}
\end{align*}
\end{remark}

\item \begin{remark}
Note that the sum of $n$ i.i.d. random variables with exponential distribution $\exp(1/b)$ have the Gamma distribution $\Gamma(n, 1/b)$. So \emph{\textbf{the sub-gamma distributed}} random variable follows  \emph{\textbf{the sub-exponential distribution}} as well (with shape parameter $=1$).
\end{remark}

\item \begin{proposition} (\textbf{Equivalent Definitions for Sub-Exponential Random Variables}).  \citep{vershynin2018high}\\
Let $X$ be a random variable. Then the following properties are \textbf{equivalent}; the parameters $K_i > 0$ appearing in these
properties differ from each other by at most an absolute constant factor.
\begin{enumerate}
\item The \textbf{tails} of $X$ satisfy
\begin{align*}
\bP\set{\abs{X} \ge t} \le 2 \exp\paren{-t/K_1}\quad\text{ for all }t \ge 0.
\end{align*}

\item The \textbf{moments} of $X$ satisfy
\begin{align*}
\norm{X}{L^p} = \paren{\E{}{\abs{X}^p}}^{1/p} \le K_2\, p\quad \text{ for all }p \ge 1.
\end{align*}

\item The \textbf{moment-generating function (MGF)} of $\abs{X}$ satisfies
\begin{align*}
\E{}{\exp(\lambda \abs{X})} \le \exp(K_3\;\lambda) \quad \text{ for all $\lambda$ such that $0 \le \lambda \le \frac{1}{K_3}$}
\end{align*}

\item The \textbf{MGF} of $\abs{X}$ is \textbf{bounded} at some point, namely
\begin{align*}
\E{}{\exp(\abs{X} / K_4)} \le 2.
\end{align*}
Moreover, if $\E{}{X} = 0$ then properties $(1)$-$(4)$ are also \textbf{equivalent} to the following one.

\item The \textbf{MGF} of $X$ satisfies
\begin{align*}
\E{}{\exp(\lambda X)} \le  \exp(K_5^2\,\lambda^2)\quad\text{ for all }\lambda \text{ such that } \abs{\lambda} \le \frac{1}{K_5}.
\end{align*}
\end{enumerate}
\end{proposition}



\item \begin{definition} (\textbf{\emph{Sub-Exponential Norm}}) \\
The \underline{\emph{\textbf{sub-exponential norm}}} of $X$, denoted $\norm{X}{\psi_1}$, is defined
to be the \emph{\textbf{smallest}} $K_4$ that satisfies 
\begin{align*}
\E{}{\exp(\abs{X} / K_4)} \le 2.
\end{align*} In other words, we define
\begin{align}
\norm{X}{\psi_1}= \inf\set{t > 0: \E{}{\exp(\abs{X} / t)} \le 2}.  \label{eqn: sub_gaussian_norm}
\end{align}
\end{definition}

\item \begin{remark}
Sub-gaussian and sub-exponential distributions are closely related. 
\begin{enumerate}
\item First, \emph{any sub-gaussian distribution is clearly sub-exponential}. 
\item Second, \emph{the \textbf{square} of a \textbf{sub-gaussian random variable} is \textbf{sub-exponential}}:
\begin{lemma} (\textbf{Sub-exponential is Sub-gaussian Squared}).  \citep{vershynin2018high}\\
A random variable  $X$ is \textbf{sub-gaussian} if and only if $X^2$ is \textbf{sub-exponential}. Moreover,
\begin{align*}
\norm{X^2}{\psi_1} = \norm{X}{\psi_2}^2
\end{align*}
\end{lemma}

More generally, \emph{the \textbf{product} of two \textbf{sub-gaussian random variables} is \textbf{sub-exponential}}:
\begin{lemma} (\textbf{Product of Sub-Gaussians is Sub-Exponential}).  \citep{vershynin2018high}\\
Let $X$ and $Y$ be \textbf{sub-gaussian random variables}. Then $XY$ is \textbf{sub-exponential}. Moreover,
\begin{align*}
\norm{XY}{\psi_1} \le \norm{X}{\psi_2} \norm{Y}{\psi_2}.
\end{align*}
\end{lemma}
\end{enumerate}
\end{remark}

\item \begin{proposition} (\textbf{Moment Characterization of Sub-Exponential Random Variables})  \citep{boucheron2013concentration}\\
Let $X$ be a nonnegative random variable. If $X$ is sub-exponential distributed with parameter $a > 0$ 
then for every integer $q \ge 1$,
\begin{align}
\E{}{X^{q}} \le 2^{q+1}\frac{q!}{a^q}. \label{eqn: sub_exponential_power_moment}
\end{align}
\textbf{Conversely}, if there exists a constant $a > 0$ in order that for every positive integer $q$,
\begin{align*}
\E{}{X^{q}} \le  \frac{q!}{a^q},
\end{align*} then $X$ is sub-exponential. More precisely, for any $0 < \lambda < a$, 
\begin{align*}
\E{}{e^{\lambda X}} \le \frac{1}{1 - \lambda /a}.
\end{align*}
\end{proposition}

\item \begin{remark} (\textbf{\emph{Concentration Inequalities for Sub-Gamma Distribution}})\\
Similarly to the \emph{sub-Gaussian property}, the \emph{\textbf{sub-gamma property}} can be characterized in terms of \emph{tail or moment conditions}. We start by computing \emph{\textbf{the Fenchel-Legendre dual function}} of
\begin{align*}
\psi(\lambda) = \frac{\lambda^2 \nu}{2(1 - c\lambda)}.
\end{align*}
Setting
\begin{align*}
h_1(u) = 1 + u - \sqrt{1 + 2u}\text{ for }u > 0,
\end{align*}
it follows by elementary calculation that for every $t > 0$,
\begin{align*}
\psi^{*}(t)  = \sup_{\lambda \in (0, 1/c)}\set{t\lambda - \frac{\lambda^2 \nu}{2(1 - c\lambda)}} = \frac{\nu}{c^2}h_1\paren{\frac{c\,t}{\nu}}.
\end{align*}
Since $h_1$ is an increasing function from $(0, \infty)$ onto $(0, \infty)$ with \textbf{\emph{inverse function}}
\begin{align*}
h^{-1}(u) = u + \sqrt{2u}\text{ for }u > 0,
\end{align*} we finally get
\begin{align*}
\psi^{*-1}(u) = \sqrt{2\nu u} + c u.
\end{align*}
Hence, \emph{Chernoff's inequality} implies that whenever $X$ is a \emph{sub-gamma random variable on the right tail} with \emph{variance factor} $\nu$ and \textit{scale parameter} $c$, for every $t > 0$, we have
\begin{align}
\bP\set{X > t} \le \exp\paren{– \frac{\nu}{c^2}h_1\paren{\frac{c\,t}{\nu}}}, \label{ineqn: tail_sub_gamma}
\end{align}
or equivalently, for every $t > 0$,
\begin{align}
\bP\set{X > \sqrt{2\nu t}+ c t} \le e^{-t}.
\end{align} Therefore, if $X$ belongs to $\Gamma(\nu, c)$, then for every $t > 0$,
\begin{align*}
\bP\set{X > \sqrt{2\nu t}+ c t} \lor \bP\set{-X > \sqrt{2\nu t}+ c t}  \le e^{-t}. \qed
\end{align*}
\end{remark}
\end{itemize}

\subsection{Hoeffding's Inequality}
\begin{itemize}
\item \begin{remark} (\textbf{\emph{Bounded Variables}})\\
Bounded variables are an important class of \emph{sub-Gaussian random variables}. The \emph{sub-Gaussian property} of \emph{bounded random variables} is established by the
following lemma:
\end{remark}

\item \begin{lemma} (\textbf{Hoeffding's Lemma}) \citep{boucheron2013concentration} \\
Let $X$ be a random variable with $\E{}{X} = 0$, taking values in a \textbf{bounded interval} $[a, b]$ and let $\psi_{X}(\lambda) := \log  \E{}{e^{\lambda X}}$. Then
\begin{align*}
\psi_{X}''(\lambda) \le \frac{(b - a)^2}{4}
\end{align*}
and $X \in \cG((b - a)^2/4)$.
\end{lemma}

\item \begin{proposition} (\textbf{Hoeffding's inequality}) \citep{boucheron2013concentration} \\
Let $X_1 \xdotx{,} X_n$ be independent random variables such that $X_i$ takes its values in $[a_i, b_i]$ \textbf{almost surely} for all $i \le n$. Let
\begin{align*}
S &= \sum_{i=1}^{n}\paren{X_i - \E{}{X_i}}.
\end{align*}
Then for every $t > 0$,
\begin{align}
\bP\set{S \ge  t} \le \exp\paren{- \frac{2t^2}{\sum_{i=1}^{n}(b_i - a_i)^2}}. \label{ineqn: hoeffding_inequality}
\end{align}
\end{proposition}

\item 
 \begin{proposition} (\textbf{General Hoeffding's inequality}) \citep{vershynin2018high} \\
Let $X_1 \xdotx{,} X_n$ be \textbf{independent} \textbf{sub-gaussian} random variables. Let
\begin{align*}
S &= \sum_{i=1}^{n}\paren{X_i - \E{}{X_i}}.
\end{align*}
Then for every $t > 0$,
\begin{align}
\bP\set{S \ge  t}  \le \exp\paren{- \frac{c\,t^2}{\sum_{i=1}^{n}\norm{X_i}{\psi_2}}}. \label{ineqn: general_hoeffding_inequality}
\end{align}
\end{proposition}
\end{itemize}

\subsection{Bernstein's Inequality}
\begin{itemize}
\item \begin{definition} (\textbf{\emph{Bernstein's Condition}}) \\
Given a \emph{random variable} $X$ with mean $\mu = \E{}{X}$ we say that \underline{\emph{\textbf{Bernstein's condition}}} \emph{with parameter $\nu$, $c$} holds if the \emph{variance} $\text{Var}(X) = \E{}{X^2} - \mu^2 \le \nu$, and
\begin{align*}
\sum_{i=1}^{n}\E{}{(X - \mu)_{+}^q} \le \frac{q!}{2} \nu c^{q-2}, \quad \text{ for all integers }q \ge 2,
\end{align*} where $(x)_{+} = \max\set{x, 0}$. 
\end{definition}

\item \begin{remark}
If $X$ is \emph{bounded}, then it satisfies \emph{the Bernstein's condition}.

If $X$ satisfies \emph{the Bernstein's condition}, $X$ follows a \emph{\textbf{sub-gamma distribution}}. 
\end{remark}


\item \begin{proposition} (\textbf{Bernstein's Condition $\Rightarrow$  Sub-Gamma Distribution}). \citep{boucheron2013concentration}\\
Let $X_1 \xdotx{,} X_n$ be independent real-valued random variables and each $X_i$ satisfies \textbf{the Bernstein's condition} with parameter $\nu$ and $c$.  If $S = \sum_{i=1}^{n}\paren{X_i - \E{}{X_i}}$, then for all $\lambda \in (0, 1/c)$ and $t > 0$
\begin{align*}
\psi_S(\lambda) \le \frac{\lambda^2 \nu}{2(1 - c\lambda)} 
\end{align*} and
\begin{align*}
\psi_S^{*}(t) \ge \frac{\nu}{c^2}h_1\paren{\frac{ct}{\nu}},
\end{align*} where $h_1(u) = 1+ u - \sqrt{1 + 2u}$ for $u >0$. In particular, for all $t > 0$, 
\begin{align}
\bP\set{S \ge \sqrt{2\nu t} + ct } \le e^{-t}. \label{ineqn: bernstein_inequality_gammar_dist_0}
\end{align}
\end{proposition}

\item \begin{proposition}(\textbf{Bernstein's Inequality}).  \citep{boucheron2013concentration}\\
Let $X_1 \xdotx{,} X_n$ be independent real-valued random variables satisfying \textbf{the Bernstein's conditions} above and let $S = \sum_{i=1}^{n}\paren{X_i - \E{}{X_i}}$. Then for all $t > 0$,
\begin{align}
\bP\set{S \ge t } \le \exp\paren{- \frac{t^2}{2(\nu + ct)}}.  \label{ineqn: bernstein_inequality_gammar_dist}
\end{align}
\end{proposition}

\item \begin{corollary} (\textbf{Bernstein's Inequality for Bounded Distributions}).  \citep{vershynin2018high}\\
Let $X_1 \xdotx{,} X_n$ be \textbf{independent}, \textbf{mean zero} random variables, such that $\abs{X_i} \le b$ all $i$. Then, for every $t \ge 0$, we have
\begin{align}
\bP\set{\abs{\frac{1}{n}\sum_{i=1}^{n}X_i} \ge t} \le 2\exp\paren{- \frac{t^2}{2(\nu + bt/3)}}. \label{ineqn: bernstein_inequality_bounded}
\end{align}
Here $\nu = \sum_{i=1}^{n}\E{}{X_i^2}$ is the variance of the sum.
\end{corollary}


\item \begin{corollary} (\textbf{Bernstein's Inequality}).  \citep{vershynin2018high}\\
Let $X_1 \xdotx{,} X_n$ be \textbf{independent}, \textbf{mean zero}, \textbf{sub-exponential random variables}. Then, for every $t \ge 0$, we have
\begin{align}
\bP\set{\abs{\sum_{i=1}^{n}X_i} \ge t} \le 2 \exp\brac{- c \min\set{\frac{t^2}{\sum_{i=1}^{n}\norm{X_i}{\psi_2}^2},  \frac{t}{\max_i \norm{X_i}{\psi_1}}}} \label{ineqn: bernstein_inequality}
\end{align}
where $c > 0$ is an absolute constant.
\end{corollary}

\item 
\begin{proposition} (\textbf{Bernstein's Inequality, Linear Combination Form}).  \citep{vershynin2018high}\\
Let $X_1 \xdotx{,} X_n$ be \textbf{independent}, \textbf{mean zero}, \textbf{sub-exponential random variables}, and $a = (a_1 \xdotx{,} a_n) \in \bR^n$. Then, for every $t \ge 0$, we have
\begin{align}
\bP\set{\abs{\sum_{i=1}^{n}a_i X_i} \ge t} \le 2 \exp\brac{- c \min\set{\frac{t^2}{K^2\, \norm{a}{2}^2},  \frac{t}{K \norm{a}{\infty}}}} \label{ineqn: bernstein_inequality_linear_comb}
\end{align}
where $c > 0$ is an absolute constant and $K = \max_i \norm{X_i}{\psi_1}$.
\end{proposition}

\item \begin{corollary}(\textbf{Bernstein's Inequality, Average Form}).  \citep{vershynin2018high}\\
Let $X_1 \xdotx{,} X_n$ be \textbf{independent}, \textbf{mean zero}, \textbf{sub-exponential random variables}. Then, for every $t \ge 0$, we have
\begin{align}
\bP\set{\abs{\frac{1}{n}\sum_{i=1}^{n}X_i} \ge t} \le 2 \exp\brac{- c \min\set{\frac{t^2}{K^2},  \frac{t}{K}}n} \label{ineqn: bernstein_inequality_average}
\end{align} where $K = \max_i \norm{X_i}{\psi_1}$.
\end{corollary}

\end{itemize}
\subsection{Bennett's Inequality}
\begin{itemize}
\item \begin{remark}
Our starting point is the fact that  \emph{the logarithmic moment-generating function} of an \emph{independent sum} equals \emph{the sum of the logarithmic moment-generating functions of the centered summands}, that is,
\begin{align*}
\psi_{S}(\lambda) &=  \sum_{i=1}^{n}\paren{\log\E{}{e^{\lambda X_i}} - \lambda \E{}{X_i}}.
\end{align*}
Using $\log u \le u - 1$ for $u > 0$,
\begin{align}
\psi_{S}(\lambda) &\le \sum_{i=1}^{n}\E{}{e^{\lambda X_i} - \lambda X_i - 1}. \label{ineqn: log_mgf_bound}
\end{align}
Both Bennett's and Bernstein's inequalities may be derived from this bound, under different integrability conditions for the $X_i$.
\end{remark}

\item \begin{proposition} (\textbf{Bennett's Inequality}) \citep{boucheron2013concentration}\\
Let $X_1 \xdotx{,} X_n$ be independent random variables with \textbf{finite variance} such that $X_i \le b$ for some $b > 0$ \textbf{almost surely} for all $i \le n$. Let
\begin{align*}
S &= \sum_{i=1}^{n}\paren{X_i - \E{}{X_i}}
\end{align*} and $\nu = \sum_{i=1}^{n}\E{}{X_i^2}$. If we write $\phi(u) = e^u - u - 1$ for $u \in \bR$, then, for all $\lambda > 0$,
\begin{align*}
\log \E{}{e^{\lambda S}} \le n \log\paren{1 + \frac{\nu}{n b^2}\phi(b \lambda)} \le \frac{\nu}{b^2}\,\phi(b\lambda),
\end{align*}
and for any $t > 0$,
\begin{align}
\bP\set{S \ge t} \le \exp\paren{-\frac{\nu}{b^2}h\paren{\frac{b\,t}{\nu}}} \label{ineqn: bennett_inequality}
\end{align}
where $h(u) = (1 + u) \log(1 + u) - u$ for $u > 0$.
\end{proposition}


\item \begin{remark} This bound can be analyzed in two different regimes:
\begin{enumerate}
\item \emph{In the \textbf{small deviation regime}}, where $u := b t/ \nu \ll 1$, we have \emph{asymptotically} $h(u) \approx u^2$ and \emph{Bennett's inequality} gives \emph{approximately the Gaussian tail bound} $\approx \exp(-t^2/\nu)$. 

\item \emph{In the \textbf{large deviations regime}}, say where $u := b t/ \nu \ge 2$, we have $h(u) \ge \frac{1}{2}u \log u$, and \emph{Bennett's inequality} gives a \emph{\textbf{Poisson-like tail}} $(\nu/bt)^{t/2 b}$.
\end{enumerate}
\end{remark}
\end{itemize}

\subsection{The Johnson-Lindenstrauss Lemma}


\section{Martingale Method}
\subsection{Martingale and Martingale Difference Sequence}
\begin{itemize}
\item \begin{definition} (\emph{\textbf{Martingale}}) \citep{resnick2013probability}\\
Let $\set{X_n, n \ge 0}$ be a stochastic process on $(\Omega, \srF)$ and $\set{\srF_n, n \ge 0}$ be a \underline{\textbf{\emph{filtration}}}; that is, $\set{\srF_n, n \ge 0}$ is an \emph{increasing sub $\sigma$-fields} of $\srF$
\begin{align*}
\srF_0 \subseteq \srF_1 \subseteq \srF_2 \xdotx{\subseteq} \srF.
\end{align*} Then $\set{ (X_n, \srF_n),  n \ge 0}$ is a \underline{\emph{\textbf{martingale (mg)}}} if
\begin{enumerate}
\item  $X_n$ is \emph{\textbf{adapted}} in the sense that for each $n$, $X_n \in \srF_n$; that is, $X_n$ is $\srF_n$-measurable.
\item  $X_n \in L_1$; that is $\E{}{\abs{X_n}} < \infty$ for $n \ge 0$.
\item For $0 \le m < n$
\begin{align}
\E{}{X_n \;|\; \srF_m} &= X_m, \quad \text{a.s.} \label{def: martingale}
\end{align}
\end{enumerate}
If the equality of \eqref{def: martingale} is replaced by $\ge$; that is, things are getting better on the average:
\begin{align}
\E{}{X_n \;|\; \srF_m} &\ge X_m, \quad \text{a.s.} \label{def: sub_martingale}
\end{align} then $\set{X_n}$ is called a \underline{\emph{\textbf{sub-martingale (submg)}}} while if things are getting worse on
the average
\begin{align}
\E{}{X_n \;|\; \srF_m} &\le X_m, \quad \text{a.s.} \label{def: sup_martingale}
\end{align}  $\set{X_n}$ is called a \underline{\emph{\textbf{super-martingale (supermg)}}}.
\end{definition}

\item \begin{remark}
$\set{X_n}$ is \emph{\textbf{martingale}} if it is \emph{both} a \emph{\textbf{sub}} and \emph{\textbf{supermartingale}}. $\set{X_n}$ is a \emph{\textbf{supermartingale}} if and only if $\set{-X_n}$ is a \emph{\textbf{submartingale}}.
\end{remark}

\item \begin{remark}
If $\set{X_n}$ is a \emph{\textbf{martingale}}, then $\E{}{X_n}$ is \emph{constant}. In the case of a \emph{\textbf{submartingale}}, \emph{the mean increases} and for a \emph{\textbf{supermartingale}}, \emph{the mean decreases}.
\end{remark}

\item \begin{proposition} \citep{resnick2013probability}\\
If  $\set{ (X_n, \srF_n),  n \ge 0}$ is a \textbf{(sub, super) martingale}, then 
\begin{align*}
\set{ (X_n, \sigma\paren{X_0, X_1 \xdotx{,} X_n}),  n \ge 0}
\end{align*} is also a \textbf{(sub, super) martingale}.
\end{proposition}

\item \begin{definition} (\textbf{\emph{Martingale Differences}}).  \citep{resnick2013probability}\\
$\set{(d_j, \srB_j), j \ge 0}$ is a \underline{\emph{\textbf{(sub, super) martingale difference sequence}}} or a \textit{\textbf{(sub, super) fair sequence}} if
\begin{enumerate}
\item For $j \ge 0$,  $\srB_j \subset \srB_{j+1}$.
\item For $j \ge 0$,  $d_j \in L_1$,  $d_j \in \srB_j$; that is, $d_j$ is \emph{absolutely integrable} and \emph{$\srB_j$-measurable}.
\item For $j \ge 0$,
\begin{align*}
\E{}{d_{j+1} | \srB_j} &= 0, && \text{(\emph{martingale difference / fair sequence})};\\
& \ge 0, && \text{(\emph{submartingale difference / subfair sequence})};\\
& \le 0, && \text{(\emph{supmartingale difference / supfair sequence})}
\end{align*}
\end{enumerate}
\end{definition}

\item \begin{proposition} (\textbf{Construction of Martingale From Martingale Difference})\citep{resnick2013probability}\\
If $\set{(d_j, \srB_j), j \ge 0}$ is \textbf{(sub, super) martingale difference sequence}, and
\begin{align*}
X_n = \sum_{j=0}^{n} d_j, 
\end{align*} then $\set{(X_n, \srB_n), n \ge 0}$ is a \textbf{(sub, super) martingale}.
\end{proposition}

\item \begin{proposition} (\textbf{Construction of Martingale Difference From Martingale}) \citep{resnick2013probability}\\
Suppose $\set{(X_n, \srB_n), n \ge 0}$ is a \textbf{(sub, super) martingale}. Define
\begin{align*}
d_0&:= X_0 - \E{}{X_0}\\
d_j &:= X_j - X_{j-1}, \quad j\ge 1.
\end{align*}
Then $\set{(d_j, \srB_j), j \ge 0}$ is a \textbf{(sub, super) martingale difference sequence}.
\end{proposition}

\item \begin{proposition} (\textbf{Orthogonality of Martingale Differences}). \citep{resnick2013probability}\\
If $\set{(X_n, \srB_n), n \ge 0}$ is a \textbf{martingale} where $X_n$ can be decomposed as
\begin{align*}
X_n = \sum_{j=0}^{n} d_j, 
\end{align*}  $d_j$ is $\srB_j$-measurable and  $\mathds{E}[d_j^2] < \infty$ for $j \ge 0$, then $\set{d_j}$ are \textbf{orthogonal}:
\begin{align*}
\E{}{d_i\,d_j} = 0 \quad i \neq j.
\end{align*}
\end{proposition}

\item \begin{example} (\textbf{\emph{Smoothing as Martingale}})\\
Suppose $X \in L_1$ and $\set{\srB_n, n \ge 0}$ is an increasing family of sub $\sigma$-algebra of $\srB$. Define for $n \ge 0$
\begin{align*}
X_n &:= \E{}{X | \srB_n}.
\end{align*}
Then $(X_n, \srB_n)$ is a \emph{\textbf{martingale}}. From this result, we see that $\set{(d_n, \srB_n), n \ge 0}$ is a \emph{\textbf{martingale difference sequence}} when 
\begin{align}
d_n &:= \E{}{X | \srB_n} - \E{}{X | \srB_{n-1}}, \quad n\ge 1. \label{eqn: smoothing_martingale_difference}
\end{align}
\end{example}

\item \begin{example}(\emph{\textbf{Sums of Independent Random Variables}}) \\
Suppose that $\set{Z_n, n \ge 0}$ is an \emph{\textbf{independent} sequence of integrable random variables} satisfying for $n \ge 0$, 
$\E{}{Z_n} = 0$.  Set
\begin{align*}
X_0 &:= 0,\\
X_n &:= \sum_{i=1}^{n}Z_i, \quad n \ge 1 \\
\srB_n &:= \sigma\paren{Z_0 \xdotx{,} Z_n}.
\end{align*} Then $\set{(X_n, \srB_n), n \ge 0}$ is a \emph{\textbf{martingale}} since $\set{(Z_n, \srB_n), n \ge 0}$ is a \emph{\textbf{martingale difference sequence}}.
\end{example}

\item \begin{example} (\emph{\textbf{Likelihood Ratios}}).\\ 
Suppose $\set{Y_n, n \ge 0}$ are \emph{\textbf{independent identically distributed}} random variables and suppose \emph{the true density} of $Y_n$ is $f_0$· (The word ``\emph{density}" can be understood with respect to some fixed reference measure $\mu$.)  Let $f_1$ be \emph{some other probability density}. For simplicity suppose $f_0(y) > 0$, for all $y$.  For $n \ge 0$, define the likelihood ratio
\begin{align*}
X_n &:= \frac{\prod_{i=0}^{n}f_1(Y_i)}{\prod_{i=0}^{n}f_0(Y_i)}\\
\srB_n &:= \sigma\paren{Y_0 \xdotx{,} Y_n}
\end{align*} Then $(X_n, \srB_n)$ is a \emph{\textbf{martingale}}.
\end{example}
\end{itemize}
\subsection{Bernstein Inequality for Martingale Difference Sequence}
\begin{itemize}
\item \begin{proposition} (\textbf{Bernstein Inequality, Martingale Difference Sequence Version}) \citep{wainwright2019high}\\
Let $\set{(D_k, \srB_k), k \ge 1}$ be a \textbf{martingale difference sequence}, and suppose that 
\begin{align*}
\E{}{\exp\paren{\lambda D_k} | \srB_{k-1}} \le \exp\paren{\frac{\lambda^2 \nu_k^2}{2} }
\end{align*} almost surely for any $\abs{\lambda} < 1/\alpha_k$. Then the following hold:
\begin{enumerate}
\item The sum $\sum_{k=1}^{n}D_k$ is \textbf{sub-exponential} with \textbf{parameters} $\paren{\sqrt{\sum_{k=1}^{n}\nu_k^2}\;  , \;\alpha_{*}}$ where $\alpha_{*} := \max_{k=1 \xdotx{,} n} \alpha_k$. That is, for any $\abs{\lambda} < 1/\alpha_{*}$, 
\begin{align*}
\E{}{\exp\set{\lambda \paren{\sum_{k=1}^{n}D_k}}} \le \exp\paren{\frac{\lambda^2\sum_{k=1}^{n}\nu_k^2}{2} }
\end{align*}
\item The sum satisfies \textbf{the concentration inequality}
\begin{align}
\bP\set{\abs{\sum_{k=1}^{n}D_k} \ge t } &\le \left\{ \begin{array}{cc}
2 \exp\paren{- \frac{t^2}{2 \sum_{k=1}^{n}\nu_k^2}} & \text{ if } 0 \le t \le \frac{\sum_{k=1}^{n}\nu_k^2}{\alpha_{*}} \\[15pt]
2 \exp\paren{- \frac{t}{\alpha_{*}}} &\text{ if } t > \frac{\sum_{k=1}^{n}\nu_k^2}{\alpha_{*}}.
\end{array}\right. \label{ineqn: bernstein_inequality_martingale}
\end{align}
\end{enumerate}
\end{proposition}
\end{itemize}
\subsection{Azuma-Hoeffding Inequality}
\begin{itemize}
\item \begin{corollary} (\textbf{Azuma-Hoeffding Inequality})\citep{wainwright2019high}\\
Let $\set{(D_k, \srB_k), k \ge 1}$ be a \textbf{martingale difference sequence} for which there are constants $\set{(a_k, b_k)}^{n}_{k=1}$ such that $D_k \in [a_k, b_k]$ almost surely for all $k = 1 \xdotx{,} n$. Then, for all $t \ge 0$,
\begin{align}
\bP\set{\abs{\sum_{k=1}^{n}D_k} \ge t } &\le  2 \exp\paren{- \frac{2 t^2}{ \sum_{k=1}^{n}(b_k - a_k)^2}} \label{ineqn: hoeffding_inequality_martingale}
\end{align}
\end{corollary}
\end{itemize}
\subsection{Bounded Difference Inequality}
\begin{itemize}
\item An important application of \emph{Azuma-Hoeffding Inequality} concerns functions that satisfy a \emph{bounded difference property}. 
\begin{definition} (\textbf{\emph{Functions with Bounded Difference Property}})\\
Given vectors $x, x' \in \cX^n$ and an index $k \in \set{1, 2 \xdotx{,} n}$, we define a new vector $x^{(-k)} \in \cX^n$ via
\begin{align*}
x_j^{(-k)} &= \left\{\begin{array}{cc}
x_j & j \neq k\\
x_k'& j = k
\end{array}
\right.
\end{align*}
With this notation, we say that $f: \cX^n \to \bR$ satisfies \underline{\textbf{\emph{the bounded difference inequality}}} with parameters $(L_1 \xdotx{,} L_n)$ if, for each index $k = 1, 2 \xdotx{,} n$,
\begin{align}
\abs{f(x) - f(x^{(-k)})} \le L_k, \quad\text{ for all }x, x' \in \cX^n. \label{eqn: bounded_difference_property}
\end{align}
\end{definition}


\item \begin{corollary} (\textbf{McDiarmid's Inequality / Bounded Differences Inequality})\citep{wainwright2019high}\\
Suppose that $f$ satisfies \textbf{the bounded difference property} \eqref{eqn: bounded_difference_property} with parameters $(L_1 \xdotx{,} L_n)$ and that the random vector $X = (X_1, X_2 \xdotx{,} X_n)$ has \textbf{independent} components. Then
\begin{align}
\bP\set{\abs{f(X) - \E{}{f(X)}} \ge t } &\le  2 \exp\paren{- \frac{2 t^2}{ \sum_{k=1}^{n}L_k^2}}. \label{ineqn: macdiarmid_bounded_difference_inequality}
\end{align}
\end{corollary}
\end{itemize}

\section{Bounding Variance}
\subsection{Mean-Median Deviation}
\begin{itemize}
\item \begin{definition} (\emph{\textbf{Median of Random Variable}})\\
\emph{The \textbf{median}} of a random variable $X \in \cX$ with distribution $\bP$ is a constant $m$ such that
\begin{align*}
\bP\set{X \ge m } \ge \frac{1}{2} \quad \land \quad \bP\set{X \le m } \ge \frac{1}{2}
\end{align*}
\end{definition}

\item \begin{proposition} (\textbf{Mean-Median Deviation, Variance Bound}) \citep{boucheron2013concentration} \\
Let $X \in \cX$ be a random variable with distribution $\bP$, $m$ be the \textbf{median} of $X$ and $\mu = \E{}{X}$ be the \textbf{mean} of $X$. If $\text{Var}(X) = \sigma^2 < \infty$,  then 
\begin{align}
\abs{m - \mu} &\le \sqrt{\text{Var}(X)} = \sigma \label{ineqn: median_mean_variance_bound}
\end{align}
\end{proposition} (proof by Jenson's inequality $\abs{m - \mu} = \abs{\E{}{X - m}} \le \E{}{\abs{X - m}} \le \E{}{\abs{X - \mu}} \le \sqrt{\E{}{\abs{X - \mu}^2}}$)

\item \begin{exercise}  (\textbf{Mean-Median Deviation via Concentration Inequality}) \citep{boucheron2013concentration} \\
Let $X$ be a random variable with \textbf{median} $m$ such that positive constants $a$ and $b$ exist so that for all $t > 0$,
\begin{align*}
\bP\set{\abs{X - m} \ge t} \le a \exp\paren{- \frac{t^2}{b}}
\end{align*} Show that 
\begin{align*}
\abs{m - \mu} &\le \min\set{\sqrt{ab}, \;\frac{a}{2}\sqrt{b\pi}}.
\end{align*}
\end{exercise}

\item \begin{exercise} (\textbf{Concentration Inequality Around Medians and Means}) \citep{wainwright2019high} \\
Given a scalar random variable $X$, suppose that there are positive constants $c_1$, $c_2$ such that for all $t \ge 0$, 
\begin{align}
\bP\set{\abs{X - \E{}{X}} \ge t} &\le c_1 \exp\paren{- c_2 t^2} \label{exe: mean_concentration_bound}
\end{align}
\begin{enumerate}
\item Prove that $\text{Var}(X) \le \frac{c_1}{c_2}$
\item Let $m_X$ be the a \textbf{median} of $X$. Show that \textbf{whenever the mean concentration bound \eqref{exe: mean_concentration_bound} holds}, then for \textbf{any median} $m_X$, we have, for all $t \ge 0$, \textbf{the median concentration}
\begin{align}
\bP\set{\abs{X -  m_X} \ge t} &\le c_3 \exp\paren{- c_4 t^2} \label{exe: median_concentration_bound}
\end{align} where $c_3 := 4c_1$ and $c_4 := \frac{c_2}{8}$. 
\item Conversely, show that \textbf{whenever the median concentration bound \eqref{exe: median_concentration_bound} holds}, then \textbf{mean concentration} \eqref{exe: mean_concentration_bound} holds with $c_1 = 2 c_3$ and $c_2 = \frac{c_4}{4}$.
\end{enumerate}
\end{exercise}
\end{itemize}
\subsection{The Efron-Stein Inequality and Jackknife Estimation}
\begin{itemize}
\item \begin{remark} (\textbf{\emph{Variance of Smoothing Martingale Difference Sequence}})\\
Suppose $X \in L_1$ and $\set{\srB_n, n \ge 0}$ is an increasing family of sub $\sigma$-algebra of $\srB$ formed by 
\begin{align*}
\srB_n &:= \sigma\paren{Z_1 \xdotx{,} Z_n}.
\end{align*} For $n \ge 1$, define 
\begin{align*}
d_0 &:= \E{}{X} \\ 
d_n &:= \E{}{X | \srB_n} - \E{}{X | \srB_{n-1}} \\
&= \E{}{X | Z_1 \xdotx{,} Z_n} -  \E{}{X | Z_1 \xdotx{,} Z_{n-1}}.
\end{align*} From \eqref{eqn: smoothing_martingale_difference} we see that $(d_n, \srB_n)$ is a martingale difference sequence. By \emph{orthogonality of martingale difference}, we see that 
\begin{align*}
\E{}{d_i\, d_j} =0 \quad i\neq j.
\end{align*} Therefore, based on the decomposition
\begin{align*}
X - E{}{X} &= \sum_{i=1}^{n}d_i
\end{align*}
we have 
\begin{align}
\text{Var}(X) &= \E{}{\paren{\sum_{i=1}^{n}d_i}^2} = \sum_{i=1}^{n}\E{}{d_i^2} + 2 \sum_{i > j}\E{}{d_i\,d_j}\nonumber\\
&=  \sum_{i=1}^{n}\E{}{d_i^2}. \label{eqn: martingale_smoothing}
\end{align}
\end{remark}

\item \begin{remark}(\textbf{\emph{Variance of General Functions of Independent Random Variables}})\\
Then above formula \eqref{eqn: martingale_smoothing} holds when $X = f\paren{Z_1 \xdotx{,} Z_n}$ for general function $f: \bR^n \to \bR$ with $n$ independent random variables $(Z_1 \xdotx{,} Z_n)$. By \emph{Fubini's theorem},
\begin{align*}
 \E{}{X | Z_1 \xdotx{,} Z_i} &= \int_{\cZ^{n-i}} f(Z_1 \xdotx{,} Z_i, z_{i+1} \xdotx{,} z_n) \;\;d\mu_{i+1}(z_{i+1})  \xdotx{} d\mu_{n}(z_{n})
\end{align*} where $\mu_j$ is the probability distribution of $Z_j$ for $j \ge 1$. 

Let $Z_{(-i)} := (Z_1 \xdotx{,} Z_{i-1}, Z_{i+1} \xdotx{,} Z_n)$ be all random variables $(Z_1 \xdotx{,} Z_n)$ \emph{\textbf{except for}} $Z_i$ . Denote $\E{(-i)}{\cdot}$ as the conditional expectation of $X$ given $Z_{(-i)}$
\begin{align*}
 \E{(-i)}{X} &:= \E{}{X | Z_1 \xdotx{,} Z_{i-1}, Z_{i+1} \xdotx{,} Z_n} \\
 &= \int_{\cZ} f(Z_1 \xdotx{,} Z_{i-1}, z_{i}, Z_{i+1} \xdotx{,} Z_n) \;\;d\mu_{i}(z_{i}).
\end{align*} Then, again by \emph{Fubini's theorem} (\emph{smoothing properties of conditional expectation}),
\begin{align}
\E{}{ \E{(-i)}{X} | Z_1 \xdotx{,} Z_i} &= \E{}{X | Z_1 \xdotx{,} Z_{i-1}} \label{eqn: martingale_smoothing_expectation}
\end{align} 
\end{remark}

\item \begin{proposition}(\textbf{Efron-Stein Inequality}) \citep{boucheron2013concentration} \\
Let $Z_1 \xdotx{,} Z_n$ be \textbf{independent random variables} and let $X = f(Z)$ be a square-integrable function of $Z = (Z_1 \xdotx{,} Z_n)$. Then
\begin{align}
\text{Var}(X) &\le  \sum_{i=1}^{n}\E{}{\paren{X - \E{(-i)}{X}}^2} := \nu.  \label{ineqn: efron_stein_inequality}
\end{align}
Moreover, if $Z_1' \xdotx{,} Z_n'$ are \textbf{independent} copies of $Z_1 \xdotx{,} Z_n$ and if we define, for every $i = 1 \xdotx{,} n$,
\begin{align*}
X_i' &:= f\paren{Z_1 \xdotx{,} Z_{i-1}, Z_{i}' ,Z_{i+1} \xdotx{,} Z_n},
\end{align*}
then
\begin{align*}
\nu &= \frac{1}{2}\sum_{i=1}^{n}\E{}{\paren{X -  X_i'}^2} = \sum_{i=1}^{n}\E{}{\paren{X -  X_i'}_{+}^2} = \sum_{i=1}^{n}\E{}{\paren{X -  X_i'}_{-}^2}
\end{align*}
where $x_{+} = \max\set{x, 0}$ and $x_{-} = \max\set{-x, 0}$ denote the \textbf{positive} and \textbf{negative} parts of a real number $x$. Also,
\begin{align*}
\nu &= \inf_{X_i}\;\sum_{i=1}^{n}\E{}{\paren{X -  X_i}^2},
\end{align*}
where the infimum is taken over the class of all $Z_{(-i)}$-measurable and square-integrable variables $X_i$, $i = 1 \xdotx{,} n$.
\end{proposition}


\item \begin{example} (\textbf{\emph{The Jackknife Estimate}}) \\
We should note here that the Efron-Stein inequality was first motivated by the study of the so-called \underline{\emph{\textbf{jackknife estimate} of \textbf{statistics}}}. 

To describe this estimate, assume that $Z_1 \xdotx{,} Z_n$ are i.i.d. random variables and one wishes to\emph{ estimate a functional $\theta$ of the distribution} of the $Z_i$ by a function $X = f(Z_1 \xdotx{,} Z_n)$ of the data. The quality of the estimate is often measured by its bias $\E{}{X} - \theta$ and its variance $\text{Var}(X)$. Since the distribution of the $Z_i$'s is unknown, one needs to \emph{estimate} the bias and variance \emph{\textbf{from the same sample}}. \underline{\emph{\textbf{The jackknife estimate}} of \emph{\textbf{the bias}}} is defined by
\begin{align}
(n-1)\paren{\frac{1}{n}\sum_{i=1}^{n}X_i - X} \label{eqn: jackknive_estimate_bias}
\end{align} where $X_i$ is an appropriately defined function of $Z_{(-i)} := (Z_1 \xdotx{,} Z_{i-1}, Z_{i+1} \xdotx{,} Z_n)$. $Z_{(-i)}$ is often called \emph{\textbf{the $i$-th jackknife sample}} while $X_i$ is the so-called \emph{\textbf{jackknife replication} of $X$}. In an analogous way, \underline{\emph{\textbf{the jackknife estimate} of the \textbf{variance}}} is defined by
\begin{align}
\sum_{i=1}^{n}\paren{X - X_i}^2 \label{eqn: jackknive_estimate_bias}
\end{align}
Using this language, \emph{\textbf{the Efron-Stein inequality}} simply states that \emph{\textbf{the jackknife estimate of the variance} is \underline{\textbf{always positively biased}}}. In fact, this is how Efron and Stein originally formulated their inequality.
\end{example}
\end{itemize}
\subsection{Functions with Bounded Differences}
\begin{itemize}
\item \begin{corollary} \citep{boucheron2013concentration}\\
If $f$ has the \textbf{bounded differences property} with parameters $(L_1 \xdotx{,} L_n)$, then
\begin{align*}
\text{Var}(f(Z)) &\le \frac{1}{4}\sum_{i=1}^{n}L_i^2.
\end{align*}
\end{corollary}
\end{itemize}
\subsection{Convex Poincar{\'e} Inequality}
\begin{itemize}
\item \begin{theorem} (\textbf{Convex Poincar{\'e} Inequality}) \citep{boucheron2013concentration} \\
Let $Z_1 \xdotx{,} Z_n$ be \textbf{independent} random variables taking values in the interval $[0, 1]$ and let $f : [0, 1]^n \to \bR$ be a \textbf{separately convex function} whose partial derivatives exist; that is, for every $i = 1 \xdotx{,} n$ and fixed $z_1 \xdotx{,} z_{i-1}, z_{i+1} \xdotx{,} z_n$,  $f$ is a convex function of its $i$-th variable. Then $f(Z) = f(Z_1 \xdotx{,} Z_n)$ satisfies
\begin{align}
\text{Var}(f(Z)) &\le \E{}{\norm{\nabla f(Z)}{2}^2}. \label{ineqn: convex_poincare_inequality}
\end{align}
\end{theorem}
\end{itemize}
\subsection{Gaussian Poincar{\'e} Inequality}
\begin{itemize}
\item \begin{theorem} (\textbf{Gaussian Poincar{\'e} Inequality}) \citep{boucheron2013concentration} \\
Let $Z = (Z_1 \xdotx{,} Z_n)$ be a vector of \textbf{i.i.d. standard Gaussian} random variables (i.e. $Z$ is a Gaussian vector with \textbf{zero mean} vector and \textbf{identity covariance matrix}). Let $f : \bR^n \to \bR$ be any \textbf{continuously differentiable} function. Then
\begin{align}
\text{Var}(f(Z)) &\le \E{}{\norm{\nabla f(Z)}{2}^2}. \label{ineqn: gaussian_poincare_inequality}
\end{align}
\end{theorem}
\end{itemize}


\section{Entropy Method}
\subsection{Entropy Functional and $\Phi$-Entropy}
\begin{itemize}
\item \begin{definition}(\textbf{\emph{$\Phi$-Entropy}})\citep{boucheron2013concentration}\\
Let $\Phi: [0, \infty) \rightarrow \bR$ be a \textbf{\emph{convex}} function, and assign, to every \emph{\textbf{non-negative} integrable random variable} $X$, \underline{\textbf{\emph{the $\Phi$-entropy}}} of $X$ is defined as 
\begin{align}
H_{\Phi}(X) &= \E{}{\Phi(X)} - \Phi(\E{}{X}). \label{def: phi_entropy}
\end{align}
\end{definition}

\item \begin{remark}
The $\Phi$-entropy is a \emph{\textbf{functional}} of \emph{distribution} $P_{X}$ instead of a function of $X$.
\end{remark}

\item \begin{remark}
By Jenson's inequality, the $\Phi$-entropy is \emph{non-negative}
\begin{align*}
\Phi(\E{}{X}) &\le \E{}{\Phi(X)}\\
\Rightarrow H_{\Phi}(X) &= \E{}{\Phi(X)} - \Phi(\E{}{X}) \ge 0.
\end{align*} 
\end{remark}

\item \begin{example} (\emph{\textbf{Special Examples for $\Phi$-Entropy}})
\begin{enumerate}
\item For $\Phi(x) = x^2$, \emph{the $\Phi$-entropy} of $X$ is \emph{the \textbf{variance}} of $X$:
\begin{align*}
H_{\Phi}(X) &= \E{}{X^2} - (\E{}{X})^2 = \text{Var}(X).
\end{align*}
\item For $\Phi(x) = -\log(x)$, \emph{the $\Phi$-entropy} of $Y=e^{\lambda X}$ is \emph{the \textbf{logarithm of moment generating function}} of $X - \E{}{X}$:
\begin{align}
H_{\Phi}(e^{\lambda X}) &= -\lambda \E{}{X} + \log\paren{\E{}{e^{\lambda X}}} = \log\E{}{e^{\lambda(X - \E{}{X})}} := \psi_{X- \E{}{X}}(\lambda). \label{eqn: phi_entropy_log_mgf}
\end{align}
\item For $\Phi(x) = x\log x$, \emph{the $\Phi$-entropy} of $X$ is defined as the \underline{\emph{\textbf{entropy functional}} of $X$}
\begin{align}
H_{\Phi}(X) = \text{Ent}(X)&:=  \E{}{X\log X} - \E{}{X}\log\paren{\E{}{X}}. \label{def: phi_entropy_kl}
\end{align} Let $(\Omega, \srB)$ be measurable space, and $P$ and $Q$ are probability measures on $\Omega$ with $P \ll Q$. Define a random variable $X$ by the \emph{Radon-Nikodym derivative} of $P$ with respect to $Q$; that is,
\begin{align*}
X(\omega) &:= \left\{ \begin{array}{cc}
\frac{dP}{dQ}(\omega) & Q(\omega) > 0\\
0 &\text{o.w.}
\end{array}
\right. .
\end{align*} We see that $X$ is $Q$-measurable and $dP = X\,dQ$ with $\E{Q}{X} = 1$. Then the entropy of $X$ is the relative entropy of $P$ with respect to $Q$.
\begin{align}
\text{Ent}(X)&= \kl{P}{Q} \label{eqn: phi_entropy_kl_divg}
\end{align}
\end{enumerate}
\end{example}
\end{itemize}
\subsection{Dual Formulation}
\begin{itemize}
\item \begin{lemma}
The \textbf{Legendre transform} (or \textbf{convex conjugate}) of $\Phi(x) = x\log(x)$ is $e^{u-1}$. That is,
\begin{align*}
\sup_{x > 0}\set{u\,x - x\log(x)} &= e^{u-1}
\end{align*}
\end{lemma}

\item \begin{proposition}(\textbf{Duality Formula of Entropy}) \citep{boucheron2013concentration}\\
Let $X$ be a non-negative random variable defined on a probability space $(\Omega, \srA, P)$ such that $\E{}{\Phi(X)} < \infty$. Then we have \textbf{the duality formula}
\begin{align}
\text{Ent}(X) &= \sup_{U \in \cU}\E{}{U\,X}  \label{eqn: duality_entropy}
\end{align} where the supremum is taken over the set $\cU$ of all random variables $U: \Omega \to \bR \cup \set{\infty}$ with $\E{}{e^{U}} = 1$. Moreover, if $U$ is such that $\E{}{U X} \le \text{Ent}(X)$ for all non-negative random variable $X$ such that $\Phi(X)$ is integrable and $\E{}{X} = 1$, then $\E{}{e^{U}} \le 1$. 
\end{proposition}

\item  \begin{corollary} \label{thm: duality_entropy_2}(\textbf{Alternative Duality Formula of Entropy}) \citep{boucheron2013concentration}
\begin{align}
\text{Ent}(X) &= \sup_{T}\E{}{X\paren{\log(T) - \log\paren{\E{}{T}}}}  \label{eqn: duality_entropy_2}
\end{align} where the supremum is taken over all non-negative and integrable random variables.
\end{corollary}

\item \begin{corollary} \label{coro: dual_log_mgf}  (\textbf{Duality Formula of Log-MGF}) \citep{thomas2006elements, boucheron2013concentration}\\
Let $X$ be a real-valued integrable random variable. Then for every $\lambda \in \bR$, 
\begin{align}
\log \E{\bP}{e^{\lambda\paren{X - \E{}{X}}}} &= \sup_{\bQ \ll \bP}\set{\lambda\paren{\E{\bQ}{X} - \E{\bP}{X}} - \kl{\bQ}{\bP} }, \label{eqn: duality_log_mgf}
\end{align} where the supremum is taken over all probability measures $\bQ$ absolutely continuous with respect to $\bP$.
\end{corollary}

\item \begin{corollary}  (\textbf{Duality Formula of Kullback-Leibler Divergence}) \citep{thomas2006elements, boucheron2013concentration}\\
Let $\bP$ and $\bQ$ be two probability distributions on the same space.Then
\begin{align}
\kl{\bQ}{\bP} &= \sup_{X}\set{\E{\bQ}{X} - \log \E{\bP}{e^{X}} }, \label{eqn: duality_kl_divg}
\end{align} where the supremum is taken over all random variables such that $\E{\bP}{\exp\paren{X}} < \infty$.
\end{corollary}

\item \begin{definition} (\textbf{\emph{Bregman Divergence}}) \\
Let $F: \cX \rightarrow \bR$ be a \emph{continuously-differentiable}, \emph{\textbf{strictly convex}} function defined on a convex set $\cX$. The \underline{\textbf{\emph{Bregman divergence}}} associated with $F$ for points $p,q \in \cX$ is the difference between the value of $F$ at point $p$ and the value of the \emph{first-order Taylor expansion} of F around point $q$ evaluated at point $p$:
\begin{align}
\divg{F}{p}{q} &= F(p) - F(q) - \inn{\grad{}{F}(q)}{p - q} \label{eqn: bregman_divg}
\end{align}
\end{definition}

\item \begin{theorem} (\textbf{The Expected Value Minimizes Expected Bregman Divergence}) \citep{boucheron2013concentration} \\
Let $I \subseteq \bR$ be an open interval and let $f: I \to \bR$ be \textbf{convex} and \textbf{differentiable}. For any $x,y \in I$, \textbf{the Bregman divergence} of $f$ from $x$ to $y$ is $f(y) - f(x) - f'(x)(y-x)$. Let $X$ be an $I$-valued random variable. Then
\begin{align}
\E{}{f(X) - f(\E{}{X}) } &= \inf_{a \in I}\E{}{f(X) - f(a) - f'(a)(X-a)}\label{eqn: bregmann_inf}
\end{align}
\end{theorem}

\item \begin{corollary} (\textbf{Duality Formula of Entropy via Bregman Divergence}) \citep{boucheron2013concentration}\\
Let $X$ be a non-negative random variable such that $\E{}{\Phi(X)} < \infty$. Then 
\begin{align}
\text{Ent}(X) &= \inf_{u > 0}\E{}{X\paren{\log(X) - \log(u)} - (X - u) } \label{eqn: duality_entropy_3_breg}
\end{align}
\end{corollary}
\end{itemize}
\subsection{Tensorization Property}
\begin{itemize}
\item \begin{proposition} (\textbf{Sub-Additivity of The Entropy / Tensorization Property}) \citep{boucheron2013concentration}\\
Let $\Phi(x) = x\log x$,  for $x >0$ and $\Phi(0) = 0$. Let $Z_1, Z_2 \xdotx{,} Z_n$ be independent random variables taking values in $\cX$, and let $f: \cX^n \to [0, \infty)$ be a measurable function. Letting $X = f(Z_1, Z_2 \xdotx{,} Z_n)$ such that $\E{}{X\log X} < \infty$, we have 
\begin{align}
\text{Ent}(X) := \E{}{\Phi(X)} - \Phi(\E{}{X}) &\le \sum_{i=1}^{n}\E{}{\E{(-i)}{\Phi(X)} - \Phi(\E{(-i)}{X})}, \label{ineqn: sub_additivity_phi_entropy}
\end{align} where $\E{(-i)}{\cdot}$ is the conditional expectation operator conditioning on $Z_{(-i)}$. Introducing the notation $\text{Ent}_{(-i)}(X) = \E{(-i)}{\Phi(X)} - \Phi(\E{(-i)}{X})$, this can be re-written as
\begin{align}
\text{Ent}(X)  &\le \E{}{\sum_{i=1}^{n} \text{Ent}_{(-i)}(X)}. \label{ineqn: sub_additivity_phi_entropy_2}
\end{align}
\end{proposition}
\end{itemize}
\subsection{Herbst's Argument}
\begin{itemize}
\item \begin{remark} (\emph{\textbf{Entropy Functional for Moment Generating Function}})\\
Let $X = e^{\lambda Z}$ where $Z$ is a random variable. The entropy function of $X$ becomes
\begin{align*}
\text{Ent}(e^{\lambda Z}) &= \E{}{\lambda Z e^{\lambda Z}} - \E{}{e^{\lambda Z}}\log\paren{\E{}{e^{\lambda Z}}}
\end{align*} Denote $\psi_{Z - \E{}{Z}}(\lambda) := \log\E{}{e^{\lambda (Z - \E{}{Z})}}$. Then we have
\begin{align}
\frac{\text{Ent}(e^{\lambda Z})}{\E{}{e^{\lambda Z}}} &= \lambda\; \psi_{Z - \E{}{Z}}'(\lambda) - \psi_{Z - \E{}{Z}}(\lambda). \label{eqn: entropy_log_mgf_differential}
\end{align} 

Our strategy is based on using \eqref{eqn: entropy_log_mgf_differential} \emph{\textbf{the sub-additivity of entropy}} and then univariate calculus to derive \emph{\textbf{upper bounds} for the \textbf{derivative} of $\psi(\lambda)$}. By solving the obtained \emph{\textbf{differential inequality}}, we obtain tail bounds via \emph{Chernoff's bounding}.

For example, if
\begin{align*}
\frac{\text{Ent}(e^{\lambda Z})}{\E{}{e^{\lambda Z}}} &\le \frac{\nu \lambda^2}{2}  \\
\Leftrightarrow  \lambda\; \psi_{Z - \E{}{Z}}'(\lambda) - \psi_{Z - \E{}{Z}}(\lambda) &\le \frac{\nu \lambda^2}{2},\\
\Leftrightarrow \frac{1}{\lambda}\psi_{Z - \E{}{Z}}'(\lambda) - \frac{1}{\lambda^2}\psi_{Z - \E{}{Z}}(\lambda) &\le \frac{\nu}{2}.
\end{align*}  Setting $G(\lambda) = \lambda^{-1}\psi_{Z - \E{}{Z}}(\lambda)$, we see that the differential inequality becomes
\begin{align*}
G'(\lambda) &\le \frac{\nu}{2}.
\end{align*} Since $G(\lambda) \to 0$ as $\lambda \to 0$, which implies that
\begin{align*}
G(\lambda) &\le \frac{\nu \lambda}{2},
\end{align*} and the result follows. 
\end{remark}

\item \begin{proposition} (\textbf{Herbst's Argument}) \citep{boucheron2013concentration, wainwright2019high}\\
Let $Z$ be an integrable random variable such that for some $\nu > 0$, we have, for every $\lambda > 0$,
\begin{align}
\frac{\text{Ent}(e^{\lambda Z})}{\E{}{e^{\lambda Z}}} &\le \frac{\nu \lambda^2}{2} \label{ineqn: herbst_argument}
\end{align} Then, for every $\lambda >0$, the logarithmic moment generating function of centered random variable $(Z - \E{}{Z})$ satisfies
\begin{align*}
\psi_{Z - \E{}{Z}}(\lambda) := \log\E{}{e^{\lambda (Z - \E{}{Z})}} &\le \frac{\nu \lambda^2}{2} .
\end{align*}
\end{proposition}
\end{itemize}

\subsection{Association Inequalities}
\begin{itemize}
\item \begin{proposition} (\textbf{Chebyshev's Association Inequalities}) \citep{boucheron2013concentration}\\
Let $f$ and $g$ be \textbf{nondecreasing} real-valued functions defined on the real line. If $X$ is a real-valued random variable and $Y$ is a \textbf{nonnegative} random variable, then
\begin{align}
\E{}{Y}\E{}{Yf(X)g(X)} \ge \E{}{Yf(X)}\E{}{Yg(X)}\label{ineqn: cheby_association_inequality_nondecreasing}
\end{align} If $f$ is \textbf{nonincreasing} and $g$ is \textbf{nondecreasing} then
\begin{align}
\E{}{Y}\E{}{Yf(X)g(X)} \le \E{}{Yf(X)}\E{}{Yg(X)}\label{ineqn: cheby_association_inequality_nonincreasing_nondecreasing}
\end{align}
\end{proposition}

\item \begin{proposition} (\textbf{Harris's Inequalities}) \citep{boucheron2013concentration}\\
Let $f, g: \bR^n \to \bR$ be \textbf{nondecreasing} functions. Let $Z_1 \xdotx{,} Z_n$ be \textbf{independent} real-valued random variables and define the random vector $Z = (Z_1 \xdotx{,} Z_n)$ taking values in $\bR^n$. Then
\begin{align}
\E{}{f(X)g(X)} \ge \E{}{f(X)}\E{}{g(X)}\label{ineqn: harris_inequality_nondecreasing}
\end{align} If $f$ is \textbf{nonincreasing} and $g$ is \textbf{nondecreasing} then
\begin{align}
\E{}{f(X)g(X)} \le \E{}{f(X)}\E{}{g(X)}\label{ineqn: harris_inequality_nonincreasing_nondecreasing}
\end{align}
\end{proposition}
\end{itemize}

\subsection{Connection to Variance Bounds}
\begin{itemize}
\item \begin{proposition} (\textbf{A Modified Logarithmic Sobolev Inequalities for Moment Generating Function}) \citep{boucheron2013concentration}\\
Consider independent random variables $Z_1 \xdotx{,} Z_n$ taking values in $\cX$, a real-valued function $f: \cX^n \to \bR$ and the random variable $X = f(Z_1 \xdotx{,} Z_n)$. Also denote $Z_{(-i)}= (Z_1 \xdotx{,} Z_{i-1}, Z_{i+1} \xdotx{,} Z_n)$ and $X_{(-i)} = f_i(Z_{(-i)})$ where $f_i: \cX^{n-1} \to \bR$ is an arbitrary function. Let $\phi(x) = e^x -x -1$.
Then for all $\lambda \in \bR$,
\begin{align}
\text{Ent}(e^{\lambda X}) := \E{}{\lambda Xe^{\lambda X}} - \E{}{e^{\lambda X}}\log\E{}{e^{\lambda X}} &\le \sum_{i=1}^{n}\E{}{e^{\lambda X}\phi(-\lambda(X - X_{(-i)}))}\label{ineqn: log_sobolev_inequality_mgf}
\end{align}
\end{proposition}


\item \begin{proposition} (\textbf{Symmetrized Modified Logarithmic Sobolev Inequalities}) \citep{boucheron2013concentration}\\
Consider independent random variables $Z_1 \xdotx{,} Z_n$ taking values in $\cX$, a real-valued function $f: \cX^n \to \bR$ and the random variable $X = f(Z_1 \xdotx{,} Z_n)$. Also denote $\widetilde{X}^{(i)} = f(Z_1 \xdotx{,} Z_{i-1}, Z_i', Z_{i+1} \xdotx{,} Z_n)$. Let $\phi(x) = e^x -x -1$.
Then for all $\lambda \in \bR$,
\begin{align}
\lambda \E{}{Xe^{\lambda X}} - \E{}{e^{\lambda X}}\log\E{}{e^{\lambda X}} &\le \sum_{i=1}^{n}\E{}{e^{\lambda X}\phi(-\lambda(X - \widetilde{X}^{(i)}))}\label{ineqn: log_sobolev_inequality_sym_mgf}
\end{align} Moreover, denoting $\tau(x) = x(e^x - 1)$, for all $\lambda \in \bR$,
\begin{align*}
\lambda \E{}{Xe^{\lambda X}} - \E{}{e^{\lambda X}}\log\E{}{e^{\lambda X}} &\le \sum_{i=1}^{n}\E{}{e^{\lambda X}\tau(-\lambda(X - \widetilde{X}^{(i)})_{+})},\\
\lambda \E{}{Xe^{\lambda X}} - \E{}{e^{\lambda X}}\log\E{}{e^{\lambda X}} &\le \sum_{i=1}^{n}\E{}{e^{\lambda X}\tau(\lambda(\widetilde{X}^{(i)} - X)_{-})}.
\end{align*}
\end{proposition}

\item \begin{remark}
In the proof, we have 
\begin{align*}
\text{Ent}_{\mu_i}(e^{\lambda X}) &\le \E{\mu_i}{e^{\lambda X}(\log e^{\lambda X} - \log e^{\lambda X_i'}) - (e^{\lambda X} - e^{\lambda X_i'})}\\
&= \E{\mu_i}{e^{\lambda X}(\lambda (X - X_i') - (e^{\lambda X} - e^{\lambda X_i'})}\\
&\le \E{\mu_i}{(e^{\lambda X} - e^{\lambda X_i'})(\lambda (X - X_i')_{+}) } \\
&\le \lambda^2 \E{\mu_i}{(X - X_i')_{+}^2 } 
\end{align*}
Using the convexity of $e^x$, we have $e^s - e^t \le e^t(s- t)$ if $s > t$. Thus
\begin{align*}
\text{Ent}_{}(e^{\lambda X}) &\le \lambda^2 \sum_{i=1}^{n} \E{}{ \paren{X - X_i'}_{+}^2 }.
\end{align*}  From Efron-Stein inequality, if we can bound
\begin{align*}
  \sum_{i=1}^{n}\E{}{\paren{X -  X_i'}_{+}^2} \le \nu,
\end{align*} then we can bound both the variance $\text{Var}(X)$ and entropy $\text{Ent}_{}(e^{\lambda X})$.
\end{remark}
\end{itemize}


\section{Transportation Method}
\subsection{Optimal Transport, Wasserstein Distance and its Dual}
\begin{itemize}
\item \begin{definition} (\textbf{\emph{Pushforward Measure}})  \citep{gabriel2019computational} \\
Let $(\cX, \srB_X)$ and $(\cY, \srB_Y)$ be two topological measurable spaces.  Denote the spaces of  \emph{general (Radon) measures} on $\cX, \cY$  as $\cM(\cX)$ and $\cM(\cY)$. Also let  $\cC(\cX)$ be space of continuous functions on $\cX$. For a \emph{continous} map $T : \cX \rightarrow \cY$,  the \underline{\textbf{\emph{push-forward operator}}} is defined as $T_{\#}: \cM(\cX) \rightarrow \cM(\cY)$ that  satisfies 
\begin{align}
\forall\, h\in \cC(\cX), \quad \int_{\cY}h(y)\; d \paren{T_{\#}\alpha}(y) &= \int_{\cX}h(T(x))\; d\alpha(x). \label{eqn: def_push_forward_opt}\\
\text{or equivalently, } \quad \paren{T_{\#}\alpha}(B)&:= \alpha\paren{\set{x: T(x) \in B \subset \cY }} = \alpha(T^{-1}(B))  \label{eqn: def_push_forward_opt2}
\end{align} where the \textbf{\emph{push-forward measure}} $\beta := T_{\#}\alpha \in \cM(\cY)$ of some $\alpha \in  \cM(\cX)$, $T^{-1}(\cdot)$ is the pre-image of $T$.
\end{definition}

\item \begin{remark} (\textbf{\emph{Density Function of Pushforward Measure}})\\
Assume that $(\alpha, \beta)$ have densities $(\rho_{\alpha}, \rho_{\beta})$ with respect to a fixed measure, and $\beta = T_{\#}\alpha$. We see that $T_{\#}$ acts on a density $\rho_{\alpha}$ linearly to a density $\rho_{\beta}$ as a change of variable, i.e. 
\begin{align}
\rho_{\alpha}(\mb{x}) &= \abs{\det(T'(\mb{x}))}\rho_{\beta}(T(\mb{x}))  \label{eqn: density of push-forward distribution}\\
\abs{\det(T'(\mb{x}))} &= \frac{\rho_{\alpha}(\mb{x}) }{\rho_{\beta}(T(\mb{x})) } \nonumber
\end{align}
\end{remark}


\item \begin{definition} (\textbf{\emph{Optimal Transport Problem, Monge Problem}}) \citep{villani2009optimal, santambrogio2015optimal, gabriel2019computational} \\
Let $(\cX, \srB_X)$ and $(\cY, \srB_Y)$ be two measurable spaces, where $\cX$ and $\cY$ are \emph{complete separable metric spaces}. Denote $\cP(\cX)$ and $\cP(\cY)$ as the space of probabiilty measures on $\cX$ and $\cY$. Define a \emph{\textbf{cost function}} $c: \cX \times \cY \to \bR_{+}$ as non-negative real-valued measurable functions on $\cX \times \cY$. \underline{\emph{\textbf{The optimal transport problem}}} by \emph{Monge} (i.e. \underline{\emph{\textbf{Monge Problem}}}) is defined as follows: given two probability measures $\bP \in \cP(\cX)$ and $\bQ \in \cP(\cY)$, find a \emph{continuous measurable map} $T: \cX \to \cY$ so that 
\begin{align*}
\inf_{T} & \int_{\cX} c(x, T(x)) d\bP(x) \\
\text{s.t. }& \bQ = T_{\#}\bP
\end{align*} The optimal solution $T$ is also called an \emph{\textbf{optimal transportation plan}}.
\end{definition}

\item \begin{definition} (\textbf{\emph{Optimal Transport Problem, Kantorovich Relaxation}}) \citep{villani2009optimal, santambrogio2015optimal, gabriel2019computational} \\
 \underline{\emph{\textbf{The optimal transport problem}}} by \emph{Kantorovich} (i.e. \underline{\emph{\textbf{Kantorovich Relxation}}}) is defined as follows: given two probability measures $\bP \in \cP(\cX)$ and $\bQ \in \cP(\cY)$, find a \emph{joint probability measure} $\gamma \in \Pi(\bP, \bQ)$ so that 
\begin{align*}
\inf_{\gamma}  & \int_{\cX \times \cY} c(x, y) d\gamma(x, y) \\
\text{s.t. }&\gamma \in \Pi(\bP, \bQ) := \set{\gamma \in \cP(\cX \times \cY): \pi_{\cX, \#}\gamma = \bP, \; \pi_{\cY, \#}\gamma = \bQ }
\end{align*} where $\cP(\cX \times \cY)$ is the space of joint probability measure on $\cX \times \cY$, $\pi_{\cX}$ and $\pi_{\cY}$ are the coordinate projection onto $\cX$ and $\cY$. $\pi_{\cX, \#}\gamma = \bP$ means that $\bP$ is the marginal distribution of $\gamma$ on $\cX$. Similarly $\bQ$ is the marginal distribution of $\gamma$ on $\cY$.

Equivalently, let $X$ and $Y$ are \emph{random variables} taking values in $\cX$ and $\cY$. The \emph{joint distribution} of $(X, Y)$ is $\gamma$ with marginal distribution of $X$ and $Y$ being $\bP$ and $\bQ$. Then the problem is
\begin{align*}
\min_{\gamma \in \Pi(\bP, \bQ)} \E{\gamma}{c(X, Y)}
\end{align*} The joint distribution $\gamma \in \Pi(\bP, \bQ)$ such that $X_{\#}\gamma = \bP$ and $Y_{\#}\gamma = \bQ$ is called \emph{\textbf{a coupling}}.
\end{definition}

\item \begin{definition} (\textbf{\emph{Dual Problem of Kantorovich Problem}}) \citep{villani2009optimal, santambrogio2015optimal, gabriel2019computational} \\
The \underline{\textbf{\emph{dual problem}}} of \emph{Kantorovich problem} is described as below:
\begin{align*}
\cL_{c}(\bP, \bQ) = \max_{(\varphi,  \psi) \in \cC(\cX)\times \cC(\cY)} & \int_{\cX}\varphi(x) d\bP(x) + \int_{\cY}\psi(y) d\bQ(y) \\
\text{s.t. }&  \varphi(x) + \psi(y) \le c(x, y),\quad \forall x\in \cX, y \in \cY, 
\end{align*} Here, $(\varphi, \psi)$ is a pair of \emph{continuous functions} on $\cX$ and $\cY$ respectively and they are also the \textbf{\emph{Kantorovich potentials}}. The feasible region is 
 \begin{align*}
\cR(c) := \set{(\varphi,  \psi) \in  \cC(\cX)\times \cC(\cY): \varphi \oplus \psi \le c} 
\end{align*} where $( \varphi \oplus \psi)(x, y)=  \varphi(x) + \psi(y)$. 

In other words, the dual optimization problem is
\begin{align*}
\max_{(\varphi,  \psi) \in \cR(c)} \E{\bP}{\varphi(X)} + \E{\bQ}{\psi(Y)}
\end{align*}
\end{definition}

\item \begin{proposition} (\textbf{Strong Duality})  \citep{santambrogio2015optimal}\\
Let $\cX, \cY$ be \textbf{complete separable spaces}, and $c: \cX \times \cY \to \bR_{+}$ be \textbf{lower semi-continuous and bounded from below}. Then the optimal value of \emph{primal} and \emph{dual problems} are the same
\begin{align*}
\min_{X \sim \bP, Y \sim \bQ} \E{}{c(X, Y)} = \cL_{c}(\bP, \bQ) = \max_{(\varphi,  \psi) \in \cR(c)} \E{\bP}{\varphi(X)} + \E{\bQ}{\psi(Y)}.
\end{align*}
\end{proposition}

\item \begin{definition} (\textbf{\emph{Wasserstein Distance}}) \\
Let $((\cX, d), \srB)$ be \emph{a metric measurable space} with \emph{Borel $\sigma$-algebra} induced by metric $d$. Let $X, Y$ be two random variables taking values in $\cX$ with distribution $\bP$ and $\bQ$. \emph{\textbf{The Wasserstein distance}} between \emph{probability distributions} $\bP$ and $\bQ$ induced by $d$ is defined as
\begin{align}
\cW_1(\bP, \bQ)  \equiv  \cW_d(\bP, \bQ) := \min_{X \sim \bP, Y \sim \bQ} \E{}{d(X, Y)} \label{def: wasserstein_dist}
\end{align} In general, for $p \in [1, \infty)$, we can define \emph{\textbf{Wasserstein $p$-distance}} as
\begin{align}
\cW_p(\bP, \bQ)  \equiv \cW_{p,d}(\bP, \bQ)  := \paren{\min_{X \sim \bP, Y \sim \bQ} \E{}{(d(X, Y))^p}}^{1/p}. \label{def: wasserstein_dist_p_norm}
\end{align}
\end{definition}

\item \begin{remark}
Not to confuse the \emph{\textbf{$2$-Wasserstein distance}} with \emph{\textbf{the Wasserstein distance induced by $L_2$ norm}}:
\begin{align*}
\cW_{\norm{\cdot}{2}}(\bP, \bQ) \equiv  \cW_{1, \norm{\cdot}{2}}(\bP, \bQ)  &:= \min_{X \sim \bP, Y \sim \bQ} \E{}{\norm{X - Y}{2}} \\
\cW_{2}(\bP, \bQ)   \equiv  \cW_{2, d}(\bP, \bQ)  &:= \sqrt{\min_{X \sim \bP, Y \sim \bQ} \E{}{d(X, Y)^2}} 
\end{align*} 
\end{remark}

\item \begin{remark} (\emph{\textbf{Wasserstein $p$-Distance is a Metric in $\cP(\cX)$}}) \\
The \underline{\textbf{\emph{Wasserstein $p$-distance}}} $\cW_{p,d}(\bP, \bQ)  := \paren{\min_{X \sim \bP, Y \sim \bQ} \E{}{(d(X, Y))^p}}^{1/p}$ is a well-defined \emph{metric} in $\cP(\cX)$: for all $\bP, \bQ, \bM \in \cP(\cX)$, 
\begin{enumerate}
\item (\emph{Non-Negativity}):\; $\cW_{p,d}(\bP, \bQ) \ge 0$.
\item (\emph{Definiteness}):\; $\cW_{p,d}(\bP, \bQ) = 0 $ iff $\bP = \bQ$
\item (\emph{Symmetric}):\; $\cW_{p,d}(\bP, \bQ) = \cW_{p,d}(\bQ, \bP)$
\item (\emph{Triangular inequality}): \; $\cW_{p,d}(\bP, \bQ)  \le \cW_{p,d}(\bP, \bM )  + \cW_{p,d}(\bM , \bQ) $
\end{enumerate}
\end{remark}

\item \begin{definition} (\emph{\textbf{Total Variation / Variational Distance}})\\
Let $P, Q$ be two probability measures on measurable space $(\Omega, \srF)$. The \underline{\emph{\textbf{total variation}}} or \underline{\emph{\textbf{variational distance}}} between $P$ and $Q$ is defined by
\begin{align}
V(P,Q) &:= \sup_{A \in \srF}\abs{P(A) - Q(A)} \label{def: total_variation_prob}
\end{align}
\end{definition}

\item \begin{remark} (\emph{\textbf{Equivalent Formulation of Total Variation}})\\
It is a well-known and simple fact that the total variation is half the $L_1$-distance, that is, if $\mu$ is a \emph{common dominating measure} of $P$ and $Q$ and $p(x) = dP/d\mu$ and $q(x) = dQ /d\mu$ denote their respective densities, then
\begin{align}
V(P,Q) &:= P(A^{*}) - Q(A^{*}) = \frac{1}{2}\int_{\Omega}\abs{p(x) - q(x)} d\mu(x), \label{def: total_variation_l1}
\end{align} where $A^{*} = \set{x: p(x) \ge q(x)}$.
\end{remark}

\item \begin{remark} (\emph{\textbf{Total Variation via Optimal Coupling of Two Measures}})\\
We note that another important interpretation of \emph{the variational distance} is related to \emph{the best coupling of the two measures}
\begin{align}
V(P,Q) &= \min P\set{X \neq Y}
\end{align} where the minimum is taken over \emph{all pairs of joint distributions} for the random variables $(X, Y)$ whose marginal distributions are $X \sim P$ and $Y \sim Q$. 
\end{remark}

\item \begin{proposition} (\textbf{Pinsker's Inequality}) \citep{thomas2006elements, boucheron2013concentration} \\
Let $P, Q$ be two probability distributions on measurable space $(\Omega, \srF)$ such that $P \ll Q$. Then
\begin{align}
V(P, Q)^2 &\le \frac{1}{2}\kl{P}{Q}. \label{ineqn: pinsker_inequality}
\end{align}
\end{proposition}

\item \begin{remark} (\textbf{\emph{Total Variation as $1$-Wasserstein Distance}})\\
\emph{The total variation} between $P$ and $Q$ is \emph{\textbf{the Wasserstein distance}} induced by \emph{\textbf{the Hamming distance}} $d(x, y) = \#\set{i: x_i \neq y_i}$.
\begin{align*}
V(P, Q) &= \cW_1(P, Q).
\end{align*} Thus \emph{the Pinsker's inequality} \eqref{ineqn: pinsker_inequality} is the special case of \emph{transportation cost inequality} \eqref{ineqn: information_inequality_general}.
\end{remark}

\item \begin{theorem} (\textbf{Kantorovich-Rubenstein Duality}) \citep{villani2009optimal}\\
Let $\cX$ be a \textbf{Polish space}, i.e. $\cX$ a \textbf{complete separable} \textbf{metric} space equipped with a Borel $\sigma$-algebra induced by metric $d$, and $\bP$ and $\bQ$ be probability measures on $\cX$. For fixed $p \in [1, \infty)$, let $Lip_1$ be the space of all 
$1$-\textbf{Lipschitz} function with respect to metric $d$  such that
\begin{align*}
\norm{f}{L}  := \sup_{x, y \in \cX}\set{\frac{\abs{f(x) - f(y)}}{d(x, y)}} \le 1.
\end{align*}
Then 
\begin{align}
\cW_{d}(\bP, \bQ) \equiv \cW_{1,d}(\bP, \bQ) &= \sup_{f \in Lip_1}\set{\E{\bP}{f(X)} - \E{\bQ}{f(Y)}}. \label{eqn: wass_dist_dual}
\end{align} 
\end{theorem}
\end{itemize}
\subsection{Concentration via Transportation Cost}
\begin{itemize}
\item \begin{lemma} (\textbf{Transportation Lemma}) \citep{boucheron2013concentration}\\
Let $X$ be a real-valued integrable random variable. Let $\phi$ be a \textbf{convex} and \textbf{continuously differentiable} function on a (possibly unbounded) interval $[0, b)$ and assume that $\phi(0) = \phi'(0) = 0$. Define, for every $x \ge 0$, \textbf{the Legendre transform} $\phi^{*}(x) = \sup_{\lambda \in (0,b)}(\lambda x - \phi(\lambda))$, and let, for every $t \ge 0$, $\phi^{*-1}(t) = \inf\{x \ge 0: \phi^{*}(x) > t\}$, i.e. the \textbf{the generalized inverse} of $\phi^{*}$. Then the following two statements are equivalent:
\begin{enumerate}
\item for every $\lambda \in (0,b)$,
\begin{align*}
\psi_{X - \E{}{X}}(\lambda) &\le \phi(\lambda)
\end{align*} where $\psi_{X}(\lambda):= \log\E{Q}{e^{\lambda X}}$ is the logarithm of moment generating function;
\item for any probability measure $P$ absolutely continuous with respect to $Q$ such that $\kl{P}{Q} < \infty$,
\begin{align}
\E{P}{X} - \E{Q}{X} &\le \phi^{*-1}\paren{\kl{P}{Q}}. \label{ineqn: information_inequality_general}
\end{align} 
\end{enumerate}
In particular, given $\nu > 0$, $X$ follows a sub-Gaussian distribution, i.e.
\begin{align*}
\psi_{X - \E{}{X}}(\lambda) &\le \frac{\nu\lambda^2}{2}
\end{align*} for every $\lambda >0$ \textbf{if and only if} for any probability measure $P$ absolutely continuous with respect to $Q$ and such that $\kl{P}{Q} < \infty$, 
\begin{align}
\E{P}{X} - \E{Q}{X} &\le \sqrt{2\nu\kl{P}{Q}}. \label{ineqn: information_inequality_sub_gaussian}
\end{align}
\end{lemma}

\item \begin{remark} (\textbf{\emph{Transportation Method}})\\ 
Let $\bP = \otimes_{i=1}^{n}\bP_i$ be the product measure for $Z := (Z_1 \xdotx{,} Z_n)$ on $\cX^n$ and $f: \cX^n  \to \bR$ be \emph{$1$-Lipschitz function}.  Consider a probability measure $\bQ$ on $\cX^n$, absolutely continuous with respect to $\bP$ and let $Y$ be a random variable (defined on the same probability space as $\cX$) such that $Y$ has distribution $\bQ$.

The lemma above suggests that one may prove \emph{sub-Gaussian concentration inequalities} for $X = f(Z_1 \xdotx{,} Z_n)$ by proving a ``\emph{transportation}" \emph{inequality} as above. The key to achieving this relies on \emph{coupling}. In particular, \emph{the Kantorovich-Rubenstein duality} for $\cW_{1,d}$  suggests that 
\begin{align*}
\E{\bQ}{f(Y)} - \E{\bP}{f(Z)} \le \min_{\gamma \in \Pi (\bQ, \bP)} \E{\gamma}{d(Y, Z)} := \cW_{1,d}( \bQ, \bP)
\end{align*} Thus, it suffices to \emph{upper bound} the \emph{$1$-Wasserstein distance} between $\bQ$ and $\bP$.
\end{remark}

\item \begin{definition} (\emph{\textbf{$d$-Transportation Cost Inequality}}) \citep{wainwright2019high}\\
Let $(\cX, d)$ be a \emph{metric space} with metric $d$,  and $(\cX, \srB)$ be a \emph{measurable space}, where $\srB$ is \emph{the Borel $\sigma$-algebra} induced by metric $d$, \emph{\textbf{the probability measure}} $\bP$ is said to satisfy a \underline{\emph{\textbf{$d$-transportation cost inequality}}} with parameter $\nu > 0$ if
\begin{align}
\cW_{1,d}( \bQ, \bP) &\le \sqrt{2\nu\kl{\bQ}{\bP}}  \label{def: transportation_cost_inequality}
\end{align} for all probability measure $\bQ \ll \bP$ on $\srB$.
\end{definition}

\item \begin{theorem} (\textbf{Isoperimetric Inequality via Transportation Cost})\citep{wainwright2019high}\\
Consider a metric measure space $(\cX, \srB, \bP)$ with metric $d$, and suppose that $\bP$ satisfies the \textbf{$d$-transportation cost inequality} in \eqref{def: transportation_cost_inequality} Then its \textbf{concentration function} satisfies the bound
\begin{align}
\alpha_{\bP, (\cX, d)}(t) &\le  \exp\paren{- \frac{(t -t_0)_{+}^2}{2 \nu}}, \text{ for }t  \ge t_0 \label{ineqn: concentration_function_transport_cost}
\end{align} where $t_0 := \sqrt{2\nu \log 2}$.  Moreover, for any $Z \sim \bP$ and any $L$-Lipschitz function $f : \cX \to \bR$, we have the \textbf{concentration inequality}
\begin{align}
\bP\set{\abs{f(Z) - \E{}{f(Z)}} \ge t} &\le 2 \exp\paren{- \frac{t^2}{2 \nu L^2}}.  \label{ineqn: lipschitz_concentration_transport_cost}
\end{align}
\end{theorem}
\end{itemize}
\subsection{Tensorization for Transportation Cost}
\begin{itemize}
\item \begin{proposition} (\textbf{Tensorization for Transportation Cost}) \citep{boucheron2013concentration}\\
Suppose that, for each $k = 1, 2 \xdotx{,} n$, the univariate distribution $\bP_k$ satisfies a \textbf{$d_k$-transportation cost inequality} with parameter $\nu_k$. Then \textbf{the product distribution} $\bP = \otimes_{k=1}^n \bP_k$ satisfies the transportation cost inequality
\begin{align}
\cW_{1, d}(\bQ, \bP) &= \sqrt{2\paren{\sum_{k=1}^n \nu_k} \kl{\bQ}{\bP}}, \quad \text{ for all distributions $\bQ \ll \bP$ } \label{ineqn: tensorization_transport_cost}
\end{align}  where the Wasserstein metric is defined using the distance $d(x, y) :=  \sum_{k=1}^{n}d_k(x_k, y_k)$.
\end{proposition}
\end{itemize}

\subsection{Induction Lemma}
\subsection{Marton's Transportation Inequality}
\begin{itemize}
\item \begin{theorem} (\textbf{Marton's Transportation Inequality}) \citep{boucheron2013concentration}\\
Let $\bP = \otimes_{k=1}^n \bP_k$ be a product probability measure on $\cX^n$, and let $\bQ$ be a probability measure absolutely continuous with respect to $\bP$. Define two random vectors $X = (X_1 \xdotx{,} X_n), Y = (Y_1 \xdotx{,} Y_n)$ in $\cX^n$ with distribution $\bP$ and $\bQ$ respectively.  Then
\begin{align}
\cW_{2, d_H}(\bQ, \bP) := \sqrt{\min_{\gamma \in \Pi(\bQ, \bP)}\sum_{i=1}^n\gamma^2\set{X_i \neq Y_i}} &\le \sqrt{\frac{1}{2}\kl{\bQ}{\bP}} \label{ineqn: marton_transport_cost_inequality} \\
\Leftrightarrow  \min_{\gamma \in \Pi(\bQ, \bP)}\sum_{i=1}^n \gamma^2\set{X_i \neq Y_i} &\le \frac{1}{2}\kl{\bQ}{\bP} \nonumber
\end{align}
\end{theorem}

\item \begin{theorem} (\textbf{Marton's Conditional Transportation Inequality}) \citep{boucheron2013concentration}\\
Let $\bP = \otimes_{k=1}^n \bP_k$ be a product probability measure on $\cX^n$, and let $\bQ$ be a probability measure absolutely continuous with respect to $\bP$. Define two random vectors $X = (X_1 \xdotx{,} X_n), Y = (Y_1 \xdotx{,} Y_n)$ in $\cX^n$ with distribution $\bP$ and $\bQ$ respectively.  Then
\begin{align}
 \min_{\gamma \in \Pi(\bQ, \bP)}\E{\gamma}{\sum_{i=1}^n(\gamma^2\set{X_i \neq Y_i | X_i} + \gamma^2\set{X_i \neq Y_i | Y_i})} &\le 2\kl{\bQ}{\bP} \label{ineqn: marton_conditional_transport_cost_inequality}
\end{align}
\end{theorem}

\item \begin{proposition} (\textbf{Concentration of Lipschitz Function with Function Weighted Hamming Distance}) \citep{boucheron2013concentration}\\
Let $f: \cX^n \to \bR$ be a measurable function and let $Z_1 \xdotx{,} Z_n$ be independent random variables taking their values in $\cX$. Define $X = f(Z_1 \xdotx{,} Z_n)$. Assume that there exist \textbf{measurable functions} $c_i: \cX_n \to [0, \infty)$ such that for all $x, y \in \cX^n$, 
\begin{align*}
f(y) - f(z) \le \sum_{i=1}^{n}c_i(z) \ind{y_i \neq z_i}.
\end{align*} Setting
\begin{align*}
\nu = \E{}{\sum_{i=1}^{n}c_i^2(Z)} \qquad \text{ and }\qquad \nu_{\infty} = \sup_{z \in \cX^n}\sum_{i=1}^{n}c_i^2(z)
\end{align*} for all $\lambda  > 0$, we have 
\begin{align*}
\psi_{X - \E{}{X}}(\lambda) \le \frac{\nu \lambda^2}{2} \qquad \text{ and }\qquad \psi_{- X + \E{}{X}}(\lambda) \le \frac{\nu_{\infty} \lambda^2}{2}
\end{align*} In particular, for all $t > 0$,
\begin{align}
\bP\set{X \ge \E{}{X}  + t} &\le \exp\paren{-\frac{t^2}{2\nu}} \nonumber\\
\bP\set{X \le \E{}{X}  - t}   &\le \exp\paren{-\frac{t^2}{2\nu_{\infty}}}. \label{ineqn: weakly_self_bounding_sub_gaussian}
\end{align}
\end{proposition}

\item \begin{remark}
The condition in above proposition covers 
\begin{enumerate}
\item \emph{Lipschitz functions} such as \emph{functions with bounded difference},  
\item \emph{\textbf{self-bounding functions}} including \emph{\textbf{configuration functions}}:
Let $f$ be such a configuration function. For any $z \in \cX^n$, fix a \emph{maximal sub-sequence} $(z_{i,1} \xdotx{,} z_{i,m})$ satisfying property $\Pi$ (so that $f(z) = m$). Let $c_i(z)$ denote \emph{the indicator that $z_i$ belongs to the sub-sequence} $(z_{i,1} \xdotx{,} z_{i,m})$. Thus,
\begin{align*}
\sum_{i=1}^n c_i^2(z) = \sum_{i=1}^n c_i(z) = f(z).
\end{align*} It follows from the definition of a configuration function that for all $z, y \in \cX^n$,
\begin{align*}
f(y) \ge f(z) -  \sum_{i=1}^n c_i(z)\ind{z_i \neq y_i}
\end{align*} So $g = -f$ satisfies the condition in above proposition. 
\item  \emph{\textbf{weakly self-bounding functions}}
\item \emph{\textbf{convex distance function}}
\begin{align*}
d_{T}(z, A) &:= \sup_{\alpha \in \bR_{+}^{n}: \norm{\alpha}{2} = 1}\inf_{y \in A}\sum_{i=1}^n\alpha_i \ind{z_i \neq y_i}
\end{align*} Denote by $c(z) = (c_1(z) \xdotx{,} c_n(z)) = \alpha^{*}$ the vector of nonnegative components \emph{in the unit ball} for which \emph{the supremum is achieved}.  Thus
\begin{align*}
d_{T}(z, A) - d_{T}(y, A) &\le  \inf_{z' \in A}\sum_{i=1}^n c_i(z) \ind{z_i \neq z_i'} -  \inf_{y' \in A}\sum_{i=1}^n c_i(z) \ind{y_i \neq y_i'}\\
&\le \sum_{i=1}^n c_i(z) \ind{z_i \neq y_i}
\end{align*}
\end{enumerate}
\end{remark}
\end{itemize}
\subsection{Talagrand's Gaussian Transportation Inequality}
\begin{itemize}
\item \begin{theorem} (\textbf{Talagrand's Gaussian Transportation Inequality}) \citep{boucheron2013concentration}\\
Let $\bP$ be be the standard Gaussian probability measure on $\bR^n$, and let $\bQ$ be a probability measure absolutely continuous with respect to $\bP$. Define two random vectors $X = (X_1 \xdotx{,} X_n), Y = (Y_1 \xdotx{,} Y_n)$ in $\cX^n$ with distribution $\bP$ and $\bQ$ respectively.  Then
\begin{align}
\cW_{2, d}(\bQ, \bP) := \sqrt{\min_{\gamma \in \Pi(\bQ, \bP)}\sum_{i=1}^n\E{\gamma}{(X_i - Y_i)^2}   } &\le \sqrt{2\kl{\bQ}{\bP}} \label{ineqn: talagrand_gaussian_transport_cost_inequality} \\
\Leftrightarrow  \min_{\gamma \in \Pi(\bQ, \bP)}\sum_{i=1}^n\E{\gamma}{(X_i - Y_i)^2}   &\le 2\kl{\bQ}{\bP} \nonumber
\end{align}
\end{theorem}
\end{itemize}
\section{Proofs of Bounded Difference Inequality}
\begin{itemize}
\item \begin{theorem} (\textbf{McDiarmid's Inequality / Bounded Differences Inequality})\citep{boucheron2013concentration, wainwright2019high}\\
Suppose that $f$ satisfies \textbf{the bounded difference property} \eqref{eqn: bounded_difference_property} with parameters $(L_1 \xdotx{,} L_n)$ i.e. for each index $k = 1, 2 \xdotx{,} n$,
\begin{align*}
\abs{f(x_1 \xdotx{,} x_n) - f(x_1 \xdotx{,} x_{i-1},x'_{i}, x_{i+1} \xdotx{,} x_n)} \le L_k, \quad\text{ for all }x, x' \in \cX^n. 
\end{align*} Assume that the random vector $X = (X_1, X_2 \xdotx{,} X_n)$ has \textbf{independent} components. Then
\begin{align*}
\bP\set{\abs{f(X) - \E{}{f(X)}} \ge t } &\le  2 \exp\paren{- \frac{2 t^2}{ \sum_{k=1}^{n}L_k^2}}. \
\end{align*}
\end{theorem}
\end{itemize}
\subsection{Martingale Method}
\begin{itemize}
\item \begin{proof}
Consider the associated \emph{martingale difference sequence}
\begin{align*}
D_k := \E{}{f(X) | X_1 \xdotx{,} X_k} - \E{}{f(X) | X_1 \xdotx{,} X_{k-1}}.
\end{align*}
We claim that $D_k$ lies in \emph{an interval of length at most $L_k$ almost surely}. In order to prove this claim, define the random variables
\begin{align*}
A_k &:= \inf_x \set{\E{}{f(X) | X_1 \xdotx{,} X_{k-1}, x}} - \E{}{f(X) | X_1 \xdotx{,} X_{k-1}} \\
B_k &:= \sup_{x} \set{\E{}{f(X) | X_1 \xdotx{,} X_{k-1}, x}} - \E{}{f(X) | X_1 \xdotx{,} X_{k-1}}.
\end{align*}
On one hand, we have
\begin{align*}
D_k - A_k &=  \E{}{f(X) | X_1 \xdotx{,} X_k} - \inf_x \set{\E{}{f(X) | X_1 \xdotx{,} X_{k-1}, x}},
\end{align*}
so that $D_k \ge  A_k$ \emph{almost surely}. A similar argument shows that $D_k \le B_k$ \emph{almost surely}.
We now need to show that $B_k - A_k \le L_k$ \emph{almost surely}. Observe that by the independence of $\set{X_k}^n_{k=1}$, we have
\begin{align*}
\E{}{f(X) \,|\, x_1 \xdotx{,} x_k } &= \E{(k+1)}{f(x_1 \xdotx{,} x_k, X_{k+1} \xdotx{,} X_n)},\text{ for any }(x_1 \xdotx{,} x_k),
\end{align*}  where $\E{(k+1)}{\cdot}$ denote the expectation over $(X_{k+1} \xdotx{,} X_n)$.
Consequently, we have
\begin{align*}
B_k - A_k &=  \sup_{x} \E{(k+1)}{f(X_1 \xdotx{,} X_{k-1}, x, X_{k+1} \xdotx{,} X_n)}  \\
&\quad - \inf_x  \E{(k+1)}{f(X_1 \xdotx{,} X_{k-1}, x, X_{k+1} \xdotx{,} X_n)} \\
&\le  \sup_{x,y}\set{\E{(k+1)}{f(X_{1:k-1}, x, X_{k+1:n})} - \E{(k+1)}{f(X_{1:k-1}, y, X_{k+1:n})}} \\
&\le L_k,
\end{align*}
using \emph{the bounded differences assumption}. Thus, the variable $D_k$ lies within an interval of length $L_k$ at most surely, so that the claim follows as a corollary of \emph{the Azuma-Hoeffding inequality}. \qed
\end{proof}
\end{itemize}
\subsection{Entropy Method}
\begin{itemize}
\item \begin{proof}
Recall that for a random variable $Y$ taking its values in $[a, b]$, then we know from \emph{Hoeffding's Lemma} that the logarithmic moment generating functions $\psi(\lambda)$ satisfies
\begin{align*}
\psi(\lambda)'' = \text{Var}(Y) \le \frac{(b-a)^2}{4}
\end{align*} for every $\lambda \in \bR$. Hence, Hoeffding's inequality is obtained since
\begin{align*}
\frac{\text{Ent}(e^{\lambda Y})}{\E{}{e^{\lambda Y}}} = \lambda \psi'(\lambda) - \psi(\lambda) = \int_{0}^{\lambda} s \psi''(s) ds \le \frac{(b-a)^2}{4}  \int_{0}^{\lambda} s ds =  \frac{(b-a)^2 \lambda^2}{8},
\end{align*} Note that by the bounded differences assumption, given $X_{(-i)}$, $f(X)$ is a random variable whose range is in an interval of length at most $L_i$, so 
\begin{align*}
\frac{\text{Ent}_{(-i)}(e^{\lambda f(X)})}{\E{(-i)}{e^{\lambda f(X)}}} &\le  \frac{L_i^2 \lambda^2}{8} 
\end{align*} 
From the  tensorization property of entropy, we can bound the entropy of total function
\begin{align*}
\text{Ent}(e^{\lambda f(X)}) \le \E{}{\sum_{i=1}^{n}\text{Ent}_{(-i)}(e^{\lambda f(X)})} &\le \sum_{i=1}^{n} \frac{L_i^2 \lambda^2}{8} \E{}{\E{(-i)}{e^{\lambda f(X)}}}\\
\frac{\text{Ent}(e^{\lambda f(X)}) }{ \E{}{e^{\lambda f(X)}} } &\le \frac{\sum_{i=1}^{n}L_i^2 \lambda^2}{8} \equiv \frac{\nu \lambda^2}{2},
\end{align*} where  $\nu := \frac{1}{4}\sum_{i=1}^{n}L_i^2$.  Using \emph{Herbst's argument}, it leads to the bound of logarithmic moment generating function:
\begin{align*}
\psi_{f(X)}(\lambda) &\le \frac{\nu \lambda^2}{2}.
\end{align*} Finally, we apply \emph{the Chernoff's inequality}
\begin{align*}
\bP\set{f(X) - \E{}{f(X)} \ge t } &\le  \inf_{\lambda >0 }\exp\paren{\psi_{f(X)}(\lambda) - \lambda t} \le \exp\paren{- \frac{t^2}{2 \nu}}.  \qed
\end{align*}
\end{proof}
\end{itemize}
\subsection{Isoperimetric Inequality on Binary Hypercube}
\begin{itemize}
\item \begin{definition} (\emph{\textbf{Vertex Boundary of Graph}}) \citep{boucheron2013concentration}\\
Consider a graph $\cG = (\cV, \cE)$ and let $\cA \subset \cV$ be a set of its vertices. \underline{\emph{\textbf{The vertex boundary}}} of $\cA$ is defined as \emph{the set of those vertices}, \emph{\textbf{not in $\cA$}}, which are \emph{\textbf{connected}} to \emph{some vertex in $\cV$} by \emph{an edge}. We denote \emph{the vertex boundary of $\cA$} by $\partial V(\cA)$.
\end{definition}

\item \begin{remark} (\textbf{\emph{Binary Hypercube as Nearest Neighbor Graph with Hamming Distance}}) \\
Consider the graph as binary hypercube $\set{-1, +1}^n$ in which two vertices are connected by an edge if and only if their \emph{\textbf{Hamming distance} equals $1$}.  Define the \emph{norm} as the \emph{Hamming distance} to $-1^n= (-1 \xdotx{,} -1)$
\begin{align*}
\norm{x}{H} := \sum_{i=1}^{n}\ind{x_i = 1} = d_{H}(x, -1^n)
\end{align*}
\end{remark}

\item \begin{definition} (\textbf{\emph{Simplicial Order}})\\
We define the so-called \emph{\underline{\textbf{simplicial order}} of the elements} of the binary hypercube. We say that $x = (x_1 \xdotx{,} x_n) \in \set{-1, 1}^n$ \emph{\textbf{precedes}} $y = (y_1 \xdotx{,} y_n) \in \set{-1, 1}^n$  \emph{in the simplicial order} if either $\norm{x}{H} < \norm{y}{H}$ (where $\norm{x}{H} := \sum_{i=1}^{n}\ind{x_i = 1} = d_{H}(x, -1^n)$) or $\norm{x}{H} = \norm{y}{H}$ and $x_i = 1$ and $y_i = -1$ for the smallest $i$ for which $x_i \neq y_i$.  That is
\begin{align*}
&x \prec y \\
\Leftrightarrow &\set{(x,y): \norm{x}{H} < \norm{y}{H} \; \lor (\norm{x}{H} = \norm{y}{H} \, \land (x_i =1 \land y_i = -1, \text{ where }i=\min\set{k: x_k \neq x_k} )) }
\end{align*} In other words, the vector with \emph{\textbf{less}} $1$'s \emph{\textbf{precedes}} the vector with more $1$'s. If the number of $1$'s are the same, then the first $1$'s on the leftmost position is preferred.
\end{definition} 

\begin{theorem} (\textbf{Harp's Vertex Isoperimetric Theorem}) \citep{boucheron2013concentration}\\
For $N = 1 \xdotx{,} 2^n$, let $S_N$ denote the set of \textbf{first $N$ elements} of $\set{-1, +1}^n$ in the \textbf{simplicial order}. For any subset $A \subset \set{-1, +1}^n$, where $\abs{A} = N$, 
\begin{align*}
\abs{\partial V(A)} \ge \abs{\partial V(S_N)} 
\end{align*}
\end{theorem}

\item \begin{remark}
Note that if $N = \sum_{i=0}^{k}{n \choose i}$, for $k=0, \xdotx{,} n$, then 
\begin{align*}
S_N = \set{x \in \set{-1, +1}^n:  d_{H}(x, -1^n) \le k} = B_{H}(-1^n, k)
\end{align*} In other words, $S_N$ is a \emph{\textbf{Hamming ball} centered at the vector $-1^n= (-1 \xdotx{,} -1)$}.
\end{remark}

\item \begin{definition} (\textbf{\emph{$t$-Blowup of Set $A$ in Binary Hypercube}}) \\
For any $A \subset \set{-1, +1}^n$ and $x \in \set{-1, +1}^n$, let $d_{H}(x, A) = \min_{y \in A} d_{H}(x, y)$ be the \emph{Hamming distance} of $x$ to the set $A$. Also, denote by
\begin{align*}
A_t := \set{x \in \set{-1, +1}^n: d_{H}(x, A) < t}
\end{align*} the \emph{$t$-blowup of the set $A$}, that is, the set of points whose Hamming distance from $A$ is at most $t$.
\end{definition}

\item \begin{corollary} (\textbf{Isoperimetric Inequality in Binary Hypercube}) \citep{boucheron2013concentration}\\ 
Let $A \subset \set{-1, +1}^n$ such that $\abs{A} \ge \sum_{i=0}^{k}{n \choose i}$. Then for any $t=1, 2 \xdotx{,} n-k+1$, 
\begin{align}
\abs{A_t} \ge  \sum_{i=0}^{k+1 - t}{n \choose i}. \label{ineqn: harp_vertex_isoperimetric_inequality}
\end{align} In particular, if $\abs{A}/2^n \ge 1/2$ then we may take $k = \floor{n/2}$ in the corollary above and
\begin{align}
\frac{\abs{A_t}}{2^n} \ge \bP\set{X  < \E{}{X} + t } \ge 1 - \exp\paren{- \frac{2t^2}{n}}
\end{align} where $X \sim \text{Ber}(1/2)$ is a symmetric Bernoulli random variable taking values in $\set{-1, +1}$ with $\bP\set{X= 1} = \bP\set{X = -1} = 1/2$.
\end{corollary}

\item \begin{proof} (\textbf{\emph{Proof of Bounded Difference Inequality on Binary Hypercube}}) \\
Note that any  function with \emph{\textbf{bounded difference property}} is \emph{\textbf{Lipschitz function}} with respect to \emph{\textbf{Hamming distance}}. 
\begin{align*}
&\sup_{x \in \cX^n, y_i \in \cX}\abs{f(x_1 \xdotx{,} x_n) - f(x_1 \xdotx{,} x_{i-1}, y_i, x_{i+1} \xdotx{,} x_{n})} &\\
&\le c_i = c_i\,d_{H}((x_1 \xdotx{,} x_n), (x_1 \xdotx{,} x_{i-1}, y_i, x_{i+1} \xdotx{,} x_{n})), \quad 1\le i \le n\\
\Rightarrow \abs{f(x) - f(y)} &=  \abs{\sum_{i=1}^{n}(f(x_{(i-1)}) - f(x_{(i)}) )}\\
&\quad \le  \sum_{i=1}^{n}\abs{f(x_{(i-1)}) - f(x_{(i)}) }\\
&\quad \le \sum_{i=1}^{n}L_i \ind{x_{(i-1)}[i] \neq x_{(i)}[i]} \\
&\quad \le \paren{\sum_i^{n}L_i^2}^{1/2} d_{H}(x, y)
\end{align*}   where $x_{(i)}$ is replicate of $x_{(i-1)}$ except for $i$-th component, which is replaced by $y_i$. Note that $x_{(0)} = x$ and $x_{(n)} = y$. 

The Harp's isoperimetric theorem suggests that the concentration function 
\begin{align*}
\alpha_{\bP, (\set{-1, +1}^n, d_{H, L})}(t) := \sup_{A: \bP\set{A} \ge 1/2}\bP\set{A_t} &\le \exp\paren{- \frac{2t^2}{\sum_i^{n}L_i^2}}
\end{align*} where $\bP$ is uniform distribution on $\set{-1, +1}^n$. Thus by \emph{Levy's inequality}, we prove that for $Z \in \set{-1, 1}^n$ and Lipschitz function $f: \set{-1, 1}^n \to \bR$ with respect to weighted Hamming distance $d_{H, L}$, 
\begin{align*}
\bP\set{\abs{f(Z) - \text{Med}(f(Z))} \ge t} &\le 2 \exp\paren{- \frac{2t^2}{\sum_i^{n}L_i^2}}. \qed
\end{align*} 
\end{proof}
\end{itemize}
\subsection{Transportation Method}
\begin{itemize}
\item \begin{proof}
Any  function with \emph{\textbf{bounded difference property}} is \emph{\textbf{Lipschitz function}} with respect to \emph{\textbf{Hamming distance}}. This implies that for all $x, y \in \cX^n$,
\begin{align*}
f(y) - f(x) \le \sum_{i=1}^{n}L_i \ind{x_i \neq y_i} \equiv d_{H,L}(x, y).
\end{align*} Note that for coupling $\gamma \in \Pi(\bQ, \bP)$ where $Y \sim \bQ$ and $X \sim \bP$, 
\begin{align*}
\E{\bQ}{f(Y)} - \E{\bP}{f(X)} &= \E{\gamma}{f(Y) - f(X)} \\
&\le \sum_{i=1}^n L_i \E{\gamma}{\ind{X_i \neq Y_i}} \\
&\le \paren{\sum_{i=1}^{n}L_i^2}^{1/2}\paren{\sum_{i=1}^{n}(\E{\gamma}{\ind{X_i \neq Y_i}})^2}^{1/2}
\end{align*}

We want to prove the concentration using transportation cost inequality. That is, to bound the term
\begin{align*}
\min_{\gamma \in \Pi(\bQ, \bP)}\sum_{i=1}^{n}(\E{\gamma}{\ind{X_i \neq Y_i}})^2 = \min_{\gamma \in \Pi(\bQ, \bP)}\sum_{i=1}^{n}\gamma^2\set{X_i \neq Y_i}.
\end{align*} We have shown that 
\begin{align*}
\min_{\gamma \in \Pi(\bQ, \bP)}\gamma\set{X \neq Y} = \cW_{1, d_H}(\bQ, \bP) = \sup_{A \in \cX}\abs{\bQ(A) - \bP(A)}  \equiv \norm{\bQ - \bP}{TV}.
\end{align*}  For each independent variable $X_i, Y_i$, and their marginal distribution $\bP_i, \bQ_i$ where $\bQ_i \ll \bP_i$, by Pinsker's inequality,
\begin{align*}
 \brac{\min_{\gamma \in \Pi(\bQ, \bP)}\gamma\set{X_i \neq Y_i}}^2 = \norm{\bQ - \bP}{TV}^2 &\le \frac{1}{2}\kl{\bQ_i}{\bP_i} \\
\min_{\gamma \in \Pi(\bQ, \bP)} \gamma^2\set{X_i \neq Y_i} \le \brac{\min_{\gamma \in \Pi(\bQ, \bP)}\gamma\set{X_i \neq Y_i}}^2 &\le \frac{1}{2}\kl{\bQ_i}{\bP_i}
\end{align*} Thus by induction lemma, 
\begin{align*}
\min_{\gamma \in \Pi(\bQ, \bP)}\sum_{i=1}^{n}\gamma^2\set{X_i \neq Y_i} &\le \frac{1}{2}\kl{\bQ}{\bP}
\end{align*} which is the \emph{Marton's transportation inequality}. Finally, we have
\begin{align*}
\E{\bQ}{f(Y)} - \E{\bP}{f(X)} &\le \paren{\sum_{i=1}^{n}L_i^2}^{1/2}\paren{\sum_{i=1}^{n}(\E{\gamma}{\ind{X_i \neq Y_i}})^2}^{1/2} \\
&\le \sqrt{\frac{\paren{\sum_{i=1}^{n}L_i^2}}{2}\kl{\bQ}{\bP}}.
\end{align*} Then we can apply the transportation lemma with $\nu := \frac{1}{4}\sum_{i=1}^{n}L_i^2$, which proves the bounded difference inequality. \qed 
\end{proof}
\end{itemize}





\newpage
\bibliographystyle{plainnat}
\bibliography{reference.bib}
\end{document}