\documentclass[11pt]{article}
\usepackage[scaled=0.92]{helvet}
\usepackage{geometry}
\geometry{letterpaper,tmargin=1in,bmargin=1in,lmargin=1in,rmargin=1in}
\usepackage[parfill]{parskip} % Activate to begin paragraphs with an empty line rather than an indent %\usepackage{graphicx}
\usepackage{amsmath,amssymb, mathrsfs,  mathtools, dsfont}
\usepackage{tabularx}
\usepackage{tikz-cd}
\usepackage[font=footnotesize,labelfont=bf]{caption}
\usepackage{graphicx}
\usepackage{xcolor}
%\usepackage[linkbordercolor ={1 1 1} ]{hyperref}
%\usepackage[sf]{titlesec}
\usepackage{natbib}
%\usepackage{tikz-cd}

\usepackage{../../Tianpei_Report}

%\usepackage{appendix}
%\usepackage{algorithm}
%\usepackage{algorithmic}

%\renewcommand{\algorithmicrequire}{\textbf{Input:}}
%\renewcommand{\algorithmicensure}{\textbf{Output:}}



\begin{document}
\title{Lecture 5: Concentration of Measure and Isoperimetry}
\author{ Tianpei Xie}
\date{Jan. 19th., 2023 }
\maketitle
\tableofcontents
\newpage
\section{The Classic Isoperimetry Inequalities}
\subsection{Brunn-Minkowski Inequality}
\begin{itemize}
\item \begin{theorem} (\textbf{Brunn-Minkowski Inequality}) \citep{boucheron2013concentration, vershynin2018high, wainwright2019high}\\
Let $A, B \subset \bR^n$ be \textbf{non-empty compact sets}. Then for all $\lambda \in [0, 1]$,
\begin{align}
\text{Vol}\paren{ \lambda A + (1- \lambda) B }^{\frac{1}{n}} &\ge \lambda\text{Vol}(A)^{\frac{1}{n}} + (1- \lambda)\text{Vol}(B)^{\frac{1}{n}}.   \label{ineqn: brunn_minkowski_inequality}
\end{align}
\end{theorem}
Note:  a convex body in $\bR^n$ is closed and compact set.

\item \begin{theorem} (\textbf{The Pr{\'e}kopa-Leindler Inequality}). \citep{boucheron2013concentration, wainwright2019high} \\
Let $\lambda \in (0, 1)$, and let $f, g, h : \bR^n \to [0, \infty)$ be \textbf{non-negative measurable functions} such that for all $x, y \in \bR^n$,
\begin{align*}
h\paren{\lambda x + (1- \lambda) y} &\ge f(x)^{\lambda}g(y)^{1-\lambda}.
\end{align*} Then
\begin{align}
\int_{\bR^n} h(x) dx &\ge \paren{\int_{\bR^n} f(x) dx }^{\lambda}\paren{\int_{\bR^n} g(x) dx}^{1-\lambda}.   \label{ineqn: prekopa_leindler_inequality}
\end{align}
\end{theorem}

\item \begin{corollary} (\textbf{Weaker Brunn-Minkowski Inequality}) \citep{boucheron2013concentration, wainwright2019high}\\
Let $A, B \subset \bR^n$ be \textbf{non-empty compact sets}. Then for all $\lambda \in [0, 1]$,
\begin{align}
\text{Vol}\paren{ \lambda A + (1- \lambda) B } &\ge \text{Vol}(A)^{\lambda}\text{Vol}(B)^{1- \lambda}.   \label{ineqn: brunn_minkowski_inequality_weaker}
\end{align}
\end{corollary}
\end{itemize}
\subsection{Blow-Up of Sets}
\subsection{The Classical Isoperimetry Theorem}

\section{Concentration via Isoperimetry}
\subsection{Levy's Inequalities and Concentration Function}
\subsection{Isoperimetric Inequalities on the Unit Sphere}
\begin{itemize}
\item \begin{remark} (\emph{\textbf{Volume Ratio of Unit Balls and its Interior}})  \citep{vershynin2018high}\\
Let $B(0, 1):= \set{x \in \bR^n: \norm{x}{} \le 1}$ be the unit ball in $\bR^n$. The volume ratio between $B(0,1)$ and its $\epsilon$-interior $B(0, 1-\epsilon)$ is
\begin{align*}
\frac{\text{Vol}(B(0, 1-\epsilon))}{\text{Vol}(B(0,1))} &= (1 - \epsilon)^n \le \exp\paren{-n \epsilon}
\end{align*} The inequality is due to $1 - x\le e^{-x}$.  

As $n \to \infty$, the above ratio goes to $0$. In other words, most of volume in $B(0,1)$ is \emph{\textbf{concentrated}} in the \emph{\textbf{boundary}} $\partial B = \bS^{n-1} := \set{x \in \bR^n: \norm{x}{} = 1}$. This phenomenon is called ``\emph{\textbf{the curse of dimensionality}}".
\end{remark}

\item \begin{definition}
\end{definition}
\end{itemize}
\subsection{Gaussian Isoperimetric Inequalities and  Concentration of Gaussian Measure}
\subsection{Edge Isoperimetric Inequality on the Binary Hypercube}
\subsection{Vertex Isoperimetric Inequality on the Binary Hypercube}
\subsection{Convex Distance Inequality}


%\section{The Classic Isoperimetry Inequalities}
%\subsection{Concentration of Lipschitz Function for the Sphere}






\newpage
\bibliographystyle{plainnat}
\bibliography{reference.bib}
\end{document}