\documentclass[11pt]{article}
\usepackage[scaled=0.92]{helvet}
\usepackage{geometry}
\geometry{letterpaper,tmargin=1in,bmargin=1in,lmargin=1in,rmargin=1in}
\usepackage[parfill]{parskip} % Activate to begin paragraphs with an empty line rather than an indent %\usepackage{graphicx}
\usepackage{amsmath,amssymb, mathrsfs, dsfont}
\usepackage{tabularx}
\usepackage[font=footnotesize,labelfont=bf]{caption}
\usepackage{graphicx}
\usepackage{xcolor}
%\usepackage[linkbordercolor ={1 1 1} ]{hyperref}
%\usepackage[sf]{titlesec}
\usepackage{natbib}
\usepackage{../../Tianpei_Report}

%\usepackage{appendix}
%\usepackage{algorithm}
%\usepackage{algorithmic}

%\renewcommand{\algorithmicrequire}{\textbf{Input:}}
%\renewcommand{\algorithmicensure}{\textbf{Output:}}



\begin{document}
\title{Lecture 7:  Calculus on Manifolds}
\author{ Tianpei Xie}
\date{ Sep. 21st., 2022 }
\maketitle
\tableofcontents
\newpage
\section{Rates of change under variations}

\section{The chain rule, velocities and tangent vectors}
\subsection{Coordinate independence}

\section{Differentials and $1$-forms}

\section{Vector fields}

\subsection{Coordinate calculations}

\section{Manifolds}
\subsection{Coordinates and geometry}
\newpage
\bibliographystyle{plainnat}
\bibliography{book_reference.bib}
\end{document}