\documentclass[11pt]{article}
\usepackage[scaled=0.92]{helvet}
\usepackage{geometry}
\geometry{letterpaper,tmargin=1in,bmargin=1in,lmargin=1in,rmargin=1in}
\usepackage[parfill]{parskip} % Activate to begin paragraphs with an empty line rather than an indent %\usepackage{graphicx}
\usepackage{amsmath,amssymb, mathrsfs,  mathtools, dsfont}
\usepackage{tabularx}
\usepackage{tikz-cd}
\usepackage[font=footnotesize,labelfont=bf]{caption}
\usepackage{graphicx}
\usepackage{xcolor}
%\usepackage[linkbordercolor ={1 1 1} ]{hyperref}
%\usepackage[sf]{titlesec}
\usepackage{natbib}
\usepackage{../../Tianpei_Report}

%\usepackage{appendix}
%\usepackage{algorithm}
%\usepackage{algorithmic}

%\renewcommand{\algorithmicrequire}{\textbf{Input:}}
%\renewcommand{\algorithmicensure}{\textbf{Output:}}



\begin{document}
\title{Lecture 7: Lie Groups}
\author{ Tianpei Xie}
\date{Oct. 16th., 2022}
\maketitle
\tableofcontents
\newpage
\section{Basic Definitions}

\section{Lie Group Homomorphisms}
\subsection{Definitions}
\subsection{The Universal Covering Group}

\section{Lie Subgroups}

\section{Group Actions and Equivariant Maps}
\subsection{Group Actions}

\subsection{Equivariant Maps}

\subsection{Semidirect Products}

\subsection{Representations}



\newpage
\bibliographystyle{plainnat}
\bibliography{book_reference.bib}
\end{document}