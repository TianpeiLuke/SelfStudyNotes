\documentclass[11pt]{article}
\usepackage[scaled=0.92]{helvet}
\usepackage{geometry}
\geometry{letterpaper,tmargin=1in,bmargin=1in,lmargin=1in,rmargin=1in}
\usepackage[parfill]{parskip} % Activate to begin paragraphs with an empty line rather than an indent %\usepackage{graphicx}
\usepackage{amsmath,amssymb, mathrsfs,  mathtools, dsfont}
\usepackage{tabularx}
\usepackage{tikz-cd}
\usepackage[font=footnotesize,labelfont=bf]{caption}
\usepackage{graphicx}
\usepackage{xcolor}
%\usepackage[linkbordercolor ={1 1 1} ]{hyperref}
%\usepackage[sf]{titlesec}
\usepackage{natbib}
\usepackage{../../Tianpei_Report}

%\usepackage{appendix}
%\usepackage{algorithm}
%\usepackage{algorithmic}

%\renewcommand{\algorithmicrequire}{\textbf{Input:}}
%\renewcommand{\algorithmicensure}{\textbf{Output:}}



\begin{document}
\title{Lecture 0: Summary (part 1)}
\author{ Tianpei Xie}
\date{Oct. 24th., 2022}
\maketitle
\tableofcontents
\newpage
\section{Topology}
\subsection{Topological Space}
\begin{itemize}
\item 
\begin{definition} 
Let $X$ be a set. \underline{\emph{A \textbf{topology}}} on $X$ is \emph{a collection} $\mathscr{T}$ of \emph{subsets} of X, called \emph{\textbf{open subsets}}, satisfying
\begin{enumerate}
\item $X$ and $\emptyset$ are \emph{open}.
\item The \emph{\textbf{union}} of \emph{\textbf{any family}} of open subsets is open.
\item The \emph{\textbf{intersection}} of \emph{any \textbf{finite} family} of open subsets is open.
\end{enumerate}
A pair $(X, \mathscr{T})$ consisting of a set $X$ together with a topology $\mathscr{T}$ on $X$ is called \emph{\textbf{a topological space}}.
\end{definition}

\item \begin{definition}
A map $F: X \rightarrow Y$ is said to be \underline{\emph{\textbf{continuous}}} if for every open subset $U \subseteq Y$, the \emph{\textbf{preimage}} $F^{-1}(U)$ is \emph{\textbf{open}} in $X$.
\end{definition}

\item \begin{definition}
A \emph{\textbf{continuous bijective}} map $F: X \rightarrow Y$ with \emph{\textbf{continuous inverse}} is called a \underline{\emph{\textbf{homeomorphism}}}. If there exists a \emph{homeomorphism} from $X$ to $Y$, we say that X and Y are \emph{\textbf{homeomorphic}}.
\end{definition}

\item \begin{definition}
A map $F: X \rightarrow Y$ (continuous or not) is said to be \emph{\textbf{an open map}} if for every \emph{open} subset $U \subseteq X$, the image set $F(U)$ is \emph{open} in $Y$, and  \emph{\textbf{a closed map}} if for every \emph{closed} subset $K \subseteq X$, the image $F(K)$ is \emph{closed} in Y . 
\end{definition}



\item \begin{definition}
A topological space $X$ is said to be a \underline{\emph{\textbf{Hausdorff space}}} if for every pair of \emph{\textbf{distinct}} points $p,q \in X$, there exist \emph{\textbf{disjoint open subsets}} $U,V \subseteq X$ such that $p \in U$ and $q \in V$.
\end{definition}

\item \begin{definition}
Suppose $X$ is a topological space. A collection $\mathscr{B}$ of open subsets of $X$ is said to be \emph{\textbf{a basis}} for \emph{the topology of $X$} (plural: \emph{\textbf{bases}}) if every open subset of $X$ is the \emph{union of some collection of elements} of $\mathscr{B}$.

More generally, suppose $X$ is merely a set, and $\mathscr{B}$ is a collection of \emph{subsets} of $X$ satisfying the following conditions:
\begin{enumerate}
\item $X = \bigcup_{B \in \mathscr{B}}B$.
\item If $B_1, B_2 \in \mathscr{B}$ and $x \in B_1 \cap B_2$, then there exists $B_3 \in \mathscr{B}$ such that $x \in B_3 \subseteq B_1 \cap B2$.
\end{enumerate}
Then \emph{the collection of \textbf{all unions} of elements of $\mathscr{B}$} is a \emph{topology} on X, called \emph{\textbf{the topology generated by $\mathscr{B}$}}, and $\mathscr{B}$ is a \underline{\emph{\textbf{basis}} for this \emph{topology}}.
\end{definition}

\item \begin{definition} See the following definitions
\begin{enumerate}
\item A set is said to be \emph{\textbf{countably infinite}} if it admits a \emph{bijection} with the set of \emph{positive integers}, and 
\item \emph{\textbf{countable}} if it is \emph{finite} or \emph{countably infinite}. 
\item A topological space $X$ is said to be \emph{\textbf{first-countable}} if there is a \emph{\textbf{countable neighborhood basis}} at each point, and 
\item \underline{\emph{\textbf{second-countable}}} if there is \emph{\textbf{a countable basis}} for its topology.
\end{enumerate}
\end{definition}

\end{itemize}

\subsection{Subspaces and Quotients}
\begin{itemize}
\item \begin{definition}
If $X$ is a topological space and $S \subseteq X$ is an arbitrary subset, we define \emph{\textbf{the subspace topology}} on $S$ (sometimes called \emph{the \textbf{relative topology}}) by declaring a subset $U \subseteq S$ to be \emph{open} in $S$ \emph{if and only} if there exists an open subset $V \subseteq X$ such that $U = V \cap S$. 

Any subset of $X$ endowed with the subspace topology is said to be \emph{\textbf{a subspace of $X$}}.
\end{definition}

\item \begin{definition}
If $X$ and $Y$ are topological spaces, a continuous injective map $F: X \rightarrow Y$ is called a \underline{\emph{\textbf{topological embedding}}} if it is a \emph{\textbf{homeomorphism}} onto its image $F(X) \subseteq Y$ in the subspace topology.
\end{definition}

\item \begin{definition}
If $X$ is a \emph{topological space}, $Y$ is a set, and $\pi: X \rightarrow Y$ is a \textbf{surjective} map, \emph{\textbf{the \underline{quotient topology}}} on $Y$ determined by $\pi$ is defined by declaring a subset $U \subseteq Y$ to be \emph{open} \emph{if and only} if $\pi^{-1}(U)$ is \emph{open} in $X$. 

If $X$ and $Y$ are topological spaces, a map $\pi: X \rightarrow Y$ is called \emph{\textbf{a quotient map}} if it is \emph{\textbf{surjective}} and \emph{\textbf{continuous}} and $Y$ has the quotient topology determined by $\pi$.
\end{definition}

\item \begin{definition}
The following construction is the most common way of producing quotient maps. \emph{A \textbf{relation}} on a set $X$ is called \emph{\textbf{an equivalence relation}} if it is 
\begin{enumerate}
\item \emph{\textbf{reflexive}}: $x \sim x$ for all $x \in X$,
\item \emph{\textbf{symmetric}}: $x \sim y$ implies $y \sim x$,
\item \emph{\textbf{transitive}}: $x \sim y$ and $y \sim z$ imply $x \sim z$.
\end{enumerate} 

If $R \subseteq X \times X$ is any \emph{\textbf{relation}} on $X$, then \emph{\textbf{the intersection of all equivalence
relations}} on $X$ \emph{\textbf{containing}} $R$ is \emph{an equivalence relation}, called \emph{\textbf{the equivalence relation generated by $R$}}. 
\end{definition}

\begin{remark}
If is an equivalence relation on $X$, then for each $x\in X$, \emph{\textbf{the equivalence class} of $x$}, denoted by $[x]$, is the \emph{set of all $y \in X$ such that $y\sim x$}. The set of \emph{all equivalence classes} is a \emph{\textbf{partition}} of $X$: a collection of disjoint nonempty subsets whose union is $X$.
\end{remark}

\item \begin{definition}
Suppose $X$ is a topological space and $\sim$ is an equivalence relation on $X$. Let $X/\sim$ denote \emph{\textbf{the set of equivalence classes}} in $X$, and let $\pi: X \rightarrow X/\sim$ be the \emph{\textbf{natural projection}} sending each \emph{point} to its \emph{equivalence class}. Endowed with \emph{\textbf{the quotient topology}} determined by $\pi$, the space $X/\sim$ is called \underline{\emph{\textbf{the quotient space}}} (or \emph{identification space}) of $X$ determined by $\pi$.
\end{definition}

\item \begin{definition}
If $\pi: X \rightarrow Y$ is a map, a subset $U \subseteq X$ is said to be \emph{\textbf{saturated}} with respect to $\pi$ if $U$ is the \textbf{\emph{entire preimage}} of its \emph{\textbf{image}}: $U =\pi^{-1}(\pi(U))$. 

Given $y \in Y$, the \underline{\emph{\textbf{fiber}}} of $\pi$ over $y$ is the set $\pi^{-1}(y)$. 
\end{definition}
\end{itemize}


\subsection{Connectedness and Compactness}
\begin{itemize}
\item  \begin{definition} 
A topological space $X$ is said to be \emph{\textbf{disconnected}} if it has two \emph{\textbf{disjoint nonempty open subsets}} whose union is $X$, and it is \emph{\textbf{connected}} otherwise.  Equivalently, $X$ is connected if and only if the only subsets of $X$ that are \emph{\textbf{both open and closed}} are $\emptyset$
and $X$ itself.
\end{definition}

\item \begin{definition}
Recall that a topological space $X$ is
\begin{itemize}
\item \underline{\emph{\textbf{connected}}} if there do not exist two \emph{disjoint}, \emph{nonempty}, \emph{open} subsets of $X$ whose union is $X$;
\item \underline{\emph{\textbf{path-connected}}} if every pair of points in $X$ can be \emph{\textbf{joined by a path}} in $X$, and
\item \emph{\textbf{locally path-connected}} if $X$ has a \emph{\textbf{basis}} of \emph{path-connected open subsets}.
\end{itemize}
\end{definition}

\item \begin{definition}
A \emph{\textbf{maximal connected subset}} of $X$ (i.e., a connected subset that is not properly contained in any larger connected subset) is called a \emph{\textbf{component}} (or \emph{\textbf{connected component}}) of $X$.
\end{definition}

\item \begin{definition}
A topological space $X$ is said to be \underline{\emph{\textbf{compact}}} if every open cover of $X$ has a \emph{\textbf{finite} subcover}. A \emph{\textbf{compact subset}} of a topological space is one that is a compact space in the subspace topology. 
\end{definition}

\item \begin{definition}
If $X$ and $Y$ are topological spaces, a map $F: X \rightarrow Y$ (continuous or not) is said to be \emph{\textbf{proper}} if for every \textbf{\emph{compact}} set $K \subseteq Y$, the \emph{\textbf{preimage}} $F^{-1}(K)$ is \emph{\textbf{compact}}.
\end{definition}

\item \begin{definition}
A topological space $X$ is said to be \underline{\emph{\textbf{locally compact}}} if every point has a \emph{neighborhood} contained in a \emph{\textbf{compact subset}} of $X$. 

A subset of $X$ is said to be \emph{\textbf{precompact}} in $X$ if its \emph{\textbf{closure}} in $X$ is \emph{compact}.
\end{definition}

\item 

\item For a \emph{\textbf{Hausdorff space}} $X$,  the following are equivalent:
\begin{enumerate}
\item $X$ is \emph{\textbf{locally compact}}.
\item Each point of $X$ has a \emph{\textbf{precompact}} neighborhood. 
\item $X$ has a basis of \emph{\textbf{precompact}} open subsets.
\end{enumerate}
\end{itemize}


\newpage
\section{Smooth Manifolds and Smooth Maps}
\subsection{From Topological Manifolds to Smooth Manifolds}
\begin{itemize}
\item \begin{definition}
Suppose $M$ is a \emph{\textbf{topological space}}. We say that $M$ is a \underline{\emph{\textbf{topological manifold}}} of \emph{dimension $n$} or a \emph{\textbf{topological $n$-manifold}} if it has the following properties:
\begin{enumerate}
\item $M$ is a \underline{\emph{\textbf{Hausdorff space}}}: for every pair of distinct points $p, q \in M$, there are disjoint open subsets $U, V \subseteq M$ such that $p \in U$ and $q \in V$.
\item $M$ is \underline{\emph{\textbf{second-countable}}}: there exists a \emph{\textbf{countable basis}} for the topology of $M$.
\item $M$ is \underline{\emph{\textbf{locally Euclidean of dimension $n$}}}: each point of $M$ has a neighborhood that is \emph{\textbf{homeomorphic}} to \emph{an open subset of $\bR^n$}. 
\end{enumerate}
\end{definition}

\item The third property means, more specifically, that for each $p \in M$ we can find
\begin{itemize}
\item an open subset $U \subseteq M$ containing $p$,
\item an open subset $\widehat{U}\subseteq \bR^n$, and
\item a \emph{homeomorphism} $\varphi: U\rightarrow \widehat{U}$.
\end{itemize}

\item  \begin{proposition} (\textbf{Manifolds Are Locally Compact}).\\
Every topological manifold is \textbf{locally compact}.
\end{proposition}

\item \begin{definition}
Let $M$ be a \emph{topological n-manifold}. A \underline{\emph{\textbf{coordinate chart}}} (or just a \emph{chart}) on $M$ is a \emph{\textbf{pair}} $(U, \varphi)$, where $U$ is an open subset of $M$ and $\varphi: U \rightarrow \widehat{U}$ is a \emph{\textbf{homeomorphism}} from $U$ to an open subset $\widehat{U} = \varphi(U) \subseteq \bR^n$. 
\end{definition}

\item \begin{definition}
Given a chart $(U, \varphi)$, we call the set $U$ a \emph{\textbf{coordinate domain}}, or a \emph{\textbf{coordinate neighborhood}} of each of its points. The map $\varphi$ is called a \emph{\textbf{(local) coordinate map}}, and the \emph{\textbf{component functions}} $(x^1,\ldots, x^n)$ of $\varphi$, defined by $\varphi(p) = (x^1(p), \ldots, x^n(p))$, are called \underline{\emph{\textbf{local coordinates}}} on $U$. 
\end{definition}

\item \begin{remark}
We sometimes write things such as ``$(U, \varphi)$ is a chart containing p" as shorthand for ``$(U, \varphi)$ is a chart whose domain $U$ contains p." If we wish to emphasize the coordinate function $(x^1,\ldots, x^n)$ instead of coordinate map $\varphi$, we sometimes denote the chart by $(U, (x^1,\ldots, x^n))$ or $(U, (x^i))$.
\end{remark}

\item \begin{definition}
If $U$ and $V$ are open subsets of Euclidean spaces $\bR^n$ and $\bR^m$, respectively, a function $F: U \rightarrow V$ is said to be \emph{\textbf{smooth}} (or $\cC^{\infty}$ , or \emph{\textbf{infinitely differentiable}}) if each of its component functions has continuous partial derivatives of \emph{\textbf{all orders}}. 

If in addition $F$ is \emph{\textbf{bijective}} and has a \emph{\textbf{smooth inverse map}}, it is called a \underline{\emph{\textbf{diffeomorphism}}}. A \emph{diffeomorphism} is, in particular, a \emph{homeomorphism}.
\end{definition}

\item \begin{definition}
Let $M$ be a topological $n$-manifold. If $(U, \varphi)$, $(V, \psi)$ are two charts such that $U \cap V \neq \emptyset$, the composite map $\psi \circ \varphi^{-1}: \varphi(U \cap V ) \rightarrow \psi(U \cap V)$ is called the \underline{\emph{\textbf{transition map}}} from $\varphi$ to $\psi$. It is a \emph{homeomorphism}. 

Two charts $(U, \varphi)$ and $(V, \psi)$ are said to be \emph{\textbf{smoothly compatible}} if either $U \cap V = \emptyset$ or \emph{the transition map \underline{$\psi \circ \varphi^{-1}$ is a \textbf{diffeomorphism}}}. 
\end{definition}

\item \begin{definition}
We define an \emph{\textbf{atlas}} $\mathcal{A}$ for $M$ to be a collection of charts whose domains \emph{\textbf{cover}} $M$, i.e. $\mathcal{A} := \set{(U_{\alpha}, \varphi_{\alpha})}_{\alpha \in A}$ such that  $M = \bigcup_{\alpha \in A}U_{\alpha}$.  An atlas $\mathcal{A}$ is called a \emph{\textbf{smooth atlas}} if any two charts in $\mathcal{A}$ are \emph{\textbf{smoothly compatible}} with each other.
\end{definition}

\item \begin{definition}
A \emph{smooth atlas} $\cA$ on $M$ is \emph{\textbf{maximal}} if it is \textbf{not} properly contained in \emph{\textbf{any larger} smooth atlas}. This just means that any chart that is \emph{smoothly compatible} with \emph{every chart} in $\cA$ is already in $\cA$. (Such a smooth atlas is also said to be \emph{\textbf{complete}}.)
\end{definition}

\item 
\begin{definition}
If $M$ is a topological manifold, a \underline{\emph{\textbf{smooth structure}}} on $M$ is a \emph{\textbf{maximal smooth atlas}}. 

A \underline{\emph{\textbf{smooth manifold}}} is a pair $(M, \cA)$, where $M$ is a \emph{\textbf{topological manifold}} and $\cA$ is a \emph{smooth structure} on $M$.
\end{definition}

\item \begin{remark}
When the smooth structure is understood, we usually omit mention of it and just say ``\emph{$M$ is a smooth manifold.}" Smooth structures are also called \emph{\textbf{differentiable structures}} or \textbf{\emph{$\cC^{\infty}$ structures}} by some authors. We also use the term \emph{\textbf{smooth manifold structure}} to mean a manifold topology together with a smooth structure.
\end{remark}

\item \begin{remark}
When defining smooth manifold, we ask for any two coordinate charts $(U, \varphi)$, $(V, \psi)$, \emph{\textbf{the transition map}} $\psi \circ \varphi^{-1}: \varphi(U \cap V ) \rightarrow \psi(U \cap V)$ is a diffeomorphism. This is \textbf{stronger} than \emph{the coordinate map} $\varphi$ itself being a  diffeomorphism.
\end{remark}


\item \begin{remark}
In practice, instead of specifying the maximal smooth atlas, it is sufficient to specify \emph{some} smooth atlas to verify if $M$ is smooth manifold
\begin{proposition}
Let $M$ be a topological manifold.
\begin{enumerate}
\item Every smooth atlas $\cA$ for $M$ is contained in a \textbf{unique} maximal smooth atlas, called \textbf{the smooth structure determined by $\cA$}.
\item Two smooth atlases for $M$ determine the same smooth structure if and only if \textbf{their union is a smooth atlas}.
\end{enumerate}
\end{proposition}
\end{remark}

\item \begin{definition}
If $M$ is a smooth manifold, any chart $(U, \varphi)$ contained in the given maximal smooth atlas is called a \underline{\emph{\textbf{smooth chart}}}, and the corresponding coordinate map $\varphi$ is called a \underline{\emph{\textbf{smooth coordinate map}}}. 

It is useful also to introduce the terms \emph{\textbf{smooth coordinate domain}} or \emph{\textbf{smooth coordinate neighborhood}} for the domain of a smooth coordinate chart.
\end{definition}

\item The following result is a generalization of \emph{second-countable} definition for the general topological manifold.
\begin{proposition}
Every smooth manifold has a \textbf{countable} basis of \textbf{regular coordinate balls}.
\end{proposition}

\item \begin{remark}
Each coordinate map $\varphi$ maps a smooth neighborhood $U \subseteq M$ to a neighborhood in Euclidean space $\widetilde{U} \subseteq \bR^{n}$. Under this map, we can \textbf{\emph{(locally)  represent}} a point $p \in U$ by its \emph{\textbf{coordinates}} $(x^1,\ldots, x^n) = \varphi(p)$, and think of this $n$-tuple as \emph{being} the point p. We typically express this by saying ``\emph{$(x^1,\ldots, x^n) $ is \textbf{the (local) coordinate representation} for $p$}" or ``\textbf{\emph{$p = (x^1,\ldots, x^n)$ in local coordinates.}}"
\end{remark}

\item \begin{remark}
As we see, a smooth manifold $M$ does not comes with any predetermined \emph{\textbf{choice of coordinates}}. Thus any objects we wish to define globally on a manifold \emph{\textbf{shall not dependent on a particular choice of coordinates}}.

There are generally two ways of doing this: 
\begin{itemize}
\item either by writing down a \emph{\textbf{coordinate-dependent definition}} and then proving that the \emph{\textbf{definition gives the same results in any coordinate chart}}, 
\item or by writing down \emph{a definition that is \textbf{manifestly coordinate-independent}} (often called an \emph{\textbf{invariant definition}}).
\end{itemize}
\end{remark}
\end{itemize}

\subsection{Manifolds with boundary}
\begin{itemize}
\item \begin{definition}
The \emph{\textbf{closed $n$-dimensional upper half-space}} $\mathbb{H}^n \subseteq \bR^n$ is defined as
\begin{align*}
\mathbb{H}^n &= \set{(x^1, \ldots, x^n) \in \bR^n: x^n \ge 0}. 
\end{align*} We will use the notations $\text{Int}\, \mathbb{H}^n$ and $\partial\, \mathbb{H}^n$ to denote the \emph{\textbf{interior}} and \emph{\textbf{boundary}} of $\mathbb{H}^n$, respectively, as a subset of $\bR^n$. When $n > 0$, this means
\begin{align*}
\text{Int}\, \mathbb{H}^n &= \set{(x^1, \ldots, x^n) \in \bR^n: x^n > 0}, \\
\partial\, \mathbb{H}^n &= \set{(x^1, \ldots, x^n) \in \bR^n: x^n = 0}.
\end{align*}
\end{definition}

\item 
\begin{definition}
An \underline{\emph{\textbf{$n$-dimensional topological manifold with boundary}}} is a \emph{second-countable Hausdorff space} $M$ in which every point has a neighborhood \emph{\textbf{homeomorphic}} either to \emph{an open subset} of $\bR^n$ or to a (\emph{relatively}) \emph{open subset} of $\bH^n$.
\end{definition}

\item \begin{proposition}
Let $M$ be a topological $n$-manifold with boundary.
\begin{enumerate}
\item $\text{Int}\, M$ is an \textbf{open subset} of $M$ and a topological n-manifold \textbf{without boundary}.
\item $\partial\,M$ is a \textbf{closed subset} of M and a \textbf{topological $(n - 1)$-manifold without boundary}.
\item $M$ is a topological manifold if and only if $\partial\,M = \emptyset$.
\item If $n = 0$,then $partial\,M = \emptyset$ and $M$ is a $0$-manifold.
\end{enumerate}
\end{proposition}

\item \begin{definition}
Now let $M$ be a topological manifold \textbf{\emph{with boundary}}. As in the manifold case, a \emph{\textbf{smooth structure}} for $M$ is defined to be a \emph{maximal smooth atlas} -- a collection of charts whose domains cover $M$ and whose transition maps (and their inverses) are smooth in the sense just described. With such a structure, $M$ is called a \emph{\textbf{smooth manifold with boundary}}. 
\end{definition}

\item \begin{remark}
Note that, despite their name, \emph{\textbf{manifolds with boundary are not in general manifolds}}, because boundary points do not have locally Euclidean neighborhoods. Moreover, a manifold with boundary might have \emph{\textbf{empty boundary}} -- there is nothing in the definition that requires the boundary to be a nonempty set. 

On the other hand, \emph{\textbf{a manifold is also a manifold with boundary}}, whose boundary is empty. Thus, every manifold is a manifold with boundary, but a manifold with boundary is a manifold if and only if its boundary is empty.
\end{remark}

\item  \begin{remark}
We will often use redundant phrases such as \emph{\textbf{manifold without boundary}} if we wish to emphasize that we are talking about a manifold in the original sense, and \emph{\textbf{manifold with or without boundary}} to refer to a manifold with boundary if we wish emphasize that the boundary might be empty.
\end{remark}
\end{itemize}

\subsection{Smooth Maps on Manifolds}
\begin{itemize}
\item 
\begin{definition}
Suppose $M$ is a smooth $n$-manifold, $k$ is a nonnegative integer, and $f: M \rightarrow \bR^k$ is any function. We say that $f$ is a \underline{\emph{\textbf{smooth function}}} if for every $p \in M$, there exists a \emph{\textbf{smooth chart}} $(U, \varphi)$ for $M$ whose domain contains $p$ and such that the \emph{\textbf{composite function}} $f \circ \varphi^{-1}$ is smooth on the open subset $\widehat{U} = \varphi(U) \subseteq \bR^n$.

If $M$ is a smooth manifold \emph{\textbf{with boundary}}, the definition is exactly the same, except that $\varphi(U)$ is now an open subset of either 
$\bR^n$ or $\bH^n$, and in the latter case we interpret smoothness of $f \circ \varphi^{-1}$ to mean that each point of $\varphi(U)$ has a neighborhood (in $\bR^n$) on which $f \circ \varphi^{-1}$ \emph{\textbf{extends to a smooth function}} in the ordinary sense.
\end{definition}

\item \begin{definition}
Given a function $f: M \rightarrow \bR^k$ and a chart $(U, \varphi)$ for $M$, the function $\widehat{f}: \varphi(U) \rightarrow \bR^k$ defined by $\widehat{f}(x) = f \circ \varphi^{-1}(x)$ is called the \underline{\emph{\textbf{(local) coordinate representation}} of $f$}. 
\end{definition}

\item \begin{remark}
With the help of coordinate chart $(U, \varphi)$, we can generalize a lot of concepts from Euclidean space to Manifolds. The process of applying $\varphi^{-1}$ is called  \emph{\textbf{(local) parameterization}} and $f \circ \varphi^{-1}$ is the local coordinate representation of the function $f$, which is \emph{\textbf{parametric}}. 

Even though  many objects in this course is \emph{\textbf{independent} of the choice of coordinates},  in practice, one need to use \emph{the coordinate representation} of these objects to \textbf{compute} associated quanitites.
\end{remark}

\item The definition of smooth functions generalizes easily to maps \emph{between} manifolds.
\begin{definition}
Let $M, N$ be \emph{smooth manifolds}, and let $F: M \rightarrow N$ be any map. We say that $F$ is a \underline{\emph{\textbf{smooth map}}} if for every $p \in M$, there exist \emph{smooth charts} $(U, \varphi)$ containing $p$ and $(V, \psi)$ containing $F(p)$ such that $F(U) \subseteq V$ and \underline{\emph{\textbf{the composite map}} $\psi \circ F \circ \varphi^{-1}$ is \emph{\textbf{smooth}}} from $\varphi(U)$ to $\psi(V)$. 

If $M$ and $N$ are smooth manifolds \emph{\textbf{with boundary}}, smoothness of $F$ is defined in exactly the same way, with the usual understanding that a map whose domain is a subset of $\bH^n$ is smooth if it admits an extension to a smooth map in a neighborhood of each point, and a map whose codomain is a subset of $\bH^n$ is smooth if it is smooth as a map into $\bR^n$.
\end{definition}

\item 
\begin{definition}
If $F: M \rightarrow N$ is a \emph{smooth map}, and $(U, \varphi)$ and $(V, \psi)$ are any smooth charts for $M$ and $N$, respectively, we call  \underline{$\widehat{F} = \psi \circ F \circ \varphi^{-1}: \varphi(U\cap F^{-1}(V)) \rightarrow \psi(V)$} the \emph{\textbf{coordinate representation}} of $F$ with respect to the given coordinates. 
\end{definition}

\item \begin{remark}
Use  $F^{-1}(V) \cap U$ is a safer way to make sure the neighborhood $V$ in coordindate chart in $N$ is covered, as compared to using $F(U)\cap V$.
\end{remark}

\item \begin{remark}
In practice, $\widehat{F} = \psi \circ F \circ \varphi^{-1}$ is the function we used in computation involving $F$.
\end{remark}

\item 
\begin{proposition} (\textbf{Equivalent Characterizations of Smoothness}) \citep{lee2003introduction}\\
Suppose $M$ and $N$ are smooth manifolds with or without boundary, and $F: M \rightarrow N$ is a map. Then $F$ is \textbf{smooth} if and only if either of the following conditions is satisfied:
\begin{enumerate}
\item For every $p \in M$, there exist \textbf{smooth charts} $(U, \varphi)$ containing $p$ and $(V, \psi)$ containing $F(p)$ such that $U \cap F^{-1}(V)$ is \textbf{open} in $M$ and the composite map $\psi \circ F \circ \varphi^{-1}$ is \textbf{smooth} from $\varphi(U\cap F^{-1}(V))$ to $\psi(V)$.
\item $F$ is \textbf{continuous} and there exist \textbf{smooth atlases} $\set{(U_{\alpha}, \varphi_{\alpha})}$ and $\set{(V_{\beta}, \psi_{\beta})}$ for $M$ and $N$, respectively, such that for \textbf{each} $\alpha$ and $\beta$, $\psi_{\beta} \circ F \circ \varphi_{\alpha}^{-1}$ is a smooth map from $\varphi_{\alpha}(U_{\alpha}\cap F^{-1}(V_{\beta}))$ to $\psi_{\beta}(V_{\beta})$.
\end{enumerate}
\end{proposition}

\item \begin{definition}
If $M$ and $N$ are smooth manifolds with or without boundary, a \underline{\emph{\textbf{diffeomorphism}}} from $M$ to $N$ is a \emph{\textbf{smooth bijective map}} $F: M \rightarrow N$ that has a \emph{\textbf{smooth inverse}}. We say that $M$ and $N$ are \emph{\textbf{diffeomorphic}} if there exists a \emph{diffeomorphism} between them. Sometimes this is symbolized by $M \approx N$.
\end{definition}
\end{itemize}

\subsection{Einstein summation convention}
\begin{itemize}
\item \begin{remark}
We interpret any such expression according to the \emph{following rule}, called the \emph{\textbf{Einstein summation convention}}:
\begin{itemize}
\item if the \textbf{\emph{same index name}} (such as $i$ in the expression above) appears exactly \textbf{\emph{twice}} in any \emph{\textbf{monomial term}}, \emph{once} as \emph{\textbf{an upper index}} and once as \emph{\textbf{a lower index}}, that term is understood to be summed over \emph{all possible values of that index}, generally from $1$ to the dimension of the space in question. 

\item \emph{\textbf{Basis (vector, functions)}} (such as $E_i$) uses  \emph{\textbf{lower indices}}. The coordinate vector of tangent space $(\dfrac{\partial}{\partial x^{i}})$ are considered as lower indices. 

\item \emph{\textbf{Cooefficients}} of a linear combination, \emph{\textbf{coordinate or component (functions)}} of a vector with respect to a basis (such as $x^i$) use \emph{\textbf{upper indices}}

\item \emph{\textbf{Basis of covectors (cotangent vectors)}} (such as $\epsilon^i, dx^i$) use \emph{\textbf{upper indices}}, while the component of covector with respect to a basis (such as $\omega_i$) use  \emph{\textbf{lower indices}}. That is, vector and covector notation rules are switched. 

\item Any index that is implicitly summed over is a "\emph{\textbf{dummy index}}," meaning that the value of such an expression is \emph{unchanged} if \emph{a different name is substituted for each dummy index}. For example, $x^i E_i$ and $x^j E_j$ mean exactly the same thing. 
\end{itemize}
\end{remark}
\end{itemize}

\section{Submersions, Immersions and Embeddings}
\begin{itemize}
\item \begin{remark}
The following two chapters concerns about the local properties of smooth function $F$ on manifolds by analyzing its differentials $dF_{p}$. ``\emph{\textbf{The differential of $dF_{p}$ of a function $F$ is its best linear approximation in the neighborhood of $p$ .}}" This statement can be generalized from $\bR^{n}$ to $n$-dimensional smooth manifold. These two chapters discuss results that centered around this idea.  
\begin{enumerate}
\item The first theorem is \emph{\textbf{the Inverse Function Theorem for Manifolds}}. This is a direct generalization of the existing results from Euclidean space, since this theorem only concerns the property of the function \emph{\textbf{locally}} and the manifolds is diffeomorphic to Euclidean space locally. The result of this theorm confirms that for a smooth map with invertible differential at a point $p$, in the neighborhood of this point, this map is \emph{\textbf{an open map}} and \emph{\textbf{has smooth inverse}}. 

\item The second theorm concerns about the smooth map \emph{\textbf{with constant rank}}.  Note that the rank of a smooth map is \emph{\textbf{a local property}} since it is about the rank of $dF_p$ at point $p$. But when we enforce the rank to be constant all over the space, we generalize the local properties to \emph{global}. 

\item \emph{\textbf{The Rank Thoerem}} for smooth map \emph{with constant rank} reflects a \emph{\textbf{stronger}} result than \emph{the Inverse Function Theorem}. It states that we \emph{only need to know the \textbf{rank}} of the differential $dF_p$ to determine \emph{\textbf{a local representation of the smooth map $F$}}, \emph{\textbf{regardless}} of \emph{the form of function itself}. In fact, all smooth maps \emph{\textbf{with constant rank}} can be \emph{locally} represented as \underline{\emph{\textbf{coordinate projection with zero padding}}}.  The \emph{\textbf{rank}} of map determines \emph{\textbf{the number of coordinate maintained}} and the others are all zero-padded.

\item \emph{The Rank Theorem} confirms that the smooth function $F$ with constant rank is \emph{\textbf{locally linear}}, thus is best represented by the differential $dF_p$.
\end{enumerate}
\end{remark}

\item  \begin{remark}
 Another important topic that is primiarily discussed in the following two chapters is two special type of smooth map with constant ranks: the  \emph{\textbf{smooth submersion}} and  \emph{\textbf{smooth immersion}}. Both of these properties characterize \underline{\emph{\textbf{the local differential properties}}} of a smooth function $F$ \underline{\emph{\textbf{with full rank}}} (the \emph{shape} of the differential matrix $dF_p$ is determined by the dimension of the manifolds in domain and codomain).
\begin{enumerate}
\item The  \emph{\textbf{smooth submersion}} corresponds to $F$ with \emph{\textbf{surjective differential}} $dF_p \in \bR^{m \times n}$ with $m \ge n$, i.e. the ``\emph{fat}" differential matrix. $dF_p$ has \emph{full column rank}. 

\emph{A smooth submersion} can be \emph{locally} represented as \underline{\emph{\textbf{a natural coordinate projection}}}. Therefore it is critical when studying \emph{\textbf{projections}} and its \emph{\textbf{section}} between manifolds. It is also closely associated with \emph{\textbf{quotient map}} and \emph{\textbf{the quotient manifolds}}.  The \emph{pre-image} of a smooth submersion is a submanifold itself, making it important to define new subspace structure. 

\item Similarly, \emph{\textbf{a smooth immersion}} corresponds to $F$ with \emph{\textbf{injective differential}} $dF_p \in \bR^{m \times n}$ with $m \le n$, i.e. the differential matrix is a ``\emph{tall}" matrix with \emph{full row rank}. 

A smooth immersion can be locally represented as \underline{\emph{\textbf{a natural coordinate inclusion}}}, or a zero-padding function. It is critical when we try to \emph{put a manifold inside another manifold} and makes it a submanifold of the latter. 

\begin{itemize}
\item A sub-class of smooth immersion is \emph{\textbf{smooth embedding}}. In addition being smooth immersion, a smooth embedding is also \emph{\textbf{locally homemorphic} to its image}. In other word, it allows one to build an \emph{\textbf{equivalence relationship}} between two \emph{local regions}  in \emph{topological sense}. 

\item Like smooth immersion, a smooth embedding locally is an \emph{\textbf{inclusion}} but the embedding map globally is an \emph{\textbf{injective}} map with \emph{\textbf{smooth inverse}} \emph{locally}.

\item \emph{Smooth embedding map} is commonly used to generate $k$-dimensional \emph{\textbf{embedded submanifolds}}, which can be seen as a subset equipped with the subspace topology and \emph{\textbf{locally homemorphic to a $k$-dimensional subspace in $\bR^{n}$.}}
\end{itemize}

\end{enumerate}
\end{remark}
\end{itemize}
\newpage
\subsection{Definitions}
\begin{itemize}
\item \begin{definition}
Suppose $M$ and $N$ are smooth manifolds with or without boundary. Given a smooth map $F: M \rightarrow N$ and a point $p \in M$, we define \underline{\emph{the \textbf{rank} of $F$ at $p$}} to be \emph{\textbf{the rank of the linear map} $dF_p: T_{p}M \rightarrow T_{F(p)}N$}; it is \underline{\emph{\textbf{the rank of the Jacobian matrix}}} of F in any smooth chart, or \emph{\textbf{the dimension of}} $\text{Im }dF_p \subseteq T_{F(p)}N$. If $F$ has the same rank $r$ \emph{at every point}, we say that it has constant rank, and write \underline{$\text{rank}\,F = r$}.
\end{definition}

\item \begin{definition}
Note that $\text{rank }dF_{p} \le \min\set{\text{dim }M, \; \text{dim }N}$. If the rank of $dF_p$ is equal to this upper bound, we say that \emph{\textbf{$F$ has full rank at $p$}}, and if $F$ has full rank everywhere, we say \underline{\emph{\textbf{$F$ has full rank}}}.
\end{definition}

\item \begin{definition}
The most important \emph{constant-rank maps} are those \emph{of full rank}. A smooth map $F: M \rightarrow N$ is called \underline{\emph{\textbf{a smooth submersion}}} if its differential is \underline{\emph{\textbf{surjective}}} \emph{at each point} (or \emph{\textbf{equivalently}}, if \underline{$\text{rank }F = \text{dim }N$}). 

It is called \underline{\emph{\textbf{a smooth immersion}}} if its differential is \underline{\emph{\textbf{injective}}} at each point (\emph{\textbf{equivalently}}, \underline{$\text{rank }F = \text{dim }M$}).
\end{definition}

\item \begin{proposition}
Suppose $F: M \rightarrow N$ is a smooth map and $p \in M$ . If $dF_p$ is \textbf{surjective}, then $p$ has a neighborhood $U$ such that $F|_{U}$ is a \textbf{submersion}. If $dF_p$ is \textbf{injective}, then $p$ has a neighborhood $U$ such that $F|_{U}$ is an \textbf{immersion}.
\end{proposition}

\item \begin{remark}
$F$ is a \emph{\textbf{surjective/injective}} is different from $F$ is a smooth \emph{\textbf{submersion/immersion}}. The latter is the property of the differential map $dF_p$ at each $p$ not the property of the map itself.  But $F$ is \emph{\textbf{a smooth embedding}} $\Rightarrow$ $F$ is \emph{\textbf{an injective smooth immersion}}. The converse is not true since $F$ also need to \emph{have \textbf{continous inverse}} from its image to its domain.
\end{remark}
\end{itemize}

\subsection{Local Diffeomorphisms}
\begin{itemize}
\item \begin{definition}
If $M$ and $N$ are smooth manifolds with or without boundary, a map $F: M \rightarrow N$ is called \underline{\emph{\textbf{a local diffeomorphism}}} if every point $p \in M$ has a neighborhood $U$ such that $F(U)$ is \emph{\textbf{open}} in $N$ and the restriction $F|_{U}: U \rightarrow F(U)$ is a \emph{\textbf{diffeomorphism}}. 
\end{definition}

\item The next theorem is the key to the most important properties of local diffeomorphisms.
\begin{theorem} (\textbf{Inverse Function Theorem for Manifolds}). \citep{lee2003introduction} \\
Suppose $M$ and $N$ are smooth manifolds, and $F: M \rightarrow N$ is a \textbf{smooth map}. If  $p \in M$ is a point such that $dF_p$ is \textbf{invertible}, then there are \textbf{connected} neighborhoods $U_0$ of $p$ and $V_0$ of $F(p)$ such that $F|_{U_0}: U_0 \rightarrow V_0$ is a \textbf{diffeomorphism}.
\end{theorem}
\end{itemize}

\subsection{The Rank Theorem}
\begin{itemize}
\item \begin{theorem} (\textbf{Rank Theorem}). \citep{lee2003introduction} \\
Suppose $M$ and $N$ are smooth manifolds of dimensions $m$ and $n$, respectively, and  $F: M \rightarrow N$ is a smooth map \underline{\textbf{with constant rank $r$}}. For each $p \in M$ there exist smooth charts $(U, \varphi)$ for $M$ centered at $p$ and $(V,\psi)$ for $N$ centered at $F(p)$ such that $F(U) \subseteq V$, in which $F$ has a \underline{\textbf{coordinate representation}} of the form
\begin{align}
\widehat{F}(x^1,\ldots, x^r, x^{r+1},\ldots, x^m) &= (x^1,\ldots, x^r, 0,\ldots, 0).  \label{eqn: rank_theorem_coordinate_rep}
\end{align} In particular, if $F$ is a \underline{\textbf{smooth submersion}}, this becomes
\begin{align}
\widehat{F}(x^1,\ldots, x^n, x^{n+1},\ldots, x^m) &=  (x^1,\ldots, x^n).  \label{eqn: rank_theorem_coordinate_rep_subm}
\end{align}
and if $F$ is a \underline{\textbf{smooth immersion}}, it is
\begin{align}
\widehat{F}(x^1,\ldots, x^m) &=  (x^1,\ldots, x^m, 0,\ldots, 0).  \label{eqn: rank_theorem_coordinate_rep_imm}
\end{align}
\end{theorem}

\item \begin{corollary}
Let $M$ and $N$ be smooth manifolds, let  $F: M \rightarrow N$ be a smooth map, and suppose $M$ is connected. Then the following are \textbf{equivalent}:
\begin{enumerate}
\item For each $p \in M$ there exist smooth charts containing $p$ and $F(p)$ in which \textbf{the coordinate representation of $F$ is linear}.
\item $F$ has \textbf{constant rank}.
\end{enumerate}
\end{corollary}

\item The rank theorem is a purely \emph{\textbf{local statement}}. However, it has the following powerful \emph{\textbf{global} consequence}.
\begin{theorem} (\textbf{Global Rank Theorem}). \citep{lee2003introduction}\\
Let $M$ and $N$ be smooth manifolds, and suppose $F: M \rightarrow N$ is a smooth map of \textbf{constant rank}.
\begin{enumerate}
\item If $F$ is \textbf{surjective}, then it is a \textbf{smooth submersion}.
\item If $F$ is \textbf{injective}, then it is a \textbf{smooth immersion}.
\item If $F$ is \textbf{bijective}, then it is a \textbf{diffeomorphism}.
\end{enumerate}
\end{theorem}
\end{itemize}
\subsection{Embeddings}
\begin{itemize}
\item  One special kind of \emph{\textbf{immersion}} is particularly important.
\begin{definition}
If $M$ and $N$ are smooth manifolds with or without boundary, a  \underline{\emph{\textbf{smooth embedding}}} of $M$ into $N$ is a \underline{\emph{\textbf{smooth immersion}}} $F: M \rightarrow N$ that is \emph{\textbf{also}} \underline{\emph{\textbf{a topological embedding}}}, i.e., \emph{a \textbf{homeomorphism} onto its image} $F(M) \subseteq N$ in the \emph{subspace topology}. 
\end{definition}

\item \begin{remark}
\emph{\textbf{A smooth embedding}} is a map that is \emph{\textbf{both}} a \emph{\textbf{topological embedding}} and a \emph{\textbf{smooth immersion}}, not just a topological embedding that happens to be smooth. Also the map need to be \emph{\textbf{injective}} and its inverse from $F(U)$ to $U$ needs to be continuous.
\end{remark}

\item \begin{proposition}
Suppose $M$ and $N$ are smooth manifolds with or without boundary, and $F: M \rightarrow N$ is an \textbf{injective smooth immersion}. If \textbf{any} of the following holds, then $F$ is a\textbf{ smooth embedding}.
\begin{enumerate}
\item $F$ is an \textbf{open} or \textbf{closed} map. (i.e. it maps an open/closed set to an open/closed set)
\item $F$ is a \textbf{proper map}. (i.e. the preimage of every compact set is compact)
\item $M$ is \textbf{compact}.
\item $M$ has empty boundary and $\text{dim }M = \text{dim }N$
\end{enumerate}
\end{proposition}

\item \begin{theorem} (\textbf{Local Embedding Theorem}).\\
Suppose $M$ and $N$ are smooth manifolds with or without boundary, and $F: M \rightarrow N$ is a smooth map. Then $F$ is a \textbf{smooth immersion} if and only if every point in $M$ has a neighborhood $U\subseteq M$ such that $F|_{U}: U \rightarrow N$ is a \textbf{smooth embedding}.
\end{theorem}
\end{itemize}

\subsection{Submersions}
\begin{itemize}
\item \begin{definition}
If $\pi: M \rightarrow N$ is any continuous map, a \underline{\emph{\textbf{section}}} of $\pi$ is a \underline{\emph{\textbf{continuous right inverse}}} for $\pi$, i.e., a continuous map $\sigma: N \rightarrow M$ such that $\pi \circ \sigma = \text{Id}_N$:
\[
  \begin{tikzcd}
     M  \arrow{r}{\pi}  & N   \arrow[l, bend left,  "\sigma"] 
  \end{tikzcd}
\] 
\end{definition}

\item \begin{definition}
A \emph{\textbf{local section}} of $\pi$ is a continuous map $\sigma: U \rightarrow M$ defined on some open subset $U \subseteq N$ and satisfying the analogous relation $\pi \circ \sigma = \text{Id}_U$
\end{definition}

\item \begin{theorem} (\textbf{Local Section Theorem}). \citep{lee2003introduction}\\
Suppose $M$ and $N$ are smooth manifolds and $\pi: M \rightarrow N$ is a smooth map. Then $\pi$ is a \textbf{smooth submersion} if and only if every point of $M$ is in the \textbf{image} of a \textbf{smooth local section} of $\pi$.
\end{theorem}

\item \begin{proposition} (\textbf{Properties of Smooth Submersions}). \\
Let $M$ and $N$ be smooth manifolds, and suppose $\pi: M \rightarrow N$ is a smooth submersion. Then $\pi$ is \textbf{an open map}, and if it is \textbf{surjective} it is a \textbf{quotient map}.
\end{proposition}

\item The next three theorems provide important tools that we will use frequently
when studying submersions. 
\begin{theorem} (\textbf{Characteristic Property of Surjective Smooth Submersions}).\\
Suppose $M$ and $N$ are smooth manifolds, and $\pi: M \rightarrow N$ is a \textbf{surjective smooth submersion}. For any smooth manifold $P$ with or without boundary, a map $F: N \rightarrow P$ is \textbf{smooth} if and only if $F \circ \pi$ is \textbf{smooth}:
\[
  \begin{tikzcd}
     M  \arrow[swap]{d}{\pi} \arrow[dr, "F \circ \pi"]  & \\
     N   \arrow[r, swap,  "F"]  & P.
  \end{tikzcd}
\] 
\end{theorem}


\begin{theorem} (\textbf{Passing Smoothly to the Quotient}).\\
Suppose $M$ and $N$ are smooth manifolds and  $\pi: M \rightarrow N$  is a \textbf{surjective smooth submersion}. If $P$ is a
smooth manifold with or without boundary and $F: M \rightarrow P$ is a smooth map that is \textbf{constant} on \textbf{the fibers of $\pi$}, then there exists a \textbf{unique smooth map} $\widetilde{F}: N \rightarrow P$ such that $\widetilde{F} \circ \pi = F$:
\[
  \begin{tikzcd}
     M  \arrow[swap]{d}{\pi} \arrow[dr, "F"]  & \\
     N   \arrow[r, swap, dashed,  "\widetilde{F}"]  & P.
  \end{tikzcd}
\] 
\end{theorem}

\begin{theorem} (\textbf{Uniqueness of Smooth Quotients}). \\
Suppose that $M, N_1$, and $N_2$ are smooth manifolds, and $\pi_1: M \rightarrow N_1$ and $\pi_2: M \rightarrow N_2$ are \textbf{surjective smooth submersions} that are \textbf{constant} on \textbf{each other's fibers}. Then there exists a \textbf{unique diffeomorphism} $F: N_1 \rightarrow N_2$ such that $F\circ \pi_1 = \pi_2$:
\[
  \begin{tikzcd}
     & M  \arrow[swap]{dl}{\pi_1} \arrow[dr, "\pi_2"]  & \\
     N_1   \arrow[rr, swap, dashed,  "F"]  & & N_2.
  \end{tikzcd}
\] 
\end{theorem}
\end{itemize}
\section{Submanifolds}
\begin{itemize}
\item \begin{remark}
A great number of manifolds of interest can be considered as a submanifold of some other (simpler) manifolds. This chaper mainly concerns about \emph{\textbf{the embedded submanifolds}} and then extends to \emph{\textbf{immersed submanifolds}}.  
\end{remark}

\item \begin{remark}
The definition of \emph{\textbf{embedded submanifold}} $S$ of $M$ consists of three parts:
\begin{enumerate}
\item $S \subseteq M$ has \emph{\textbf{the subspace topology}} (inherited from the topology of the ambient manifold $M$);
\item $S$ is endowed with \emph{\textbf{a smooth structure}};
\item Under this smooth structure, \emph{\textbf{the inclusion map}} $S \xhookrightarrow{} M$ is a \emph{\textbf{smooth embedding}}, i.e. it has injective differential everywhere and is \emph{locally homemorphic} to its image.
\end{enumerate}
\end{remark}

\item \begin{remark}
Besides the definition, there are two other ways to identify embedded submanifold:
\begin{enumerate}
\item \emph{\textbf{Local Slice Criterion}}: If a subset $S$ under local representation of $M$ is \emph{homemorphic} to an open subset of $\bR^{k} \subseteq \bR^{n}$. \emph{The Local Slice Criterion} relies on \emph{the \textbf{topology} and \textbf{the smooth structure} of \textbf{the ambient manifold $M$}} to determine both the topology and the smooth structure of submanifold $S$.
\item \emph{\textbf{Level Set Criterion}}: If every point of subset $S$ has a neighborhood in $M$ so that $S \cap U$ is a level set of some \emph{\textbf{smooth submersion}}, then this set is a  $k$-dimensional embedded submanifold.  In particular,  \emph{every embedded submanifold admits a local defining function in a neighborhood of each of its points}. This result provides \emph{\textbf{a constructive way}} to build embedded submanifold by defining map.
\end{enumerate}
\end{remark}

\item \begin{remark}
An important result of \emph{smooth map with constant rank} is that each of its \emph{\textbf{level set}} is a \emph{properly embedded submanifold} with \emph{\textbf{codimension}} equal to the \emph{\textbf{rank}} of smooth map.  In particular,
\begin{enumerate}
\item If the map is a \emph{\textbf{submersion}}, then the level set is \emph{\textbf{a regular level set}}, which is a properly embedded submanifold. Moreover, the \emph{\textbf{dimension}} of the submanifold is determined by the difference of dimension between domain and codomain.
\item The \emph{\textbf{image of immersion}} is also a submanifold, \emph{\textbf{the immersed submanifold}}. An immersed submanifold may not be an embedded submanifold. But \emph{\textbf{locally}} it is \emph{an embedded submanifold}.
\end{enumerate}
\end{remark}

\item \begin{remark} The final important remark on embedded submanifold is that it has a \emph{\textbf{unique}} \emph{\textbf{topology and smooth structure}}, which is \emph{\textbf{the subspace topology}} and the coordinate map that makes the \emph{$S\cap U$ is a \textbf{$k$-slice} of $U$}. 
\end{remark}
\end{itemize}
\subsection{Embedded Submanifolds}
\subsubsection{Definitions}
\begin{itemize}
\item \begin{definition}
Suppose $M$ is a smooth manifold with or without boundary. \emph{\textbf{An \underline{embedded} \underline{submanifold}}} of $M$ is a subset $S \subseteq M$ that is a \emph{manifold} (without boundary) in the \emph{\textbf{subspace topology}}, endowed with a \emph{\textbf{smooth structure}} with respect to which the \emph{\textbf{inclusion map}} $S \xhookrightarrow{} M$ is a \underline{\emph{\textbf{smooth embedding}}}. Embedded submanifolds are also called \emph{\textbf{regular submanifolds}}.
\end{definition}

\item \begin{definition}
If $S$ is an \emph{embedded submanifold} of $M$, the \emph{\textbf{difference}} $\text{dim }M - \text{dim }S$ is called
\underline{\emph{\textbf{the codimension}}} of $S$ in $M$, and the \emph{\textbf{containing manifold}} $M$ is called \emph{\textbf{the \underline{ambient} manifold}} for $S$. 

An embedded \emph{\textbf{hypersurface}} is an embedded submanifold of codimension $1$. The \emph{empty set} is an embedded submanifold of \emph{any dimension}.
\end{definition}

\item \begin{proposition}(\textbf{Open Submanifolds}). \citep{lee2003introduction}\\
Suppose $M$ is a smooth manifold. The embedded submanifolds of \textbf{codimension 0} in $M$ are exactly the \textbf{open submanifolds}.
\end{proposition}

\item There are several other ways to create submanifolds:
\begin{proposition} (\textbf{Images of Embeddings as Submanifolds}). \citep{lee2003introduction} \\
Suppose $M$ is a smooth manifold with or without boundary, $N$ is a smooth manifold, and $F: N \rightarrow M$ is a \textbf{smooth embedding}. Let $S = F(N)$. With the subspace topology, $S$ is a topological manifold, and it has a \textbf{unique smooth structure} making it into an \textbf{embedded submanifold} of $M$ with the property that $F$ is a \textbf{diffeomorphism} onto its image.
\end{proposition}

\item \begin{proposition} (\textbf{Slices of Product Manifolds}). \citep{lee2003introduction}\\
Suppose $M$ and $N$ are smooth manifolds. For each $p \in N$, the subset $M \times \set{p}$ (called \textbf{a slice of the product manifold}) is an \textbf{embedded submanifold} of $M \times N$ diffeomorphic to $M$.
\end{proposition}

\item \begin{proposition} (\textbf{Graphs as Submanifolds}). \citep{lee2003introduction}\\
Suppose $M$ is a smooth $m$-manifold (without boundary), $N$ is a smooth $n$-manifold with or without boundary, $U \subseteq M$ is open, and $f: U \rightarrow N$ is a smooth map. Let $\Gamma(f) \subseteq M \times N$ denote \textbf{the graph of $f$}:
\begin{align*}
\Gamma(f) &= \set{ (x,y) \in M \times N: x \in U, y = f(x)}.
\end{align*} Then $\Gamma(f)$ is an \textbf{embedded $m$-dimensional} submanifold of $M \times N$
\end{proposition}

\item \begin{definition}
An embedded submanifold $S \subseteq M$ is said to be \emph{\textbf{properly embedded}} if the inclusion $S \xhookrightarrow{} M$ is a \emph{\textbf{proper map}}.
\end{definition}

\item \begin{proposition}
Suppose $M$ is a smooth manifold with or without boundary and $S \subseteq M$ is an embedded submanifold. Then $S$ is \textbf{properly} embedded if and only if it
is a \textbf{closed} subset of $M$.
\end{proposition}

\item \begin{corollary}
Every \textbf{compact} embedded submanifold is \textbf{properly} embedded.
\end{corollary}

\item \begin{proposition} (\textbf{Global Graphs Are Properly Embedded}). \citep{lee2003introduction}\\
Suppose $M$ is a smooth manifold, $N$ is a smooth manifold with or without boundary, and $f: M \rightarrow N$ is a \textbf{smooth} map. With the smooth manifold structure as above, the graph of $f$ $\Gamma(f)$ is properly embedded in $M \times N$.
\end{proposition}
\end{itemize}

\subsubsection{Slice Charts for Embedded Submanifolds}
\begin{itemize}
\item \begin{definition}
if $U$ is an open subset of $\bR^n$ and $k\in \set{0,\ldots,n}$, a \underline{\emph{\textbf{$k$-dimensional slice}}} of $U$ (or simply a $k$-slice) is any subset of the form 
\begin{align*}
S = \set{(x^1,\ldots,x^k, x^{k+1},\ldots, x^n) \in U: x^{k+1} = c^{k+1},\ldots, x^n = c^n}
\end{align*} for some constants $c^{k+1},\ldots,c^n$. (When $k = n$, this just means $S = U$.)  Clearly, \emph{\textbf{every $k$-slice is \underline{homeomorphic to an open subset of $\bR^k$}}}. 
\end{definition}

\item \begin{definition}
Let $M$ be a smooth $n$-manifold, and let $(U, \varphi)$ be a smooth chart on $M$. If $S$ is a subset of $U$ such that $\varphi(S)$ is a $k$-slice of $\varphi(U)$, then we say that \emph{\textbf{$S$ is a $k$-slice of $U$}}. 
\end{definition}

\item \begin{definition}
Given a subset $S \subseteq M$ and a nonnegative integer $k$, we say that \emph{\textbf{$S$ satisfies the local $k$-slice condition}} if \emph{each point} of $S$ is contained
in the domain of a smooth chart  $(U, \varphi)$ for $M$ such that \emph{$S \cap U$ is a \textbf{single $k$-slice} in $U$}. Any such chart is called \emph{\textbf{a slice chart for $S$ in $M$}} , and the corresponding coordinates $(x^1,\ldots,x^n)$ are called \emph{\textbf{slice coordinates}}.
\end{definition}

\item \begin{remark}
The key to understand the \underline{\emph{\textbf{the local $k$-slice condition}}} for $S \subseteq M$:
\begin{enumerate}
\item It is a condition on the \emph{\textbf{subset}} $S$ only; it does \emph{\textbf{not presuppose}} any particular \emph{\textbf{topology}} or \emph{\textbf{smooth structure}} on $S$. All it needs is the topology and smooth structure from the ambient manifold $M$.
\item The \emph{local neighborhood} $U \subseteq M$ is a \underline{\emph{\textbf{neighborhood} of $p$ in the \textbf{ambient manifold $M$}}} not a neigborhood in $S$ (since we do not define such topology);
\item The $k$-slice representation is for the \emph{\textbf{intersection}} \underline{$S\cap U$} under \emph{\textbf{the smooth chart}} $(U, \varphi)$ of \emph{\textbf{the ambient manifold $M$.}}
\end{enumerate}
\end{remark}

\item \begin{theorem} (\textbf{Local Slice Criterion for Embedded Submanifolds})  \citep{lee2003introduction}.\\
Let $M$ be a smooth n-manifold. If $S \subseteq M$ is an embedded $k$-dimensional submanifold, then $S$ satisfies the local $k$-slice condition. \textbf{Conversely}, if $S \subseteq M$ is a subset that \textbf{satisfies the local $k$-slice condition}, then with the subspace topology, $S$ is a topological manifold of dimension $k$, and it \textbf{has a smooth structure} making it into a $k$-dimensional embedded submanifold of $M$ .
\end{theorem}

\item \begin{theorem}
If $M$ is a smooth $n$-manifold with boundary, then with the subspace topology, $\partial M$ is a topological $(n-1)$-dimensional manifold (without boundary), and has a smooth structure such that it is a properly \textbf{embedded submanifold} of $M$.
\end{theorem}
\end{itemize}
\subsubsection{Level Sets}
\begin{itemize}
\item \begin{remark}
In practice, embedded submanifolds are most often presented as \emph{\textbf{solution sets}} of \emph{equations or systems of equations}.
\end{remark}


\item \begin{definition}
If $\Phi: M \rightarrow N$ is any map and $c$ is any point of N, we call the set \emph{\textbf{$\Phi^{-1}.(c)$ a level set of $\Phi$}} (Fig. \ref{fig: level_sets}). (In the special case $N = \bR^k$ and $c = 0$, the level set $\Phi^{-1}(0)$ is usually called \emph{\textbf{the zero set of $\Phi$}}.)
\end{definition}

\item \begin{remark}
It is easy to find \emph{level sets of smooth functions} that are \emph{not smooth submanifolds}. 
\begin{align*}
\Theta(x,y)  = x^2 - y, \quad \Phi(x,y) = x^2 - y^2, \quad  \Psi(x, y) = x^2 -y^3.
\end{align*} (Note that the zero set $\Theta^{-1}(0)$ is an embedded submanifolds in $\bR^{2}$ but not for others.) In fact, \emph{\textbf{every closed subset of $M$}} can be expressed as \emph{\textbf{the zero set}} of some smooth real-valued function.
\end{remark}

\item \begin{theorem} (\textbf{Constant-Rank Level Set Theorem}). \citep{lee2003introduction} \\
Let $M$ and $N$ be smooth manifolds, and let $\Phi: M \rightarrow N$ be a smooth map \textbf{with constant rank $r$}. \textbf{Each level set} of $\Phi$ is a properly embedded submanifold of \textbf{codimension $r$} in $M$.
\end{theorem}

\item \begin{corollary} (\textbf{Submersion Level Set Theorem}). \citep{lee2003introduction} \\
 If $M$ and $N$ are smooth manifolds and $\Phi: M \rightarrow N$ is a \textbf{smooth submersion}, then each level set of $\Phi$ is a \textbf{properly embedded} submanifold whose \textbf{codimension} is equal to the \textbf{dimension of $N$}.
\end{corollary}

\item \begin{remark}
This result should be compared to the corresponding result in linear algebra: if $L: \bR^m \rightarrow \bR^r$ is a surjective linear map, then the kernel of $L$ is a linear subspace of codimension $r$ by \emph{\textbf{the rank-nullity law}}.  The vector equation $Lx = 0$ is equivalent to $r$ linearly independent scalar equations, each of which can be thought of as cutting down one of the degrees of freedom in $\bR^m$, leaving a subspace of codimension $r$. 

In the context of smooth manifolds, the analogue of \emph{a surjective linear map} is \emph{\textbf{a smooth submersion}}, \emph{\textbf{each}} of whose \emph{\textbf{(local) component functions cuts down the dimension by one}}.
\end{remark}

\item \begin{definition}
If $\Phi: M \rightarrow N$ is a smooth map, a point $p \in M$ is said to be \underline{\emph{\textbf{a regular point}}} of $\Phi$ if $d\Phi_{p}: T_{p}M \rightarrow
T_{\Phi(p)}N$ is \emph{\textbf{surjective}}; it is \underline{\emph{\textbf{a critical point}}} of $\Phi$ otherwise. 

This means, in particular, that \emph{\textbf{every point}} of $M$ is \emph{\textbf{critical}} if \underline{$\text{dim }M < \text{dim }N$}, and every point is \emph{\textbf{regular}} if and only if $\Phi$ is a \emph{\textbf{submersion}}. 
\end{definition}

\item \begin{definition}
A point $c \in N$ is said to be \emph{\textbf{a regular value}} of  $\Phi$ if \emph{\textbf{every point} of the level set $\Phi^{-1}(c)$  is a regular point}, and \emph{\textbf{a critical value}} otherwise. In particular, if $\Phi^{-1}(c) = \emptyset$, then $c$ is \emph{a regular value}. Finally, a level set $\Phi^{-1}(c)$  is called \emph{\textbf{a regular level set}} if $c$ is a regular value of $\Phi$; in other words, a regular level set is a level set consisting \emph{\textbf{entirely} of regular points} of $\Phi$ (points $p$ such that $d\Phi_{p}$ is surjective).
\end{definition}

\item \begin{remark}
\emph{Every properly embedded submanifold $M = \Phi^{-1}(c)$ is a regular level set}. The following theorem shows that the converse is true as well. 
\end{remark}

\item \begin{theorem} (\textbf{Regular Level Set Theorem}).  \citep{lee2003introduction}\\
\textbf{Every regular level set} of a smooth map between smooth manifolds is a \textbf{properly embedded} submanifold whose codimension is equal to the dimension of the codomain.
\end{theorem}

\item \begin{proposition} (\textbf{Local Level Set Criterion for Smooth Embedded Submanifolds})\\
Let $S$ be a subset of a smooth $m$-manifold $M$. Then $S$ is an \textbf{embedded $k$-submanifold} of $M$ if and only if every point of $S$ has a neighborhood
$U$ in $M$ such that $U \cap S$ is a \textbf{level set} of a \textbf{smooth submersion} $\Phi: U \rightarrow \bR^{m-k}$.
\end{proposition}

\item \begin{proposition} (\textbf{Local Level Set Criterion for Smooth Embedded Submanifolds})\\
Let $S$ be a subset of a smooth $m$-manifold $M$. Then $S$ is an \textbf{embedded $k$-submanifold} of $M$ if and only if every point of $S$ has a neighborhood
$U$ in $M$ such that $U \cap S$ is a \textbf{level set} of a \textbf{smooth submersion} $\Phi: U \rightarrow \bR^{m-k}$.
\end{proposition}

\item \begin{definition}
If $S \subseteq M$ is an embedded submanifold, \emph{a smooth map} $\Phi: M \rightarrow N$ such that $S$ is \emph{\textbf{a regular level set of $\Phi$}} is called \underline{\emph{\textbf{a defining map for S}}}. In the special case $N = \bR^{m-k}$ (so that $\Phi$ is a real-valued or vector-valued function), it is usually called \emph{\textbf{a defining function}}. 

More generally, if $U$ is an open subset of $M$ and $\Phi: U \rightarrow N$ is a smooth map such that $S \cap U$ is a regular level set of $\Phi$, then $\Phi$ is called \emph{\textbf{a local defining map (or local defining function) for $S$}}.
\end{definition}
\end{itemize}

\subsection{Immersed Submanifolds}
\subsubsection{Definitions and Examples}
\begin{itemize}
\item \begin{definition}
Let $M$ be a smooth manifold with or without boundary. \textit{\textbf{An \underline{immersed} submanifold}} of $M$ is a subset $S \subseteq M$ endowed with a \emph{topology} (\emph{not necessarily \textbf{the subspace topology}}) with respect to which it is \emph{\textbf{a topological manifold}} (without boundary), and \emph{a smooth structure} with respect to which \emph{\textbf{the inclusion map}} $S \xhookrightarrow{} M$ is \emph{\textbf{a smooth immersion}}. 

As for embedded submanifolds, we define the \emph{\textbf{codimension}} of $S$ in $M$ to be $\text{dim }M - \text{dim }S$.
\end{definition}

\item \begin{remark}
Immersed submanifolds \emph{do not require \textbf{the submanifold topology} to be \textbf{the subspace topology}} which is more general than embedded submanifold. 
\end{remark}

\item The immersed submanifolds arise in natural way:
\begin{proposition} (\textbf{Images of Immersions as Submanifolds}). \citep{lee2003introduction} \\
Suppose $M$ is a smooth manifold with or without boundary, $N$ is a smooth manifold, and $F: N \rightarrow M$ is an \textbf{injective smooth immersion}. Let $S = F(N)$. Then $S$ has a unique topology and smooth structure such that it is a \textbf{smooth submanifold} of $M$ and such that $F: N \rightarrow S$ is a \textbf{diffeomorphism} onto its image.
\end{proposition}

\item \begin{example} (\emph{\textbf{Immersed Submanifold but Not an Embedded Submanifold}})\\
Both examples of \emph{\textbf{The Figure-Eight}} and \emph{\textbf{the Dense Curve on the Torus}} are \emph{images of injective smooth immersions}, they are \emph{\textbf{immersed submanifolds}} when given appropriate topologies and smooth structures. As smooth manifolds, they are \emph{\textbf{diffeomorphic}} to $\bR$. \emph{They are \textbf{not embedded submanifolds}}, because \emph{\textbf{neither}} one has \emph{\textbf{the subspace topology}}. In fact, their image sets cannot be made into embedded submanifolds even if we are allowed to change their topologies and smooth structures. \qed
\end{example}

\item \begin{remark}
Suppose $M$ is a smooth manifold and $S \subseteq M$ is \emph{\textbf{an immersed submanifold}}. It can be shown that every subset of $S$ that is \emph{\textbf{open}} in the \emph{\textbf{subspace topology}} is also \emph{\textbf{open}} in its given \emph{\textbf{submanifold topology}}; and the \textbf{converse} is true if and only if $S$ is \textbf{\emph{embedded}}.
\end{remark} 


\item \begin{proposition} (\textbf{Immersed Submanifolds Are Locally Embedded}). \citep{lee2003introduction} \\
If $M$ is a smooth manifold with or without boundary, and $S \subseteq M$ is an \textbf{immersed submanifold}, then for each $p \in S$ there exists a neighborhood $U$ of $p$ \underline{\textbf{in $S$}} that is an \textbf{embedded submanifold} of $M$.
\end{proposition} Note that a smooth immersion is locally a smooth embedding.


\item \begin{remark}
It is important to be clear about what this proposition does and does not say: given an immersed submanifold $S \subseteq M$ and a point $p \in S$,  it is possible to find \emph{a neighborhood $U$ of $p$ \underline{\textbf{in $S$}}} such that \underline{$U$} is \emph{embedded}; but it may not be possible to find \emph{a neighborhood $V$ of $p$ \underline{\textbf{in $M$}}} such that \underline{$V \cap S$} is embedded. 
\end{remark}

\item \begin{definition}
Suppose $S \subseteq M$ is an immersed $k$-dimensional submanifold. \emph{\textbf{A local parametrization}} of $S$ is a continuous map $X: U \rightarrow M$ whose domain is an \emph{\textbf{open subset}} $U \subseteq \bR^k$, whose image is an \emph{\textbf{open subset}} of $S$, and which, considered as a map into $S$, is a \emph{\textbf{homeomorphism} onto its image}. It is called a \underline{\emph{\textbf{smooth local parametrization}}} if it is a \emph{\textbf{diffeomorphism}} onto its image (with respect to $S$’s smooth manifold structure). If the image of $X$ is all of $S$, it is called \underline{\emph{\textbf{a global parametrization}}}.
\end{definition}

\item \begin{remark}
For \emph{a smooth chart} $(U, \varphi)$ of $M$, $\varphi: U \rightarrow \widehat{U} \subseteq \bR^{n}$ is a \emph{diffeomorphism}, \emph{its inverse} $\varphi^{-1}: \widehat{U} \rightarrow U \subseteq M$ is \emph{\textbf{a smooth local parameterization}} (in fact $X= \text{Id}_{M} \circ \varphi^{-1}$).
\end{remark}

\item \begin{proposition}
Suppose $M$ is a smooth manifold with or without boundary, $S \subseteq M$ is an immersed $k$-submanifold, $\iota: S \xhookrightarrow{} M$ is the inclusion map, and $U$ is an open subset of $\bR^k$. A map $X: U \rightarrow M$ is a \textbf{smooth local parametrization} of $S$ \textbf{if and only if} there is a smooth coordinate chart $(V, \varphi)$ for $S$ such that \underline{$X = \iota \circ \varphi^{-1}$}. Therefore, every point of $S$ is in the image of some local parametrization.
\end{proposition}
\end{itemize}
\subsection{Restricting Maps to Submanifolds}
\subsubsection{Restricting Domains and Codomains}
\begin{itemize}
\item \begin{remark}
Given a smooth map $F: M \rightarrow N$, it is important to know whether $F$ is still smooth when its \emph{\textbf{domain}} or \emph{\textbf{codomain}} is restricted to a submanifold. See Fig. \ref{fig: restricting_domain}.
\begin{enumerate}
\item \emph{\textbf{Restricting Domains}}: The answer is \emph{\textbf{yes}}.
\begin{theorem} (\textbf{Restricting the Domain of a Smooth Map}). \citep{lee2003introduction}\\
If $M$ and $N$ are smooth manifolds with or without boundary, $F: M \rightarrow N$ is a smooth map, and $S \subseteq M$ is an \textbf{immersed or embedded submanifold}, then $F|_{S}: S \rightarrow N$ is smooth.
\end{theorem}

\item  \emph{\textbf{Restricting Codomains}}: We provides \emph{\textbf{sufficient conditions}}: \emph{the \textbf{image} of the smooth map should be \textbf{contained in the submanifold}}.
\begin{itemize}
\item \emph{\textbf{Immersed Submanifolds}}:
\begin{theorem} (\textbf{Restricting the Codomain of a Smooth Map}). \citep{lee2003introduction}\\ 
Suppose $M$ is a smooth manifold (without boundary), $S \subseteq M$ is an immersed submanifold, and $F: N \rightarrow M$ is a smooth map whose \textbf{image is contained in $S$}. If $F$ is \textbf{continuous} as a map from $N$ to $S$, then $F: N \rightarrow S$ is smooth.
\end{theorem}
\item \emph{\textbf{Embedded Submanifolds}}:
\begin{corollary} (\textbf{Embedded Case}). \\
Let $M$ be a smooth manifold and $S \subseteq M$ be an embedded submanifold. Then every smooth map $F: N \rightarrow M$ whose \textbf{image is
contained in $S$} is also \textbf{smooth} as a map from $N$ to $S$.
\end{corollary}
\end{itemize}

\item We can generalize the corollary above as the definition of weakly embedded submanifold.
\begin{definition}
If $M$ is a smooth manifold and $S \subseteq M$ is an immersed submanifold, then $S$ is said to be \underline{\emph{\textbf{weakly embedded}}} in $M$ if every smooth map $F: N \rightarrow M$ \emph{\textbf{whose image lies in $S$}} is \textbf{\emph{smooth}} as a map from $N$ to $S$. (\emph{Weakly embedded submanifolds} are called \emph{\textbf{initial submanifolds}} by some authors.) 
\end{definition}
\end{enumerate}
\end{remark}
\end{itemize}

\subsubsection{Uniqueness of Smooth Structures on Submanifolds}
\begin{itemize}
\item \begin{theorem}
Suppose $M$ is a smooth manifold and $S \subseteq M$ is an \textbf{embedded submanifold}. The subspace topology on $S$ and the smooth structure from the local $k$-slice condition are \textbf{the only topology and smooth structure} with respect to which $S$ is an embedded or immersed submanifold.
\end{theorem}

\item \begin{remark}
Thanks to this uniqueness result, we now know that a \emph{subset} $S \subseteq M$ is an \emph{embedded submanifold} \emph{\textbf{if and only if}} it satisfies \emph{the local slice condition}, and if so, its topology and smooth structure are \emph{\textbf{uniquely determined}}.

Because the local slice condition is \emph{\textbf{a local condition}}, if every point $p \in S$ has a neighborhood \underline{$U \subseteq M$} such that \underline{$U \cap S$} is an embedded $k$-submanifold \underline{\emph{of $U$}}, then $S$ is an embedded $k$-submanifold of $M$.
\end{remark}

\item \begin{theorem}
Suppose $M$ is a smooth manifold and $S \subseteq M$ is an \textbf{immersed submanifold}. For the \textbf{given topology} on $S$, there is \textbf{only one smooth structure} making $S$ into an immersed submanifold.
\end{theorem}

\item \begin{theorem} If $M$ is a smooth manifold and $S \subseteq M$ is a \textbf{weakly embedded submanifold}, then $S$ has \textbf{only one topology and smooth structure} with respect to which it is an immersed submanifold.
\end{theorem}
\end{itemize}

\subsubsection{Extending Functions from Submanifolds}
\begin{itemize}
\item \begin{remark}
Complementary to the restriction problem is the problem of extending smooth functions from a submanifold to the ambient manifold. Here we say $f \in \cC^{\infty}(S)$ for submanifold $S\subseteq M$, when $f$ is considered as a function on the manifold $S$.
\end{remark}

\item \begin{lemma} (\textbf{Extension Lemma for Functions on Submanifolds}). \\
Suppose $M$ is a smooth manifold, $S\subseteq M$ is a smooth submanifold, and $f \in \cC^{\infty}(S)$.
\begin{enumerate}
\item  If $S$ is \textbf{embedded}, then there exist a \textbf{neighborhood} $U$ of $S$ in $M$ and a smooth
function $\widetilde{f} \in  \cC^{\infty}(U)$ such that $\widetilde{f}|_{S} = f$.
\item If $S$ is \textbf{properly embedded}, then the neighborhood U above can be taken to be \textbf{all} of $M$. 
\end{enumerate}
\end{lemma}
\end{itemize}

\subsection{The Tangent Space to a Submanifold}
\begin{itemize}
\item \begin{remark}
The \emph{\textbf{tangent space to a smooth submanifold}} of an abstract smooth manifold can be viewed as \emph{a \underline{\textbf{subspace}}} of \emph{\textbf{the tangent space to the ambient manifold}}, once we make appropriate identifications. The following proof is based on the \underline{\emph{\textbf{differential}}} of \underline{\emph{\textbf{the inclusion map}}} as a smooth immersion.
\end{remark}
 \begin{proof} 
Let $M$ be a smooth manifold with or without boundary, and let $S\subseteq M$ be an immersed or embedded submanifold. Since the inclusion map $\iota: S \xhookrightarrow{} M$ is a \emph{\textbf{smooth immersion}}, at each point $p \in S$ we have an \emph{injective linear map} $d\iota_{p}: T_{p}S \rightarrow T_{p}M$.
In terms of \emph{\textbf{derivations}}, this injection works in the following way: for any vector $v \in T_{p}S$, the image vector $\widetilde{v} = d\iota_{p}(v) \in T_{p}M$ acts on smooth functions on $M$ by
\begin{align*}
\widetilde{v}f = d\iota_{p}(v)f = v\paren{f \circ \iota} = v\paren{f|_{S}}.
\end{align*}
We adopt the convention of \emph{\textbf{identifying}} $T_{p}S$ with \emph{\textbf{its image} under this map}, thereby
thinking of $T_{p}S$ as a certain linear subspace of $T_{p}M$. This identification makes sense regardless of whether $S$ is \emph{embedded or immersed}. \qed
\end{proof}

\item There are several \emph{alternative} ways to \emph{characterize} the tangent space of a submanifold 
\begin{enumerate}
\item \underline{\emph{\textbf{Smooth curve} on \textbf{submanifold}}}. 
\begin{proposition}
Suppose $M$ is a smooth manifold with or without boundary, $S\subseteq M$ is an immersed or embedded submanifold, and $p \in S$. A vector $v \in T_{p}M$ is
in $T_{p}S$ if and only if there is a smooth curve  $\gamma: J \rightarrow M$ whose \textbf{image is contained in $S$}, and which is also \textbf{smooth} as a map into $S$, such that $0 \in J$, $\gamma(0) = p$, and $\gamma'(0) = v$.
\end{proposition}

\item \emph{\textbf{Derivations} on \underline{functions whose \textbf{restriction on submanifold are constant zero}}}.
\begin{proposition}
Suppose $M$ is a smooth manifold, $S\subseteq M$ is an embedded submanifold, and $p \in S$. As a subspace of $T_{p}M$, the tangent space $T_{p}S$ is characterized
by
\begin{align*}
T_{p}S &= \set{v \in T_{p}M: vf = 0\text{ \textbf{whenever} }f \in \cC^{\infty}(M)\text{ \textbf{and} }f|_{S} = 0}.
\end{align*}
\end{proposition}

\item \underline{\emph{\textbf{Kernel subspace} of \textbf{differential of local defining map}}}.
\begin{proposition}
Suppose $M$ is a smooth manifold and $S \subseteq M$ is an embedded submanifold. If $\Phi: U \rightarrow N$ is any \textbf{local defining map} for $S$, then $T_{p}S = 
\text{\textbf{Ker} }(d\Phi_{p}): T_{p}M \rightarrow T_{\Phi(p)}N$ for each $p \in S \cap U$.
\end{proposition} Note that $S\cap U = (\Phi \circ \iota)^{-1}(c)$ is the level set of $\Phi \circ \iota$ thus it is constant for $\Phi \circ \iota$. So $d\Phi_p \circ d\iota_p = 0$.

\begin{corollary}
Suppose $S \subseteq M$ is a \textbf{level set} of a \textbf{smooth submersion} $\Phi = (\Phi^1,\ldots,\Phi^k): M \rightarrow \bR^k$. A vector $v \in T_{p}M$ is tangent to $S$ if and only if $v\Phi^1 = \ldots = v\Phi^k = 0$.
\end{corollary}
\end{enumerate}

\item \begin{remark}
If $M$ is a smooth manifold \emph{\textbf{with boundary}} and $p \in \partial M$, the vectors in $T_{p}M$ can be separated into \emph{\textbf{three classes}}:  
\begin{enumerate}
\item those \emph{\textbf{tangent to the boundary}};
\item those pointing \emph{\textbf{inward}}; 
\item those pointing \emph{\textbf{outward}}.
\end{enumerate}

 \begin{definition}
If $p \in \partial M$, a vector $v \in T_{p}M \setminus T_{p}\partial M$ is said to be \emph{\textbf{inward-pointing}} if for some $\epsilon > 0$ there exists a smooth curve $\gamma: [0, \epsilon)\rightarrow M$ such that $\gamma(0) = p$ and $\gamma'(0) = v$, and it is \emph{\textbf{outward-pointing}} if there exists such a curve whose domain is $(-\epsilon, 0]$.
\end{definition}

 \begin{proposition} (\textbf{Characterization of Tangent Vectors on Boundary using Component Functions})\\
Suppose $M$ is a smooth $n$-dimensional manifold with boundary, $p \in \partial M$, and $(x^i)$ are any smooth boundary coordinates defined on a neighborhood of $p$. The \textbf{inward-pointing vectors} in $T_{p}M$ are precisely those with \textbf{positive $x^n$-component}, the \textbf{outward-pointing} ones are those with \textbf{negative $x^n$-component}, and the ones \textbf{tangent to $\partial M$} are those with \textbf{zero $x^n$-component}. Thus, $T_{p}M$ is the \textbf{disjoint union} of $T_{p}\partial M$, the set of inward-pointing vectors, and the set of outwardpointing vectors, and $v \in T_{p}M$ is inward-pointing if and only if $-v$ is outward-pointing.
\end{proposition}
\end{remark}
\end{itemize}

\subsection{Vector Fields and Submanifolds}
\begin{itemize}
\item \begin{remark}
If $S \subseteq M$ is an immersed or embedded submanifold (with or without boundary), \emph{a vector field $X$} on $M$ does \emph{\textbf{not necessarily} restrict to a vector field on $S$}, because $X_p$ may not lie in the \emph{subspace} $T_{p}S \subseteq T_{p}M$ at a point $p \in S$.
\end{remark}


\item \begin{definition}
Given a point $p \in S$, a vector field $X$ on $M$ is said to \underline{\emph{\textbf{be tangent to}}} $S$ at $p$ if $X_p \in T_{p}S \subseteq T_{p}M$. It \emph{i\textbf{s tangent to}} $S$ if it is tangent to $S$ at every point of $S$.
\end{definition}

\item \begin{proposition}
Let $M$ be a smooth manifold, $S\subseteq M$ be an \textbf{embedded submanifold} with or without boundary, and $X$ be a smooth vector field on $M$. Then $X$ is
\textbf{tangent} to $S$ if and only if $(Xf)|_{S} = 0$ for \textbf{every} \underline{$f \in \cC^{\infty}(M)$}  \textbf{such that} \underline{$f|_{S}\equiv 0$}.
\end{proposition}

\item \begin{remark}
Suppose $S\subseteq M$ is an \textbf{\emph{immersed submanifold}} with or without boundary, and $Y$ is a smooth vector field on $M$. If there is a vector field $X \in \mathfrak{X}(S)$ that is \underline{\emph{\textbf{$\iota$-related to $Y$}}}, where $\iota: S \xhookrightarrow{} M$ is the inclusion map, then clearly \emph{\textbf{$Y$ is tangent to $S$}}, because $Y_p = d\iota_{p}(X_p)$ is in the image of $d\iota_p$ for each $p \in S$. 

The converse is true as well.
\begin{proposition} (\textbf{Restricting Vector Fields to Submanifolds}). \citep{lee2003introduction}\\
Let $M$ be a  smooth manifold, let $S\subseteq M$ be an \textbf{immersed submanifold} with or without boundary, and let $\iota: S \xhookrightarrow{} M$ denote the inclusion map. If $Y \in \mathfrak{X}(M)$ is \textbf{tangent to} $S$, then there is a \textbf{unique smooth vector field} on $S$, denoted by $Y|_{S}$ , that is \textbf{$\iota$-related to $Y$}.
\end{proposition}
\end{remark}
\end{itemize}

\subsection{Restricting Covector Fields to Submanifolds}
\begin{itemize}
\item \begin{remark}
Compare to restricting vector fields to submanifolds, the restriction of covector fields to submanifolds is much simpler.
\end{remark}

\item \begin{remark} (\emph{\textbf{The Pullback of Covector Field by the Inclusion Map is a Covector Field on Submanifold}})\\
Suppose $M$ is a smooth manifold with or without boundary, $S \subseteq M$ is an \emph{\textbf{immersed submanifold}} with or without boundary, and $\iota: S \xhookrightarrow{} M$ is \emph{the inclusion map}. If $\omega$ is any smooth covector field on $M$, \emph{\textbf{the pullback by $\iota$ yields a smooth covector
field $\iota^{*}\omega$ on $S$}}. 

To see what this means, let $v \in T_{p}S$ be arbitrary, and compute
\begin{align*}
(\iota^{*}\omega)_p(v) = \omega_p\paren{d\iota_p(v)} = \omega_p(v).
\end{align*} since $d\iota_p: T_{p}S \rightarrow T_{p}M$ is just the inclusion map, under our usual identification of $T_{p}S$ with a subspace of $T_{p}M$. Thus, $\iota^{*}\omega$ is just the restriction of $\omega$ to vectors tangent to $S$. For this reason, $\iota^{*}\omega$ is often called \underline{\emph{\textbf{the restriction of $\omega$ to $S$}}}. 

Be warned, however, that $\iota^{*}\omega$ might equal \textbf{\emph{zero}} at a given point of $S$, even though \emph{\textbf{considered as a covector field on $M$}}, \emph{\textbf{$\omega$ might not vanish there}}. 
\end{remark}

\item \begin{example} ($\omega \neq 0$ but $\iota^{*}\omega = 0$)\\
Let $\omega = dy$ on $\bR^2$, and let $S$ be the $x$-axis, considered as an embedded submanifold of $\bR^2$. As a covector field on $\bR^2$, $\omega$ is \textbf{\emph{nonzero}} everywhere, because one of its component functions is \emph{\textbf{always} $1$}. However, the restriction $\iota^{*}\omega$ is
\emph{\textbf{identically zero}}, because $y$ vanishes identically on $S$:
\begin{align*}
\iota^{*}\omega = \iota^{*} dy = d(y \circ \iota) = 0.
\end{align*}
\end{example}

\item \begin{remark}
One usually says that ``\emph{\textbf{$\omega$ vanishes along $S$}}" or ``\emph{\textbf{$\omega$ vanishes at points of $S$}}"
if $\omega_p = 0$ for every point $p \in S$. 

The \emph{\textbf{weaker condition}} that $\iota^{*}\omega = 0$ is expressed by saying that ``\emph{\textbf{\underline{the restriction of} $\omega$ to $S$ vanishes}}", or ``\emph{\textbf{\underline{the pullback of $\omega$} to $S$ vanishes}}".
\end{remark}

\end{itemize}

\newpage
\bibliographystyle{plainnat}
\bibliography{book_reference.bib}
\end{document}
