\documentclass[11pt]{article}
\usepackage[scaled=0.92]{helvet}
\usepackage{geometry}
\geometry{letterpaper,tmargin=1in,bmargin=1in,lmargin=1in,rmargin=1in}
\usepackage[parfill]{parskip} % Activate to begin paragraphs with an empty line rather than an indent %\usepackage{graphicx}
\usepackage{amsmath,amssymb, mathrsfs,  mathtools, dsfont}
\usepackage{tabularx}
\usepackage{tikz-cd}
\usepackage[font=footnotesize,labelfont=bf]{caption}
\usepackage{graphicx}
\usepackage{xcolor}
%\usepackage[linkbordercolor ={1 1 1} ]{hyperref}
%\usepackage[sf]{titlesec}
\usepackage{natbib}
\usepackage{../../Tianpei_Report}

%\usepackage{appendix}
%\usepackage{algorithm}
%\usepackage{algorithmic}

%\renewcommand{\algorithmicrequire}{\textbf{Input:}}
%\renewcommand{\algorithmicensure}{\textbf{Output:}}



\begin{document}
\title{Lecture 6: Sard's Theorem}
\author{ Tianpei Xie}
\date{Oct. 17th., 2022}
\maketitle
\tableofcontents
\newpage
\section{Sard's Theorem}
\subsection{Sets of Measure Zero}
\begin{itemize}
\item \begin{definition} 
A set $A \subseteq \bR^n$ to have \emph{\textbf{measure zero}} if for any $\delta > 0$,  $A$ can be \emph{\textbf{covered}} by \emph{\textbf{a countable collection of open rectangles}}, the \emph{\textbf{sum}} of whose \emph{\textbf{volumes}} is less than $\delta$ (Fig.).
\end{definition}

\item \begin{lemma}\label{lemma: compact_intersection_measure_zero}
Suppose $A \subseteq \bR^n$ is a \textbf{compact} subset whose intersection with $\set{c} \times \bR^{n-1}$ has  $(n-1)$-dimensional measure zero for \textbf{every} $c \in \bR$. Then $A$ has $n$-dimensional measure zero.
\end{lemma}

\item \begin{proposition}
Suppose $A$ is an open or closed subset of $\bR^{n-1}$ or $\bH^{n-1}$, and $f: A \rightarrow \bR$ is a \textbf{continuous} function. Then the \textbf{graph} of $f$ has measure zero in $\bR^n$.
\end{proposition}

\item \begin{corollary}
Every \textbf{proper affine subspace} of $\bR^n$ has measure zero in $\bR^n$.
\end{corollary}

\item \begin{proposition}
Suppose $A \subseteq \bR^n$ has \textbf{measure zero} and $F: A \rightarrow \bR^n$ is a \textbf{smooth} map. Then $F(A)$ has measure zero.
\end{proposition}

\item \begin{definition}
Let $M$ be a smooth $n$-manifold with or without boundary. A subset $A \subseteq M$ has \emph{\textbf{measure zero}} in $M$ if for \emph{\textbf{every smooth chart}} $(U, \varphi)$ for $M$, the \emph{subset} $\varphi(A \cap U ) \subseteq  \bR^n$ has \emph{\textbf{$n$-dimensional measure zero}}. 
\end{definition}

\item The following lemma shows that we need only check this condition for \emph{a single collection of smooth charts} whose \emph{domains cover} $A$. 
\begin{lemma}
Let $M$ be a smooth $n$-manifold with or without boundary and $A \subseteq M$. Suppose that for \textbf{some collection} $\set{(U_{\alpha}, \varphi_{\alpha})}$ of smooth charts whose domains \textbf{cover} $A$, $\varphi_{\alpha}(A \cap U_{\alpha})$ has \textbf{measure zero} in $\bR^n$ for each $\alpha$. Then $A$ has measure zero in $M$.
\end{lemma}

\item \begin{proposition}
Suppose $M$ is a smooth manifold with or without boundary and $A \subseteq M$ has \textbf{measure zero} in $M$. Then $M \setminus A$ is \textbf{dense} in $M$.
\end{proposition}

\item \begin{theorem}
Suppose $M$ and $N$ are smooth $n$-manifolds with or without boundary, $F: M \rightarrow N$ is a \textbf{smooth} map, and $A \subseteq M$ is a subset of \textbf{measure zero}. Then $F(A)$ has measure zero in $N$.
\end{theorem}
\end{itemize}
\subsection{Proof of Sard's Theorem}
\begin{itemize}
\item \emph{The Sard's theorem} underlies all of our results about embedding, approximation, and transversality.

\item \begin{theorem}(\textbf{Sard's Theorem}). \\
Suppose $M$ and $N$ are smooth manifolds with or without boundary and $F: M \rightarrow N$ is a \textbf{smooth map}. Then the set of \textbf{critical
values} of $F$ has \textbf{measure zero} in $N$.
\end{theorem}
\begin{proof}

\end{proof}
\end{itemize}

\subsection{Corollaries}
\begin{itemize}
\item \begin{corollary}
Suppose $M$ and $N$ are smooth manifolds with or without boundary, and $F: M \rightarrow N$ is a \textbf{smooth map}. If $\text{dim}\,M < \text{dim}\,N$ , then $F(M)$ has measure
zero in $N$.
\end{corollary}

\item \begin{remark}
It is important to be aware tha Corollary above is false if $F$ is merely assumed to be \textbf{continuous}. For example, there is a continuous map $F: [0,1] \rightarrow \bR^2$ whose image is the entire unit square $[0,1] \times [0, 1]$. (Such a map is called \emph{\textbf{a space-filling curve}}).
\end{remark}

\item \begin{corollary}
Suppose $M$ is a smooth manifold with or without boundary, and $S \subseteq M$ is an \textbf{immersed submanifold} with or without boundary. If $\text{dim}\,S < \text{dim}\,M$,  then $S$ has measure zero in $M$.
\end{corollary}
\end{itemize}

\section{The Whitney Embedding Theorem}
\begin{itemize}
\item Our first application of \emph{Sard's theorem} is to show that \emph{\textbf{every smooth manifold can be embedded into a Euclidean space}}. In fact, we will show that every smooth $n$-manifold with or without boundary is \emph{diffeomorphic} to a \emph{properly embedded submanifold} (with or without boundary) of $\bR^{2n+1}$.

\item \begin{theorem} (\textbf{Whitney Embedding Theorem}).\\
Every smooth $n$-manifold with or without boundary admits a proper \textbf{smooth embedding} into $\bR^{2n+1}$.
\end{theorem}


\item \begin{theorem} (\textbf{Whitney Immersion Theorem}). \\
Every smooth $n$-manifold with or without boundary admits a \textbf{smooth immersion} into $\bR^{2n}$.
\end{theorem}

\item \begin{theorem}  (\textbf{Strong Whitney Embedding Theorem}). \\
If $n > 0$, every smooth $n$-manifold admits a \textbf{smooth embedding} into $\bR^{2n}$.
\end{theorem}

\item \begin{theorem} (\textbf{Strong Whitney Immersion Theorem}). \\
If $n > 1$, every smooth $n$-manifold admits a \textbf{smooth immersion} into $\bR^{2n-1}$.
\end{theorem} Because of these results, the first two theorems are sometimes called the \emph{easy} or \emph{weak Whitney embedding and immersion theorems}.
\end{itemize}
\section{The Whitney Approximation Theorems}
\subsection{Whitney Approximation Theorem for Functions}
\begin{itemize}
\item We begin with a theorem about \emph{\textbf{smoothly approximating functions} into Euclidean spaces}. Our first theorem shows, in particular, that \emph{any continuous function} from a smooth manifold $M$ into $\bR^k$ can be \emph{uniformly approximated} by a smooth function.

\item \begin{theorem} (\textbf{Whitney Approximation Theorem for Functions}). \\
Suppose $M$ is a smooth manifold with or without boundary, and $F: M \rightarrow \bR^k$ is a \textbf{continuous} function. Given any \textbf{positive continuous} function $\delta: M \rightarrow \bR$, there exists a \textbf{smooth} function $\widetilde{F}: M \rightarrow \bR^k$ that is $\delta$-close to $F$. If $F$ is smooth on a closed subset $A \subseteq M$, then $\widetilde{F}$ can be chosen to be equal to $F$ on $A$.
\end{theorem}
\end{itemize}
\subsection{Tubular Neighborhoods}

\subsection{Smooth Approximation of Maps Between Manifolds}
\begin{itemize}
\item Now we can extend \emph{the Whitney approximation theorem} to \emph{maps between manifolds}. This extension will have important applications to \emph{line integrals}.

\item \begin{theorem} (\textbf{Whitney Approximation Theorem}). \\
Suppose $N$ is a smooth manifold with or without boundary, $M$ is a smooth manifold (without boundary), and $F: N \rightarrow M$ is a \textbf{continuous} map. Then $F$ is \textbf{homotopic} to a smooth map. If $F$ is already smooth on a closed subset $A \subseteq N$, then the \textbf{homotopy} can be taken to be relative to $A$.
\end{theorem}
\end{itemize}

\section{Transversality}
\begin{itemize}
\item As our final application of \emph{Sard's theorem}, we show how \emph{submanifolds} can be \emph{perturbed} so that \emph{\textbf{they intersect "nicely."}} To explain what this means, we introduce the concept of \emph{\textbf{transversality}}.

\item \begin{definition}
Suppose $M$ is a smooth manifold. Two \emph{embedded submanifolds} $S, S' \subseteq M$ are said to \underline{\emph{\textbf{intersect transversely}}} if for each $p \in S \cap S'$, the tangent spaces \underline{$T_{p}S$} and \underline{$T_{p}S'$} \emph{together} \underline{\emph{\textbf{span}} $T_{p}M$} (where we consider $T_{p}S$ and $T_{p}S'$ as subspaces of $T_{p}M$).
\end{definition}

\item \begin{definition}
If $F: N \rightarrow M$ is a smooth map and $S \subseteq M$ is an \emph{embedded submanifold}, we say that \underline{$F$ \emph{\textbf{is transverse to $S$}}} if for every $x \in F^{-1}(S)$, the spaces \underline{$T_{F(x)}S$} and \underline{$dF_{x}\paren{T_{x}N}$} \emph{together} \underline{\emph{\textbf{span}} $T_{F(x)}M$}.
\end{definition}
\end{itemize}




\newpage
\bibliographystyle{plainnat}
\bibliography{book_reference.bib}
\end{document}