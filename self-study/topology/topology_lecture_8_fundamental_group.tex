\documentclass[11pt]{article}
\usepackage[scaled=0.92]{helvet}
\usepackage{geometry}
\geometry{letterpaper,tmargin=1in,bmargin=1in,lmargin=1in,rmargin=1in}
\usepackage[parfill]{parskip} % Activate to begin paragraphs with an empty line rather than an indent %\usepackage{graphicx}
\usepackage{amsmath,amssymb, mathrsfs,  mathtools, dsfont}
\usepackage{tabularx}
\usepackage{tikz-cd}
\usepackage[font=footnotesize,labelfont=bf]{caption}
\usepackage{graphicx}
\usepackage{xcolor}
%\usepackage[linkbordercolor ={1 1 1} ]{hyperref}
%\usepackage[sf]{titlesec}
\usepackage{natbib}
\usepackage{../../Tianpei_Report}

%\usepackage{appendix}
%\usepackage{algorithm}
%\usepackage{algorithmic}

%\renewcommand{\algorithmicrequire}{\textbf{Input:}}
%\renewcommand{\algorithmicensure}{\textbf{Output:}}



\begin{document}
\title{Lecture 5: The Fundamental Group}
\author{ Tianpei Xie}
\date{Nov. 7th., 2022}
\maketitle
\tableofcontents
\newpage
\section{Homotopy of Paths}
\section{The Fundamental Group}
\subsection{Definitions}
\subsection{The Fundamental Group of the Circle}
\subsection{The Fundamental Group of $\bS^n$}
\subsection{Fundamental Groups of Some Surfaces}

\section{Covering Spaces}
\section{Retractions and Fixed Points}
\subsection{Definitions}
\subsection{Deformation Retracts and Homotopy Type}
\newpage
\bibliographystyle{plainnat}
\bibliography{book_reference.bib}
\end{document}