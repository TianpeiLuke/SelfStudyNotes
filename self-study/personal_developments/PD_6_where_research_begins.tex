\documentclass[11pt]{article}
\usepackage[scaled=0.92]{helvet}
\usepackage{geometry}
\geometry{letterpaper,tmargin=1in,bmargin=1in,lmargin=1in,rmargin=1in}
\usepackage[parfill]{parskip} % Activate to begin paragraphs with an empty line rather than an indent %\usepackage{graphicx}
\usepackage{amsmath,amssymb, mathrsfs,  mathtools, dsfont}
\usepackage{tabularx}
\usepackage{tikz-cd}
\usepackage[font=footnotesize,labelfont=bf]{caption}
\usepackage{graphicx}
\usepackage{xcolor}
%\usepackage[linkbordercolor ={1 1 1} ]{hyperref}
%\usepackage[sf]{titlesec}
\usepackage{natbib}
%\usepackage{tikz-cd}

\usepackage{../../Tianpei_Report}

%\usepackage{appendix}
%\usepackage{algorithm}
%\usepackage{algorithmic}

%\renewcommand{\algorithmicrequire}{\textbf{Input:}}
%\renewcommand{\algorithmicensure}{\textbf{Output:}}


\begin{document}
\title{Book Reading Summary: Where Research Begins: choosing a research topic that matters to you (and the world)}
\author{ Tianpei Xie}
\date{Dec. 10th., 2023}
\maketitle
\tableofcontents
\newpage
\section{Summary}
\subsection{Research Phases: Inside-Out Self-Centered Approach}
\emph{Where Research Begins: choosing a research topic that matters to you (and the world)}, Thomas S. Mullaney and Christopher Rea, The University of Chicago Press.

\begin{itemize}
\item In this book, the authors proposed the \emph{\textbf{Self-Centered Research}} process, an inside-out self-motivated research \emph{\textbf{practice}}, \emph{\textbf{ethic}} and \emph{\textbf{a state of mind}}. It discuss, at \emph{the beginning phase of research}, how the research problem is \emph{identified}, \emph{refined} and \emph{evaluated}, through \emph{\textbf{inward-focus}} \emph{first} and \emph{\textbf{outward-focus}} later methodology. 

\item This book focus on \emph{\textbf{uncovering the internal curiosity}} of researcher before the research process started, and in the process to gain \emph{self-awareness}, \emph{self-trust} and \emph{affirmation on the direction of research} on their own. It stress the importance of ``\emph{\textbf{finding your center}}" -- the matters that really motivates \emph{you} to start the research.

\item By using ``\emph{\textbf{introversion first, extroversion second}}" approach, the authors help the early researchers to \emph{\textbf{avoid}} being \emph{\textbf{distracted}} by perspectives, ideas, suggestions from \emph{\textbf{others} (authorities, mentors, colleagues)} in research communities, as well as to \emph{avoid} \emph{\textbf{the judgemental thoughts}} from \emph{\textbf{their own mind}}, which could hurt the inner exploration processing at early stage of research. 

\item The key messages to early researchers:
\begin{enumerate}
\item Be  \emph{\textbf{vulnerable}}
\item \emph{\textbf{Listening to yourself}}
\item \emph{\textbf{Writing things down}}
\end{enumerate}

\item This books covers several phases of early research:
\begin{enumerate}
\item \emph{\textbf{Question Brainstorming with Self-Observation}}: at this earliest phase, the task is to generate questions relevant to you, \emph{\textbf{from the best of your knowledge}}. 

\begin{itemize}
\item This part covers the traditional method of ``\emph{\textbf{from topics to questions}}". But in this book, the emphasis is on self-observing on your own reactions to specific matters in the topic and your own perspective and related questions. 

\item While scanning through entities under a topic, ask yourself  ``\emph{Why does that interest you?}" or ``\emph{Why i do not care about it?}" \emph{\textbf{Take notes}} on questions generated and \emph{flag} those with unconsciously effect (interest/bordom) on you. 

\item The point is being \emph{\textbf{honest}}, \emph{\textbf{self-observing}} but not \emph{\textbf{not be judgemental}}. 


\item Your questions are for yourself. They are meant to be \emph{\textbf{unpolished}}. Use normal worlds, not jargons. You are not meant to impress anyone. 

\item \emph{Your questions} need to be \emph{\textbf{specific and concrete}}, \emph{not vague}, not \emph{abstracting}. 
\end{itemize}

\item  \emph{\textbf{Question Refinement}}: 
\begin{itemize}
\item Rewrite your questions, making it clear, precise, jargon-free.

\item Avoid biases, prejudice; avoid too much assumptions

\item \emph{\textbf{Refine keywords} and \textbf{Search} for \textbf{other primary sources}} that mentioned in \emph{your primary sources}. These related primary sources can help you ask better questions. They also contains \emph{\textbf{new keywords}}, which helps to find more and better primary sources.

Keep track of all keywords and searches. 

\item Make the \emph{\textbf{Assumption}} visable. \emph{\textbf{Categorize and Sort}} them according to a) the assumption that you want to work with; b) the assumption that you want to discard; c) the assumption you are uncertain about for now. During this categorization process, write down your thoughts. Modify weak, unfounded, prejudical assumptions.
\end{itemize}

\item \emph{\textbf{Problem Identification}}: This is the part where we move ``\emph{\textbf{from questions to problems}}".
\begin{itemize}
\item In this part, we found \emph{\textbf{internal connections}} between questions. 

\item  \emph{\textbf{Generalize}} from previous questions to higher level.

\item Identify \emph{\textbf{the shared concerns}} among these questions.
\end{itemize}

\item \emph{\textbf{Primary Source Engagement}}: After identifying your problems, we need to revisit the primary sources we have and to identify, filter and refine your list of primary sources. You need to consider how to answer your questions with the primary sources you have. 
\begin{itemize}
\item You need to distinguish primary and secondary sources.
\item You need to deep dive and discover the full potential of these primary sources.  This helps you to look beyond obvious questions and to arrive at something original.
\item Envision imaginative primary sources that best answer your question. Search for it.
\item You need to pinpoint these sources to your problem; determine if they are relevant, reduandent, reliable. 
\end{itemize}

\item \emph{\textbf{Argument Construction}}: You need to make proper argument from these sources. 
\begin{itemize}
\item Find the dots. 
\item Figure out which dots belong to your picture. See what is there? See what is missing? 
\item Figure out which dots are not dots at all. Not all materials can be used as sources. 
\item Determine when you have enough.
\item Connect the dots. 
\end{itemize}

\item  \emph{\textbf{Research Project Design}}: Plan the project by answering following questions
\begin{itemize}
\item What outcome do you want to achieve? 
\item What primary sources do you possess? 
\item What resources (time, computational, people, other responsibilities) can you utilize? What constraints? 
\item What is the deadline? 
\item What timeline are you planning?
\item Understand my personality
\end{itemize} Finally, write it down as a research proposal for yourself.

\item  \emph{\textbf{Problem Collective Identification}}:

\item  \emph{\textbf{Rewriting for Problem Collective}}:

\item  \emph{\textbf{Field Grouping via Problem Collectives}}:

\item  \emph{\textbf{Rewriting for Field}}:

\item  \emph{\textbf{Assembling into Draft}}:
\end{enumerate}
\end{itemize}

\subsection{To-Do List}
\subsubsection{Brainstorming Questions}
\begin{itemize}
\item \begin{exercise}[Search Yourself]
\end{exercise}

\item \begin{exercise}[Let Boredom Be Your Guide]
\end{exercise}

\item \begin{exercise}[Go Small or Go Home]
\end{exercise}
\end{itemize}
\subsubsection{From Questions to Problems}
\begin{itemize}
\item \begin{exercise}[Run a Diagnostic Test on Your Questions]
\end{exercise}

\item \begin{exercise}[Use Primary Sources to Educate Your Questions]
\end{exercise}

\item \begin{exercise}[Make Your Assumptions Visible]
\end{exercise}

\item \begin{exercise}[Identify the Problem That Connects Your Questions]
\end{exercise}
\end{itemize}
\subsubsection{From Problems to Research Projects}
\begin{itemize}
\item  \begin{exercise}[Treat Your Primary Source Like a Cereal Box]
\end{exercise}

\item \begin{exercise}[Envision Your Primary Sources]
\end{exercise} 

\item \begin{exercise}[Connect the Dots Using Your Sources]
\end{exercise}

\item \begin{exercise}[Decision Matrix]
\end{exercise}

\item \begin{exercise}[Prepare a Formal Research Proposal]
\end{exercise}
\end{itemize}
\subsubsection{Identifying Your Problem Collective}
\begin{itemize}
\item  \begin{exercise}[Change One Variable]
\end{exercise}

\item  \begin{exercise}[Before and After]
\end{exercise}

\item  \begin{exercise}[Map Out Your Collective (Secondary Source Search)]
\end{exercise}
\end{itemize}
\subsubsection{Rewriting for Your Collective}
\begin{itemize}
\item \begin{exercise}[Find and Replace All ``Insider Language"]
\end{exercise}
\end{itemize}
\subsubsection{Organizing Your Field into Problem Collectives}
\begin{itemize}
\item \begin{exercise}[Start Your Own ``What’s Your Problem?" Bookstore]
\end{exercise}

\item  \begin{exercise}[Change Their Variables]
\end{exercise} 
\end{itemize}
\subsubsection{Rewriting for Your Field}
\begin{itemize}
\item \begin{exercise}[Rewrite for Your Field]
\end{exercise} 
\end{itemize}
\subsubsection{Formal Drafting and Revision}
\begin{itemize}
\item \begin{exercise}[Create ``Draft 0"]
\end{exercise}

\item \begin{exercise}[Move from 0 to 1]
\end{exercise}
\end{itemize}
\subsubsection{What is Next?}
\begin{itemize}
\item \begin{exercise}[Find a New Problem and Start a New Project]
\end{exercise}

\item \begin{exercise}[Help Someone Else]
\end{exercise}
\end{itemize}
\subsection{Sounding Board and Research Network}
\subsubsection{Building Research Network}
\begin{itemize}
\item  \begin{exercise}[Start Building Your Research Network]
\end{exercise}
\end{itemize}
\subsubsection{Identifying Primary Sources}
\begin{itemize}
\item  \begin{exercise}[Get Leads on Primary Sources]
\end{exercise}
\end{itemize}
\subsubsection{Decision Consulting and Proposal Sharing}
\begin{itemize}
\item \begin{exercise}[Is Your Decision Matrix Complete?]
\end{exercise}
\item \begin{exercise}[Share Your Proposal with a Trusted Mentor (Who Understands How Preliminary This Is)]
\end{exercise}
\end{itemize}
\subsubsection{Evaluation of Proposal from Outsider}
\begin{itemize}
\item \begin{exercise}[Does the Lay Version of My Proposal Make Sense?]
\end{exercise}
\end{itemize}
\subsubsection{Finding a Sounding Board in Your Field}
\begin{itemize}
\item \begin{exercise}[Find a Sounding Board in Your Field]
\end{exercise}
\end{itemize}
\subsubsection{Being Your Own Sounding Board}
\begin{itemize}
\item \begin{exercise}[Talk to Yourself]
\end{exercise}
\end{itemize}
\newpage
\section{Introduction}
\subsection{Self-Centered Research: A Manifesto}
\begin{itemize}
\item In this book, we advocate a ``self-centered" approach to research. 

\item \emph{\textbf{Self-Centered Research}} is the following:
\begin{itemize}
\item A \emph{\textbf{practice}} of research that emphasizes the importance of setting out on the research journey from exactly where you are right now, and maintaining close contact with your own self -- your instincts, your curiosities, and your biases-- throughout the process.

To be a ``self- centered" researcher is to maintain your center of gravity over your own two feet at all times, and to \emph{\textbf{avoid pursuing topics and questions} that you \textbf{imagine} might \textbf{please} some \textbf{imaginary, external judge}}.

\item An \emph{\textbf{ethic}} of research that involves \emph{\textbf{consciously} \textbf{acknowledging} and \textbf{assessing} your \textbf{abilities} and your \textbf{limitations}} as a researcher. It involves being \emph{centered}: \emph{\textbf{knowing} who you are}, \emph{\textbf{listening to your own instincts}}, \emph{\textbf{trusting} them} even when they sound naive or inarticulate, and \emph{\textbf{refining} them during the research process}.

\item A \emph{\textbf{state}} of mind that \emph{\textbf{affirms the value} of your ideas}, \emph{assumptions}, and concerns in shaping your agenda and \emph{\textbf{the direction of your research}}. It presumes that the better (and faster) you figure out your own concerns and motivations as a researcher, the better (and faster) you will discover a research problem that is deeply meaningful both to you and to the world at large. But the first person who must be deeply concerned with your research problem is you, the \emph{researcher}.
\end{itemize}

\item \emph{Self-Centered Research} 
\begin{itemize}
\item \emph{does not} mean \emph{unleashing (or inflating)} your \emph{ego}. Being self-centered is not being self-absorbed, self-obsessed, self-congratulatory, self-consumed, self-indulgent, self-involved, self-serving, or self-ish.

Quite the opposite: self-centered researchers are \emph{\textbf{self-reflexive}}, and always \emph{\textbf{self-critical}}; \emph{honest} and \emph{probing} about their own interests, motivations, and abilities; but also \emph{\textbf{open}} and \emph{\textbf{confident}} enough to \emph{assess} the validity of others’. This means having the wherewithal to \emph{\textbf{challenge}} received wisdom, including unfounded ideas you are probably carrying around without realizing it.

\item is also \emph{\textbf{not autobiographical}}.

It does not imply that the papers, articles, reports, or books you write will tell the story of your life. Or that every documentary you produce, or painting you paint, will be a self-portrait.
\end{itemize}

\item \emph{\textbf{The end goal}} of the Self-Centered Research process is, just like conventional research processes, one in which the researcher produces empirical, grounded, theoretically informed, and compelling scholarship about some aspect of the world around us, and does so in a way that \emph{changes} how \emph{other} people think. 

\item In order to identify and solve a problem that truly matters to other people, however, the \emph{Self-Centered Research process} insists that \emph{this problem must matter, first and foremost, to \textbf{you}}.
\end{itemize}
\subsection{Centered Research Is the Best Research}
\begin{itemize}
\item \emph{\textbf{Where to begin?}} The answer is: \emph{\textbf{Exactly where you are}, \textbf{right now}}.

\item Core to this book are two propositions. 
\begin{itemize}
\item First, research can be a \emph{life-changing experience}, if you get a few things \emph{right} from the \emph{start}. 
\item Second, the most important part of beginning a research project is \emph{\textbf{finding your center}}. 
\end{itemize}

\item Research is a process \emph{not just} of \emph{solving problems} but of \emph{\textbf{finding problems that you}} -- and other people -- \emph{\textbf{didn’t know existed}}.

It’s a process of \emph{discovery}, \emph{analysis}, and \emph{creation} that can \emph{generate its own momentum} and \emph{create a virtuous cycle of inspiration}.

Deep-seated problems only reveal themselves through \emph{\textbf{self-trust}}, \emph{\textbf{exposure} to \textbf{primary sources}}, and \emph{\textbf{time}}. 

\item \emph{\textbf{Only you}} -- not anyone else -- can tell you \emph{what you were meant to research}. Answering the question ``What to research?" requires a \emph{hard look in the mirror}.

\item The \emph{\textbf{goal}} of this book, then, is 
\begin{itemize}
\item to help you \emph{\textbf{create the ideal conditions}} to start a fire in your mind.

\item But at the same time, it will show you how to \emph{\textbf{maintain balance and clarity}} in situations of complexity, uncertainty, and ambiguity. 

\item And it will teach you ways to tell the difference between \emph{\textbf{unproductive uncertainty}} -- that is, when you’re \emph{\textbf{on the wrong path}}, and should probably turn back -- and \emph{\textbf{productive uncertainty}} -- that is, when it may feel like you’re lost, but where your \emph{\textbf{inner instinct}} and \emph{\textbf{wisdom}} are \emph{encouraging} you to keep on going.
\end{itemize}
\end{itemize}
\subsection{How to Use this Book}
\begin{itemize}
\item \emph{\textbf{Try This Now}}

In each chapter, you will work through practical exercises and games designed to help you achieve a specific set of goals: 
\begin{itemize}
\item generating questions, 
\item refining questions, 
\item discovering the patterns that connect your questions, and 
\item identifying the problem that motivates you. 
\end{itemize}

\item All of the exercises rely on a core set of principles. These include
\begin{itemize}
\item \emph{\textbf{attentive}}, \emph{\textbf{nonjudgmental self-observation}};
\item giving oneself permission and \emph{encouragement} to say \emph{inarticulate, tentative, and vulnerable things} out loud;
\item getting things \emph{\textbf{down on paper}}.
\end{itemize}

\item Whether or not you tackle everything in sequence on your first pass, the only way to get the benefit of our advice is by completing the exercises, and, as mentioned above, by \emph{\textbf{writing things down}}.

\item The point of all this \emph{continual writing} is to \emph{produce} what we term ``evidence of self," or ``\emph{\textbf{self-evidence}}."

You can think of self-evidence as clues that will help you figure out the answers to the most important questions that a researcher must answer during this early phase:
\begin{itemize}
\item \emph{Why am I concerned with this topic? }
\item \emph{What is it about this subject that I think holds the key to some larger issue? }
\item \emph{Why does this primary source jump out at me? }
\item \emph{Why, out of all possible topics that I could be working on, do I keep coming back to this one? }
\item \emph{What is my Problem?}
\end{itemize}

\item \emph{\textbf{Self-evidence}} is a valuable form of \emph{note-taking} that we believe many researchers neglect.

\item We advocate making \emph{\textbf{introspection}} a \emph{habitual part} of your research method. 

\item The pieces of \emph{self-evidence} you produce during \emph{the Self-Centered Research} process are cousins to the kind of \emph{notes} experienced researchers routinely make when they read primary sources, conduct interviews, carry out ethnographic fieldwork, or copy down bibliographic information.

We call them \emph{\textbf{self-evidence}} because, \emph{\textbf{during this early phase of research}}, \emph{\textbf{these notes} will possess a \textbf{value}} that goes far \emph{beyond} the \emph{recording} of facts, quotes, observations, and other evidence about the world around you. They will \emph{\textbf{provide evidence about you yourself}}. 

With these clues you will be able to \emph{uncover the \textbf{hidden} questions and problems} you carry around inside you. 

Discover them early in the research process and not only will you \emph{save yourself time and frustration}, but, more importantly, you will be more likely to \emph{arrive at the research project that is \textbf{right for you}}.

\item \emph{\textbf{Commonly Made Mistakes}}

A list of these follows each ``\emph{Try This Now}" exercise. Most of these mistakes fall into one of three categories:
\begin{itemize}
\item Not letting yourself be \emph{\textbf{vulnerable}}
\item Not \emph{\textbf{listening to yourself}}
\item Not \emph{\textbf{writing things down}}
\end{itemize}

\item \emph{\textbf{Sounding Board}}
 
A \emph{\textbf{Sounding Board}} is someone who helps you to gain \emph{alternative perspectives} on your ideas and writings and to \emph{step outside yourself}. 
\begin{itemize}
\item A \emph{Sounding Board} helps you to \emph{\textbf{self-reflect}} and \emph{make better decisions}, so we recommend that you make \emph{\textbf{talking to someone you trust}} a habit early in the research process. 

\item Ultimately, \emph{the Self-Centered Research} process will empower you to \emph{\textbf{become your own Sounding Board}}.
\end{itemize}

\item Well-meaning suggestions from a teacher, adviser, or other \emph{authority} figure -- suggestions as to what you ``could" or ``should" work on -- can have a \emph{major impact} on a researcher during the early phases of research. If you feel lost, or uncertain about the value of your nascent ideas, these suggestions can feel a lot like a \emph{\textbf{command}}. Or it may become your fallback, your ``Well, I can’t come up with anything better, so I might as well go with that!" 

What if you \emph{\textbf{skipped}} all that messy \emph{\textbf{introspection}} and snapped up the ready-made idea that your trusted adviser has told you is important? Unfortunately, the effect can be \emph{\textbf{inhibitory}} and \emph{\textbf{counterproductive}}.

\item The point of research is \emph{not to fall back}, it’s to \emph{\textbf{move forward}} -- to take a risk and discover or create something original. 

\item A \textbf{mentor} can \emph{\textbf{offer advice}} that \emph{\textbf{saves you from retracing others’ paths}} to the same conclusion. 

But when a student comes with an idea for a research project and asks, ``Is this what you want?" a true mentor’s response is always
the same: ``\emph{\textbf{Is this what you want?}}"

\item In our experience, if a research question is \emph{not} one that you’re \emph{truly motivated} to spend your time answering, you’ll find it a \emph{challenge} to do a good job, or even to finish. 
\end{itemize}

\subsection{Introversion, First. Extroversion Second}
\begin{itemize}
\item The \emph{two-part process} of starting a research project involves
\begin{itemize}
\item  looking first \emph{\textbf{inward}} and
\item   then \emph{\textbf{outward}}.
\end{itemize}

\item 
\begin{itemize}
\item Part 1 takes you through the \emph{\textbf{inward-focused process}} of becoming a self-centered researcher.  You will \emph{\textbf{reflect}} on the experiences, interests, priorities, and assumptions you bring with you -- and assess how to make best use of them in charting out a research direction. 

This process goes beyond \emph{conventional \textbf{brainstorming}} because it requires \emph{taking stock of your \textbf{values}}. It involves \emph{\textbf{distinguishing}} between
\begin{itemize}
\item \emph{\textbf{what doesn’t matter to you}}, 
\item \emph{\textbf{what you think matters to you}}, and 
\item \emph{\textbf{what really matters to you}}. 
\end{itemize} We believe that you are best off \emph{starting} this process \emph{before} you field-test your ideas against the wisdom of the research community. 

\item Part 2 focuses on this process of \emph{\textbf{extroversion}}. It helps you to navigate the often bewildering process of coming to terms with the research communities conventionally known as ``\emph{fields}" and ``\emph{disciplines}," as well as how to identify researchers who may not be in the same field as you but \emph{\textbf{who are interested in similar problems}} -- what we call your \emph{\textbf{Problem Collective}}.
\end{itemize}

\end{itemize}

\section{Become a Self-Centered Researcher}
\subsection{Questions}
\subsection{What's Your Problem?}
\subsection{Designing a Project that Works}


\section{Get Over Yourself}
\subsection{How to Find Your Problem Collective}
\subsection{How to Navigate Your Field}
\subsection{How to Begin}

\section{What's Next in Your Research Journey?}


\newpage
\bibliographystyle{plainnat}
\bibliography{reference.bib}
\end{document}