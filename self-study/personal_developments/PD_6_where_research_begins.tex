\documentclass[11pt]{article}
\usepackage[scaled=0.92]{helvet}
\usepackage{geometry}
\geometry{letterpaper,tmargin=1in,bmargin=1in,lmargin=1in,rmargin=1in}
\usepackage[parfill]{parskip} % Activate to begin paragraphs with an empty line rather than an indent %\usepackage{graphicx}
\usepackage{amsmath,amssymb, mathrsfs,  mathtools, dsfont}
\usepackage{tabularx}
\usepackage{tikz-cd}
\usepackage[font=footnotesize,labelfont=bf]{caption}
\usepackage{graphicx}
\usepackage{xcolor}
%\usepackage[linkbordercolor ={1 1 1} ]{hyperref}
%\usepackage[sf]{titlesec}
\usepackage{natbib}
%\usepackage{tikz-cd}

\usepackage{../../Tianpei_Report}

%\usepackage{appendix}
%\usepackage{algorithm}
%\usepackage{algorithmic}

%\renewcommand{\algorithmicrequire}{\textbf{Input:}}
%\renewcommand{\algorithmicensure}{\textbf{Output:}}


\begin{document}
\title{Book Reading Summary: Where Research Begins: choosing a research topic that matters to you (and the world)}
\author{ Tianpei Xie}
\date{Dec. 10th., 2023}
\maketitle
\tableofcontents
\newpage
\section{Summary}
\subsection{Research Phases: Inside-Out Self-Centered Approach}
\emph{Where Research Begins: choosing a research topic that matters to you (and the world)}, Thomas S. Mullaney and Christopher Rea, The University of Chicago Press.

\begin{itemize}
\item In this book, the authors proposed the \emph{\textbf{Self-Centered Research}} process, an inside-out self-motivated research \emph{\textbf{practice}}, \emph{\textbf{ethic}} and \emph{\textbf{a state of mind}}. It discuss, at \emph{the beginning phase of research}, how the research problem is \emph{identified}, \emph{refined} and \emph{evaluated}, through \emph{\textbf{inward-focus}} \emph{first} and \emph{\textbf{outward-focus}} later methodology. 

\item This book focus on \emph{\textbf{raising the internal motivations}} of researcher before the research process started, and in the process to gain \emph{self-awareness}, \emph{self-trust} and \emph{affirmation on the direction of research} on their own. It stress the importance of ``\emph{\textbf{finding your center}}" -- the matters that really motivates \emph{you} to start the research.  It focus on the question of ``\emph{Why?}"

\item By using ``\emph{\textbf{introversion first, extroversion second}}" approach, the authors help the early researchers to \emph{\textbf{avoid}} being \emph{\textbf{distracted}} by perspectives, ideas, suggestions from \emph{\textbf{others} (authorities, mentors, colleagues)} in research communities, as well as to \emph{avoid} \emph{\textbf{the judgemental thoughts}} from \emph{\textbf{their own mind}}, which could hurt the inner exploration processing at early stage of research. 

\item This book has a favor of using \emph{\textbf{psychological} treatement} to examine the researcher's own interest/boredom on research questions.

\item The key messages to early researchers:
\begin{enumerate}
\item Be  \emph{\textbf{vulnerable}}
\item \emph{\textbf{Listening to yourself}}
\item \emph{\textbf{Writing things down}}
\end{enumerate}

\item This book is divided into two stages:
\begin{itemize}
\item \emph{\textbf{inward stage}}: the \emph{\textbf{goal}} of this stage is the \emph{\textbf{raise} the \textbf{self-awareness}} of the researcher ourselves on our motivations and values. This way to make sure we are confident on our direction and priorites. This stage is also the stage we accumulate knowledge, source materials and raise our arguments.

In this stage, we transform
\begin{itemize}
\item from \textbf{topics} into \textbf{questions}: including \emph{question brainstorming}, \emph{interest self-observation}, \emph{boredom self-observation}, and \emph{question narrow-down and specification}.

\item from \textbf{questions} into \textbf{problems}: including \emph{question linguistic check}, \emph{question source check}, \emph{question expansion}, \emph{assumption uncover and modification}, and \emph{problem identification}. 

\item we \textbf{engage sources} and \textbf{assemble our arguments}.
\item finally, we \textbf{plan} our project and write our \textbf{research proposal}.
\end{itemize}

\item \emph{\textbf{outward stage}}: the goal of this stage is to \emph{\textbf{raise awareness}} on the \emph{external} ideas and perspectives in your \emph{\textbf{field}} and your \emph{\textbf{Problem Collectives}}, and to make most out of this \emph{\textbf{relationship}}. 

Part 2 moves your research journey into a \emph{broader} and deeper \emph{\textbf{engagement}} with other people’s ideas.  This stage focus on \emph{\textbf{engaging}} with external world and their perspectives, ideas, esp. with your \emph{field} and your Problem Collectives. 

We are also going to revise our draft based on their feedbacks multiple rounds.
\end{itemize}
\end{itemize}

\subsection{Summary of Chapters}
\begin{itemize}
\item This books covers several phases of early research:
\begin{enumerate}
\item \emph{\textbf{Question Brainstorming with Self-Observation}}: at this earliest phase, the task is to generate questions relevant to you, \emph{\textbf{from the best of your knowledge}}. 

\begin{itemize}
\item This part covers the traditional method of ``\emph{\textbf{from topics to questions}}". But in this book, the emphasis is on self-observing on your own reactions to specific matters in the topic and your own perspective and related questions. 

\item While scanning through entities under a topic, ask yourself  ``\emph{Why does that interest you?}" or ``\emph{Why i do not care about it?}" \emph{\textbf{Take notes}} on questions generated and \emph{flag} those with unconsciously effect (interest/bordom) on you. 

\item The point is being \emph{\textbf{honest}}, \emph{\textbf{self-observing}} but not \emph{\textbf{not be judgemental}}. 


\item Your questions are for yourself. They are meant to be \emph{\textbf{unpolished}}. Use normal worlds, not jargons. You are not meant to impress anyone. 

\item \emph{Your questions} need to be \emph{\textbf{specific and concrete}}, \emph{not vague}, not \emph{abstracting}. 
\end{itemize}

\item  \emph{\textbf{Question Refinement}}: In this step we \emph{\textbf{improve the questions}} you have already generated. 
There are two ways for stress-testing your questions:
\begin{itemize}
\item Focus on \emph{\textbf{language}}; By the end of this process, your questions should meet these criteria:
\begin{itemize}
\item \textbf{\emph{They should be clear, precise, and jargon-free}}. 

\item They should be \textbf{\emph{rooted in verifiable and falsifiable data}}. Your research questions should have \textbf{integrity}. This means that they should be inspired by \textbf{fact}, rather than by speculation, prejudice, or opinion. 

\item They should be \emph{\textbf{indifferent to the outcome}}. A research question \emph{should not presume a certain answer}. If yours does, rewrite it to \emph{eliminate that presumption}.

\item They should be \emph{\textbf{clear} about the \textbf{subject}}. Be as specific as you can be about the \emph{who} in your question.

\item They should be \emph{\textbf{raw}} and \emph{\textbf{undisciplined}}. At least for now.
\end{itemize}

\item Focus on \emph{\textbf{sources}}. Rather than trying to use primary sources to start answering the questions you’ve come up with, we want you to use them to \emph{develop}, \emph{refine}, and \emph{expand those questions}. 
\begin{itemize}
\item \emph{\textbf{Refine}} and \emph{\textbf{Expand}} your \emph{\textbf{keyword}} used for searching the primary sources. Pay close attentions to new keywords in titles, abstract, index, introduction, conclusion. Keep track of all keywords, the primary sources and searches found during the process. 

\item Reveal the hidden \emph{\textbf{assumption}} behind these questions. Also for assumptions that are bias, do not discard them directly; instead modify them and improve them. Keep a note on all of assumptions discovered and then treat them accordingly based on your evaluation and your intent. 

During this categorization process, write down your thoughts and reasons for the categorization.
\end{itemize}
\end{itemize}

\item \emph{\textbf{Problem Identification}}: This is the part where we move ``\emph{\textbf{from questions to problems}}". At this step, we use \emph{sources} to \emph{identify the problem} by understanding \emph{\textbf{internal connections}} between questions. We also learn how to use our \emph{Problem} to \emph{generate new and better questions}.
\begin{itemize}
\item In particular, we need to \emph{\textbf{generalize}} from previous questions to higher level. 

Also we need to \emph{\textbf{group}} these questions and to identify \emph{\textbf{the shared concerns}} among these questions.

\item Note that it is not time to answer these questions. 

\item How do I know when I’ve truly discovered my Problem? 

A problem is never a fleeting thing. Rather, it is something that is \emph{\textbf{sustained}} and \emph{\textbf{enduring}}.
\end{itemize}

\item \emph{\textbf{Primary Source Engagement}}: We need to identify, filter and refine your list of primary sources \emph{\textbf{based on the identifed problems}}. \emph{\textbf{Reversely}}, we also need to \emph{\textbf{milk our sources}} to generate \emph{\textbf{a genre of new related questions}} to the same problem. We will connect to different projects and then connect with different sources.
\begin{itemize}
\item You need to distinguish primary and secondary sources.

Primary Sources are the \emph{\textbf{evidence}} that you use to \emph{\textbf{develop and test claims}}, \emph{\textbf{hypotheses}}, and \emph{\textbf{theories} \textbf{about reality}}. 

\item Note that a source’s \emph{type} is determined solely by \emph{\textbf{its relationship with the questions}} you are trying to answer, and \emph{the problem} you are trying to solve. A source is never inherently primary or secondary.

\item \emph{\textbf{how you treat this source}} will lead you down either a narrow path or a broad avenue of potential research questions. We can
\begin{itemize}
\item jump to obvious candidates: direct questions related to the source or the elements of the source; or
\item think about \emph{\textbf{different projects}} that might include this source;
\item think about \emph{\textbf{other primary sources}}.
\end{itemize} We can name \emph{the genre of questions} and considering them connected to an underlying problem. 

\item You need to deep dive and discover the full potential of these primary sources.  This helps you to look beyond obvious questions and to arrive at something original.

\item \emph{\textbf{Envision} imaginative primary sources} that best answer your question. Search for it.

\item You need to pinpoint these sources to your problem; determine if they are relevant, reduandent, reliable. 
\end{itemize}

\item \emph{\textbf{Argument Construction}}: We need to make proper argument from these sources. Before, we need to \emph{prepare enough primary sources} to make a good argument. 
\begin{itemize}
\item In particular, we need to 
\begin{enumerate}
\item Find the dots. 
\item Figure out which dots \emph{belong to your picture}, and which dots belong to some other picture. 
\item Figure out which dots are not dots at all. \emph{Not all materials can be used as sources}. 
\item \emph{Do all of the above in real time.} Do not expect linear progression. 
\item Determine when you have enough.
\end{enumerate}

\item Be aware of the \emph{\textbf{ethical} issues} on how to use the source materials. Be honest for what you see and what you don't see. Not forcing a story. Represent the source accurately. 

\item Source has its own agency so treat them with \emph{critical mindset}. Even if the sources contain flaws, you can still use them in question-generation process to clarify your assumptions and problems. 
\end{itemize}
 



\item  \emph{\textbf{Research Project Design}}: Plan the project by answering following questions
\begin{itemize}
\item What outcome do you want to achieve? 
\item What primary sources do you possess? 
\item What resources (time, computational, people, other responsibilities) can you utilize? What constraints? 
\item What is the deadline? 
\item What timeline are you planning?
\item Understand my personality
\end{itemize} Finally, write it down as a research proposal for yourself.

\item  \emph{\textbf{Problem Collective Identification}}:

\item  \emph{\textbf{Rewriting for Problem Collective}}:

\item  \emph{\textbf{Field Grouping via Problem Collectives}}:

\item  \emph{\textbf{Rewriting for Field}}:

\item  \emph{\textbf{Assembling into Draft}}:
\end{enumerate}
\end{itemize}


\newpage
\section{Try This Now}
\subsection{From Topics to Questions}
\begin{itemize}
\item \begin{exercise}[Search Yourself]

The \textbf{goal}: To use a list of \textbf{primary-source search results} to figure out \textbf{the aspects of your topic} that \textbf{most interest you}, and \textbf{draft questions} based on these interests.

You already know how to search the internet. This exercise prompts you to use the results of an internet search to search yourself.

This exercise offers one way to \textbf{get from a topic to questions}
\begin{enumerate}
\item  Based on the “Try This Now” exercise you completed in the introduction, \textbf{write down any and all of the research topics you are drawn to}. Feel free to be as general as possible, and to include more than one.

\item \textbf{Select} one of the topics on your list and run a search using at least three (or more) of the webbased databases listed below. 

\item Click on a few of the \textbf{search} results that \textbf{interest} you -- say, five to ten.

\item Don’t read the search results in depth. Instead, your goal is to dedicate 
\begin{itemize}
\item  perhaps 20 percent of your mental energy to \textbf{scanning} the list of search results (and perhaps the contents of a few) and
\item the remaining 80 percent of your mental energy to \textbf{self-observation}. 
\end{itemize} You want to \textbf{read yourself} as you \textbf{read the results}.

\item In particular, pay close attention to how your \textbf{mind} and \textbf{body} are \textbf{responding} to different search results:
\begin{itemize}
\item Which ones seem to \textbf{jump out at you}? 

\item Which ones cause you to \textbf{linger} just a split second longer?

\item Which ones quicken your \textbf{pulse}, even slightly?
\end{itemize}

\item \textbf{Write} down at least ten entries that \textbf{attract} you, without worrying about why they do.

\item Based on this list of ten entries, \textbf{answer} the three questions above about those entries, to \textbf{generate self-evidence}.

\item Sleep on it (take a break of at least twenty-four hours).

\item Return to the answers you wrote out and ask yourself: If I didn’t know the person who wrote these answers, or \emph{flagged} these search results as ``\emph{interesting}," 

what kinds of guesses would I make about this researcher? 

What story does this ``self-evidence" seem to tell about the researcher, in terms of their concerns and interests?

\item \textbf{Write} down your \textbf{thoughts} on these questions, getting as much down on paper as possible.
\end{enumerate}
\end{exercise}
Common \textbf{Mistakes}:
\begin{itemize}
\item Not writing things down
\item Getting bogged down in \textbf{individual sources too soon}
\item Excluding ``fluke" search results that \textbf{seem unrelated} to the keywords you entered in the database or unrelated to your topic
\item Feigning interest in a search result that seems ``\textbf{important}," even if it doesn’t really interest you
\item Only registering interest in search results for which \textbf{you think you know why} you’re interested in them, instead of being more \textbf{inclusive}
\item Trying to make a list of noticings that is \textbf{coherent} and fits together
\item When speculating about why a search result jumped out at you, worrying about whether or not the reason is ``\textbf{important}," based on some \textbf{imagined external standard}
\end{itemize}

\vspace{20pt}
\item \begin{exercise}[Let Boredom Be Your Guide]

The \textbf{goal}: To become \textbf{attentive} to your \textbf{active dislikes}, identifying questions that you ``should" (in theory) be interested in based on your topic of interest, but aren’t. By \textbf{understanding what you don’t care about} regarding your topic, you accelerate the process of figuring out what you do care about.

After all, the most common reaction human beings have to \textbf{boredom} is \textbf{avoidance}. We try to dismiss or ignore things that bore us.

But how would you explain why something bores you -- especially something that seems like it should align with your topic of interest?
Here’s what to do:
\begin{enumerate}
\item Go back to your search results, and scan them again.
\item Pay close attention to your EKG readout, focusing this time on the results that \textbf{bore} you. 

In the very same way that we spoke of not ``outsmarting" yourself regarding your interests, you will need to be cautious during this process as well. 

\item Choose a few ``\textbf{boring}" results and \textbf{write} down answers to the same questions you answered before -- this time for these different, boring search results:
\begin{itemize}
\item What does this make me think of?

\item  If I had to venture a guess, why did this one \textbf{not jump out at me}?

\item What \textbf{questions} come to mind for me when I look at this search result?
\end{itemize}

\item Now, for each search result, write some version of this sentence: ``I’m \textbf{more interested in} [something else] \textbf{than} [search result]."
\end{enumerate}
\end{exercise}
Common \textbf{Mistakes}:
\begin{itemize}
\item Denying boredom, or feigning interest in something because you feel it’s ``on topic" and demands your interest because it’s ``important."
\item Engaging in circular logic. 
\end{itemize}


\vspace{20pt}
\item \begin{exercise}[Go Small or Go Home]

The \textbf{goal}: To generate \textbf{specific, fact-focused questions} about your topic before you’ve done in-depth research. These will lead to bigger questions later on.

You have a set of notes about two things:
\begin{itemize}
\item What you noticed about sources on a topic, and your best guesses as to why you noticed those things

\item What, among the ``logical" or ``obvious" aspects of your proposed topic, bored you and why
\end{itemize} Using all of this as inspiration, try the following -- as always, in writing.
\begin{enumerate}
\item In a stream of consciousness, \textbf{write} out a minimum of \textbf{twenty questions} related to your topic. 

The \textbf{key} is to make your questions as \textbf{specific} as possible, using the following prompts:
\begin{itemize}
\item What facts do you \textbf{wish to know} about your topic?
\item Which \textbf{data} or \textbf{information} about your topic might you need to satisfy your curiosity?
\item What telling \textbf{details} about your topic do you \textbf{imagine} might exist?
\end{itemize}
\end{enumerate}
Asking \textbf{precise factual questions} is one \textbf{key} to \textbf{escaping} \textbf{Topic Land}.
\end{exercise}
Common \textbf{Mistakes}:
\begin{itemize}
\item Asking vague, grand, abstract, or big-picture questions about ``meaning" or ``significance,“ instead of specific and \textbf{precise factual questions}
\item Not asking actual questions (with a question mark), but instead writing statements or sentence fragments -- \textbf{topics masquerading as questions}
\item Not asking a question because \textbf{you think you couldn’t answer it}, perhaps because you think that the data doesn’t exist or is unattainable
\item Asking \textbf{too few questions}, resulting in an inadequate quantity of self-evidence
\end{itemize}
\end{itemize}
\subsection{From Questions to Problems}
\begin{itemize}
\item \begin{exercise}[Run a Diagnostic Test on Your Questions]

The \textbf{goal}: To ensure that the vocabulary, grammar, and phrasing of your questions are \textbf{specific} and \textbf{unprejudiced} so that they do not presume a certain outcome.

Rewrite your research questions with particular attention to the following:
\begin{enumerate}
\item \textbf{Punctuation}. Do your questions actually end in a question mark? Or have you phrased them in more general, and vaguer term? \textbf{Be more specific}, and add a question mark.

\item \textbf{Adjectives and adverbs}. Do your questions rely on \textbf{broad}, \textbf{generic}, \textbf{imprecise}, or \textbf{sweeping} adjectives? Try to cut such adjectives and adverbs out entirely.

\item \textbf{Collective nouns.} Do your questions depend upon collective nouns? If so, do your best to replace these nouns with
more precise categories. You do not need to take into account all possible variables, but you do want to try to include all of those that might make a difference to your project.

\item \textbf{Verbs}. Do your questions contain verbs like ``influence," ``affect," ``shape," or ``impact" or passive constructions such as ``was affected by," ``responded to," or ``reacted to"? In such cases, chances are high that you are building your questions in such a way that they \textbf{rule out} an
entire set of \textbf{possible} answers and outcomes. \textbf{Rephrase to avoid presumptions} that could result in confirmation bias.
\end{enumerate}
\end{exercise}
Common \textbf{Mistakes}:
\begin{itemize}
\item Asking \emph{\textbf{leading questions}}, which are phrased so as to \emph{predetermine the answer}.

These questions are motivated by \emph{unproven assumptions}, andr result in \emph{\textbf{confirmation bias}}. The result of leading questions is that you inevitably find what you are looking for. 
 
\item Asking \emph{\textbf{advocacy questions}}, which promote a certain ideology (taken-for-granted worldview) or course of action.

These questions \emph{\textbf{take a position}} and encourage others to \emph{adopt} it, irrespective of the actual facts of the case or which interpretations the evidence suggests are plausible. 

\item \emph{\textbf{Forcing}} all your questions to ``make sense" or ``add up." Don’t worry. That part will come soon.
\end{itemize}

\vspace{20pt}
\item \begin{exercise}[Use Primary Sources to Educate Your Questions]

The \textbf{goal}: To learn how to \textbf{run keyword searches} designed to \textbf{enhance} or ``educate" the \textbf{questions} you are asking about
your topic. 

These searches \textbf{uncover primary sources} relevant to your research that themselves contain \textbf{new keywords} you were previously unaware of (thereby enabling you to run follow-up searches to reveal even more, and more useful, primary sources).

This next exercise requires you to delve back into your \textbf{specific subject matter} and into \textbf{primary sources}. Rather than trying to use primary sources to start answering the questions you’ve come up with, we want you to use them to \textbf{develop}, \textbf{refine}, and \textbf{expand those questions}.

Here are some \textbf{techniques} to help you use primary sources to refine your keyword searches.
\begin{itemize}
\item \textbf{Take Advantage of Category Searches}. In certain databases, you might be fortunate to come across materials that are accompanied by \textbf{metadata} (data about data), crafted by librarians and archivists whose goal it is to make sources more discoverable to researchers like you. 

This is one way to get from a source that contains only the keywords you used in your search to another source that contains none of the keywords you used. Here’s what to do: 
\begin{itemize}
\item after you run your search, and receive your results, \textbf{sort} the results chronologically
\item As you scan through these \textbf{titles}, take notice: What words show up in the title? 
\item If you are able to read the work online, scan the \textbf{table of contents}, the \textbf{preface}, the \textbf{introduction}, and the \textbf{index}.  What words, terms, and vocabulary are used? These are your new keywords. Write them down.
\end{itemize}

\item \textbf{Locate Self-Reflexive Sources}. In some cases, you might be fortunate to find a primary source, like a historical dictionary, that explicitly addresses the shifting nomenclature surrounding the very topic the primary source is about, outlining for you the varied ways a given idea, place, community, practice, or the like has been named and renamed across time and space. 

The goal for now is to determine if this source will lead you in the \textbf{direction of further primary sources} that you wouldn’t have been able to find otherwise.

\item \textbf{Keep Track of Your Keywords and Searches}.  As you discover and try out more and more keywords -- and even a smaller-scale project can produce hundreds -- it’s easy to lose track of them and get overwhelmed. 

Fortunately, there’s a simple solution: \textbf{track} your searches using a \textbf{table}. Here’s how, in three steps:
\begin{itemize}
\item In the rows on the left side, enter the \textbf{keywords} you plan to use.
\item In the column headers, enter all of the electronic \textbf{databases} or library catalogues you plan to explore.
\item Inside each cell, keep track of when you ran a particular search. Enter the \textbf{date of your search}, and perhaps also a brief \textbf{note} on the number of results you found.
\end{itemize}
\end{itemize}
\end{exercise}

\vspace{20pt}
\item \begin{exercise}[Make Your Assumptions Visible]

The \textbf{goal}: To become aware of the \textbf{assumptions} you bring to your research project and use them to identify the \textbf{problem} that motivates your research questions.

So let’s get to work on making your assumptions visible, and vulnerable. Here’s what to do:
\begin{enumerate}
\item \textbf{Review} your most recent set of \textbf{questions} and ask yourself: 

For each of these questions, what has to be true in advance in order for me to ask this question in the first place?

\item List the small questions/things you \textbf{noticed}, and \textbf{write down the assumptions} you may hold that helped you notice each in the first place.

\item \textbf{Make a list of the assumptions} you bring to \textbf{this particular question}, and sort them into the following categories:
\begin{enumerate}
\item Assumptions you \textbf{want to} work with, for now
\item Assumptions you want to \textbf{discard} \textbf{right away}
\item Assumptions you are \textbf{unsure} or ambivalent about
\end{enumerate}

\item Write two lines to \textbf{justify your choice} for putting each assumption in a particular category.

\item Now go back to all of the questions in your list whose underlying assumptions fall into \textbf{category (a)}. 

Since these are built on assumptions that you, having reflected on them, feel safe in \textbf{maintaining}, then these questions are good as they are.

\item What about questions whose underlying assumptions fall into \textbf{category (b)}? Although you might be tempted to, do not throw them away just yet! If you find them to be based on weak, prejudicial, or unfounded assumptions, \textbf{try to rephrase them} so that they aren’t. 

Can they be \textbf{rebuilt} as more \textbf{grounded}, \textbf{open-ended questions}?

Try to \textbf{improve} them before you \textbf{discard} them.

\item As for questions built on \textbf{category (c)} assumptions, these fall somewhere in between. Most likely you
would want to \textbf{keep} them in your list, but perhaps \textbf{flag} them, as reminders to yourself that you want to keep an eye on them, and revisit them as your research deepens.
\end{enumerate}
To keep things organized, try \textbf{creating a chart} like the one in table 3 for each question, in which you identify and analyze underlying assumptions and revise the question as needed.
\end{exercise}
Common \textbf{Mistakes}:
\begin{itemize}
\item \emph{Not identifying or divulging the assumptions motivating your research questions} -- for any reason, including embarrassment or self-consciousness. 

\item \emph{Not attempting to \textbf{revise or restructure} a research question based on \textbf{category (b) assumptions}}.

\item \emph{\textbf{Dismissing} or throwing out \textbf{category (c) assumptions}}, instead of \textbf{examining} them as a type of self-evidence. 
\end{itemize}

\vspace{20pt}
\item \begin{exercise}[Identify the Problem That Connects Your Questions]

The \textbf{goal}: To \textbf{identify the problem} underlying your multiple draft research questions

Try this procedure:
\begin{enumerate}
\item Lay all of your questions out in front of you.

\item Do not try to answer all those questions for now. Instead, ask yourself: 

What are \textbf{the shared concerns} that connect these \textbf{questions}?

\item \textbf{Step outside yourself}. If you were \textbf{someone else} looking at these questions, 

what might you speculate are the \textbf{deeper questions} that \textbf{connect} these \textbf{small questions}?

\item \textbf{Write down those questions}.

\item If necessary, \textbf{prioritize} your questions by degree of \textbf{specificity} or \textbf{generality}, as \textbf{medium-level} or \textbf{high-level} questions. 

These questions should be \textbf{more general} than the \textbf{specific factual questions} you generated earlier.
\end{enumerate}
\end{exercise}
Common \textbf{Mistakes}:
\begin{itemize}
\item \textbf{\emph{Trying to answer your multiple questions}}, instead of focusing on \textbf{\emph{identifying the shared concern}} that underlies them. 

\item Not thinking \textbf{\emph{beyond the particular topic or case}}, and ignoring a \textbf{\emph{more fundamental concern}}.
\end{itemize}


\end{itemize}
\subsection{Engaging Primary Sources}
\begin{itemize}
\item  \begin{exercise}[Treat Your Primary Source Like a Cereal Box]

The \textbf{goal}: To adopt the habit of \textbf{asking multiple genres of questions} about each of your \textbf{primary sources} so as to \textbf{identify problems that are not self-evident} and thus might easily be \textbf{overlooked}. 

This technique will both enable you to decide which problem interests you most, and enhance your ability to conduct original research.

\begin{enumerate}
\item Using the search techniques you learned in chapters 1 and 2, track down and obtain a single source. 

it should be a source that you instinctively feel must be ``primary" with regard to your emerging research concerns.

\item Create a table (namely table 5 in the book) with 4 columns:
\begin{enumerate}
\item \textbf{What I notice} about the source 
\item \textbf{Questions}/concerns I might have
\item The very n\textbf{ext primary source} I might \textbf{want} to find
\item \textbf{Broader} subjects and/or genres of \textbf{questions} that might be \textbf{related to my problem}
\end{enumerate}

\item Using table 5 as your guide, take notice of as \textbf{many different features} of your \textbf{source} as possible. 
\begin{itemize}
\item \textbf{Disaggregate} the source into its different \textbf{elements}. 
\item  \textbf{Identify} as many elements as possible, but no fewer than ten.
\item  You will need to abstract and extrapolate from observations. 
\item Fill in first column.
\end{itemize}


\item As you fill in the first column, try to imagine the kinds of \textbf{questions} that could be asked by \textbf{focusing} on one or another \textbf{specific feature} of your source.   Think expansively. Add these questions to column 2.

\item Now imagine what a potential ``\textbf{very next source}" might be for each of these feature- question pairings, and fill in column 3.

\item Finally, return to your increasingly skilled faculties of introspection, asking yourself:
\begin{itemize}
\item Which of these feature-question-source lights my fire?
\item Which excites me the most? Why, if I had to venture a guess?
\item Which of these bore me? Why, if I had to venture a guess?
\item What does this suggest about what my primary concerns might be?
\item How is this source ``primary" with respect to my questions and concerns?
\end{itemize} Write them down.
\end{enumerate}
\end{exercise}
Common \textbf{Mistakes}:
\begin{itemize}
\item Asking only \emph{obvious} or \emph{self-evident questions} related to the ostensible topic of the source, instead of \emph{\textbf{multiple genres of questions}}

\item Asking questions that are vague and general instead of \emph{\textbf{specific} and \textbf{factual}}

\item \emph{Asking \textbf{too few genres of questions}} -- aim for at least ten. Err on the side of being creative, even far-fetched

\item For ``\emph{the very next primary source I might want to find}," \emph{thinking only of sources \textbf{within your Field}}

\item After completing the table of noticings, questions, next sources, and genres of questions, \emph{skipping the steps of (a) \textbf{gauging your relative interest} in those results, and (b) \textbf{writing} down the result.}
\end{itemize}

\vspace{20pt}
\item \begin{exercise}[Envision Your Primary Sources]

The \textbf{goal}: To identify places you might not have originally considered looking for primary sources. This will \textbf{enhance the comprehensiveness}, \textbf{originality}, and \textbf{significance} of your research.

The steps for this exercise are straightforward:
\begin{enumerate}
\item Write down your \textbf{research questions}, as always, with as much precision as you can.
\item \textbf{Brainstorm}: What sources \textbf{might exist} that would be primary with respect to my research questions?
\item Write down as many \textbf{types} of sources as possible.
\item \textbf{Optional}: If you have time to spare, and as long as it doesn’t distract you from steps 1 through 3, try to \textbf{find} such sources. If you find any of them, put them through the Cereal Box Challenge.
\end{enumerate}
\end{exercise} 
Common \textbf{Mistakes}:
\begin{itemize}
\item During brainstorming, thinking only in terms of your \emph{specific case} and not in terms of \emph{the general categories or institutional structures} in which the world might have arranged sources related to your case

\item \emph{Excluding} sources because they \emph{do not appear to be related to your topic or keywords}

\item \emph{Worrying} about whether or not you can \emph{actually obtain} the sources you envision

\item \emph{Not writing things down}
\end{itemize}

\end{itemize}
\subsection{Assembling Arguments}
\begin{itemize}
\item \begin{exercise}[Connect the Dots Using Your Sources]

The \textbf{goal}: To start thinking about source criticism early in the research process, while remaining flexible and inclusive.

The steps are simple: 
\begin{enumerate}
\item Identify my \textbf{primary sources}. Draw on what you wrote down for the ``Envision Your Primary Sources" exercise.

\item Determine which \textbf{primary sources belong to my problem}, and which to someone else’s. 

Be as honest as possible about the problem that motivates you.

\item Determine which of the things I have are \textbf{actually (primary) sources}.

\item Find the best way to form a \textbf{narrative} by arranging the primary sources.  

Trying out some narrative possibilities by \textbf{structuring} and \textbf{ordering} your sources in different \textbf{configurations}, to see how they speak to one another. 

The key, of course, is not to force any pieces of the puzzle together.

\item Determine when \textbf{how many} primary sources will I \textbf{need} to answer my questions, solve my Problem, and complete my project.

This is a question that only \textbf{you} can answer, although your Sounding Board might be able to help you make this assessment.
\end{enumerate}
\end{exercise}
Common \textbf{Mistakes}:
\begin{itemize}
\item Thinking that you \textbf{\emph{have to have all}} your sources in hand \emph{before you start this process}. You will need to have multiple sources (dots) to begin this process, \emph{but not all}.

\item Writing in \emph{pen} rather than \emph{pencil}. 

Recognize that the connections you make between sources right now are necessarily \emph{\textbf{tentative} and \textbf{speculative}}.

Expect that you will have to \emph{\textbf{reassess your judgments later on}}, and don’t think that you have to ``stick to" your original thought.
\end{itemize}

\end{itemize}
\subsection{Planning Research Projects}
\begin{itemize}
\item \begin{exercise}[Decision Matrix]
\end{exercise}

\vspace{20pt}
\item \begin{exercise}[Prepare a Formal Research Proposal]
\end{exercise}
\end{itemize}
\subsection{Identifying Your Problem Collective}
\begin{itemize}
\item  \begin{exercise}[Change One Variable]
\end{exercise}

\vspace{20pt}
\item  \begin{exercise}[Before and After]
\end{exercise}

\vspace{20pt}
\item  \begin{exercise}[Map Out Your Collective (Secondary Source Search)]
\end{exercise}
\end{itemize}
\subsection{Rewriting for Your Collective}
\begin{itemize}
\item \begin{exercise}[Find and Replace All ``Insider Language"]
\end{exercise}
\end{itemize}
\subsection{Organizing Your Field into Problem Collectives}
\begin{itemize}
\item \begin{exercise}[Start Your Own ``What’s Your Problem?" Bookstore]
\end{exercise}

\vspace{20pt}
\item  \begin{exercise}[Change Their Variables]
\end{exercise} 
\end{itemize}
\subsection{Rewriting for Your Field}
\begin{itemize}
\item \begin{exercise}[Rewrite for Your Field]
\end{exercise} 
\end{itemize}
\subsection{Formal Drafting and Revision}
\begin{itemize}
\item \begin{exercise}[Create ``Draft 0"]
\end{exercise}

\vspace{20pt}
\item \begin{exercise}[Move from 0 to 1]
\end{exercise}
\end{itemize}
\subsection{What is Next?}
\begin{itemize}
\item \begin{exercise}[Find a New Problem and Start a New Project]
\end{exercise}

\vspace{20pt}
\item \begin{exercise}[Help Someone Else]
\end{exercise}
\end{itemize}
\section{Sounding Board and Research Network}
\subsection{Building Research Network}
\begin{itemize}
\item  \begin{exercise}[Start Building Your Research Network]

Start building your research network -- a community of people you can consult with and seek advice from during the research process. 
\begin{enumerate}
\item Make a list of teachers, colleagues, students, and fellow travelers you think would be willing and available to discuss ideas with you on a periodic basis. 

\item Circle a couple of names on your list of potential Sounding Boards.

\item Choose a few of the questions you’ve generated while reading this chapter, and make a meeting to discuss them. 
\begin{itemize}
\item  Tell them you’re not trying to settle on a research question just yet. 
 
\item  The goal is to get on their radar, and to start \textbf{the process of communicating} about your research ideas \textbf{orally} -- since you’ve already done some writing.

\item  You could send them your questions in advance, but strive to make it a casual conversation.

\item Spend some time generating questions about a topic together.
\end{itemize}

\item And say Thank you. You may well be seeking them out again.
\end{enumerate}
\end{exercise}
\end{itemize}
\subsection{Identifying Primary Sources}
\begin{itemize}
\item  \begin{exercise}[Get Leads on Primary Sources]
\begin{itemize}
\item When you are searching for your Problem, or verifying that the problem you’ve been working with is the right one for you, it might still be \textbf{too early} to talk to a Sounding Board about your assumptions. 

\item What your Sounding Board can help you with at this stage is \textbf{finding primary sources} that you can use to educate your questions. 
\end{itemize}

\textbf{Describe those exercises} to your Sounding Board, and \textbf{ask for suggestions} of other databases or archive catalogues or \textbf{repositories} of \textbf{primary sources} you might use.
\end{exercise}
\end{itemize}
\subsection{Project Planning Consulting and Proposal Sharing}
\begin{itemize}
\item \begin{exercise}[Is Your Decision Matrix Complete?]
\end{exercise}

\vspace{20pt}
\item \begin{exercise}[Share Your Proposal with a Trusted Mentor (Who Understands How Preliminary This Is)]
\end{exercise}
\end{itemize}
\subsection{Evaluation of Proposal from Outsider}
\begin{itemize}
\item \begin{exercise}[Does the Lay Version of My Proposal Make Sense?]
\end{exercise}
\end{itemize}
\subsection{Finding a Sounding Board in Your Field}
\begin{itemize}
\item \begin{exercise}[Find a Sounding Board in Your Field]
\end{exercise}
\end{itemize}
\subsection{Being Your Own Sounding Board}
\begin{itemize}
\item \begin{exercise}[Talk to Yourself]
\end{exercise}
\end{itemize}
\newpage
\section{Introduction}
\subsection{Self-Centered Research: A Manifesto}
\begin{itemize}
\item In this book, we advocate a ``self-centered" approach to research. 

\item \emph{\textbf{Self-Centered Research}} is the following:
\begin{itemize}
\item A \emph{\textbf{practice}} of research that emphasizes the importance of setting out on the research journey from exactly where you are right now, and maintaining close contact with your own self -- your instincts, your curiosities, and your biases-- throughout the process.

To be a ``self- centered" researcher is to maintain your center of gravity over your own two feet at all times, and to \emph{\textbf{avoid pursuing topics and questions} that you \textbf{imagine} might \textbf{please} some \textbf{imaginary, external judge}}.

\item An \emph{\textbf{ethic}} of research that involves \emph{\textbf{consciously} \textbf{acknowledging} and \textbf{assessing} your \textbf{abilities} and your \textbf{limitations}} as a researcher. It involves being \emph{centered}: \emph{\textbf{knowing} who you are}, \emph{\textbf{listening to your own instincts}}, \emph{\textbf{trusting} them} even when they sound naive or inarticulate, and \emph{\textbf{refining} them during the research process}.

\item A \emph{\textbf{state}} of mind that \emph{\textbf{affirms the value} of your ideas}, \emph{assumptions}, and concerns in shaping your agenda and \emph{\textbf{the direction of your research}}. It presumes that the better (and faster) you figure out your own concerns and motivations as a researcher, the better (and faster) you will discover a research problem that is deeply meaningful both to you and to the world at large. But the first person who must be deeply concerned with your research problem is you, the \emph{researcher}.
\end{itemize}

\item \emph{Self-Centered Research} 
\begin{itemize}
\item \emph{does not} mean \emph{unleashing (or inflating)} your \emph{ego}. Being self-centered is not being self-absorbed, self-obsessed, self-congratulatory, self-consumed, self-indulgent, self-involved, self-serving, or self-ish.

Quite the opposite: self-centered researchers are \emph{\textbf{self-reflexive}}, and always \emph{\textbf{self-critical}}; \emph{honest} and \emph{probing} about their own interests, motivations, and abilities; but also \emph{\textbf{open}} and \emph{\textbf{confident}} enough to \emph{assess} the validity of others’. This means having the wherewithal to \emph{\textbf{challenge}} received wisdom, including unfounded ideas you are probably carrying around without realizing it.

\item is also \emph{\textbf{not autobiographical}}.

It does not imply that the papers, articles, reports, or books you write will tell the story of your life. Or that every documentary you produce, or painting you paint, will be a self-portrait.
\end{itemize}

\item \emph{\textbf{The end goal}} of the Self-Centered Research process is, just like conventional research processes, one in which the researcher produces empirical, grounded, theoretically informed, and compelling scholarship about some aspect of the world around us, and does so in a way that \emph{changes} how \emph{other} people think. 

\item In order to identify and solve a problem that truly matters to other people, however, the \emph{Self-Centered Research process} insists that \emph{this problem must matter, first and foremost, to \textbf{you}}.
\end{itemize}
\subsection{Centered Research Is the Best Research}
\begin{itemize}
\item \emph{\textbf{Where to begin?}} The answer is: \emph{\textbf{Exactly where you are}, \textbf{right now}}.

\item Core to this book are two propositions. 
\begin{itemize}
\item First, research can be a \emph{life-changing experience}, if you get a few things \emph{right} from the \emph{start}. 
\item Second, the most important part of beginning a research project is \emph{\textbf{finding your center}}. 
\end{itemize}

\item Research is a process \emph{not just} of \emph{solving problems} but of \emph{\textbf{finding problems that you}} -- and other people -- \emph{\textbf{didn’t know existed}}.

It’s a process of \emph{discovery}, \emph{analysis}, and \emph{creation} that can \emph{generate its own momentum} and \emph{create a virtuous cycle of inspiration}.

Deep-seated problems only reveal themselves through \emph{\textbf{self-trust}}, \emph{\textbf{exposure} to \textbf{primary sources}}, and \emph{\textbf{time}}. 

\item \emph{\textbf{Only you}} -- not anyone else -- can tell you \emph{what you were meant to research}. Answering the question ``What to research?" requires a \emph{hard look in the mirror}.

\item The \emph{\textbf{goal}} of this book, then, is 
\begin{itemize}
\item to help you \emph{\textbf{create the ideal conditions}} to start a fire in your mind.

\item But at the same time, it will show you how to \emph{\textbf{maintain balance and clarity}} in situations of complexity, uncertainty, and ambiguity. 

\item And it will teach you ways to tell the difference between \emph{\textbf{unproductive uncertainty}} -- that is, when you’re \emph{\textbf{on the wrong path}}, and should probably turn back -- and \emph{\textbf{productive uncertainty}} -- that is, when it may feel like you’re lost, but where your \emph{\textbf{inner instinct}} and \emph{\textbf{wisdom}} are \emph{encouraging} you to keep on going.
\end{itemize}
\end{itemize}
\subsection{How to Use this Book}
\begin{itemize}
\item \emph{\textbf{Try This Now}}

In each chapter, you will work through practical exercises and games designed to help you achieve a specific set of goals: 
\begin{itemize}
\item generating questions, 
\item refining questions, 
\item discovering the patterns that connect your questions, and 
\item identifying the problem that motivates you. 
\end{itemize}

\item All of the exercises rely on a core set of principles. These include
\begin{itemize}
\item \emph{\textbf{attentive}}, \emph{\textbf{nonjudgmental self-observation}};
\item giving oneself permission and \emph{encouragement} to say \emph{inarticulate, tentative, and vulnerable things} out loud;
\item getting things \emph{\textbf{down on paper}}.
\end{itemize}

\item Whether or not you tackle everything in sequence on your first pass, the only way to get the benefit of our advice is by completing the exercises, and, as mentioned above, by \emph{\textbf{writing things down}}.

\item The point of all this \emph{continual writing} is to \emph{produce} what we term ``evidence of self," or ``\emph{\textbf{self-evidence}}."

You can think of self-evidence as clues that will help you figure out the answers to the most important questions that a researcher must answer during this early phase:
\begin{itemize}
\item \emph{Why am I concerned with this topic? }
\item \emph{What is it about this subject that I think holds the key to some larger issue? }
\item \emph{Why does this primary source jump out at me? }
\item \emph{Why, out of all possible topics that I could be working on, do I keep coming back to this one? }
\item \emph{What is my Problem?}
\end{itemize}

\item \emph{\textbf{Self-evidence}} is a valuable form of \emph{note-taking} that we believe many researchers neglect.

\item We advocate making \emph{\textbf{introspection}} a \emph{habitual part} of your research method. 

\item The pieces of \emph{self-evidence} you produce during \emph{the Self-Centered Research} process are cousins to the kind of \emph{notes} experienced researchers routinely make when they read primary sources, conduct interviews, carry out ethnographic fieldwork, or copy down bibliographic information.

We call them \emph{\textbf{self-evidence}} because, \emph{\textbf{during this early phase of research}}, \emph{\textbf{these notes} will possess a \textbf{value}} that goes far \emph{beyond} the \emph{recording} of facts, quotes, observations, and other evidence about the world around you. They will \emph{\textbf{provide evidence about you yourself}}. 

With these clues you will be able to \emph{uncover the \textbf{hidden} questions and problems} you carry around inside you. 

Discover them early in the research process and not only will you \emph{save yourself time and frustration}, but, more importantly, you will be more likely to \emph{arrive at the research project that is \textbf{right for you}}.

\item \emph{\textbf{Commonly Made Mistakes}}

A list of these follows each ``\emph{Try This Now}" exercise. Most of these mistakes fall into one of three categories:
\begin{itemize}
\item Not letting yourself be \emph{\textbf{vulnerable}}
\item Not \emph{\textbf{listening to yourself}}
\item Not \emph{\textbf{writing things down}}
\end{itemize}

\item \emph{\textbf{Sounding Board}}
 
A \emph{\textbf{Sounding Board}} is someone who helps you to gain \emph{alternative perspectives} on your ideas and writings and to \emph{step outside yourself}. 
\begin{itemize}
\item A \emph{Sounding Board} helps you to \emph{\textbf{self-reflect}} and \emph{make better decisions}, so we recommend that you make \emph{\textbf{talking to someone you trust}} a habit early in the research process. 

\item Ultimately, \emph{the Self-Centered Research} process will empower you to \emph{\textbf{become your own Sounding Board}}.
\end{itemize}

\item Well-meaning suggestions from a teacher, adviser, or other \emph{authority} figure -- suggestions as to what you ``could" or ``should" work on -- can have a \emph{major impact} on a researcher during the early phases of research. If you feel lost, or uncertain about the value of your nascent ideas, these suggestions can feel a lot like a \emph{\textbf{command}}. Or it may become your fallback, your ``Well, I can’t come up with anything better, so I might as well go with that!" 

What if you \emph{\textbf{skipped}} all that messy \emph{\textbf{introspection}} and snapped up the ready-made idea that your trusted adviser has told you is important? Unfortunately, the effect can be \emph{\textbf{inhibitory}} and \emph{\textbf{counterproductive}}.

\item The point of research is \emph{not to fall back}, it’s to \emph{\textbf{move forward}} -- to take a risk and discover or create something original. 

\item A \textbf{mentor} can \emph{\textbf{offer advice}} that \emph{\textbf{saves you from retracing others’ paths}} to the same conclusion. 

But when a student comes with an idea for a research project and asks, ``Is this what you want?" a true mentor’s response is always
the same: ``\emph{\textbf{Is this what you want?}}"

\item In our experience, if a research question is \emph{not} one that you’re \emph{truly motivated} to spend your time answering, you’ll find it a \emph{challenge} to do a good job, or even to finish. 
\end{itemize}

\subsection{Introversion, First. Extroversion Second}
\begin{itemize}
\item The \emph{two-part process} of starting a research project involves
\begin{itemize}
\item  looking first \emph{\textbf{inward}} and
\item   then \emph{\textbf{outward}}.
\end{itemize}

\item 
\begin{itemize}
\item Part 1 takes you through the \emph{\textbf{inward-focused process}} of becoming a self-centered researcher.  You will \emph{\textbf{reflect}} on the experiences, interests, priorities, and assumptions you bring with you -- and assess how to make best use of them in charting out a research direction. 

This process goes beyond \emph{conventional \textbf{brainstorming}} because it requires \emph{taking stock of your \textbf{values}}. It involves \emph{\textbf{distinguishing}} between
\begin{itemize}
\item \emph{\textbf{what doesn’t matter to you}}, 
\item \emph{\textbf{what you think matters to you}}, and 
\item \emph{\textbf{what really matters to you}}. 
\end{itemize} We believe that you are best off \emph{starting} this process \emph{before} you field-test your ideas against the wisdom of the research community. 

\item Part 2 focuses on this process of \emph{\textbf{extroversion}}. It helps you to navigate the often bewildering process of coming to terms with the research communities conventionally known as ``\emph{fields}" and ``\emph{disciplines}," as well as how to identify researchers who may not be in the same field as you but \emph{\textbf{who are interested in similar problems}} -- what we call your \emph{\textbf{Problem Collective}}.
\end{itemize}

\item \begin{exercise}  Writting Exercise -- \emph{\textbf{Try This Now}}

All of the writing you do with (and in) this book will help the research process by
\begin{itemize}
\item creating an \emph{\textbf{evolving record}} of your ideas— your ``\emph{\textbf{self-evidence}}";
\item \emph{\textbf{continually} \textbf{externalizing} your thoughts}, as an aid to memory and to your research collaborators;
\item building your project \emph{step-by-step} through different types of writing, \emph{focused} on discrete aspects of the early phases of research;
\item making writing a \emph{regular research habit}
\end{itemize}
\end{exercise}
\end{itemize}

\section{Become a Self-Centered Researcher}
\begin{itemize}
\item Part 1 of this book guides you through the process of centering your research questions, of aligning them with the concerns that you carry inside you.

\item You will \emph{not be writing about yourself}, but rather from yourself, instead of \emph{from external sources}. This is a \emph{process of self-reflective decision-making} that is crucial at the inception stage of a research project.

\item The \emph{\textbf{goal}} of this stage is to make sure that you are fully aware of your own \emph{motivations} and \emph{values}, are confident of your \emph{priorities}, and have taken stock of your \emph{assets}, \emph{capabilities}, and \emph{limitations}. 

\item Part 1 teaches you how to avoid the ever- present risk of outsmarting yourself.

\end{itemize}

\subsection{Questions}
\begin{itemize}
\item This chapter helps you navigate the first challenge you will face in your research process: 

\emph{How do you transform broad and vague ``\textbf{topics}" of interest into a set of concrete and (for you, at least) fascinating \textbf{questions}?} 

\item In the earliest phases of research, most people don’t have specific questions in mind. They have \emph{topics of interest}. 

\item The main challenge is not identifying potential topics of interest, but in moving from these \emph{\textbf{generic topics}} to a \emph{s\textbf{pecific set of questions}}. 
\end{itemize}
\subsubsection{A Topic Is Not a Question}
\begin{itemize}
\item A \emph{\textbf{topic}} suggests a \emph{field} or \emph{scope of inquiry}. 

\item  Having a topic makes one feel \emph{solid}, \emph{self-aware}, \emph{oriented}.

\item Topics can be deceptive, however. They are immense and \emph{\textbf{abstract}} categories.

\item But their use to the researcher is limited for one very obvious reason: \emph{\textbf{a topic is not a question}}.

\item You can see already how topics can even be \emph{\textbf{obstacles}} to the research process.  Simply put, when we speak about
topics, we could be speaking about \emph{anything} (and thus \emph{nothing}) at all.

\item By themselves, topics are not very good guides for the research process. That’s why they can be dangerous.

\item When you have a \emph{topic} and are struggling to \emph{turn it into a project}, the common advice you will hear is ``\emph{\textbf{Narrow it down.}}"

\item We call this \emph{\textbf{the Narrow-Down-Your-Topic Trap}}.

A more discrete scope that reduces the volume of sources you need to analyze can, to be sure, \emph{answer the \textbf{when} and \textbf{where} questions}. But \emph{a topic alone} -- even a ``narrow" one -- is \emph{\textbf{insufficient}}, because it still leaves \emph{unanswered the how and why questions}. Tell someone your ``narrow" topic, and they may still have \emph{no clue what you’re doing}. \emph{\textbf{Even a ``narrow" topic cannot tell you what to do}}.

Simply put, you cannot ``narrow" your way out of \emph{Topic Land}.

\item Every researcher needs to figure out \emph{\textbf{what to do}} and \emph{\textbf{how to do it}}. 

the question that comes before \emph{what} and \emph{how} is \emph{\textbf{why}}.

\item Besides, ``gaps" in human knowledge are infinite. \emph{Why fill this particular gap?}

\item Most researchers (even seasoned ones) instinctually try to justify their incipient research ideas using the \emph{vocabulary} of \emph{``\textbf{importance}"} or \textit{``\textbf{significance}"} -- \emph{as defined by an \textbf{imaginary, external judge}}. 

\item \emph{Out of the infinite number of potential topics of interest, why am I drawn to this one?} 

\emph{If I had to guess, what is my connection with this topic?} 

\emph{Why is it so magnetic to me?}

\item Here’s how we’d \emph{\textbf{phrase}} them for a researcher trying to \emph{move from a \textbf{topic to questions}}:
\begin{enumerate}
\item \emph{\textbf{Make yourself vulnerable.}} The questions one generates during this early phase are \emph{\textbf{not final products}}.

at this stage, our questions \emph{don’t need to be polished} or even \emph{coherent}. All they have to be is \emph{\textbf{honest}}, \emph{\textbf{to the best of our knowledge}}. \emph{\textbf{Trust} yourself}.

\item \emph{\textbf{Keep the conversation affirmative and nonjudgmental.}} \emph{Neither} the researcher \emph{nor} the \emph{Sounding Board} said anything to \emph{denigrate} the researcher’s \textit{assumptions} about rationality.

At the \emph{brainstorming} stage, it’s easy to shut down lines of inquiry \emph{prematurely}. Or perhaps by chiding oneself with high-level language. \emph{Resist the temptation}. Far better is simply to \emph{allow the questions to proliferate}, no matter how seemingly \emph{unimportant}, \emph{naive}, incoherent, scattered, or biased they might seem.

Whether you’re working alone or with someone else, the goal at this point is \emph{\textbf{simply to generate questions}}.

\item \emph{\textbf{Write down your ideas.}} The researcher and \emph{Sounding Board} \emph{wrote down all the questions as they spilled out}.

As we will emphasize again and again, during this early phase of research, thinking about things is not enough. You need to \emph{\textbf{get things down in writing}}, to create \emph{\textbf{traces of thought}} that you can later use for other purposes.

\item \emph{\textbf{Generate questions internally. }} The questions you should be aiming at now are those \emph{\textbf{driven by your own knowledge, assumptions, and curiosities}}. 

At this point, don’t try to think from the ``outside in" by trying to \emph{generate questions} you think might \emph{\textbf{satisfy some imaginary judge}}.
\end{enumerate}

\item For most of us, the challenge is greater. We might be drawn to a particular topic without having any idea why. Or, perhaps more accurately, \emph{\textbf{some part of us knows why}, \textbf{but the rest of us}} -- the part of us that has to field questions like ``Why does that interest you?" -- \emph{\textbf{still has absolutely no idea}}.

\item You will learn how to bring together
\begin{itemize}
\item the \emph{\textbf{intuitive}} part of you that \emph{\textbf{knows}}, but \emph{\textbf{cannot speak}};
\item  the \emph{\textbf{executive}} part of you that \emph{\textbf{speaks}}, but \emph{\textbf{does not know}}.
\end{itemize}

\item \emph{\textbf{Questions lead us in specific directions}} -- whether toward \emph{\textbf{specific answers}} or to \emph{\textbf{primary sources}} that we need to answer the questions or to the work of fellow scholars who are grappling with similar questions (i.e., secondary sources) or, more often than not, to more and better questions.

\item Questions have another virtue. Every question a person asks about the world is \emph{a piece of ``\textbf{self-evidence}" \textbf{about the researcher}} -- evidence that helps the researcher \emph{\textbf{reflect} on their own intellectual, emotional, and personal motivations} for asking the question in the first place.

\item The \emph{goal} here is to \emph{\textbf{explain}}, rather than simply assert, \emph{\textbf{one’s interest in a topic}}.

They require you to ask \emph{probing questions} about yourself

\item \begin{exercise}[Search Yourself] (check in early sections)

The \textbf{goal}: To use a list of \textbf{primary-source search results} to figure out \textbf{the aspects of your topic} that \textbf{most interest you}, and \textbf{draft questions} based on these interests.

\begin{enumerate}
\item 
\item 
\item On most of these sites, you won’t be able to view the original source, only the catalogue entry. But even if a site does offer full text results, try not to get caught up in any one source for too long at this point. This is not yet the time for close reading.

Instead— and this is key— while scrolling through your search results, try to imagine that you are strapped to an EKG machine that is recording the electrical pulses going through your system as you read. 

Which primary sources raise your heart rate, even slightly?  Write them down.
Which ones have no effect on you one way or another?  Take note of them too (since, a bit later on, we will also be taking stock of things that bore you!).

The goal right now, as we said above, is to ``\textbf{read yourself}" as you read other things.

Why bother? How does this get a researcher any closer to discovering their research direction? 

Given how efficient we are at ignoring \emph{stimuli}, it follows that when we do take notice of something -- however small or insignificant -- we should \emph{take notice that we’re noticing}. This form of \emph{self-evidence} gives a potential clue about our underlying concerns and curiosities.

Put plainly, whenever your mind takes notice of something -- anything -- you can be certain that there is a question there, even if you are not sure what that question is.

Learn to \textbf{pay attention to these clues}, and then to uncover the questions whose presence they indicate, and you’ll be able to move quickly and effectively from generic topics to precise and generative questions.

``Noticing what you are noticing" can be surprisingly difficult. 

\item Go back to your search results. Write down, circle, or asterisk the ones that seem to \textbf{have any effect on you}, however small. 

Write a list out by hand, copy and paste the titles of the sources into a text file, or click a checkbox to save those sources in a folder or email. However you choose to do it, take notes.

\item  Once you have an initial list of at least ten items, take thirty minutes or so to ask yourself three questions
about each item, setting down your answers in writing:
\begin{itemize}
\item What does this make me think of?
\item  If I had to venture a guess, why did I notice this one?
\item  What questions come to mind for me when I look at this search result?
\end{itemize} 

Your only audience is you, so allow yourself to be inarticulate, instinctual, and honest.

\item 
\item  
\end{enumerate}

Common \textbf{Mistakes}:
\begin{itemize}
\item Not writing things down
\item Getting bogged down in \textbf{individual sources too soon}
\item Excluding ``fluke" search results that \textbf{seem unrelated} to the keywords you entered in the database or unrelated to your topic
\item Feigning interest in a search result that seems ``\textbf{important}," even if it doesn’t really interest you
\item Only registering interest in search results for which \textbf{you think you know why} you’re interested in them, instead of being more \textbf{inclusive}
\item Trying to make a list of noticings that is \textbf{coherent} and fits together
\item When speculating about why a search result jumped out at you, worrying about whether or not the reason is ``\textbf{important}," based on some \textbf{imagined external standard}
\end{itemize}
\end{exercise}

\item  \begin{exercise}[Let Boredom Be Your Guide]

The \textbf{goal}: To become \textbf{attentive} to your \textbf{active dislikes}, identifying questions that you ``should" (in theory) be interested in based on your topic of interest, but aren’t. By \textbf{understanding what you don’t care about} regarding your topic, you accelerate the process of figuring out what you do care about.

After all, the most common reaction human beings have to \textbf{boredom} is \textbf{avoidance}. We try to dismiss or ignore things that bore us.

\textbf{Boredom is a powerful teacher}, and \textbf{deserves our attention}. Boredom is not the same thing as disinterest or lack of interest. It is not a passive experience. 

Boredom is an \textbf{active sentiment}, a \textbf{rejection} of something that, like excitement, provides you with more self-evidence through which you can understand your concerns and motivations more clearly. 

By \textbf{taking note of your boredom} -- in precisely the same way you just did with your excitement -- you will gain clues about what your real research questions and problems might be.

Ask yourself: 
\begin{itemize}
\item What about your chosen topic bores you? 

\item Among the potential questions or subtopics that derive quite naturally and obviously from your stated topic, which ones repel you, perhaps even unnerve you?
\end{itemize}

Taking account of your boredom is part of your conversation with your research-self. Besides helping the process of \textbf{elimination}, steering you away from \textbf{unprofitable lines of inquiry}, boredom can also help you to ask better questions and zero in on your Problem.
\end{exercise}
Common \textbf{Mistakes}:
\begin{itemize}
\item Denying boredom, or feigning interest in something because you feel it’s ``on topic" and demands your interest because it’s ``important."
\item Engaging in circular logic. 

Boredom, like inspiration, is a dynamic process that happens between you and whatever it is you’re interacting with. The sensation of boredom is the by-product of reactions between the substance that makes you you, and the substances of the reality you’re encountering.
\end{itemize}


\item \begin{exercise}[Go Small or Go Home]

The \textbf{goal}: To generate \textbf{specific, fact-focused questions} about your topic before you’ve done in-depth research. These will lead to bigger questions later on.

Try to \textbf{avoid} posing questions that strive to be \textbf{profound} or too \textbf{big-picture}. If you find yourself asking questions about the essential ``meaning" or ``significance" of your topic, chances are you are thinking too abstractly.

Remember, too that question means question -- \textbf{with a question mark} -- and not a statement or sentence fragment masquerading as a question.

Again, your goal here is not to \textbf{justify} the significance of your project \textbf{to someone else}. You need to start with questions about \textbf{basic facts}. 

\textbf{Specificity} is the goal at this point, for two reasons
\begin{itemize}
\item First, it is only through small questions like these that you can begin to \textbf{form a picture in your mind} (and in your notebook) about the \textbf{core fundamentals} of the topic you are researching. 

To try and answer ``profound" questions at this point -- about ``meaning" and ``significance" -- is \textbf{premature}, since you don’t yet have the facts, much less the opportunity, to analyze them.  

By contrast, the more facts you know the greater command you begin to have of your subject matter. This, in turn, prepares you for the process of asking ``bigger" questions -- ``profound" questions -- when the time is right.

\item Second, lurking in one or more of those ``small" questions may be an \textbf{unexpected question} that could, when you hear yourself ask it aloud, send your research off into an entirely \textbf{new direction}. 
\end{itemize}

When you begin to ask (and then to answer) precise and seemingly \textbf{mundane} questions like these, you begin to \textbf{liberate yourself} from the confines of \textbf{vague} and \textbf{unproductive} ``\textbf{topics}," moving instead toward specific and coherent clusters of questions that will, over time, add up to something compelling, \textbf{open-ended}, and \textbf{doable}.

Asking \textbf{precise factual questions} is one \textbf{key} to \textbf{escaping} \textbf{Topic Land}.
\end{exercise}
Common \textbf{Mistakes}:
\begin{itemize}
\item Asking vague, grand, abstract, or big-picture questions about ``meaning" or ``significance,“ instead of specific and \textbf{precise factual questions}
\item Not asking actual questions (with a question mark), but instead writing statements or sentence fragments -- \textbf{topics masquerading as questions}
\item Not asking a question because \textbf{you think you couldn’t answer it}, perhaps because you think that the data doesn’t exist or is unattainable
\item Asking \textbf{too few questions}, resulting in an inadequate quantity of self-evidence
\end{itemize}
\end{itemize}

\subsubsection{You Have Questions}
\begin{itemize}
\item Most importantly, in formulating these possible research questions, you’ve set aside for the time being any \emph{concerns} about \emph{whether or not your questions are Important}, with a capital \emph{I}.

\item Your list of questions contains questions that \emph{\textbf{matter to you}}, even if you don’t know why yet. 

\item As a bonus, you also have \emph{an initial set of primary and secondary sources} from your database searches.
\end{itemize}

\subsection{What's Your Problem?}
\begin{itemize}
\item In this chapter, you will begin to \emph{find and use primary sources}, and you may find the answers to at least some of your questions. But answering questions will not be the primary focus. \emph{\textbf{Educating your questions} will be the focus}.

\item It is a fundamental part of this stage of research: your questions are \emph{underdeveloped} at this point because you have not yet had the chance to conduct research into your subject matter.

\item it is true that it takes a lot of research to \emph{\textbf{arrive at the right questions}}.  And then it \emph{takes more research} to \emph{\textbf{answer these questions}} and \emph{\textbf{generate new ones}}. 

In this early stage of research, \emph{\textbf{the goal is not}, as many assume, \textbf{the generation of answers}}. It is about \emph{the \textbf{refinement} of your existing questions} and \emph{the \textbf{generation} of \textbf{new} (and better) ones}.

\item The goal of this chapter is to help you \emph{\textbf{identify and articulate the problem}} underlying your many research questions.
\end{itemize}

\subsubsection{Don’t Jump to a Question (or You’ll Miss Your Problem)}
\begin{itemize}
\item Here’s a simple way to \emph{distinguish} a \emph{problem} from a random set of \emph{curiosities}:

 \emph{if it \textbf{changes} by the day, week, or month, chances are it’s a \textbf{passing curiosity}. If it \textbf{endures}, it just might be a \textbf{problem}.}
 
\item A problem is a nagging presence within you -- one that \emph{disturbs}, \emph{bothers}, and \emph{unsettles} you, but also \emph{attracts}, \emph{compels}, and \textbf{\emph{keeps you coming back}}. 

\emph{A problem is something that \textbf{follows you around}}. it calls out for you to try to solve it.

\item Your job is to give that problem a \emph{\textbf{name}}, to \emph{\textbf{identify}} a \emph{\textbf{case} of that problem} that you will be able to study (given your personal abilities and constraints), and to figure out \emph{\textbf{how} to study that case} so that you might arrive at a \emph{broader solution}.

\item Researching a problem requires asking questions, of course, but (again, to state the obvious) \emph{\textbf{a question is not a problem}}.

\item You can think of plenty of questions that have answers, but whose answers do not solve any problem.  So you want to make sure that \emph{your questions are indeed \textbf{problem-driven}}. This is why it is so important not to jump to a question.

\item A \emph{\textbf{problem}} has several \emph{\textbf{functions}} for the researcher, among them the following:
\begin{itemize}
\item It \textbf{\emph{motivates}} you to \emph{ask questions about your topic}.
\item It \emph{determines} \emph{\textbf{which}} questions you ask.
\item It \emph{\textbf{defines}} the \emph{what/why/when/how} of \emph{your engagement with your topic}.
\item It \emph{\textbf{guides}} the \emph{path of your \textbf{inquiry}}.
\item It \emph{\textbf{shapes the story}} you tell when the time comes to share your research results.
\end{itemize}

\item The next steps in this chapter will help you figure out
\begin{itemize}
\item  how to \emph{improve the questions} you have already generated;
\item  how to use sources to \emph{identify the problem} motivating your questions;
\item  how to use your Problem to \emph{generate new and better questions}.
\end{itemize}

\item What the early-stage researcher has to \emph{\textbf{avoid}} is \emph{\textbf{jumping to a question}}. You have generated many questions, and the risk now is that you’ll feel pressured to jump ahead and \emph{\textbf{choose one prematurely}}.

\emph{The Jump-to-a-Question Trap} can be as harmful as \emph{the Narrow-Down-Your-Topic Trap}.
\end{itemize}

\subsubsection{Stress-Testing Your Questions}
\begin{itemize}
\item Now that you have done the work of producing a multitude of questions --  small, factual questions, ideally -- you will still need to \emph{\textbf{stress-test}, \textbf{refine}, and winnow them out}, \emph{removing any dead ends}, \emph{enhancing the rest}, and \emph{adding additional questions} that will better serve your research process.

\item Here are \emph{two ways} to stress-test your questions and improve their soundness. 
\begin{itemize}
\item The first \emph{\textbf{focuses on language}}; 

\item the second is subject-specific and \emph{\textbf{focuses on sources}}.
\end{itemize} We recommend that you tackle them in that order

\item \begin{exercise}[Run a Diagnostic Test on Your Questions]

The \textbf{goal}: To ensure that the vocabulary, grammar, and phrasing of your questions are \textbf{specific} and \textbf{unprejudiced} so that they do not presume a certain outcome.
\begin{enumerate}
\item have you phrased them in more general, and vaguer, terms like 
``This is an examination of .... ,"
``I plan to explore ... ," or 
``My project is about the question of ... "? 

 If you find yourself articulating your questions as ``I want to examine how" something happened, there is a fair chance that what you have are not really questions at all, but rather \textbf{topics disguised as questions}. 

 \item Do your questions rely on broad, generic, imprecise, or sweeping adjectives Try to cut such adjectives and adverbs out entirely.
 
\item Do your questions depend upon collective nouns? If so, do your best to replace these nouns with more precise demographic categories
 
 \item Do your questions contain verbs like ``influence," ``affect," ``shape," or ``impact" or passive constructions such as ``was affected by," ``responded to," or ``reacted to"? In such cases, chances are high that you are building your questions in such a way that they \textbf{rule out} an
entire set of \textbf{possible} answers and outcomes. \textbf{Rephrase to avoid presumptions} that could result in confirmation bias.
\end{enumerate}
\end{exercise}

\item By the end of this process, your questions should meet these criteria:
\begin{itemize}
\item \textbf{\emph{They should be clear, precise, and jargon-free}}. If your questions are too hard for a colleague or mentor to understand, this means that you (and not they) still don’t get what your Problem really is.

\item They should be \textbf{\emph{rooted in verifiable and falsifiable data}}. Your research questions should have \textbf{integrity}. This means that they should be inspired by fact, rather than by speculation, prejudice, or opinion. 

\item They should be \emph{\textbf{indifferent to the outcome}}. A research question \emph{should not presume a certain answer}. If yours does, rewrite it to \emph{eliminate that presumption}.

\item They should be \emph{\textbf{clear} about the \textbf{subject}}. Your questions should not be reliant on broad categories of identity. Be as
specific as you can be about the \emph{who} in your question.

\item They should be \emph{\textbf{raw}} and \emph{\textbf{undisciplined}}. At least for now.

If the questions seem \emph{random} to you, let them be \emph{random}. If they seem \emph{unrelated} to one another, let them be \emph{unrelated}.
\end{itemize}

\item \emph{\textbf{Leading questions}} are so \emph{common} and so \emph{prejudicial} to the research process that it’s worth looking at one extended example. 

What you want to avoid is building your questions in such a way that you actually \emph{\textbf{need} this ``influence" to exist} in order for your questions to be \emph{viable}. Almost inevitably, you will end up discovering \emph{\textbf{specious} ``proof" of influence} in primary source material, misleading both your readers and yourself.

\item \begin{exercise}[Use Primary Sources to Educate Your Questions]
\begin{itemize}
\item The \textbf{goal}: To learn how to \textbf{run keyword searches} designed to \textbf{enhance} or ``educate" the \textbf{questions} you are asking about your topic. 

These searches \textbf{uncover primary sources} relevant to your research that themselves contain \textbf{new keywords} you were previously unaware of (thereby enabling you to run follow-up searches to reveal even more, and more useful, primary sources).

\item This next exercise requires you to delve back into your \textbf{specific subject matter} and into \textbf{primary sources}. Rather than trying to use primary sources to start answering the questions you’ve come up with, we want you to use them to \textbf{develop}, \textbf{refine}, and \textbf{expand those questions}.

\item How do we use primary sources to strengthen and ``educate" our questions? The answer is simple: primary sources \textbf{alert} you the \textbf{existence of other primary sources}, exposure to which helps you \textbf{ask more mature questions} about your subject.

\item In this earliest stage of finding primary sources, then, your main goal is actually not to start answering your questions, but to use the primary sources you do find to \textbf{reveal new keywords} that you did not know existed --  keywords that you can then feed back into the search process, in order to \textbf{uncover more and better primary sources}, \textbf{more and better keywords}, and most importantly of all, \textbf{more and better questions}.

\item Should you really be expected to read, notate, and cite even more sources?  \textbf{Not necessarily.} you’re \textbf{improving} your grasp of your topic by \textbf{eliminating blind spots}.

Whenever you do a keyword search, ask yourself:
\begin{itemize}
\item Are there other search terms I should be using?
\item Might there be different spellings of the search terms I already have?
\end{itemize} 

\item You need to be as \textbf{confident} as possible that the search results you are getting are \textbf{broadly representative and reflective} of available primary sources, and not the by-product of narrow or unrefined searches. 

\item In some cases, you might be fortunate to find a primary source, like a historical dictionary, that explicitly addresses the shifting nomenclature surrounding the very topic the primary source is about, outlining for you the varied ways a given idea, place, community, practice, or the like has been named and renamed across time and space. 

Even in such cases, however, remember that a primary source still is subject to its own \textbf{limitations}. But for your current purpose of finding more \textbf{generative keywords}, the source can be useful to you whether or not its data or conclusions are accurate, so you can defer judgment on those questions for the time being. 

The goal for now is to determine if this source will lead you in the \textbf{direction of further primary sources} that you wouldn’t have been able to find otherwise.

\item As you continue to use primary sources to further ``educate" your questions, two other helpful things will inevitably happen: 
\begin{itemize}
\item you will end up \textbf{answering} some of your questions along the way, 
\item and you will find that some are \textbf{not} actually worth answering.
\end{itemize} In other words, you will discover that some of your initial questions can be \textbf{scrapped}. This is precisely what you want to happen.

\item as you learn more, your \textbf{instincts} regarding your subject matter improve. In ``educating your questions" you are \textbf{educating your instincts}.
\end{itemize}
\end{exercise}

\item \begin{exercise}[Make Your Assumptions Visible]
\begin{itemize}
\item The \textbf{goal}: To become aware of the assumptions you bring to your research project and use them to identify the problem that motivates your research questions.

\item Now that you’ve analyzed your questions using the two techniques described above, there is still one more thing you need to do: \textbf{identify the assumptions} that underlie your questions, make them visible, and make peace with them.

\item You arrived at your topic and your questions with a whole mess of assumptions.  

After all, these are the reasons why you thought the topic is \textbf{interesting} and why you think your question is the \textbf{right} one for you.

\item \textbf{Dispelling misconceptions} can be useful in many pedagogical and research settings. Yet the dispelling process, however well meaning, can have an \textbf{inhibitory effect}.

\item Your assumptions about the world -- even the most naive or negative -- \textbf{serve} you at this point in the research process. 

\item They are what helped you \textbf{notice a detail} in a primary source. It was the \textbf{gaps between your assumptions and the world} as it really is that \textbf{gave rise to all those specific research questions}. 


\item \textbf{Write down every possible reason} why you might think what you think, even if you’re uncertain, \textbf{without judging} them as good or bad. 

The point here is not to ``expose" your assumptions in a negative way. Rather, the \textbf{goal} is to \textbf{bring to the surface} those parts of your thinking that remain \textbf{invisible} yet \textbf{influence} how you think.

you’re \textbf{admitting these assumptions} to yourself so that you can improve your own thinking. There is \textbf{no external judgment} here.
\end{itemize}
\end{exercise}

\item \emph{\textbf{Self-Centered Research}} is premised upon a very different approach to \emph{\textbf{assumptions}}, as follows:
\begin{itemize}
\item \emph{Assumptions should be made \textbf{visible}, and thus \textbf{vulnerable}}.

\item \emph{Assumptions \textbf{should not}, however, \textbf{be} stigmatized, \textbf{silenced}, or driven underground}, since this, counterintuitively, encourages holding on to them more tightly.

\item \emph{Assumptions are fuel \textbf{to be consumed}}. Using them, you can achieve two goals at once: 
\begin{itemize}
\item \emph{you can move in a \textbf{new direction}}, and
\item \emph{you can \textbf{exhaust your assumptions} in the process (meaning that you will eventually need new fuel)}.
\end{itemize}
\end{itemize}

\item Your assumptions \emph{shape your \textbf{expectations about reality}}. And when those expectations are \emph{not met}, it’s time to \emph{pay attention}.

\item \begin{exercise}[Identify the Problem That Connects Your Questions]
\begin{itemize}
\item The \textbf{goal}: To \textbf{identify} the problem underlying your multiple draft research questions

\item By now, you have completed several exercises to produce a large number of questions about facts related to your project. What you now want to figure out is, What is the problem that connects your questions?

\item \begin{itemize}
\item What relationships can you find between the different questions and fragments you have created and gathered thus far? 
\item What motivates your search for these particular facts? 
\item You could have asked any questions about this topic -- why these?
\item Which questions are the most compelling to you (and which seem less important)? 
\end{itemize} Figure this out, and you’ll have accomplished a major breakthrough: you’ll have identified the underlying pattern that connects all (or most) of your questions in a coherent whole. 

\item The higher-level questions might not all add up. Don’t force them to. 

What are the \textbf{parent categories} that connect two or more of your questions? 

The connective tissue might not be obvious immediately. Finding it might require thinking \textbf{counterintuitively}.
\end{itemize}
\end{exercise}
\end{itemize}
\subsubsection{You Have a Problem (in a Good Way)}
\begin{itemize}
\item You have now taken a close look at your many factual questions and \textbf{grouped} them under \textbf{parent categories} by \textbf{shared concern}. 

\item You have formulated higher-level questions motivated by these concerns.

\item The key concern that overshadows all others might have emerged in a flash or intuition. 

\item How do I know when I’ve truly discovered my Problem?

A problem is never a fleeting thing. Rather, it is something that is \textbf{\emph{sustained}} and \textbf{\emph{enduring}}.

\item They are good to have, good to worry about, good to mull over.

\item Ultimately, however, the final decision can only come from \textbf{\emph{you}}. 

Only you can know whether or not the cluster of fascinating questions you’ve generated thus far add up to a problem, or just a highly sophisticated and interesting set of curiosities.
\end{itemize}


\subsection{Designing a Project that Works}
\begin{itemize}
\item Having arrived at a problem, now you must make decisions about \emph{what you can accomplish}, \emph{given your available resources}. 

\item In particular, you need to think about \emph{\textbf{the primary sources}} you’ll need to answer your questions and solve your Problem, as well as the \emph{\textbf{resources}} you’ll need (including time!) to put together a \emph{\textbf{project}}.

\item The issues this chapter deals with are both \emph{conceptual} and \emph{practical}: 
\begin{itemize}
\item What are primary sources? 

\item Which ones can you actually access?

\item How can you discover the full potential of a source related to your topic, or look beyond the obvious questions one might ask about a source to arrive at something original? 

\item How can you use such sources to pinpoint your Problem?

\item What arguments can you make with your sources?

\item How many sources can you acquire? How much time will you have to analyze them?

\item How should you design your project, given your personal work habits, material constraints, or deadline?
\end{itemize}

\item \emph{\textbf{Project planning}} involves \emph{\textbf{self-assessment} and \textbf{visualization}}. 
\end{itemize}

\subsubsection{Primary Sources and How to Use Them}
\begin{itemize}
\item \emph{\textbf{Sources}} are essential to original research, so figuring out how to identify, evaluate, and use them is a crucial practical consideration. 

\item Researchers conventionally divide sources into two general categories:
\begin{itemize}
\item \emph{\textbf{primary sources}} and 
\item \emph{\textbf{secondary sources}}.
\end{itemize}

\item Research guides typically define \emph{\textbf{primary sources}} as ``original" or ``raw" materials. 

They are the \emph{\textbf{evidence}} that you use to \emph{\textbf{develop and test claims}}, \emph{\textbf{hypotheses}}, and \emph{\textbf{theories} \textbf{about reality}}. 

\item In machine learning situations, \emph{\textbf{the primary sources}} include:
\begin{itemize}
\item \emph{\textbf{the original datasets}};
\item \emph{\textbf{the original codes}};
\item \emph{\textbf{the original experiment notebooks}}.
\end{itemize} Both original codes and orignal datasets are key to \emph{reproduce} the experiment results, which is \emph{the main message} behind a publication and its associated research argument.

The \emph{\textbf{publication itself}} and its \emph{\textbf{experiment results}} are all \emph{\textbf{secondary sources}}.


\item  Most research guides define \emph{\textbf{secondary sources}} along similar lines. 

Secondary sources are \emph{books}, \emph{articles}, or \emph{reports} that are \emph{\textbf{based on primary sources}} and are intended for scholarly or professional audiences,

\item we also want to reinforce a point well known to veteran researchers about \emph{the \textbf{dangers} of defining ``\textbf{primary}" and ``\textbf{secondary}" \textbf{sources} in terms of \textbf{absolutes}}.

\item Absolutist definitions of sources get in the way of the process of identifying primary sources and asking research questions, for two reasons:
\begin{itemize}
\item Any source can be \emph{primary}, \emph{secondary}, or \emph{not a source} \emph{\textbf{for your project}}.
\item A source’s \emph{type} is determined solely by \emph{\textbf{its relationship with the questions}} you are trying to answer, and \emph{the problem} you are trying to solve. A source is never inherently primary or secondary.
\end{itemize}

\item A more accurate \emph{definition} of \emph{\textbf{primary source}} would be the following: 

\emph{a \textbf{source} that is \textbf{primary} with respect to a \textbf{particular question}}.

\item Let’s take this one step further. Just as the same source can be ``primary" or ``secondary," depending upon context, so too \emph{\textbf{can the same source be ``primary" in dramatically different ways}}. 

\emph{The very same source} can show up in the bibliographies of strikingly \emph{different research projects}, and can be used by different authors to \emph{pose dramatically different kinds of questions}.

\item \emph{\textbf{how you treat this source}} will lead you down either a narrow path or a broad avenue of potential research questions.
\begin{itemize}
\item The narrow route is to \emph{jump to the obvious candidates}.
\item Or, let’s brainstorm for a moment \emph{\textbf{different} kinds of \textbf{research projects}} that might conceivably include this source 
\item And go one step further and brainstorm what \emph{\textbf{other primary sources}}, depending upon the particular research project in which it appears.
\end{itemize}
Based on what we come up with, let’s then give a \emph{\textbf{name}} to \emph{the \textbf{genre} of questions} we’re asking, on the assumption that it might be \emph{connected} to \emph{an underlying problem}.

\item Take a note and create a table with columns as
\begin{itemize}
\item What I notice about the source 
\item Questions/concerns I might have
\item The very next primary source I might want to find
\item Broader subjects and/or genres of questions that might be related to my problem
\end{itemize}

\item  \begin{exercise}[Treat Your Primary Source Like a Cereal Box]

The \textbf{goal}: To adopt the habit of \textbf{asking multiple genres of questions} about each of your \textbf{primary sources} so as to \textbf{identify problems that are not self-evident} and thus might easily be \textbf{overlooked}. 

This technique will both enable you to decide which problem interests you most, and enhance your ability to conduct original research.
\end{exercise}

\item \begin{exercise}[Envision Your Primary Sources]
\begin{itemize}
\item The \textbf{goal}: To identify places you might not have originally considered looking for primary sources. This will \textbf{enhance the comprehensiveness}, \textbf{originality}, and \textbf{significance} of your research.

\item Doing original research requires looking where no one else has looked for a solution to your Problem.

\item Researchers nowadays often make two major mistakes here. They think that
\begin{itemize}
\item all of the information they need to do their research well is available online; and
\item all of the information available online is searchable
\end{itemize} In fact, digitized materials make up only a small fraction of the total number of primary source materials. 

\item Instead of letting keyword searches define the boundaries of your source base, try closing your laptop or your browser, instead \textbf{envisioning in your mind’s eye} \textbf{where} \textbf{relevant sources} about your subject might be located; what these sources might look like, in terms of their format and genre; and who or what organization might have produced them. 

In other words, rather than limiting yourself to what is (database results), \textbf{expand your search} to include sources that could be or even that must be.

\item Sometimes, in order to get to specifics, you have to think \textbf{systematically} and \textbf{institutionally}.

\item You can see now why you want to take the time to envision sources. Keyword searching is not always the place to start, nor does it turn up all of the results you might need.

Instead, you need to \textbf{envision where sources might exist}, and only \textbf{then} go back to the work of \textbf{searching}.

\item You’ll discover more primary sources, generate more useful questions, and deepen your research in ways you did not anticipate.
\end{itemize}
\end{exercise} 
\end{itemize}

\subsubsection{Connecting the Dots: Getting from Sources to Arguments}
\begin{itemize}
\item any argument can be made when based on \emph{only one source}. Even with two points, or three, the puzzle is overwhelmingly \emph{unrestricted}.

\item Yet the lesson here is not just that \emph{you need an adequate number of sources} to connect the dots of a good argument.

It’s more \emph{fundamental} than that.

\item the main \emph{challenge} becomes \emph{not solving}, but \emph{\textbf{creating puzzles}} that are nontrivial, not preordained, open-ended, and significant (no matter what the answer ends up being). 

In order to create puzzles, we need to be able to \emph{\textbf{envision} and \textbf{identify unknowns}}.

\item the researcher needs to do the following:
\begin{itemize}
\item \emph{\textbf{Find the dots.}}
 
\item \emph{Figure out which dots \textbf{belong to your picture,} and which dots belong to some other picture.} 

you need to keep an open mind and be able to envision multiple possible outcomes. 

Even if you have all of your data points in hand, you still need to know how to analyze them so as to come up with the right solution.

\item \emph{Figure out which dots \textbf{are not dots} at all.} We call these \emph{\textbf{non-sources}}. Sources are sources because they have utility for the researcher trying to answer a question or solve a problem. Their usefulness is \emph{relative} -- they may be more useful or less. 

\emph{Not everything out there is a source.} A single dot could reorient your whole research project.

\item \emph{Do all of the above in real time.}  And as more dots appear on your page, the picture becomes clearer. Each additional dot adds a constraint, limiting the number of interpretations that is viable. 

\item \emph{Determine when you have enough.} It’s during the research process itself that you’ll learn to identify thresholds of probability, confidence, and certainty
\end{itemize}
\end{itemize}

\subsubsection{Sources Cannot Defend Themselves}
\begin{itemize}
\item Before you connect some dots (not all of them, yet) on your own project, there are \emph{\textbf{ethical issues}} regarding the use of sources to consider.

\item Writing is a different art form, however, and when you construct \emph{the narrative of your arguments} and your \emph{explanations}, or when you tell the story of your research findings, \emph{\textbf{you have the choice} of connecting your dots using straight or curvy lines} -- or, most likely, some combination of the two.

\item A few key takeaways here:
\begin{itemize}
\item \emph{\textbf{Sources cannot speak for themselves}}, nor can they defend themselves \emph{against you}; thus it is \emph{your \textbf{obligation} to \textbf{represent} them \textbf{accurately}}.

As soon as you start dealing with primary sources, you have to make \emph{\textbf{ethical decisions}}, the first being to \emph{represent the sources as honestly as possible}.

\item \emph{\textbf{Research integrity} requires not just dealing in \textbf{facts} but also \textbf{not forcing} them to tell a story}. 

\item Connecting the dots from sources to arguments is always \emph{a deliberate choice} involving \emph{\textbf{ethical responsibility}}. 

The key here is not to avoid or downplay this responsibility, but to make these choices as \emph{\textbf{deliberately} and \textbf{defensibly}} as possible.
\end{itemize}

\item As you make choices about sources, be aware: even though sources cannot speak for themselves, this does not mean that sources are merely inert objects subject to the will or manipulation of the researcher. 

\item  They have a kind of \emph{\textbf{agency of their own}}, even in their seeming silence. 

A source might be any of the following:
\begin{itemize}
\item Incomplete or fragmentary. 
\item Purposefully deceptive.
\item Wrong by accident. 
\item Biased -- sincere or well-meaning in trying to tell truth, but distorted by unconscious bias. 
\item Motivated by an acknowledged or unacknowledged agenda. 
\item Inconsistent.
\end{itemize}

\item While evaluating your own sources, use the bullet points above as a checklist, and make a note of further steps you might want to take to understand them better.

\item Note that even if a source you come across is any of the things described above, it can still be useful to you, so \emph{don’t reflexively dismiss it}. 

Instead, incorporate it into your \emph{\textbf{question-generation process}}. 
\begin{itemize}
\item Why might this source be trying to deceive me? 
\item What phenomenon is this source symptomatic of?
\end{itemize}

\item \begin{exercise}[Connect the Dots Using Your Sources]
\begin{itemize}
\item The \textbf{goal}: To start thinking about source criticism early in the research process, while remaining flexible and inclusive.


\item Research, once again, is a \textbf{nonlinear process}, which is why we keep encouraging you to think through your ideas, your questions, and your sources in a subjunctive mode -- to keep thinking \textbf{What if?} We want you to take the time to chart and rechart your course as many times
as necessary before you launch your journey.

\item you will likely need to create your own research puzzle, instead of finding it lying on the ground, ready-made -- and trying out different possibilities, \textbf{without jumping to a question or forcing a project}.

\item For this exercise, try connecting some dots using your sources, but do this in pencil so that you can erase the lines and draw new ones later. 
\end{itemize}

\end{exercise}
\end{itemize}


\subsubsection{Taking Stock of Your Research Resources}
\begin{itemize}
\item 
\end{itemize}

\subsubsection{Two Types of Plan B}
\begin{itemize}
\item 
\end{itemize}

\subsubsection{Setting Up Shop}
\begin{itemize}
\item 
\end{itemize}

\subsubsection{You Have the Beginnings of a Project}
\begin{itemize}
\item 
\end{itemize}
\section{Get Over Yourself}
\begin{itemize}
\item Your project matters to you. \emph{\textbf{Does it matter to the world?}}

\item Why should we engage with other researchers?
\begin{itemize}
\item Getting over yourself is a movement from a more \emph{\textbf{narrow} understanding of self} to a \emph{more \textbf{expansive} one}.
This process of exploration, discovery, and accretion is based on \emph{\textbf{engagement}}. 

Far from losing your sense of self, seeing your ideas in relation to others’ can help you to \emph{learn even more about yourself}. 

\item Another reason to get over yourself is entirely \emph{\textbf{pragmatic}}:  none of us, even when we do much of our work alone, inhabit a research community of one.

In the creation of any new research, we rely on the \emph{ideas of predecessors and peers}.

One of the most important conversations you’ll be joining is with the broader community of researchers who work on the same topic as you— a community commonly referred to as a ``\emph{field}."

\end{itemize}

\item The overarching \emph{\textbf{goal}} for part 2 is to \emph{become aware of how \textbf{other people’s agendas and questions} \textbf{intersect with our own}}, and to \emph{\textbf{make the most of those relationships}}.

\item Research is never a monologue, and your research identity is not static.

\item You have to navigate your \emph{Field} (and might change or add Fields), which involves interacting with different \emph{Problem Collectives}.

\item Doing so requires remaining \emph{\textbf{mobile}} and \emph{\textbf{open-minded}}. Yet the key to engaging with the ideas of others is to \emph{\textbf{maintain your own sense of centeredness}}.

\item Once again, you’ll be stress-testing your ideas, assumptions, and theories, but this time you’ll be doing so using the ideas, assumptions, and theories of others.

\item You will \emph{\textbf{make other people’s ideas your own}}. Eventually, you’ll help other people \emph{\textbf{make your ideas their own}}.
\end{itemize}
   
\subsection{How to Find Your Problem Collective}
\subsection{How to Navigate Your Field}
\subsection{How to Begin}

\section{What's Next in Your Research Journey?}


\newpage
\bibliographystyle{plainnat}
\bibliography{reference.bib}
\end{document}