\documentclass[11pt]{article}
\usepackage[scaled=0.92]{helvet}
\usepackage{geometry}
\geometry{letterpaper,tmargin=1in,bmargin=1in,lmargin=1in,rmargin=1in}
\usepackage[parfill]{parskip} % Activate to begin paragraphs with an empty line rather than an indent %\usepackage{graphicx}
\usepackage{amsmath,amssymb, mathrsfs,  mathtools, dsfont}
\usepackage{tabularx}
\usepackage{tikz-cd}
\usepackage[font=footnotesize,labelfont=bf]{caption}
\usepackage{graphicx}
\usepackage{xcolor}
%\usepackage[linkbordercolor ={1 1 1} ]{hyperref}
%\usepackage[sf]{titlesec}
\usepackage{natbib}
%\usepackage{tikz-cd}

\usepackage{../../Tianpei_Report}

%\usepackage{appendix}
%\usepackage{algorithm}
%\usepackage{algorithmic}

%\renewcommand{\algorithmicrequire}{\textbf{Input:}}
%\renewcommand{\algorithmicensure}{\textbf{Output:}}



\begin{document}
\title{Book Reading Summary: The 5 Elements of Effective Thinking}
\author{ Tianpei Xie}
\date{Dec. 10th., 2023}
\maketitle
\tableofcontents
\newpage
\section{Summary of Advices}
\subsection{Understand deeply}
\begin{itemize}
\item Don’t face complex issues head-on; first understand \emph{\textbf{simple} ideas \textbf{deeply}}. 
\begin{itemize}
\item \emph{What is deep understanding?}

In everything you do, refine your skills and knowledge about fundamental concepts and simple cases. Once is never enough. As you revisit fundamentals, you will find new insights.

\item A \emph{\textbf{commonsense}} approach leads to the core. 

To learn any subject well and to create ideas beyond those that have existed before, return
to the basics repeatedly.

\item When faced with a \emph{\textbf{difficult challenge}} –- \emph{\textbf{don’t do it!}}

When the going gets tough, creative problem solvers \emph{\textbf{create an easier, simpler problem}} that they can solve.

Focus entirely on solving a subproblem that you know you can successfully resolve.
\end{itemize}

\item \emph{Clear the clutter} and expose what is really \emph{\textbf{important}}. 
\begin{itemize}
\item \emph{\textbf{Uncover the essence}}. 

By systematically \emph{\textbf{ignoring} one distraction after another}, you can turn your attention to more central (often initially invisible) themes.
\end{itemize}

\item Be brutally honest about \emph{what you \textbf{know} and \textbf{don’t know}}. 
\begin{itemize}
\item acknowledging and then letting go of bias and prejudice can lead you to see what’s truly there and (often more importantly) to discover what’s missing.
\item \emph{What \textbf{everybody believes} is \textbf{not always} what’s actually true}.

\item \emph{How do you know?} 

Becoming aware of the basis of your opinions or beliefs is an important step toward a better understanding of yourself and your world.

What is the evidence that your understanding is based upon? Become aware of the \emph{\textbf{sources of your opinions}}.
\end{itemize}

\item Then see \emph{what’s \textbf{missing}}, identify the \emph{\textbf{gaps}}, and fill them in. 
\begin{itemize}
\item One of the most profound ways to see the world more clearly is to look deliberately for the gaps -– the negative pace, as it is called in the art world; that is, the space surrounding the objects or issues of interest.

\item Add the adjective and uncover the gaps.
\end{itemize}

\item Let go of \emph{\textbf{bias}}, \emph{\textbf{prejudice}}, and \emph{\textbf{preconceived notions}}. 
\item There are \emph{\textbf{degrees} to understanding} (it’s not just a yes-or-no proposition) and you can always heighten yours. 
\item \emph{\textbf{Rock-solid understanding}} is the foundation for success.
\end{itemize}

\subsection{Make mistakes}
\begin{itemize}
\item \emph{\textbf{Fail}} to succeed. 
\begin{itemize}
\item ``Fail" is not an obscene word.

\item Any creative accomplishment evolves out of lessons learned from a long succession of missteps.

\item \emph{\textbf{Effective failure}} is an important, positive (and, as in the case of Microsoft, lucrative) step toward success.

\item \emph{Viewing failure} as an \emph{\textbf{opportunity}} for learning requires a fresh mind-set.

\item failing \emph{productively} involves two basic steps: \emph{\textbf{creating the mistake}} and then \emph{\textbf{exploiting the mistake}}.

\item Success is about \emph{\textbf{persisting}} through the process of repeatedly failing and learning from failure.

\item Give credit to failure.
\end{itemize}

\item \emph{\textbf{Intentionally}} get it wrong to \emph{inevitably} get it even \emph{more right}. 
\begin{itemize}
\item A specific mistake is an excellent source of insight and direction.

\item You may not know how to do it right, but you can certainly do it wrong.

A good way to generate useful mistakes is simply to \emph{tackle the issue at hand} by \emph{\textbf{quickly constructing the best solution} you can} with \emph{\textbf{little or no effort}}.

Don’t stare at a blank screen.

\item incremental approach: try something; see what’s wrong; learn from the defect; try again.

\item Going to the \emph{\textbf{extreme}}. 

Deliberately \emph{exaggerating} or considering \emph{extreme}, impractical scenarios often frees us to have an unforeseen insight.
\end{itemize}

\item \emph{\textbf{Mistakes}} are great teachers -- they \emph{highlight \textbf{unforeseen opportunities}} and holes in your understanding. 
\begin{itemize}
\item Learning from other’s missteps.
\end{itemize}

\item They also show you which way to \emph{\textbf{turn next}}, and they ignite your \emph{\textbf{imagination}}.
\begin{itemize}
\item Sometimes when your attempt fails to resolve one issue, you might discover that you have actually found an \emph{imaginative answer} to a totally \emph{different question}.

\item What separates the \emph{good} from the \emph{great} is \emph{how we react to that bad day}.
\end{itemize}

\item Failure and mistake has its own cost and its impact. If possible, making mistake and failure as early as possible when the cost is low.
\end{itemize}

\subsection{Raise questions}
\begin{itemize}
\item Constantly create \emph{\textbf{questions}} to \emph{\textbf{clarify}} and \emph{\textbf{extend}} your understanding. 
\begin{itemize}
\item Wisdom just for the asking. Creating questions is as important as answering them.

\item Generating questions can help direct your \emph{\textbf{next steps}} toward deeper understanding and creative problem solving.

\item Overcoming \emph{\textbf{bias}}.

\item \emph{Take another look}. \emph{\textbf{Frame questions in different ways.}} Alternative perspectives lead to new sights and new insights.

\item Try to \emph{\textbf{bridge ideas}} from one discipline or area to another.

\item A questionable habit. \emph{force} yourself to ask questions.

\item Be \emph{\textbf{thought provoking}}. Getting in the habit of asking questions will transform you into an \emph{\textbf{active listener}}.

If you are constantly engaged in asking yourself questions about what you are hearing, you will find that even boring lecturers become a bit more interesting
\end{itemize}


\item What’s the \emph{\textbf{real}} question? Working on the wrong questions can waste a lifetime.
\begin{itemize}
\item \emph{\textbf{Effective questions}} highlight \emph{hidden assumptions} and indicate \emph{directions} to take to make progress.

\item \emph{Effective questions} \emph{\textbf{lead to action}} and are \emph{\textbf{not vague}}.

\item The \emph{\textbf{right}} questions \emph{\textbf{clarify your understanding}} and \emph{\textbf{focus} your attention} on features that \emph{matter}.

\item \emph{Effective questions} expose the \emph{\textbf{real issue}}.

\item Remember to always question the questions.
\end{itemize}

\item Ideas are in the air -- the \emph{\textbf{right questions}} will bring them out and help you see \emph{\textbf{connections}} that otherwise would have been invisible.

\item Asking questions about an assignment or project before beginning work in earnest will always lead to a stronger final product.
\end{itemize}


\subsection{Follow the flow of ideas}
\begin{itemize}
\item Look back to see \emph{where \textbf{ideas} \textbf{came from}} and then \emph{look ahead} to discover \emph{where those \textbf{ideas} may \textbf{lead}}. 
\begin{itemize}
\item To better master a subject, after you have been introduced to a new concept, \emph{look beyond the new concept} and just guess what you think will come \emph{\textbf{next}}.

\item Once you understand a more advanced topic, a \emph{\textbf{look back}} makes \emph{\textbf{earlier material}} easier.

The earlier material will become easier, clearer, and more meaningful because you will
see its significance through the later work that came from it.

The more advanced work will also be easier since you will now see how it grew from
the seeds that existed in the earlier work.

\item All the new ideas we have are \emph{\textbf{only tiny variations}} of what has been thought before. The difference between those who have great insights and those who don’t is that the first group actually \emph{take those baby steps}.
\end{itemize}

\item \emph{A new idea} is a \emph{\textbf{beginning}}, not an end. 
\begin{itemize}
\item When you learn a new concept or master a skill, think about what \emph{\textbf{extensions}}, \emph{\textbf{variations}}, and \emph{\textbf{applications}} are possible.

\item The \emph{best} can get even \emph{\textbf{better}}. Starting with what is currently the best is often the ideal place to expect great improvements.

\item Often the solution to a difficult problem comes from a \emph{\textbf{struggled focus}} on the issue. You may have to struggle to finally
master an idea or a skill.  But after you have reached one level, that is where you start.

\item Once you have it, see if you can improve it.
\end{itemize}
\item Ideas are \emph{rare} -- \emph{\textbf{milk}} them. 
\item Following the \emph{\textbf{consequences}} of \emph{\textbf{small ideas}} can result in \emph{big payoffs}.
\begin{itemize}
\item Making it practical. ``What’s next?"  ``If this, then that." ``What now?" Explore the consequences several steps forward.

\item \emph{Following that flow} can \emph{\textbf{highlight}} some \emph{\textbf{fallacies}} in seemingly sound schemes.
\end{itemize}
\end{itemize}


\subsection{Change}
\begin{itemize}
\item The unchanging element is change -- by mastering the first four elements, you can change the way you think and learn. You can always improve, grow, and extract more out of your education, yourself, and the way you live your life. 
\begin{itemize}
\item Definition of Insanity: Doing the same thing and expecting a different outcome.

\item And now for something completely different.

people who \emph{perform better} can be viewed as actually doing a \emph{\textbf{different task}}, rather than \emph{doing the same task better}.

\item Instead of thinking,``Do it better," think,``\emph{\textbf{Do it differently.}}"

\item Messing things up. 

Often \emph{\textbf{the most profound advances}} you can make in your life come through \emph{\textbf{experiences that challenge the life you have}}. 

\item You can turn doubt into a comfortable and insightful guide along the road to true change.
\end{itemize}

\item Change is the universal constant that allows you to get the most out of living and learning.
\end{itemize}



\newpage
\section{Introduction -- Elements of Effective Thinking, Learning and Creating}
\begin{itemize}
\item The surprising fact is that just a few learnable strategies of thinking can make you more effective in the classroom, the boardroom, and the living room. You can personally \emph{\textbf{choose}} to become more successful by adopting five learnable habits, which, in this book, we not only explain in detail but also make concrete and practical. Here in this section we briefly introduce those important habits to come.

\item \emph{\textbf{Understand deeply}}:
\begin{itemize}
\item Don’t face complex issues head-on; first understand \emph{\textbf{simple} ideas \textbf{deeply}}. 
\item Clear the clutter and expose what is really \emph{\textbf{important}}. 
\item Be brutally honest about \emph{what you \textbf{know} and \textbf{don’t know}}. 
\item Then see \emph{what’s \textbf{missing}}, identify the \emph{\textbf{gaps}}, and fill them in. 
\item Let go of \emph{\textbf{bias}}, \emph{\textbf{prejudice}}, and \emph{\textbf{preconceived notions}}. 
\item There are \emph{\textbf{degrees} to understanding} (it’s not just a yes-or-no proposition) and you can always heighten yours. 
\item \emph{\textbf{Rock-solid understanding}} is the foundation for success.
\end{itemize}

\item \emph{\textbf{Make mistakes}}:
\begin{itemize}
\item \emph{\textbf{Fail}} to succeed. 
\item \emph{\textbf{Intentionally}} get it wrong to \emph{inevitably} get it even \emph{more right}. 
\item \emph{\textbf{Mistakes}} are great teachers -- they \emph{highlight \textbf{unforeseen opportunities}} and holes in your understanding. 
\item They also show you which way to \emph{\textbf{turn next}}, and they ignite your \emph{\textbf{imagination}}.
\end{itemize}

\item \emph{\textbf{Raise questions}}:
\begin{itemize}
\item Constantly create \emph{\textbf{questions}} to \emph{\textbf{clarify}} and \emph{\textbf{extend}} your understanding. 
\item What’s the \emph{\textbf{real}} question? Working on the wrong questions can waste a lifetime.
\item Ideas are in the air -- the \emph{\textbf{right questions}} will bring them out and help you see \emph{\textbf{connections}} that otherwise would have been invisible.
\end{itemize}

\item \emph{\textbf{Follow the flow of ideas}}:
\begin{itemize}
\item Look back to see \emph{where \textbf{ideas} \textbf{came from}} and then \emph{look ahead} to discover \emph{where those \textbf{ideas} may \textbf{lead}}. 
\item \emph{A new idea} is a \emph{\textbf{beginning}}, not an end. 
\item Ideas are \emph{rare} -- \emph{\textbf{milk}} them. 
\item Following the \emph{\textbf{consequences}} of \emph{\textbf{small ideas}} can result in \emph{big payoffs}.
\end{itemize}

\item \emph{\textbf{Change}}:
\begin{itemize}
\item The unchanging element is change -- by mastering the first four elements, you can change the way you think and learn. You can always improve, grow, and extract more out of your education, yourself, and the way you live your life. 
\item Change is the universal constant that allows you to get the most out of living and learning.
\end{itemize}
\end{itemize}

\section{Ground Your Thinking -- Understand Deeply}
\begin{itemize}
\item Understanding is not a \emph{yes-or-no proposition}; it’s not an on-or-off switch. 

\item When you learn anything, \emph{go for \textbf{depth}} and \emph{make it \textbf{rock solid}}. 

\item You can understand anything better than you currently do. Setting a \emph{\textbf{higher standard}} for yourself for \emph{what you mean by \textbf{understanding}} can revolutionize how you perceive the world. The following steps illustrate why a deep understanding is essential to a solid foundation for future thinking and learning.
\end{itemize}

\subsection{Understand simple things deeply}
\begin{itemize}
\item \emph{\textbf{The most fundamental ideas}} in any subject can be understood with \emph{\textbf{ever-increasing depth}}. 

\item Successful students continue to improve their mastery of the concepts from previous chapters and courses as they move toward the more advanced material on the horizon; successful people regularly focus on the core purpose of their profession or life. 

\item \emph{True experts \textbf{continually deepen} their \textbf{mastery} of the \textbf{basics}}.

\item \emph{\textbf{What is deep understanding?}} 

How can you realize when you don’t know something deeply? In everything you do, \emph{\textbf{refine} your skills and knowledge} about \emph{\textbf{fundamental concepts}} and \emph{\textbf{simple cases}}. \emph{Once is never enough}. As you \emph{\textbf{revisit fundamentals}}, you will find \emph{\textbf{new insights}}. It may appear that returning to basics is a step backward and requires additional time and effort; however, by \emph{building on \textbf{firm foundations}} you will soon see your true abilities soar higher and faster.

\item 
\begin{exercise} [Master the Basics]
Consider a skill you want to improve or a subject area that you wish to understand better.  Spend five minutes writing down \textbf{specific components of the skill} or \textbf{subject area} that are \textbf{basic} to that theme. Your list will be a freeflowing stream of consciousness. 

Now \textbf{pick one} of the items on your list, and spend thirty minutes \textbf{actively improving your mastery} of it.  See how working deeply on the basics makes it possible for you to hone your skill or deepen your knowledge at the higher levels you are trying to attain. Apply this exercise to other things you think you know or would like to know.
\end{exercise}

\item \emph{\textbf{A commonsense approach leads to the core}}. Many of the most complicated, subtle, and profound ideas arise from looking unmercifully clearly at simple, everyday experiences. 

To \emph{learn} any subject well and to \emph{create} ideas beyond those that have existed before, \emph{r\textbf{eturn to the basics repeatedly}}. When you look back after learning a complicated subject, the basics seem far simpler; however, \emph{those \textbf{simple basics are a moving target}}. As you learn more, the fundamentals become at once \emph{simpler} but also \emph{subtler, deeper, more nuanced, and more meaningful}. 

\item 
\begin{exercise}[Ask: What do you know?]
Do you or don’t you truly know the basics? Consider a subject you think you know or a subject you are trying to master.  Open up a blank document on your computer. Without referring to any outside sources, write a \textbf{detailed outline of the fundamentals} of the subject.

Can you write a \textbf{coherent}, \textbf{accurate}, and \textbf{comprehensive} \textbf{description} of the foundations of the subject, or does your knowledge have gaps? Do you struggle to think of \textbf{core examples}? Do you fail to see the \textbf{overall big picture} that puts the pieces together? Now compare your effort to external sources (texts, Internet, experts, your boss).

When you \textbf{discover weaknesses} in your own \textbf{understanding of the basics}, \textbf{take action}. \textbf{Methodically learn} the fundamentals. Thoroughly understand any \textbf{gap} you fill in as well as its \textbf{surrounding territory}. Make these new insights part of your base knowledge and connect them with the parts that you already understood. 

Repeat this exercise regularly as you learn more advanced aspects of the subject (and save your earlier attempts so that you can look back and see how far you’ve traveled). Every return to the basics will deepen your understanding of the entire subject.
\end{exercise}

\item When faced with a \emph{\textbf{difficult challenge}} --\emph{\textbf{don’t do it!}}

Great scientists, creative thinkers, and problem solvers \emph{\textbf{do not solve hard problems} \textbf{head-on}}. When they are faced with a daunting question, they immediately and prudently \emph{admit defeat}. 

They realize that there is no sense in wasting energy vainly grappling with complexity when, instead, they can \emph{productively grapple with \textbf{simpler cases}} that will \emph{teach them how to deal with the complexity to come}.

``\emph{If you can’t solve a problem, then \textbf{there is an easier problem you can’t solve}: find it.}" -- George Polya.

\item When the going gets tough, creative problem solvers \emph{\textbf{create an easier, simpler problem}} that they can solve. They resolve that easier issue thoroughly and then \emph{\textbf{study that simple scenario} with laser focus}. Those insights often point the way to a resolution of the original difficult problem.

\item Apply this mind-set to your work: when faced with a \emph{difficult issue or challenge}, \emph{do something else}. \emph{\textbf{Focus}} entirely on \emph{\textbf{solving a subproblem}} that you know you can successfully resolve. Be completely confident that the extraordinarily thorough work that you invest on the subproblem will later be the guide that allows you to navigate through the complexities of the larger issue. 

But don’t jump to that more complex step while you’re at work on the subissue. 

\item \begin{exercise}[Sweat the small stuff]
Consider some complex issue in your studies or life. Instead of tackling it in its entirety, find one \textbf{small element} of it and \textbf{solve} that part \textbf{completely}. Understand the subissue and its solution \textbf{backwards} and \textbf{forwards}. Understand all its \textbf{connections} and \textbf{implications}. Consider this small piece from many points of view and in great detail. Choose a \textbf{subproblem} small enough that you can
give it this level of attention. Only later should you consider how your efforts could help solve the larger issue.
\end{exercise}
\end{itemize}

\subsection{Clear the clutter -- seak the essential}
\begin{itemize}
\item \emph{\textbf{Uncover the essence}}. 

When faced with an issue that is complicated and multifaceted, attempt to \emph{\textbf{isolate} the \textbf{essential ingredients}}. The essence is n\emph{ot the whole issue}. There is a further step of understanding how the other features of the situation fit together; however, \emph{clearly \textbf{identifying} and \textbf{isolating} essential principles} can guide you through the morass. The strategy of
clearing the clutter and seeking the essential involves two steps:
\begin{enumerate}
\item \textbf{\emph{Identify}} and \emph{\textbf{ignore}} all \emph{\textbf{distracting features}} to \emph{isolate} the \emph{\textbf{essential core}}.
\item  \emph{\textbf{Analyze}} that \emph{\textbf{central issue}} and \emph{apply} those insights to the \emph{\textbf{larger whole}}.
\end{enumerate}

\item Many real questions are surrounded and obscured by history, context, and adornments. Within that cloud of vaguely related, interacting influences, you need to \emph{\textbf{pluck out} the \textbf{central themes}}.  Often you may be surprised that after you pare down a complex issue to its essentials, the essentials are much clearer and \emph{easier to face}. 

Ignoring things is difficult. Often the peripheral clutter is blinking and clanging and trying madly to draw your attention away from what is really going on. By \emph{\textbf{systematically} \textbf{ignoring} one \textbf{distraction} after another}, you can turn your \emph{attention} to more \emph{central} (often initially invisible) \emph{themes}. 

After you \emph{\textbf{clear the clutter}}, what remains will \emph{clarify understanding} and open the door to \emph{creating new ideas}. Remember, you may not be able to see everything, but you can certainly \emph{\textbf{ignore most things}}.


\item \begin{exercise}[Uncover one essential]
Consider a subject you wish to understand, and \textbf{clear the clutter} until you have \textbf{isolated} \textbf{one essential ingredient}. Each complicated issue has \textbf{several possible core ideas}. You are not seeking ``the” essential idea; you are seeking just one -- consider a subject and pare it down to one theme. 

In fact, you might perform this exercise on yourself. What do you view as essential elements of you? Isolating those elements can give a great deal of \textbf{focus to life decisions}.
\end{exercise}

\item Once you have isolated the essential, you have armed yourself with a \emph{\textbf{solid center}} upon which to build and embellish. \emph{The core is not the whole issue}, but it is a lodestar that can guide you through turbulent storms and complications. 

What’s \emph{\textbf{core}}? What’s fluff? Find what’s \emph{\textbf{at the center}} and \emph{work out from there}. You can confidently \emph{\textbf{center yourself}}.
\end{itemize}

\subsection{See what's there}
\begin{itemize}
\item Whenever you ``\emph{see}" an issue or ``\emph{understand}" a concept, be \emph{conscious of the lens} through which you’re viewing the subject. You should assume you’re \emph{\textbf{introducing bias}}. The challenge remains to identify and \emph{let go of that bias or the assumptions} you bring, and actively work to see and understand the subject anew.

Whether it be physical characteristics of what you see, emotional aspects of what you feel, or conceptual underpinnings of what you understand, acknowledging and then \emph{letting go of \textbf{bias} and \textbf{prejudice}} can \emph{lead you to \textbf{see what’s truly there}} and (often more importantly) to \emph{\textbf{discover what’s missing}}.

\item To better understand your world, \emph{consciously \textbf{acknowledge what you actually see}} -- no matter how mundane or obvious -- rather than guess at what you think you are supposed to see. Saying what you actually see forces you to become \emph{\textbf{conscious}} of \emph{what is there and also what is missing}. 

\emph{If you see it, then say it; if you don’t see it, then don’t claim to see it.}

\item \begin{exercise}[Say it like you see it]
Homework assignments, tests, and job-related assessments ask you what you know. Unfortunately, partial credit or social pressure often encourages you to pretend to know a bit more than you actually do. 

So in the privacy of your own room look at assignments or possible test questions and write down the weaknesses as well as the strengths of what you know and don’t know. Deliberately avoid glossing over any gaps or vagueness. Instead boldly assert what is tepid or missing in your understanding. 

Now take the action of filling in the gaps. Identifying and admitting your own uncertainties is an enormous step toward solid understanding.
\end{exercise}

\item If you are writing an essay, \emph{read literally what you have written} -- not what you intended to communicate.  Pretend you don’t know the argument you are making and read your actual words.  What’s confusing and what’s missing? If you think you know an idea but can’t express it clearly, then this process has identified a \emph{gap or vagueness in your understanding}. 

After you admit and address those weaknesses, your exposition will be clearer and more directed to the actual audience. When delivering an address or making a presentation, apply this same process of deliberately listening to the actual words you are speaking rather than what you imagine you are saying.

\item \emph{\textbf{What everybody believes is not always what’s actually true.}}

Commonly held opinions are frequently just plain false. Often we are \emph{persuaded by authority and repetition} rather than by \emph{evidence and reality}. This tendency to \emph{accept what surrounds us} makes it difficult to separate what we really know from what we just believe we know. 

\item Individuals tend to accept ideas if people they know or respect state or believe those ideas. You need to be very clear about the \emph{\textbf{foundations}} of \emph{\textbf{your opinions}}. If you believe something only because another person -- even a professor -- told you it was so, then you should not view your understanding as rock solid. 

\item \emph{\textbf{How do you know?}}

\emph{\textbf{Becoming aware} of the \textbf{basis} of your \textbf{opinions} or \textbf{beliefs}} is an \emph{important step} toward a better understanding of yourself and your world. Regularly consider your opinions, beliefs, and knowledge, and subject them to the ``How do I know?" test.

What is the evidence that your understanding is based upon? Become aware of \emph{the \textbf{sources of your opinions}}. If your \emph{\textbf{sources} are shaky}, then you might want to be more \emph{open-minded} to the possibility that your opinion or knowledge might be \emph{incorrect}. \emph{Regularly} \emph{find \textbf{cases}} in which you need to \emph{\textbf{rethink your views}}.

\item \emph{Opening our minds} to \emph{\textbf{counterintuitive ideas}} can be the \emph{\textbf{key}} to discovering novel solutions and building deeper understanding, but how can we take advantage of those opportunities? Certainly we are not intentionally closed-minded. 

So how can we break free of our unintended closed-mindedness and see the world with less bias? First, we can simply \emph{try out \textbf{alternative ideas} \textbf{hypothetically} and \textbf{temporarily}}. 

I’ll \emph{\textbf{pretend}} my opinions are \emph{the opposite of what I normally believe} (even though I know it’s nonsense), and see where those new beliefs take me. This strategy allows you to explore ideas without having to overcome deeply ingrained moral
or institutional prejudices. 

Even \emph{following} ideas that you know are \emph{\textbf{wrong}} can be illuminating. Because in following the consequences of those ``wrong" ideas, you might be led to better understand why your original belief is \emph{indeed correct}, or you might be led to \emph{new and unexpected insights} that run \emph{counter} to \emph{your original beliefs}.

\item \begin{exercise}[Try on alternatives and size up the fit]
Take some opinion that you hold that other people (those who clearly are wrong) do not hold. 
\begin{itemize}
\item Every other hour \textbf{accept your own current opinion} and think about its \textbf{implications}, and 

\item on the alternate hours \textbf{accept the alternative opinion} and see where that \textbf{leads}. Try \textbf{not to be judgmental}. Don’t resist the alternative views. You are not committing to any change. 
\end{itemize}
This exercise has the goal of \textbf{understanding alternatives} more \textbf{realistically}. As a result, you might change an opinion, but more likely you will simply have a better understanding of why the alternative views make sense to others. If an hour is too long a time period, try the challenge in fifteen-minute intervals.
\end{exercise}
\end{itemize}

\subsection{See what's missing}
\begin{itemize}
\item Forcing yourself to see what’s actually in front of you rather than what you believe you should see is a difficult task. 

\item However, an even greater challenge is to \emph{\textbf{see what’s missing}}. One of the most profound ways to see the world more clearly is to \emph{\textbf{look deliberately for the gaps}} -- \emph{\textbf{the negative space}}, as it is called in the art world; that is, the space surrounding the objects or issues of interest. 

In our daily and intellectual experiences there are \emph{gaps} of many sorts. If you’re studying some body of material, ask yourself to \emph{identify those concepts that you \textbf{truly do not fully understand}}. Those concepts may, in fact, be ideas that you were supposed to have mastered in an earlier class or at an earlier point in your life.

Don’t despair. \emph{\textbf{Honestly admitting} those \textbf{gaps}} in \emph{knowledge and understanding} is the first \emph{important step} in \emph{attempting} to fill them. 

\item A harder question is this: How can you see what’s truly invisible?

\item \emph{\textbf{Add the adjective and uncover the gaps}}. By just describing what was there, he was led to see the invisible.

\item \begin{exercise}[See the invisible]
Select your own object, issue, or topic of study and 
\begin{itemize}
\item \textbf{attach} an \textbf{adjective} or \textbf{descriptive phrase} (such as ``the First" before ``World War") that points out some reality of the situation, ideally some feature that is limiting or \textbf{taken for granted}. 

\item Then consider whether your \textbf{phrase} \textbf{suggests new possibilities or opportunities}. It might be helpful to think of this exercise as a word-association game. For example, if you are a student, you could consider a word such as ``semester" and then list the first few \textbf{adjectives} that come to mind -- for example ``busy," ``boring," ``tiring," ``exciting," and the like.

\item  Use your newfound adjectives to create interesting and provocative insights that might otherwise have gone unnoticed.
\end{itemize}
\end{exercise}


\end{itemize}

\subsection{Final thoughts: Deeper is better}
\begin{itemize}
\item Understanding simple things deeply means \emph{\textbf{mastering} the \textbf{fundamental} principles, ideas, and methods} that then create a \emph{\textbf{solid foundation}} on which you can build. 
\begin{itemize}
\item Seeking the \emph{essential} creates the \emph{core} or \emph{skeleton} that supports your understanding. 
\item Seeing what’s actually there \emph{without prejudice }lets you develop a less biased understanding of your world.
\item And seeing what’s missing helps you to identify the \emph{limits} of your knowledge, to reveal \emph{new possibilities}, and to create new solutions to complex problems. 
\end{itemize} 

From the physical world to society, academics, personal relations, business, abstract ideas, and even sports, a deep examination of the simple and familiar is a \emph{potent \textbf{first step}} for learning, thinking, creating, and problem solving.

\item \emph{Earth} is that which is under where we stand.

\item  Among the \emph{\textbf{goals}} of this book are to describe \emph{how you can \textbf{construct original ideas}}, to show \emph{how you can \textbf{solve old problems}}, and to reveal \emph{how you can \textbf{create new worlds}}. 

\item Here we are advocating a process that
\begin{itemize}
\item starts with your \emph{\textbf{most comfortable surroundings}}, your \emph{most familiar territory}, \emph{the basics} that you know best, 
\item and encourages you to \emph{\textbf{search deeply}} for features that you \emph{don’t ordinarily perceive}.
\end{itemize}

\item \emph{The \textbf{familiar} is full of \textbf{unseen depth} and wonder}. \emph{Clear away the distractions}, \emph{see what’s actually there}, and \emph{make the invisible visible}.
\end{itemize}

\section{Igniting Insights through Mistakes -- Fail to Succeed}
\begin{itemize}
\item \emph{\textbf{``Fail" is not an obscene word.}} 

In our society ``\emph{fail}" is viewed as another \emph{offensive} four-letter word beginning with ``f." The typical attitude that \emph{mistakes should be avoided} is patently wrong and has several detrimental consequences. The mind-set that mistakes are poisonous often \emph{freezes us into inaction}.  

If we have the healthier attitude that failure is a potent teacher and a scheduled stop along the road to success, then we find ourselves liberated to move forward sooner, because \emph{mistakes are actions we definitely can take at any time}. 

\emph{If you’re stuck, a mistake can be just the thing to unstick you}.

\item \emph{Any \textbf{creative accomplishment} evolves out of lessons learned from a \textbf{long succession of missteps}}. Failure is a \emph{critical} element of effective learning, teaching, and creative problem solving. Mistakes direct our attention in productive ways by forcing us to focus on the specific task of determining why the attempt at hand failed. 

\emph{\textbf{Effective failure} is an important, positive} (and, as in the case of Microsoft, lucrative) step toward \emph{success}.

\item Viewing failure as an \emph{\textbf{opportunity}} for learning requires a \emph{fresh mind-set}. 

Once you make the mistake, you can ask, ``\emph{\textbf{Why is that wrong?}}" Now you’re back on track, tackling the original challenge.

\item Students need to experience the arc of \emph{\textbf{starting with failure} and \textbf{ending with success}}. Teachers need to embrace the power of failure by consciously inspiring students to l\emph{earn the productive potential of making mistakes} as important steps toward understanding.

\item \begin{exercise}[Fail nine times]
The next time you face a daunting challenge, think to yourself, ``In order for me to resolve this issue, I will have to \textbf{fail nine times}, but on the tenth attempt, I will be successful." This attitude frees you and allows you to \textbf{think creatively without fear of failure}, because you understand that learning from failure is a forward step toward success. Take a risk and when you fail, no longer think, ``Oh, no, what a frustrating waste of time and effort," but instead extract a new insight from that misstep and correctly think, ``Great: one down, nine to go -- I’m making forward progress!" And indeed you are. 

After your first failure, think, ``Terrific, I’m $10\%$ done!" Mistakes, loss, and failure are all flashing lights clearly pointing the way to \textbf{deeper understanding} and \textbf{creative solutions}.
\end{exercise}

\item The \emph{\textbf{moral}} of this chapter’s story is that \emph{\textbf{mistakes} are \textbf{positive} elements of quintessential thinking} and \emph{\textbf{failure} is an important part of the \textbf{foundation} upon which to build success}.

\item Once you’re open to the positive potential of failure, \emph{failing \textbf{productively}} involves two basic steps: 
\begin{itemize}
\item \textbf{\emph{creating the mistake}} 
\item and then \textbf{\emph{exploiting the mistake}}.
\end{itemize}

\item  In this chapter we encourage you to embrace several facets of failure that can lead to success.
\begin{itemize}
\item One method is to \emph{\textbf{try your best to get it right}} and, if and when you fail, \emph{\textbf{isolate the specific failed features}} of that attempt. 
\item Alternatively, \emph{\textbf{deliberately}} try something that you know is \emph{wrong} to identify and \emph{clarify specifically \textbf{where the defects lie}}. 
\item \emph{\textbf{Analyze}} each \emph{\textbf{specific mistake}} to understand the reason it’s wrong, thus gaining new insights that may point you in the right direction.
\item Finally, \emph{examine the mistakes} to see whether the failed attempt might be a \emph{\textbf{correct solution to a different problem}}.
\end{itemize}
\end{itemize}

\subsection{Welcome accidental missteps -- let your errors be your guide}
\begin{itemize}
\item \emph{A \textbf{specific mistake} is an excellent \textbf{source of insight and direction}}, because a mistake gives you something specific to think about: ``This attempt is wrong because --." When you fill in the blank, you are forcing yourself to identify precisely what is wrong with your
attempted solution. 

This process \emph{\textbf{shifts}} the activity from trying to \emph{\textbf{think of a correct solution}}, which you \emph{don’t know at the moment}, to the activity of \emph{\textbf{correcting mistakes}}, which is often \emph{something you can do}.

\item She could have used that exact same technique by simply giving herself the same prompts: \emph{make an attempt, find a flaw, fix it, make an attempt . . .} She could have been her own teacher. Furthermore, she can apply that technique to anything she wishes.

Mary’s story illustrates one specific, practical, broadly applicable strategy for effective thinking, learning, and creating. Successful students and famously successful people have used this strategy throughout history, and you can use it for your own benefit.

\item \emph{First drafts are not just for writers}. Thomas Edison was famous for his \emph{\textbf{incremental approach}} to intentional invention: 
\begin{itemize}
\item try something; 
\item see what’s wrong; 
\item learn from the defect;
\item  try again.
\end{itemize}
When he said that invention is $1\%$ inspiration and $99\%$ perspiration, the perspiration was the process of incrementally making mistakes and learning from them to make the next attempts apt to be closer to right. 

\item Success is not about almost always succeeding. 

\emph{Success} is about \emph{\textbf{persisting}} through \emph{the process of repeatedly failing} and \emph{\textbf{learning from failure}}.

\item ``The way to get good ideas is to \emph{\textbf{get lots of ideas}} and \emph{\textbf{throw the bad ones away}}." -- Linus Pauling

\item \emph{\textbf{You may not know how to do it right, but you can certainly do it wrong.}} 

A good way to \emph{generate useful mistakes} is simply to tackle the issue at hand by \emph{\textbf{quickly} constructing \textbf{the best solution}} you can \emph{with \textbf{little or no effort}}.

\item \begin{exercise}[Don’t stare at a blank screen]
Take an issue or \textbf{problem} you are facing. For example, you may want to get organized or write a business plan or improve a course grade or write an essay or get more out of life. 
\begin{itemize}
\item \textbf{Open} up a blank document on your computer. Now just \textbf{quickly type any ideas} -- good, bad, inaccurate, or vague -- that you have about the issue. 

\textbf{Don’t hesitate to record ideas or phrases that you know are not quite right} -- no one (except you) is going to read what you write. Your ideas will be very bad in many ways.  Congratulations—excellent start!

\item You may not feel that writing down bad ideas is a worthwhile start, but one thing is certain: \textbf{writing down bad ideas is something anyone can do}.  That is not a challenge. But it’s also not the end of the story.

\item Now \textbf{read what you wrote} and focus on two features: what’s right and what’s wrong. When you just write down ideas without worrying about correctness, structure, or elegance, your thoughts about the subject often flow out freely and clearly. 

The ideas that you are trying to express are in you, so when you write without fretting about the mistakes, the surprising reality is that you will often say what you really want to say. You will include partial truths as well as some unexpected gems.

\item Now you have something to do. You can \textbf{tease out the good elements}.  Looking for \textbf{good features} in your \textbf{bad first attempt} is a \textbf{great first step} toward some creative, high-quality work.

\item Next, see if you can \textbf{recognize} and \textbf{exploit what’s wrong}. When something is bad, it’s often easy to see what’s wrong and \textbf{identify mistakes}. 

\item Now you have something to do: \textbf{correct the errors you see}. 
\end{itemize}
\end{exercise}
You have traded in the impossible task of creating something that’s perfect for the much easier task of mining gems and correcting errors. 

You are now doing something different -- you are \emph{not creating a work on a blank canvas} but instead \emph{you are \textbf{responding to a work already there}}. 

Your responses, in turn, will lead to new good ideas that you could not have created before you made the requisite mistakes. In making this action item practical, you must be sure to give yourself enough time for the required iterations. 

Thus you must commit to starting your effort (that is, creating a crummy draft or first attempt) far enough in advance to allow the necessary gestation and iteration that leads to a polished work of which you will be proud. So start early.


\item \emph{\textbf{Give credit to failure.}} 

Instructors need to celebrate students’ useful missteps, because those failed attempts lead to important epiphanies at the end. 

If at first you do succeed, try, try again (until you finally fail). If a student presents a correct solution, we will sometimes ask for another volunteer to present an \emph{erroneous} solution to the same challenge, so the class can explore the reasons behind that defect. \emph{\textbf{Understanding what doesn’t work and why is valuable knowledge}}. 

By not exploiting this great opportunity to learn from their mistakes, they’re essentially throwing away -- on average -- $20\%$ of their grade on their next exam before they’ve even taken it, and they’re building future work on a cracked foundation. 
\end{itemize}

\subsection{Finding the right question to the wrong answer}
\begin{itemize}
\item Sometimes when your attempt fails to resolve one issue, you might discover that you have actually found \emph{an \textbf{imaginative answer} to a \textbf{totally different question}}.

That is, your \emph{bad solution} to one problem might lead to a \emph{different project} altogether -- a project suggested by the accidental virtues of your mostly bad attempt.

\item Two reactions to mistakes. So when you see or make a mistake, you have at least two actions to take: 
\begin{itemize}
\item (1) let the mistake lead you to a \emph{\textbf{better attempt}}, and/or
\item (2) ask whether the mistake is a \emph{correct answer to a \textbf{different question}}.
\end{itemize}
 

\item \emph{\textbf{Have a bad day}}

Bad days happen to good people. \emph{\textbf{What separates the good from the great is how we react to that bad day}}. Bad days often include uncomfortably clear lessons about how to grow, learn, or reassess. 

So the next time you’re having a bad day, make the conscious effort to find and \emph{extract positive lessons} from those \emph{not-so-positive experiences}.
\end{itemize}

\subsection{Failing by intent}
\begin{itemize}
\item \emph{\textbf{Going to the extreme.}} 

Now we take the act of failing to its extreme: One profound way to \emph{make new discoveries} is to \emph{\textbf{intentionally fail}} along the way. \emph{Deliberately exaggerating or considering extreme}, \emph{impractical} scenarios often frees us to have an unforeseen insight.

\item \begin{exercise}[Exaggerate to generate errors]
Consider an issue or problem and
\begin{itemize}
\item  now \textbf{exaggerate some feature} of it to a ridiculous \textbf{extreme}. For example, take a political, personal, business, academic, or other issue and create an extremely exaggerated perspective on the subject. 

If you are arguing one side of an issue (whether or not it is the side you truly believe), make the \textbf{argument} so \textbf{exaggerated} that you realize that it’s way over the top. 

\item Now \textbf{study your exaggerated description} and \textbf{discover} some \textbf{underlying defect}. 

\textbf{Does that defect exist in your original, nonexaggerated perspective}? You might apply this exercise to such things as \textbf{organizational structures} or sports or any other activity or belief. As if you were conducting a stress test, you might apply this exercise to something that works well and learn how it breaks down. 
\end{itemize}
\end{exercise}
The strategy of exaggeration to extremes can be applied to any issue, from writing to marketing to product development to politics. You might perform this exercise physically or metaphorically, depending on the issue.

\item \emph{\textbf{Learning from other’s missteps. }}

Often we don’t even have to be the ones to actually make the mistake. When we see an evil or inept person in action, or we see a good or competent person make a huge blunder, we find it easy to recognize the pitfall and consciously turn that moment into a learning opportunity.
\end{itemize}

\subsection{Final thoughts: A modified mind-set}
\begin{itemize}
\item \emph{Mistakes and failure} are not signs of \emph{weakness}; instead they are \emph{\textbf{opportunities} for \textbf{future success}}. 

\item \emph{Failure is a sign of a creative mind}, \emph{of original thought and strength}.

\item A person who is \emph{\textbf{willing to fail}} is someone who is \emph{willing to \textbf{step outside the box}}. \emph{Being willing
to fail} is a liberating attribute of \emph{\textbf{transformative thinking}}.

\item \textit{Failing is \textbf{progress}}; it’s not losing ground. Often a mistake or \emph{the revelation of error} is the most important step toward \emph{success}. 

\item When you’re stuck, and \emph{you don’t know what to do}, \emph{\textbf{don’t do nothing}} -- instead, \emph{\textbf{fail}}. 

Making a \emph{specific mistake} puts you in a different and better position than you were in before you started. And it’s a forward step you know you can actually take.

\item Let’s be honest: failure can be frightening and uncomfortable -- a true trial by fire. 

\item Problems that require truly \emph{creative solutions} are problems that you simply \emph{do not yet know how to solve}. This book is all about being successful -- even if and often because you \emph{fail first}.

\item Note that failure and mistake has its own \emph{\textbf{cost}} and its \emph{\textbf{impact}}. If possible, \emph{\textbf{making mistake and failure as early as possible when the cost is low}}. 
\end{itemize}

\section{Creating Questions out of Thin Air -- Be Your Own Socrates}
\begin{itemize}
\item Questions can be an inspiring guide to insight and understanding. In fact, the very act of \emph{\textbf{creating questions}}, for yourself, is a profound step toward \emph{\textbf{understanding}} -- even if \emph{the questions are neither asked nor answered}.

\item You would certainly be astonishingly successful if you had your very own personal Socrates with you at all times, prodding you with \emph{\textbf{the right leading questions}}. 

\item You can \emph{generate} your own questions that \emph{challenge} your own \emph{assumptions} and lead to insights. You can become your own Socrates.

\item \emph{\textbf{Wisdom just for the asking}}.

Traditionally people believe that it’s \emph{in the answering of questions that progress is made}. In fact, \emph{\textbf{creating questions is as important as answering them}}, if not more so, because \emph{\textbf{framing good questions} focuses your attention on the \textbf{right issues}}. 

\emph{Constantly formulating and raising questions} is a \emph{\textbf{mind-opening habit}} that \emph{\textbf{forces} you to have a \textbf{deeper engagement}} with the world and a different inner experience. 

\emph{Asking yourself \textbf{challenging} questions} can help you 
\begin{itemize}
\item \emph{\textbf{reveal hidden assumptions}}, 
\item \emph{\textbf{avoid bias}}, 
\item \emph{\textbf{expose vagueness}},
\item \emph{\textbf{identify errors}}, 
\item  and \emph{\textbf{consider alternatives}}. 
\end{itemize} Generating questions can help \emph{direct your \textbf{next steps}} toward \emph{\textbf{deeper understanding}} and \emph{\textbf{creative problem solving}}.
\end{itemize}

\subsection{How answers can lead to questions}
\begin{itemize}
\item Even when you do know the answer, asking, ``\emph{\textbf{What if ... ?}}" is a great way to \emph{see more} and \emph{delve deeper}. 

\item If you gained nothing else from your formal education but the mind-set of always asking, ``What if ... ?" then you would have benefited tremendously from your schooling. 

\item ``What if ... ?" questions invite you to see the world differently because those questions \emph{force you to \textbf{challenge the status quo}} and to \emph{explore the \textbf{limits} of your \textbf{understanding}}. 

\item A \emph{transformative} but \emph{challenging} personal policy is to \emph{\textbf{never pretend to know more than you do}}.

\item Don’t build on \emph{\textbf{ambiguity}} and \emph{\textbf{ignorance}}. When you don’t know something, admit it as quickly as possible and immediately take action -- \emph{\textbf{ask a question}}.

\item \emph{\textbf{Overcoming bias.}} 
\begin{itemize}
\item One profound habit of thinking individuals is to first \emph{\textbf{acknowledge} their \textbf{biases}} and then \emph{\textbf{intentionally overcome}} them. 

\emph{Asking challenging questions} can help. Passionately argue an issue from the opposite point of view, and ask probing and difficult questions that \emph{\textbf{challenge your original stance}}. 

\item Be brutally \emph{\textbf{honest}} and see what’s actually there rather than what’s expected. 

Get in the habit of asking, ``\emph{\textbf{Do I really know?}}" and \emph{\textbf{refuse to accept assertions blindly}}.

\item \emph{Challenge everything and everyone} -- including your teachers. Don’t be intimidated. 

You are the best \emph{authority} on what you don’t understand -- \emph{\textbf{trust yourself}}: don’t be afraid to ask the questions you need to ask, and be brave enough to \emph{\textbf{change your thinking}} when you \emph{\textbf{uncover a blind spot}}.
\end{itemize}

\item \emph{\textbf{Take another look.}}
\begin{itemize}
\item Get in the habit of asking how the issue looks \emph{from various viewpoints}. \emph{\textbf{Frame questions in different ways}}. \emph{Alternative perspectives} lead to new sights and new insights. 

\item Moreover, we can investigate issues from an \emph{\textbf{evolutionary point of view}} and ask \emph{what is causing change}; how those influences have \emph{caused change over time}; and how they will cause change \emph{in the future}. 

\item Try to \emph{\textbf{bridge ideas}} from one discipline or area to another. 

Ask whether the skills, attitudes, techniques from one subject might be applied to another subject and to your work or life. 

Everything fits together and interacts -- take the transformative step of asking how.
\end{itemize}

\item \begin{exercise}[Teach to learn]
Consider an idea or topic you are trying to better understand, and
\begin{itemize}
\item create a list of \textbf{fundamental questions} that will guide you to a \textbf{complete explanation}, including motivation, examples, overview, and details, of that subject.

\item With those questions (and their corresponding answers) in hand, prepare a \textbf{minilecture} and consider \textbf{delivering} it to some \textbf{audience} -- family, friends, or even a teacher. 

\item Ask them \textbf{questions} to measure \textbf{how well} you \textbf{understood} and \textbf{articulated} your message.
\end{itemize}
\end{exercise}

\item \emph{\textbf{Is the standard preparation really preparing you?}}

When do we teach students how to perform well under such time pressure? The answer is \emph{never}. 

To prepare students effectively, instructors should teach students how to perform under the same conditions that they will face when the major assessment occurs. 
\end{itemize}

\subsection{Creating questions enlivens your curiosity}
\begin{itemize}
\item \emph{\textbf{A questionable habit. }}
\begin{itemize}
\item If you want to get more out of what you hear or see, \emph{\textbf{force yourself to ask questions}} -- in a lecture, at a meeting, while listening to music, watching TV, or viewing art. 

People who ask lots of \emph{\textbf{probing questions}} \emph{outperform} those who \emph{don’t engage with the ideas}. 

\emph{Constantly generate questions} and then ask them -- that mind-set will lead to a richer appreciation of the issues.

\item Instead of asking whether there are questions, tell your \emph{\textbf{listeners}} that they are to \emph{\textbf{create questions}} -- an important habit to \emph{develop for lifelong learning and curiosity}.

\item Whether or not you are asked to write down questions, constantly \emph{come up with questions \textbf{on your own}}.

\item Of course, actually \emph{\textbf{asking the questions you create}} is also an excellent exercise -- it allows for \emph{further clarity}, and it shows the presenter you’re \emph{genuinely thinking about the material}. 
\end{itemize}


\item \emph{\textbf{Be thought provoking.}}
\begin{itemize}
\item Getting in the habit of asking questions will transform you into an \emph{\textbf{active} (rather than passive) \textbf{listener}}. This practice forces you to have a different \emph{inner life experience}, since you will, in fact, be \emph{\textbf{listening more effectively}}. 

\item It’s what goes on \emph{inside your head} that makes all the difference in how well you will convert what you hear into something you learn. \emph{\textbf{Listening is not enough}}.

If you are \emph{constantly \textbf{engaged in} \textbf{asking yourself questions about what you are hearing}}, you will find that even boring lecturers become a bit more interesting, because much of the interest will be coming from what you are generating rather than what the lecturer is offering. 

\item When someone else speaks, you need to be \emph{\textbf{thought provoking}}!
\end{itemize}
\end{itemize}

\subsection{What's the real question?}
\begin{itemize}
\item Sadly, many people spend their entire lives \emph{\textbf{focusing on the wrong questions}}.

\item So \emph{before} you succumb to the temptation to \emph{immediately spring to work on the answer}, \emph{always stop and first ask}, ``\emph{\textbf{What’s the real question here?}}" 

\item \emph{\textbf{Effective questions}} turn your mind in directions that \emph{lead to new insights and solutions}. They \emph{highlight hidden assumptions} and indicate \emph{directions} to take to \emph{make progress}. 

\item \emph{\textbf{Effective questions lead to action} and are \textbf{not vague}.}

\item \emph{The \textbf{right} questions \textbf{clarify your understanding} and \textbf{focus} your attention on \textbf{features that matter}.}

\item \emph{\textbf{Effective} questions expose \textbf{the real issue}.}

\item Seeking the right question forces you to realize that there are at least two kinds of \emph{ignorance}: 
\begin{itemize}
\item cases in which you \emph{know the right question} but \emph{not the answer},
\item and cases in which you \emph{don’t even know which question to ask}.
\end{itemize}

\item  \begin{exercise}[improve the question]
From a student’s point of view, the question ``How can I get better grades?" is not the most effective route to higher grades. Questions such as ``How can I learn to think better and understand more deeply?" ``How can I learn to communicate better?" ``How can I increase my curiosity?" are far more \textbf{constructive}.

\begin{itemize}
\item For the questions below that are relevant to you, and more importantly for the ones you \emph{will create}, \textbf{craft more focused questions} that might lead to a \textbf{productive conclusion}. 


\item Try to create questions that 
\begin{itemize}
\item \emph{\textbf{expose hidden assumptions}}, 
\item \emph{\textbf{clarify issues}}, 
\item and \emph{\textbf{lead to action}}.
\end{itemize}

\item Apply this exercise whenever you are confronted with a question in your own life -- that is, constantly question your own questions.
\end{itemize}
\end{exercise}


\item \emph{\textbf{The right questions in the classroom}}. 

When a teacher gives an assignment, that instructor has the pedagogical responsibility to ask, ``What beneficial change will this exercise help foster or develop in my students?"

Realize that every time you \emph{\textbf{write}} anything, you can \emph{harness that moment as an opportunity} to improve your \emph{communication} and \emph{argumentation skills}, which can help you literally every day at home, at work, and in the world. 

Teachers should craft assignments that promote long-term goals such as communicating and thinking more effectively. By asking questions about \emph{\textbf{goals}}, you are better able to extract the \emph{advantages} from assignments rather than mindlessly checking them off your to-do list. 

\emph{Remember to always \textbf{question the questions}}.

\item \begin{exercise}[Ask meta-questions]
Whether in the classroom, the boardroom, or the living room, 
\begin{itemize}
\item \textbf{asking questions} about an assignment or project \textbf{before beginning work} in earnest will always lead to a stronger final product.  

Ask, ``What’s the goal of this task?" and ``What benefit flows from the task?" 

\item Keep that benefit in mind as you move forward. 
\end{itemize}
A by-product of this exercise is that it often saves time, because it \textbf{focuses} your attention on the \textbf{core issues} and allows you to \textbf{quickly} \textbf{clear up the initial confusion} that always is present at the start of any project or task.
\end{exercise}
\end{itemize}



\subsection{Final thoughts: The art of creating questions and active listening}
\begin{itemize}
\item \emph{The \textbf{right questions}} can be incredibly powerful tools for understanding and learning. 

\item Great questions can lead to \emph{\textbf{insights}} that will make a difference. 

\item You can \emph{\textbf{create great questions}} using \emph{concrete and straightforward techniques} -- questions that guide you and \emph{\textbf{arouse your curiosity}}. 

\item \emph{Questions} give us a breath of \emph{\textbf{inspiration}} and \emph{\textbf{insight}}; thus we associate the art of questioning with the element \emph{Air}.

\item Constantly thinking of questions is a mind-set with tremendous impact. 
\begin{itemize}
\item You become more alive and curious, because you are \emph{\textbf{actively engaged}} while you are \emph{\textbf{listening}} and living. 

\item You become \emph{\textbf{more open to ideas}}, because you are constantly discovering places where your assumptions are exposed. 

\item You \emph{\textbf{take more effective action}}, because you clarify what needs to be done.
\end{itemize}
\end{itemize}

\section{Seeing the Flow of Ideas -- Look Back, Look Forward}
\begin{itemize}
\item New ideas today are built on the ideas of yesterday and illuminate the way to the brilliant ideas of tomorrow. Innovators recognize that each new idea \emph{extends a line} that started in the past and travels through the present into the future. 

\item Successful and effective learners and innovators harness \emph{\textbf{the power of the flow of ideas}}, which suggests the element \emph{Water}.

\item There’s always more: every advance can be the \emph{launchpad to far greater advances yet to be discovered}.

\item \emph{Solutions to little problems} generate \emph{solutions to great problems}.

\item \begin{exercise}[Iterate ideas]
You don’t need an army of thousands of individuals to struggle a thousand years to address a challenge. The only person who needs to move forward little by little is \textbf{you}. \textbf{Engineer your own evolution}. 
\begin{itemize}
\item Take a homework assignment, essay, or project that you’re facing and \textbf{quickly just do it}; that is, tackle the questions, draft the essay, or move forward on the project \textbf{at a fast-forward speed} that will surely generate a work that is, at best, subpar. 

\item Now consider that poor effort as your starting point: \textbf{react} to that work and \textbf{start to improve and iterate}.
\end{itemize}
The flow of iteration will lead to a refined final product. Notice how this flowing mind-set perfectly coincides with the elements of failure we introduced earlier.
\end{exercise}

\item \emph{To \textbf{understand current ideas} through \textbf{flow}}, 
\begin{itemize}
\item first \emph{\textbf{find easier elements}} that lead to what you want to understand, and
\item then \emph{\textbf{build bridges}} from those \emph{easier elements} to the \emph{ideas} you wish to \emph{master}. 
\end{itemize}

\item \emph{To \textbf{generate new ideas} through \textbf{flow}}, 
\begin{itemize}
\item first \emph{\textbf{modify} an \textbf{existing} idea} within \emph{\textbf{its own context}} and
\item then \emph{apply that same idea in \textbf{different settings}}. 
\item Then you can construct \emph{\textbf{extensions}}, \emph{\textbf{refinements}}, and \emph{\textbf{variations}}.
\end{itemize}
 
\end{itemize}

\subsection{Understanding current ideas through the flow of ideas}
\begin{itemize}
\item To truly understand a concept, discover how it \emph{naturally \textbf{evolves} from \textbf{simpler thoughts}}. 

Recognizing that the present reality is a moment in a \emph{\textbf{continuing evolution}} makes your understanding fit into a more coherent structure.

\item Every subject is an \emph{ongoing journey of discovery and development}. It is not just a laundry list of disconnected topic, topic, topic, but \emph{\textbf{a flow of ideas}} that build upon each other. 

When we see and understand that \emph{\textbf{these ideas are connected}}, they become more interesting, more memorable, and more meaningful.

\item \begin{exercise}[Think back]
Whenever you face an issue -- whether an area of study or a decision about a future path -- \textbf{consider what came before}. 
\begin{itemize}
\item Wonder how the issue at hand landed in front of you. 

Ask where and what it was yesterday, a month ago, a year ago, and so forth.
\end{itemize}
Everything, everyone has a \textbf{history} and \textbf{evolves}. Acknowledging that reality will allow you to generate new insights as well as create fruitful directions in which to move forward.
\end{exercise}

\item \emph{\textbf{Guessing what’s next anchors what’s there.}}
\begin{itemize}
\item To better master a subject, after you have been introduced to a new concept, \textbf{\emph{look beyond the new concept}} and just \emph{\textbf{guess}} what you think will \emph{\textbf{come next}} -- in a text or in a lecture or in any presentation. 

\item Even if your conjecture is not right, it’s still important. 

Being wrong allows you to better realize \emph{\textbf{what is truly there}}, and offers \emph{\textbf{insights}} as to \emph{how the ideas might actually fit together}.

\item Even when our guesses were completely off, they still helped us to view the previous material more fully by thinking \emph{how the \textbf{earlier} material \textbf{might have looked}} in the middle of a stream of progress rather than in isolation. 
\end{itemize}

\item \emph{\textbf{A look back makes earlier material easier.}}
\begin{itemize}
\item Once you understand a more advanced topic, \emph{\textbf{look back} to see \textbf{what brought you to where you are}}. That process will improve your \emph{understanding} both of \emph{the earlier work} and of \emph{the more advanced work}. 

\item \emph{The \textbf{earlier} material} will become \emph{easier}, \emph{clearer}, and \emph{more meaningful} because you will \emph{\textbf{see its significance}} through \emph{the later work} that came from it. 

\item \emph{The \textbf{more advanced} work} will also be easier since you will now see \emph{\textbf{how it grew} from the seeds that existed in the earlier work}. 
\end{itemize}
We have seen that the most successful people regularly undertake this important \emph{\textbf{reflective exercise}}. 


\item \emph{\textbf{One small step.}} 
\begin{itemize}
\item One of the most heartening realities of human thought is that all the new ideas we have are, in fact, \emph{\textbf{only tiny variations}} of what has been thought before. 

\item The difference between those who have great insights and those who don’t is that the first group actually \emph{take those baby steps}.
\end{itemize}
Students who embrace the mind-set that better ideas are literally right next door and that ``\emph{one more small step will get me there}" outperform those who believe that only the great minds make great progress.
\end{itemize}

\subsection{Creating new ideas from old ones}
\begin{itemize}
\item When you learn a new concept or master a skill, think about what \emph{\textbf{extensions}}, \emph{\textbf{variations}}, and \emph{\textbf{applications}} are possible.

\item A new idea or solution should always be viewed as a beginning. 

Effective students and creative innovators regularly strive to \emph{\textbf{uncover} the \textbf{unintended consequences}} of a lesson learned or \emph{a new idea}.

\item ``\emph{The time to work on a problem is \textbf{after you’ve solved it}.}" -- R. H. Bing

\item \begin{exercise}[Extend ideas]
Take a good idea from any arena -- work, society, or personal life. It need not be an idea you yourself originated.
\begin{itemize}
\item Now \textbf{engage} with that idea and 
\item \textbf{extend} it.
\end{itemize} 
The key is not to wonder whether the idea has extensions; it does. Your challenge is to \textbf{find} them.
\end{exercise}

\item \emph{\textbf{The best can get even better.}}
\begin{itemize}
\item Just as our own understanding can be deeper and richer than it currently is -- no matter where it is in its evolution -- an important perspective of successful thinking is that \emph{\textbf{the best can be improved}}. 

\item In fact, starting with \emph{\textbf{what is currently the best} is often the \textbf{ideal} place to expect great \textbf{improvements}}. We limit ourselves when we think that success is an end.

\item Sometimes getting to \emph{the current highest level of perfection was so difficult} or so satisfying that we \emph{can’t imagine further heights}. 

\emph{\textbf{The newcomer}} did not experience the toil, did not live through the trials and failures and hard-won small steps. The young person or the person new to the field sees that issue in its solved condition as just the way the world is.

\item Often the solution to a difficult problem comes from a \emph{\textbf{struggled focus}} on the issue. 

\item We must get in the habit of \emph{seeing each \textbf{advance} as putting us on the \textbf{lower slope} of a \textbf{much higher peak} that has yet to be scaled}.

\item The same is true of \emph{learning new and \textbf{increasingly difficult} concepts or mastering skills at \textbf{increasingly higher} levels}. You may have to struggle to finally master an idea or a skill. Having toiled to get that far, you may think that it would be impossible to go yet further, or you may just feel worn out. But after you have reached one level, that is where you start. That is the platform from which you can proceed even further—whether that starting point is a high grade, a professional accomplishment, or a profound insight; go for it!
\end{itemize}

\item \begin{exercise}[Once you have it, see if you can improve it]
Take an essay you’ve written or a solution to an issue and \textbf{create a different, better one}. 
\begin{itemize}
\item Assume there is a \textbf{mistake} or \textbf{omission} or \textbf{missed opportunity} in your work -- there always is! 
\item Now find it (yet another example of the insights we can gain by failing).
\end{itemize}
This activity is much more \textbf{challenging} than it might at first appear.  We are biased and limited by what we already know -- especially since we know it works. 

However, \textbf{moving beyond that bias} can lead to new answers that, in turn, can lead to new insights and more effective solutions. People who make this evolutionary iteration a standard practice are far more successful in their education and in life than those who see an answer as an end.
\end{exercise}

\item \emph{\textbf{Making it practical.}} 
\begin{itemize}
\item Human beings do not instantly see far. Our field of intellectual vision is \emph{\textbf{limited} to a few steps from where we are now}. We must \textbf{\emph{acknowledge}} that however far we do see, our \emph{vision} extends merely to a horizon beyond which a far bigger world will become visible.

\item How can we start the process of exploring \emph{where new ideas can \textbf{lead} us}? Ask, ``\emph{\textbf{What’s next?}}" 

\item Explore the \emph{\textbf{connective}}, ``\emph{If this, then that.}" Follow the hypothetical results of the idea. 

\item And when you have arrived at the next step, let it settle as the new reality and only then think, “\emph{\textbf{What now?}}”
\end{itemize}
To be sure, not every sequence of consequences that we imagine will actually come to pass or lead to fertile new ground, but \emph{exploring those consequences} \emph{\textbf{several steps forward}} can have great value. 

\emph{Following that flow} can \emph{\textbf{highlight}} some \emph{\textbf{fallacies}} in seemingly sound schemes. 


\item Any example of a practice that is \emph{accepted} today but will be viewed as \emph{immoral} in the future must be a \emph{custom} that we now view as \emph{perfectly fine}. Only in the future will that cultural norm be viewed from a different angle and deemed unacceptable.

\item It is impossible to avoid bias -- it infuses itself through our upbringing, our values, our society, and our community. The first real action item for all of us is to acknowledge (unabashedly) that we are all prejudiced.
\end{itemize}

\subsection{Final thoughts: ``Under construction" is the norm}
\begin{itemize}
\item Many people believe that the ideal state of the world is one in which everything is \emph{\textbf{finished}} and \emph{\textbf{perfect}}.

\item It is more realistic and healthier to view our world as one in which construction is always under way -- \emph{\textbf{everything is a work-in-progress}}.

\item \emph{\textbf{The right dream.}}

You may dream of creating that \emph{one new idea that will solve lots of problems} (and lead to fame and fortune). But \emph{\textbf{the better dream}} is to \emph{see yourself \textbf{standing} on what seems to be the \textbf{summit}} and \emph{\textbf{climbing higher}} by taking one small step after another. 

That modest habit of effective thinking will \emph{help you accomplish things you never dreamed possible}.
\end{itemize}

\section{Engaging Change -- Transform Yourself}
\begin{itemize}
\item The fifth element of effective learning and thinking is the simplest and most difficult, the most important and most dispensable. 

\item Each of the preceding four techniques has the goal of changing you into someone who thinks and learns better.

\emph{\textbf{Change}} is really the \textbf{goal} of the whole story.

\item Of course, in reality, change seems hard -- not simple. 

\item However, like the way to happiness, the path to change is \emph{not through greater willpower} and \emph{harder work}, but rather \emph{through \textbf{thinking differently}}.

\item The first four elements of effective thinking do the heavy lifting. 
\begin{itemize}
\item They invite you to understand fundamental ideas, to look for essential elements, and to extend what you already know. 
\item They suggest pointed questions for you to pose to yourself and others that cause you to think of new ideas, and 
\item they point out the value of failure and errors on the road to success.
\end{itemize}

\item The fifth element is a \emph{meta-lesson}. It recommends that you \emph{adopt the habit of \textbf{constructive change}}.

Don’t be afraid to change any part of yourself -- you’ll still be there, only better.

\item The fifth element recommends that you actually do it. \emph{\textbf{Just do it.}} 

If the ability to change is part of who you are, then you are \emph{liberated from worry about weaknesses or defects}, because you can \emph{\textbf{adapt}} and \emph{\textbf{improve}} whenever you like.

\item \emph{\textbf{Definition of insanity: Doing the same thing and expecting a different outcome}}.

\item \emph{\textbf{And now for something completely different.}} 

There is a \emph{subtle perspective} about \emph{\textbf{improvement}} and about \emph{\textbf{better performance}} that can alter how you approach the task of \emph{changing yourself}. Namely, people who \emph{perform better} can be viewed as actually doing a \emph{\textbf{different task}}, rather than \emph{doing the same task better}.

\item Individuals who are more successful at anything are \emph{performing their task with their \textbf{eyes open}}; that is, the \emph{activity} they are doing is \emph{different} from the activity that less successful people are undertaking. 

Often people describe the distinction between the skilled practitioner and the less skilled practitioner by saying that the skilled person is better at the task. But a \emph{more useful and accurate perspective} is that the \emph{skilled} practitioner is doing a \emph{\textbf{fundamentally different task}} -- one that you could master as well.

\item To become more skillful and successful, you might think in terms of \emph{\textbf{altering what you do}}, rather than thinking in terms of how well you do it. 

Instead of thinking,``Do it better," think,``\emph{\textbf{Do it differently.}}" 
\begin{itemize}
\item If you want to learn a subject, instead of memorizing rules and facts, concentrate on truly \emph{\textbf{understanding the fundamentals deeply}}.

\item If you want to think of new ideas, don’t sit and wait for inspiration. Instead, apply strategies of \emph{\textbf{transformative thinking}} such as \emph{\textbf{making mistakes}}, \emph{\textbf{asking questions}}, and \emph{\textbf{following the flow of ideas}}.
\end{itemize}

\item \begin{exercise}[Expert change]
If you’re learning something, solving a problem, or developing a skill, 
\begin{itemize}
\item imagine in detail what a more skilled practitioner does, or what added knowledge, understanding, and previous experience the expert would bring to the task. 

In other words, describe the different task that an expert \textbf{would be doing} compared to what you are currently doing in undertaking your task.

\item Instead of thinking that you are going to be doing something that is harder -- requiring more concentration and more effort -- \textbf{think} in terms of what kind of knowledge or skill or strategy would \textbf{make the task an easier one}.
\end{itemize}
\end{exercise}

\item \emph{\textbf{You can do it}}
\begin{itemize}
\item For \emph{\textbf{effective thinking}}, differences in native ability are dwarfed by \textbf{\emph{habits}} and \emph{\textbf{methods}}. 

Those individuals who may appear to be the brightest in the sense of catching on to things immediately and being able to deal with complexity without getting confused are rarely the most \emph{productive}, \emph{imaginative}, or \emph{effective}.

\item Some very bright people can keep amazingly complicated things straight, but they may fail to try new perspectives and new ideas. 

\item Coming up with new ideas requires the habit of \emph{employing \textbf{thinking techniques} that generate new ideas}. 

Being imaginative is a \emph{learnable} skill, not an inborn characteristic like having blue eyes. 

The secret to solving problems and coming up with new ideas is not to find different parents, but to use different strategies of \emph{\textbf{transformative thinking}}.
\end{itemize}

\item \emph{\textbf{Messing things up.}} 
\begin{itemize}
\item Often \emph{\textbf{the most profound advances}} you can make in your life come through \emph{\textbf{experiences that challenge the life you have}}. 

\item The image of \emph{building a life from solid success to solid success is a wonderful dream}, but it is only a fantasy. Instead, you must \emph{let old ideas crumble} in the face of challenges in order to \emph{build yet better structures}.

\item \emph{\textbf{Don’t mute voices that challenge your beliefs}}. Listen for \emph{whispers of doubt} and turn those doubts into helpful and positive tests of assumptions, ideas, and theories. 

\item Doubt can be unsettling, but it does not have to be. You can turn \emph{doubt} into a \emph{comfortable and insightful guide} along the road to \emph{true change}.

\item If you are open to new ideas, and you allow yourself to follow your changing opinions and passions, they will lead you in directions you did not originally expect to go. 
\end{itemize}


\end{itemize}


\subsection{Final thoughts: Becoming the quintessential you}
\begin{itemize}
\item if you adopt the elements of thinking suggested in this book and make those habits part of who you are, you will develop mental strength and capacity, and you will become a more effective and creative thinker. Moreover, applying these elements to yourself leads you to clarify the core of your own self-definition—including your values, morals, ethics, and beliefs. 
\end{itemize}

\newpage
\bibliographystyle{plainnat}
\bibliography{reference.bib}
\end{document}
