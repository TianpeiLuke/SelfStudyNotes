\documentclass[11pt]{article}
\usepackage[scaled=0.92]{helvet}
\usepackage{geometry}
\geometry{letterpaper,tmargin=1in,bmargin=1in,lmargin=1in,rmargin=1in}
\usepackage[parfill]{parskip} % Activate to begin paragraphs with an empty line rather than an indent %\usepackage{graphicx}
\usepackage{amsmath,amssymb, mathrsfs,  mathtools, dsfont}
\usepackage{tabularx}
\usepackage{tikz-cd}
\usepackage[font=footnotesize,labelfont=bf]{caption}
\usepackage{graphicx}
\usepackage{xcolor}
%\usepackage[linkbordercolor ={1 1 1} ]{hyperref}
%\usepackage[sf]{titlesec}
\usepackage{natbib}
%\usepackage{tikz-cd}

\usepackage{../../Tianpei_Report}

%\usepackage{appendix}
%\usepackage{algorithm}
%\usepackage{algorithmic}

%\renewcommand{\algorithmicrequire}{\textbf{Input:}}
%\renewcommand{\algorithmicensure}{\textbf{Output:}}



\begin{document}
\title{Book Reading Summary: The 7 Habits of Highly Effective People}
\author{ Tianpei Xie}
\date{Dec. 10th., 2023}
\maketitle
\tableofcontents
\newpage
\section{Reading Summary}
\subsection{Keywords}
\begin{itemize}
\item personality ethic, character ethic, paradigms, maps, principles, values, guidelines,  perception, attitudes, behaviors
\item paradigm shift, inside-out, quantum change, conditioning effects, process,  growth, development
\item habit, knowledge, skill, desire,  principle-centered, character-based, interpersonal effectiveness
\item maturity, dependence, independence, interdependence, physically, mentally, emotionally
\item production, production capacity, P/PC balance, assets, physical, financial, human
\item private victory, principle of personal vision, principle of personal leadership, principle of personal management 
\item public victory, principle of interpersonal leadership, principle of empathic communication, principle of creative cooperation
\item principle of balanced renewal
\item self-awareness, vision, social mirror, freedom to choose, proactivity, responsibility, consistent, subordinate an impulse to a value, language
\item reactive, conditions/conditioning,  act or be acted upon, wait for something to happen, paradigm of determinism
\item circle of concern, circle of influence, direct control, indirect control, no control, choose our response
\item consequences, mistakes, principle, harmony, regret, acknowledge, learn, make and keep commitments and promises
\item unique human endowments, self-awareness, imagination, conscience, independent will
\item first creation, mental creation, second creation, physical creation
\item leadership, management
\item imagination, visalization, rescripting
\item personal mission statement, values, principles, changeless, ability to change, mission, strength
\item center, intrinsic, security, guidance, wisdom, power, interdependent
\item principle-centered, other centers, self, family, spouse, money, work, possession, pleasure, friend, enemy, church
\item personal mission statement, roles, proactivity, circle of influence, personal responsibility, review, many rewrites, introspection, analysis, expression
\item personal leadership, expand perspective, visualization, affirmation
\item organization mission statement, involvement, commitment
\item independent will, prioritization, organize, execute around balanced priorities, time management
\item time management matrix, quadrant i, crises, problems, quadrant ii, importance, urgence, proactivity
\item opportunity-minded, problem-minded 
\item tool, coherence, balance, quadrant ii focus, people dimension, flexibility, portability
\item organize, roles, goals, schedule, adapt, review, value-based decisions, natural prioritization
\item living it, integrity, commitment, effectiveness with people, efficiency with things
\item delegation, manager, effectiveness, stewardship delegation, result,  trust, human motivation 
\item emotional bank account, trust, safeness, relationship, deposit, reserve, compensate
\item deposit, understand the individual, attend to details, keep commitments, clarify expectations, personal integrity, apologize sincerely, internal security
\item withdrawal, break promises, implicit expectations, duplicity, repeated apologies, not to admit mistakes
\item the law of love, secure, validated, affirm, essential worth, identity, integrity
\item the laws of life, cooperation, contribution, self-discipline, integrity
\item P problem, PC opportunities, P/PC balance
\item win/win or no deal, agree to disagree, with standard, productive cooperation, 
\item win/loss, competition, authoritarian, loss/win, indulgence, permissiveness, giving in or giving up, dependent 
\item win/win, interpersonal leadership, character, relationships, agreements, systems, processes 
\item win/win, character, integrity, maturity, balance between courage and consideration, abundance mentality
\item win/win, relationship, trust, emotional bank, courage, proactive, transformational leadership, circle of influence
\item win/win, agreement, partners, partnership agreement, desired results, guidelines, resources, accountability, consequences
\item win/win management,  self-supervision, mutual understanding, financial consequences, psychological consequences, opportunities, responsibility
\item win/win, cooperative, reward system, environment of competition and contests, unnecessarily competitive situations
\item win/win, processes, principled negotiation, separate people from problem, focus on interests and not on positions, invent options for mutual gain, insist on objective criteria, other's perspective, key issues and concerns, fully acceptable solution,  the third option
\item interpersonal communication,  character, trust, be influenced,  not by skill alone
\item empathic listening, listen with intent to understand, diagnose before prescribe, trust, confidence, inside other's perspective, registering, reflecting, body language
\item listen with intent to reply, autobiographical response, collective monologues, own rightness, ignoring, pretending, selective listening, attentive listening
\item satisfied needs do not motivate, psychological survival, therapeutic
\item interdependent production, interpersonal PC
\item autobiographical response, evaluate, probe, advise, interpret, scripted
\item empathic listening skills, mimic content, rephrase the content, reflect feeling, transformational opportunity, patience, discerning
\item appreciate difference, perception, 
\item seek to be understood, knowing how to be understood, courage, ethos, pathos, logos, trust, feeling, logic, presentation, contextual, visual
\item patient, respectful, listen, influenceable, discerning, sensitive, aware
\item synergy, principle-centered leadership, whole, value difference, respect, build strength, compensate
\item levels of communication, low trust, defensiveness, protectiveness, legalistic language, lose/lose, win/lose
\item levels of communication, mid trust, respectful communication, politely, not empathically, avoid confrontations
\item levels of communication, high trust, synergy, opened up, creative
\item synergy, win/win, seak for the third alternative, dichotomous mentality
\item synergy, value the differences, recognize own perceptual limitations, humility, reverence, interactions
\item four dimensions of renewal, physical, spiritual, mental, social/emotional
\item sharpen the saws, exercise, balanced, regularly, consistently, Quadrant ii, proactive, personal PC
\item physical renewal, caring body, foods, rest, exercise, Quadrant ii, endurance, flexibility, strength
\item spiritual renewal, leadership in life, center, value system, commitment
\item mental renewal, education, learning, training, reading, writing, organizing and planning
\item social/emotional renewal, relationship, instrinsic security,  peace in mind, interpersonal leadership, empathic communications, creative cooperation
\item balanced renewal process, optimally synergetic, seven habits
\item renewal process, upward spiral, conscience, learn, commit, do
\item transition person, link, past, future, script, change
\end{itemize}
\subsection{Summary}
\emph{The 7 Habits of Highly Effective People: Restoring the Character Ethic}, first published in 1989, revised in 2004, is writen by Stephen R. Covey. 

\emph{\textbf{Summary}}
\begin{itemize}
\item This is book for \emph{\textbf{personal development}} and \emph{\textbf{relationship management}}. 

\item In this book, the author used a \emph{top-down}, \emph{\textbf{inside-out} approach} to help the readers to develop \emph{\textbf{effectiveness}} in their personal life and in their relationship with others. 

\item The \emph{\textbf{effectiveness}} defined in this book lies in the \emph{\textbf{balance} of \textbf{production} (\textbf{P}) and \textbf{production capacity} (\textbf{PC})}.

\item This apprach belongs to so called the \emph{\textbf{character ethic}} approaches, which emphasizes that ``there are \emph{\textbf{basic principles of effective living}}, and that people can only experience true success and enduring happiness as they learn and integrate these principles into their basic character."

\item The author's argument is that developing a \emph{\textbf{principle center} of ourselves} is essential to form an \emph{\textbf{effective}}, \emph{\textbf{mature}}, \emph{\textbf{independent}} individual, which becomes the foundation for \emph{\textbf{trustworthy}}, \emph{\textbf{interdependent  relationships}} with others.

\item The \emph{\textbf{seven principles}} and the corresponding \emph{\textbf{seven habits}} presented in this book are thus a part of a \emph{\textbf{sequential development process}}. From top to down, from inside to outside, this process allows the reader to introspect, evaluate and then change their own paradigms of life and finally to live in harmony with their principles, their values in an ever-changing, challenging, interdependent environment.

\item \emph{The \textbf{ultimate goal}} for the readers to achieve is to form \emph{an \textbf{upward spiral renewal process}} that helps the readers to continually develop themselves and to improve their relationships with other individuals. 
\end{itemize}

\subsection{Summary of Chapters}
\emph{\textbf{Summary of Chapters}}
\begin{enumerate}
\item  \textbf{Part 1: } \emph{\textbf{Paradigms and Principles}}. This part presents the main argument of the book, i.e. forming a principle-centered paradigm. 
\begin{enumerate}
\item \textbf{Chapter 1} \textbf{Inside-Out}. 

This chapter presents the main approach of this book -- the \emph{\textbf{character ethic}} approach and compare it with the popular \emph{personality ethic} approach. The author, who is restoring an old paradigm, argues that the \emph{\textbf{behaviors}} and \emph{\textbf{attitudes}}, the two main focuses on \emph{personality ethic} trait, are \emph{\textbf{secondary}} to \emph{\textbf{character}}. Without correct principles and fundamental character strength, skills on behaviors and attitudes will fade quickly and our true self will resurface. 

Then the author argues that through \emph{\textbf{paradigm shift}} from \emph{personality ethics} to \emph{\textbf{character ethics}}, we will change our perspective, understanding and interpreting to the world and circumstances. This paradigm shift is via forming a \emph{\textbf{principle-centered paradigm}}. \emph{Paragidm} is our internal model of the world and \emph{principles} are \emph{self-evident} guidelines to human conduct with \emph{enduring} \emph{values}. In this book, the author argues that one individual's internal world model should live in harmony with the correct principles so that he/she can have a correct map to maximize the personal and interpersonal effectiveness.

\item \textbf{Chapter 2} \textbf{The Seven Habits: an Overview}.

This chapter introduces the definitions of \emph{\textbf{habits}} and \emph{\textbf{effectiveness}}, two keywords in the book's title. This chapter also presents the \emph{\textbf{maturity continuum}}, which introduces the concept of dependent, \emph{\textbf{independent}} and \emph{\textbf{interdependent}}. The latter two are the focus on part 2 and part 3, respectively.  The main point of this chapter is to argue that \emph{\textbf{effectiveness} lies in \textbf{the balance of production and production capacity}}. Overall, this chapter is an introduction chapter  and it provides a high-level overviews of the topics discussed in the later chapters of the book.
\end{enumerate}


\item \textbf{Part 2: } \emph{\textbf{Private Victory}}. This part concerns about the personal \emph{\textbf{self-development}}. The ultimate goal is to form an \emph{\textbf{independent}} and \emph{\textbf{matured}} character of our own.  In particular, the author stated that a \emph{\textbf{habit}} is defined as an \emph{intersection} of \emph{knowledge}, \emph{skill} and \emph{desire}. Then in the following three chapters, the author address the issues of \emph{\textbf{desire}} (via our \emph{\textbf{personal vision}}), \emph{\textbf{knowledge}} (via our \emph{\textbf{personal leadership}}) and \emph{\textbf{skill}} (via our \emph{\textbf{personal management}}). Note that \emph{skills} come last, which reflect the author's perspective based on the \emph{character ethic approach}.

This part correlates to the \emph{self-awareness} and \emph{self-management} part in \emph{emotional intelligence 2.0}. But this book emphasize the development of \emph{characters} not only \emph{behaviors}.

\begin{enumerate}
\item \textbf{Chapter 3} \textbf{Habit 1: Be Proactive -- Principles of Personal Vision}. 

This chapter addresses the issue of ``\emph{\textbf{Desire}}" -- ``Do we \emph{\textbf{want to}} lead an effective life?". The author raised the readers' \emph{\textbf{desire}} to commit to a paradigm shift in our character. It is argued that ``We are the \emph{Creator}." In particular, 
\begin{itemize}
\item we are \emph{\textbf{responsible}} for our own life.
\item we have \emph{\textbf{freedom}} to choose our responses in our life.
\end{itemize} And the Habit 1, being \emph{\textbf{proactive}} means:
\begin{itemize}
\item \emph{\textbf{taking initiatives}}, to \emph{\textbf{act}} instead of being acted upon by others or environment;
\item recognizing and taking \emph{\textbf{responsibility}};
\item focus on \emph{\textbf{Circle of Influence}}; only act when we have \emph{\textbf{direct control}} or \emph{\textbf{indirect control}};
\item \emph{\textbf{live with}} things outside Circle of Influence, i.e. things that we have no control of;
\item making and keeping \emph{\textbf{commitments}}.
\end{itemize} The author argued that human response are not purely \emph{reactive} to the stimulus of our environment. We are also not a true reflection of our social mirror. To the opposite, we have \emph{freedom} to make choice through four unique \emph{human endowments}:  \textbf{\emph{Self-awareness}},  \emph{\textbf{Imagination}}; \emph{\textbf{Conscience}}; \emph{\textbf{Independent Will}}.  These four \emph{characteristics} are the basis of the proposed \emph{\textbf{proactivity model}}. Throughout the book, the proactive model and the four human endowments are repeatedly mentioned. 

The concept of Circle of Influence vs. Circle of Concern is also impressive. By focusing on things we have control of we are able to motivate ourselves to act on them. The Circle of Influence is a central piece in the habit of \emph{Proactivity}, as it directs our mind to consider ``Which part i can work on?". Note that not only our own personal life but also part of our interpersonal relationships lie within Circle of Influence, which is up to us to influence.

The habit of \emph{\textbf{proactivity}} helps us to make commitments to start our transition to new paradigms. It gives us the initial force to push us forward.

\item \textbf{Chapter 4} \textbf{Habit 2: Begin with the End in Mind -- Principles of Personal Leadership}. 

This chapter addresses the issue of ``\emph{\textbf{Knowledge}}" -- ``\emph{\textbf{What}} to do to lead an effective life and \emph{\textbf{Why}}?" The author used the term ``\emph{\textbf{Begin with the End in Mind}}" which means that ``\emph{envisioning a clear \textbf{destination} at the beginning of journey}."  By knowing where are you going, you will have a better understanding of where are you now and whether or not you are heading in the right direction. The author encouraged us to use \emph{\textbf{imagination}} and \emph{\textbf{conscience}} to visualize and affirm an ideal image of ourselves \emph{\textbf{in minds}}.  

In this chapter, \emph{\textbf{the main proposal}} is to develop a \emph{\textbf{personal mission statement}}, a \emph{personal constitution}. Such personal mission statement includes \emph{what you want to be}, \emph{to do} and \emph{the values or principles} upon which your identity and your actions are based. And the personal mission statement is separated according to \emph{\textbf{roles}} and \emph{\textbf{goals}}, which provides \emph{structual directions} to our personal mission. According to the author, ``\emph{with unchanging principles, we created a fundamental paradigm of effective living}". Our \emph{personal mission statement} is a manifesto of our ideal \emph{destinations} \emph{in mind}.  It is our mental creation.

Behind the personal mission statement is our \emph{\textbf{centers}}. In particular, our centers are the source of \emph{four life-support factors}: \emph{\textbf{security}, \textbf{guidance}, \textbf{wisdom} and \textbf{power}}. In order to write a personal mission statement, we need to identify our \emph{\textbf{principle center}}, which is built upon the correct principles. By \emph{centering our life} on correct principles, we put various other centers and roles in our life (\emph{self, family, spouse, work, money, friend, enemy etc.}) into perspective and develop them in a balanced way. Note that these centers are at very center of our \emph{Circle of Influence}. 

Habit 2 is based on the \emph{principle} of \emph{\textbf{Personal Leadership}}. \emph{Personal leadership} is ``to determine the right things to do". ``We are more in need of a \emph{vision or destination} and a \emph{compass} and less in need of a \emph{road map}." Personal leadership defines the criterion on \emph{prioritization}, which is the focus on \emph{Personal Management}. According to the author, ``\emph{Personal Leadership comes before Personal Management.}" Effectiveness cannot be achieved if we are heading in the wrong direction.

Finally, the author suggested us to ``\emph{take deep introspection, careful analysis, thoughtful expression, and often many rewrites}" to produce the final form of our personal mission statement, and then \emph{review} it regularly. The author also suggested us to \emph{\textbf{continually}} develop our \emph{Personal Leadership} by \emph{\textbf{expanding our perspective}} with new experience, and by \emph{\textbf{visualizing} and \textbf{affirming}} it with respect to our principles.


\item \textbf{Chapter 5} \textbf{Habit 3: Put First Things First -- Principles of Personal Management}. 

This chapter addresses the issue of ``\emph{\textbf{Skill}}" -- ``\emph{\textbf{How}} to lead an effective life?"  The answer to the question above can be summarized as
\begin{itemize}
\item \emph{\textbf{Separating importance from urgence}};
\item \emph{\textbf{Always prioritizing importance over urgence}};
\item \emph{\textbf{Focus on important but Not urgent matters (Quadrant II)}};
\item For Quadrant III or Quadrant IV, either say ``\emph{\textbf{No}}" or \emph{\textbf{Delegate}}.
\end{itemize} ``What are important matters?" \emph{\textbf{Importance}} is measured by how much the \emph{\textbf{results}} will contribute to our goals, missions and values, which are identified in Habit 2. \emph{Leadership} decides on the ``first things" and \emph{management} puts them first in routine activities.

The key in this chapter is ``\emph{\textbf{Quadrant II focus}}". \emph{\textbf{Quadrant I}} represents \emph{\textbf{Crises}}, and \emph{crises act on you} not vice versa. Thus Quadrant I is in effect \emph{out of} our Circle of Influence. However, \emph{\textbf{Quadrant II is within our Circle of Influence}}. We have \emph{freedom} to determine when to do it and how to do it. Moreover, the more we work on Quadrant II the less we will face them in Quadrant I. 

The Habit 3 can be described in one word ``\emph{\textbf{prioritization}}".  ``How to prioritize Quadrant II?" The author's answer is ``By \emph{\textbf{rejecting}} \emph{\textbf{unimportant}} matters or \emph{\textbf{delegate}} them to other skilled people." That is, by taking time from Quadrant III and IV.  Both \emph{\textbf{rejection}} and \emph{\textbf{delegation}} are result of fully understanding of our principles/values, and thus understanding what is really important to us. ``The essence of effective time and life management is to \emph{\textbf{organize} and \textbf{execute} around \textbf{balanced priorities}}." In order to master \emph{Habit 3}, we need to internalize \emph{Habit 2}.

The final product in this chapter is a \emph{\textbf{Quadrant II Self-Manager}}, who complete the work via
\begin{itemize}
\item Identifying \emph{\textbf{Roles}};
\item Selecting \emph{\textbf{Goals}};
\item \emph{\textbf{Scheduling}};
\item Daily \emph{\textbf{Adapting}}.
\end{itemize} The objective of Quadrant II management is to keep our life effective. 
\begin{itemize}
\item it should maintain \emph{integrity} and \emph{coherence} to our vision and mission. 
\item It should keep \emph{balance} in our life between multiple roles and goals. 
\item The key is ``\emph{not to prioritize what is on your schedule, but to \textbf{schedule your priorities}}."
\item It maintains a ``People" dimension, which subordinate of schedules to people.
\end{itemize}

Mastering \emph{the Skill (Habit 3)} is the result of understanding \emph{the Desire (Habit 1)} and the \emph{Knowledge (Habit 2)}. Only through deep understanding of our instrinct values and goals do we define the importance and thus the Quadrant II. By choosing to prioritize Quadrant II, we utilize our \emph{Habit 1 (Proactivity)}. Finally through \emph{Proactivity}, we reach \emph{\textbf{effectiveness}} in \emph{\textbf{time and life management}}. 
\end{enumerate}

\item  \textbf{Part 3: } \emph{\textbf{Public Victory}}. This part concerns about the interpersonal \emph{\textbf{relationship management}}. From the author's perspective, \emph{effective \textbf{interdependence} can only be built on a foundation of true \textbf{independence}}. \emph{Private Victory procedes Public Victory}. 

\emph{The ultimate goal} is to \emph{raise the level of \textbf{trusts}} among individuals. In the process of \emph{\textbf{earning trusts}}, we adopt the \emph{motives} of \emph{\textbf{Win/Win interactions (Habit 4)}} and the \emph{skills} of \emph{\textbf{Empathic Communications (Habit 5)}}. Finally, it results in a situation where all parties in a group obtain \emph{\textbf{Synergy (Habit 6)}} in \emph{creative cooperations}.

This part corresponds to the \emph{social awareness} and \emph{relationship management} in \emph{the Emotional Intelligence 2.0}. It also covers the \emph{principled negotiations} in \emph{Getting to Yes: Negotiating Agreement Without Giving In}.

\begin{enumerate}
\item \textbf{Chapter 5} \textbf{Paradigms of Interdependence}. 

This chapter discuss \emph{\textbf{the paradigm of interdependence}}. Unlike the paradigm of independence, which involves development of principled center and then working from inside out to reach effective in personal life, the interdependence is built upon \emph{\textbf{trusts}} among individuals. In this chapter, the author use the metaphor of ``\emph{\textbf{Emotional Bank Account}}" to describe \emph{\textbf{the level of trusts}} \emph{\textbf{accumulated}} between people. 

The central point in this chapter is that \emph{our \textbf{character}, \textbf{actions} and \textbf{communications} would affect the amount of trusts that the other people have towards us}. Such trusts and distrusts can be \emph{\textbf{accumulated}}, which is described as ``\emph{deposit}" and ``\emph{withdrawal}". The paradigms of interdependence is to maintain and \emph{earn people's trusts} towards us. 

``How to earn trust?" The author's suggestions can be summarized as the following 
\begin{itemize}
\item \emph{\textbf{Understanding the Individual}};
\item \emph{\textbf{Attending to the Little Things}};
\item \emph{\textbf{Keeping Commitments}};
\item \emph{\textbf{Clarifying the Expectations}};
\item \emph{\textbf{Showing Personal Integrity}};
\item \emph{\textbf{Apologizing Sincerely When You Make a Withdrawal}}.
\end{itemize} 

Finally, the author suggested that the differences between people's perspectives would give us an opportunities to build a deep relationship. ``\emph{Every P problem is a PC opportunity.}" Thus we should always appreciate our differences.

\item \textbf{Chapter 6} \textbf{Habit 4: Think Win/Win -- Principles of Interpersonal Leadership}. 

This chapter deals with people's \emph{different \textbf{mindsets}} under \emph{different \textbf{environment}}:
\begin{itemize}
\item \emph{Win/Loss};
\item \emph{Loss/Win};
\item \emph{Lose/Lose}, 
\item \emph{Win}, 
\item and \emph{Win/Win (or No Deal)}.
\end{itemize} The first three mindsets mainly deal with \emph{\textbf{competitive environments}}, with \emph{limited \textbf{resources}} and \emph{\textbf{opposite} goals}. The author, however, stressed that maintaining a competitive mindset would \emph{narrow our perspective} and \emph{cost people's trust towards us}, which consequently \emph{hurt our relationship} with others. Moreover, people with competitive mentality would treat others with \emph{\textbf{authoritarian approach}}, focusing on the \emph{\textbf{power and position}} rather than the \emph{\textbf{principle}}.

Habit 4 is a belief that \emph{\textbf{the Third Alternative exists}}, taking us out of the dichotomous mentality of Win vs Loss. Win/Win is a \emph{\textbf{cooperative mentality}}. It comes from ``our \emph{\textbf{character}}, then moves towards our \emph{\textbf{relationship}}, out of which flow \emph{\textbf{agreements}},. It is nutured in an environment where \emph{\textbf{structures} and \textbf{systems}} are based on Win/Win. And it involves \textbf{\emph{process}}, which encourage Win/Win." Win/Win provides a direction of \emph{interpersonal relationship (interpersonal leadership)}. 


We can work on building a \emph{Win/Win mindset} using our character of \emph{\textbf{integrity}}, \emph{\textbf{maturity}} in combination with an \emph{\textbf{abundance mentality}}. We should maintain \emph{high level of trusts} among people. Then we can work togther with people to form a \emph{\textbf{partnership agreements}}. Besides this, the author suggested that we also need to design better \emph{\textbf{reward systems}}. Such system designs should reflect the values and principles in our mission statement and should be aligned with our personal goals. Finally, we need to adopt various \emph{\textbf{principled negotiation} strategies} in the \emph{\textbf{process}}. This includes
\begin{itemize}
\item \emph{understanding \textbf{the other's perspective}}, 
\item \emph{identifying \textbf{key interests}},
\item \emph{seeking for \textbf{mutually beneficial solutions}}
\item and \emph{inventing \textbf{new options}}.
\end{itemize}

\emph{The \textbf{process} of obtaining Win/Win} requires not only \emph{skills} but also our \emph{characters} and \emph{the level of accumulated trusts} during our past interactions with people. That is why it is essential to start with our own personal development and earning trusts with people. Note that the process of obtaining Win/Win would also help us accumulate trusts and improve our relationship with others. ``\emph{The end and the means are the same}". 

Note that \emph{\textbf{Win/Loss}} and its \emph{\textbf{competitive mindset}} are also critical for sports, games, society and economy. It is through competitions that limited resources are distributed. Competitions and its associated Win/Loss push the new development in personal, social and economical status. The important point in this chapter is to understand \emph{which mindset to apply at which situation}. When both parties have interests to collaborate and to reach agreement, it is when \emph{Win/Win mentality} start to shrine. On the other hand, it is always up to us to decide if we want to participate in competition. \emph{\textbf{No Deal}} is also an option when there is no viable Win/Win solution between two parties.


\item \textbf{Chapter 7} \textbf{Habit 5: Seek First to Understand, Then to Be Understood -- Principles of Empathic Communication}. 

This chapter addresses an important topic in relationship management, i.e. \emph{\textbf{communications}}. In specific, it presents a powerful principle: ``\emph{\textbf{Seek first to understand, then to be understood}}." Before we rush in to provide our own diagnose, we need to obtain data from others. \emph{\textbf{Empathic listening}} is a critical communication skill in interdependent relationship management.

``Seek first to understand" is a critical \emph{paradigm shift}. Most people \emph{\textbf{listen with intent to reply}}. But we should really \emph{\textbf{listen with intent to understand}}. If you fail to understand others, you fail to \emph{\textbf{earn their trusts}}, thus fail to \emph{\textbf{influence}} them. It is hard to \emph{diagnose before prescribe}, since understanding others first opens ourselves \emph{\textbf{to be vulnerable}} to others' paradigms \emph{first}. It thus requires \emph{strong \textbf{instrinct security}} and \emph{\textbf{integrity}} in one's character to implement.  \emph{Being \textbf{influenceable} is the key to \textbf{influencing} others}. It can greatly increase \emph{the level of trusts} in relationship.

In this chapter, we discussed the four levels of listenings:
\begin{itemize}
\item \emph{\textbf{ignoring}};
\item \emph{\textbf{pretending}};
\item \emph{\textbf{selective listening}};
\item \emph{\textbf{empathic listening}}.
\end{itemize} Here we see that only empathic listening would give us \emph{accurate data} about the other people, thus allows us to understand them. Through empathic listening, we understand others' perspectives, which is the first step towards a Win/Win solution. 

We should reserve our \emph{judgement} to others before we understand them. Also we should avoid \emph{\textbf{autobiographical responses}} first, such as \emph{\textbf{evaluate}};  \emph{\textbf{probe}}; \emph{\textbf{advise}};  \emph{\textbf{interpret}}. This is not about us. In empathic listening, we \emph{\textbf{rephrase} the content and \textbf{reflect} the feeling}. This is not only a communication skill but should also come from our own \emph{desire} to understand, and our \emph{own character strength} as well as \emph{the accumulated trusts (the Emotional Bank Account)} we obtained from other people. 

Finally, we need \emph{\textbf{to be understood}}. \emph{To be understood} is also important in Habit 5. It requires us to improve \emph{our communication skills}, to present our idea \emph{\textbf{clearly}}, \emph{\textbf{empathically}}, \emph{\textbf{logically}} and to express it with \emph{\textbf{courage}}. In this chapter, the author presents the sequence ``\emph{\textbf{ethos}, \textbf{pathos}, \textbf{logos}}", i.e. 
\begin{itemize}
\item our character and the trusts we inspire (\emph{ethos});
\item our empathic feelings (\emph{pathos});
\item the logics of our presentation (\emph{logos}).
\end{itemize} In the context of deep understanding other's paradigms and concerns, we can significantly improve the \emph{\textbf{credibility}} of our ideas.

Empathic Listening is within our \emph{\textbf{Circle of Influence}}. It is our choice to \emph{put aside our autobiography} and to \emph{seak to understand first} before we present our own idea. As \emph{proactive} people, we will create opportunities. Empathic Listening requires \emph{\textbf{patience}} and \emph{\textbf{respects}}.


\item \textbf{Chapter 8} \textbf{Habit 6: Synergize -- Principles of Creative Cooperation}. 

This chapter discuss what miracle would happen if you mastered \emph{the principle-centered leadership} and is able to implement Habit 1-5 to your own personal life and your relationships. ``\emph{The highest form of synergy focus the four unique human endowments (\textbf{self-awareness}, \textbf{imagination}, \textbf{conscience}, \textbf{independent will)}, the motives of \textbf{Win/Win}, and the skills of \textbf{empathic communications} on the toughest challenges we face in life.}" In summary, this chapter depicts \emph{\textbf{the ideal destination} of \textbf{relationship management}}, i.e. reaching synergy within the group. It is the result of both \emph{Private Victory} and \emph{Public Victory}.

\emph{\textbf{Synergy}} means that \emph{the \textbf{whole} is greater than the sum of individual}. \emph{Synergy} in \emph{communications} will let you \emph{\textbf{open your minds}} to new opportunities and \emph{\textbf{inspire creativities}}. 

Synergy does not come naturally in human interactions. It is result of \emph{high level of \textbf{trusts}} and \emph{high} \emph{\textbf{cooperations}}. Human interactions 
\begin{itemize}
\item begins with \emph{\textbf{defensiveness}}, an indicator of low trust and Win/Loss mentality, 
\item to \emph{\textbf{respectful} communications}, a mature, polite interactions but not empathically,
\item to \emph{\textbf{synergistic} communications}, a level of high trust, high cooperation, high creativity.
\end{itemize}

\emph{\textbf{Valuing the differences} is the \textbf{essence} of synergy} -- the mental, the emotional, the psychological difference between people. It requires us to master Habit 5, to understand first. We need to recognize our own perceptual limitations and appreciate resources available through rich interactions.   
\end{enumerate}

\item \textbf{Part 4: } \emph{\textbf{Renewal}}. This part deals with the \emph{\textbf{continual balanced development}} of \emph{ourselves} and our \emph{relationship} in an ever-changing interdependent environment. The author in this part inspired the readers to view themselves in the middle of process. The \emph{\textbf{renewal process}} is an \emph{\textbf{upward spiral}} that involves \emph{\textbf{learning}, \textbf{commitment} and \textbf{doing}}. All \emph{habits (1-6), principles and skills} we have learned so far need to be maintained and practiced \emph{\textbf{regularly}} and consistently. The ultimate goal of renewal is to always prepare ourselves in upcoming challenges in life.

\begin{enumerate}
\item \textbf{Chapter 9} \textbf{Habit 7: Sharpen the Saw -- Principles of Balanced Self-Renewal}

There are two keywords in the principle: \emph{\textbf{balanced}} and \emph{\textbf{self-renewal}}. On the self-renewal part, the author introduced \emph{four dimensions of renewal}:
\begin{itemize}
\item \emph{\textbf{Physical Dimension}}: our body, food, exercise, rest;
\item \emph{\textbf{Spiritual Dimension}}: our principles, our values, our personal missions, and our commitments;
\item \emph{\textbf{Mental Dimension}}: our knowledge, our educations, skills in learning, reading, writing, planning etc;
\item \emph{\textbf{Emotional/Social Dimension}}: our relationship, synergy, empathic communications, our intrinsic security.
\end{itemize} 
Note that all four dimensions of self-renewal requires us to be \emph{\textbf{proactive}} since they are all \emph{\textbf{Quadrant II matters}}. In fact, \emph{\textbf{the renewal process}, as the Habit 7 suggests, is \textbf{Quadrant II activity}} -- \emph{Shapen the saws} is not urgent but important. Use our Habit 1 and Habit 3 to help in the planning and executing Habit 7. 

Finally, the author suggests us to ``\emph{show diligence in the process of renewal by \textbf{educating and obeying our conscience}}". ``Conscience is the endowment that senses our congruence or disparity with correct principles and lifts us towards them." As we renew on all four dimensions in a balanced way, we will reinforce all our Habits learned by far. 
\begin{itemize}
\item Physical dimension renewal reinforces our personal vision (Habit 1) with proactivity
\item Spiritual dimension renewal reinforces our personal leadership (Habit 2) with improved imagination, conscience to deeply understand our principles and values.
\item Mental dimension renewal reinforces our personal management (Habit 3) via Quadrant II focus and prioritization.
\item Emotional/Social dimension renewal reinforces our interpersonal leadership (Habit 4) and empathic communications (Habit 5) and then, based on the foundation of Habit 1-3 on Private Victory, it provides the source of intrinsic security that we need to Shapen the Saw (Habit 7).
\end{itemize} In the process of \emph{balanced renewal}, we would reach to Synergy (Habit 6).

In this final principle, \emph{\textbf{the Principle of Balanced Self-renewal}}, we revisit all the \emph{Habits} and \emph{Principles} we have learnd so far and we realized that they are interconnected and are reinforcing each other. \emph{\textbf{The Synergy of Seven Habits}} is also the \emph{\textbf{ultimate ideal destination}} for the readers after finishing reading the book. The process of balanced renewal is thus critical for reader to master all seven principles and seven habits together. 

At the end, it is critical to realize that \emph{\textbf{effectiveness}} is not \emph{\textbf{efficiency}}. The latter is about how we handle our work, but the former is about \emph{how we lead our life}. Effectiveness is a higher level of efficency since it is beyond one single thing but an entire process. Effectiveness is not about one single objectives. It is about \emph{reaching \textbf{the balance of multiple objectives}} in our life.

\item  \textbf{Chapter 10} \textbf{Inside-Out Again}

``\emph{\textbf{Becoming a Transition Person}}" is the last message of this book. It is an inspiring message from author to break the scripts from past generations to the future. \emph{Rescripting} yourself and \emph{scripting the others}.
\end{enumerate}

\end{enumerate} \vspace{50pt}
Side Note: It is called \emph{\textbf{Seven Habits}}, but it is actually \emph{Five Habits (1-5)} + a \emph{Looping Mechanism (Habit 7)} + an \emph{Ideal Outcome (Habit 6)}.

%\subsection{Principle-Centered Paradigm}
%\begin{itemize}
%\item 
%\end{itemize}
%\subsection{From Independence to Interdependence}
%\subsection{Seven Habits and Seven Principles}
%\subsection{Balanced Self-Renewal}
\newpage
\subsection{Comparison with Other Books}
\subsubsection{Getting Things Done: The Art of Stress-Free Productivity}
\emph{\textbf{Getting Things Done: The Art of Stress-Free Productivity}} by \emph{David Allen} \textbf{v.s.} \emph{\textbf{The 7 Habits of Highly Effective People}} by \emph{Stephen R. Covey}
\begin{itemize}
\item Both books are influential in the \emph{\textbf{self-help} and \textbf{productivity genres}}. Here's a comparison of their main points:

\begin{itemize}
\item \textbf{Approach to Productivity}:
\begin{itemize}
\item David Allen's \emph{\textbf{Getting Things Done (GTD)}}: GTD is \emph{\textbf{bottom-up approach}}. Allen focuses on \emph{practical strategies for \textbf{organizing tasks and managing workflow}}. The GTD system involves \emph{capturing, clarifying, organizing, reflecting, and engaging with tasks}.
\item Stephen R. Covey's \emph{\textbf{The 7 Habits}}: The 7 Habits is \emph{\textbf{top-down approach}}. Covey provides a \emph{\textbf{holistic approach}} to productivity by \emph{emphasizing \textbf{character development} and \textbf{habit formation}}. The seven habits cover \emph{a range of personal and professional effectiveness principles}.
\end{itemize}

\item \textbf{Task Management}:
\begin{itemize}
\item \emph{\textbf{GTD}}: Allen introduces a \emph{\textbf{systematic} approach to managing tasks}, including the use of lists, projects, and next actions. \emph{The emphasis is on \textbf{clearing mental clutter} and \textbf{organizing tasks effectively}}. 

\item \emph{\textbf{The 7 Habits}}: Covey's habits also address task management, particularly in Habit 3, which is about \emph{\textbf{prioritizing} and \textbf{executing} tasks based on \textbf{importance}}. 
\end{itemize}

\item \textbf{Time Management}:
\begin{itemize}
\item \emph{\textbf{GTD}}: Allen's system indirectly addresses time management by \emph{providing a \textbf{structure} for organizing tasks and priorities}.

\item \emph{\textbf{The 7 Habits}}: Covey's approach includes \emph{time management principles}, particularly in \emph{\textbf{Habit 3 (Put First Things First)}}, which focuses on \emph{time management and prioritization}.
\end{itemize}

\item \textbf{Proactivity and Responsiveness}:
\begin{itemize}
\item \emph{\textbf{GTD}}: Allen's approach is \emph{practical} and emphasizes \emph{\textbf{being responsive to tasks as they arise}}, but it also involves proactive planning and organization.

\item \emph{\textbf{The 7 Habits}}: Covey's \emph{habit of \textbf{being proactive}} encourages individuals to \emph{take initiative} and \emph{be responsible for their choices}.
\end{itemize}

\item \textbf{Holistic vs. Task-Specific}:
\begin{itemize}
\item \emph{\textbf{GTD}}: Allen's focus is primarily on \emph{the practical aspects of \textbf{task and project management}}, aiming to \emph{reduce stress and increase productivity}.

\item \emph{\textbf{The 7 Habits}}: Covey's book takes a \emph{\textbf{more holistic approach}}, addressing \emph{\textbf{character development}, \textbf{interpersonal relationships}, and \textbf{personal mission statements} alongside \textbf{productivity}}.
\end{itemize}

\item \textbf{Mindset and Paradigms}:
\begin{itemize}
\item \emph{\textbf{GTD:}} Allen's book is more about changing the way individuals manage tasks and projects, addressing the mindset and habits associated with productivity.

\item \emph{\textbf{The 7 Habits:}} Covey's book emphasizes paradigm shifts, encouraging readers to change their perspectives on life and work to achieve lasting effectiveness.
\end{itemize}
\end{itemize}


\item In summary, while both books offer valuable insights into productivity, Allen's \textit{\textbf{Getting Things Done}} is more \emph{\textbf{task-specific}}, \emph{providing \textbf{a detailed system} for managing workflow}, while Covey's \emph{\textbf{The 7 Habits}} takes a \emph{\textbf{holistic approach}}, incorporating \emph{\textbf{principles} for personal development and effectiveness}.
\end{itemize}

\subsubsection{How to Win Friends and Influence People}
\emph{\textbf{How to Win Friends and Influence People}} by \emph{Dale Carnegie} \textbf{v.s.} \emph{\textbf{The 7 habits of highly effective people}} \emph{by Stephen R. Covey}
\begin{itemize}
\item Both are influential \emph{\textbf{self-help books}}, they approach \emph{personal and professional development} from different perspectives. Here's a brief comparison of their main points:

\begin{itemize}
\item \textbf{Focus on Relationships}:
\begin{itemize}
\item \emph{Carnegie}'s book emphasizes building \emph{positive relationships} through \emph{\textbf{effective communication}, \textbf{active listening}, and \textbf{making others feel valued}}.
\item \emph{Covey}'s book also values relationships but places a broader focus on \emph{developing interdependent relationships} by \emph{\textbf{fostering trust and collaboration}}.
\end{itemize}

\item \textbf{Communication Styles}:
\begin{itemize}
\item \emph{Carnegie}'s book provides practical tips on \emph{communication}, including the importance of \emph{\textbf{being genuinely interested in others}, \textbf{avoiding criticism}, and \textbf{making people feel important}}.
\item \emph{Covey}'s book delves into \emph{effective communication} as one of the seven habits, emphasizing the importance of \emph{\textbf{understanding before seeking to be understood}}.
\end{itemize}

\item \textbf{Principles vs. Habits}:
\begin{itemize}
\item \emph{Carnegie}'s book revolves around \emph{\textbf{principles} of human interaction and behavior}, offering practical guidelines for applying these principles in various situations.
\item \emph{Covey}'s book \emph{outlines seven \textbf{habits}} that, when cultivated, lead to \emph{personal and professional effectiveness}. These habits are presented as a \emph{holistic approach} to living a successful and fulfilling life.
\end{itemize}

\item \textbf{Proactive vs. Reactive Approach}:
\begin{itemize}
\item \emph{Covey}'s book introduces the concept of being \emph{\textbf{proactive}}, taking responsibility for one's actions, and focusing on what can be controlled.
\item \emph{Carnegie}'s book encourages a positive and proactive approach to building relationships but \emph{doesn't explicitly use the term ``proactive."}
\end{itemize}

\item \textbf{Time Management}:
\begin{itemize}
\item \emph{Covey}'s book includes a s\emph{trong emphasis on \textbf{time management}} within the framework of his seven habits.
\item \emph{Carnegie}'s book touches on time management \emph{indirectly} through effective communication and building positive relationships.
\end{itemize}

\item \textbf{Personal Development}: Both books promote personal development, but 
\begin{itemize}
\item \emph{Carnegie}'s book focuses more on \emph{\textbf{social skills} and \textbf{interpersonal relationships}}.
\item \emph{Covey}'s book takes a \emph{\textbf{broader approach}}, covering \emph{personal and professional development}, including \emph{\textbf{character-building} and \textbf{aligning actions with core values}}.
\end{itemize}
\end{itemize}

\item In summary, while both books share the goal of personal and professional growth, \emph{Carnegie}'s work is more focused on \emph{interpersonal \textbf{skills}} and \emph{\textbf{communication}}, while Covey's book provides a \emph{\textbf{holistic framework}} for \emph{effective living} through \emph{\textbf{habits} and \textbf{principles}}.
\end{itemize}

\subsubsection{Atomic Habits: An Easy $\&$ Proven Way to Build Good Habits $\&$ Break Bad Ones}
\emph{\textbf{The 7 habits of highly effective people}} \emph{by Stephen R. Covey} \textbf{v.s.} \emph{\textbf{Atomic Habits: An Easy $\&$ Proven Way to Build Good Habits $\&$ Break Bad Ones}} by \emph{James Clear} 

\begin{itemize}
\item Both are popular \emph{\textbf{self-help} books that focus on \textbf{habits} and \textbf{personal development}}. Here's a comparison of their main points:

\begin{itemize}
\item \textbf{Framework vs. Atomic Changes:}
\begin{itemize}
\item Covey's \emph{\textbf{The 7 Habits}}: Covey presents \emph{a framework of seven habits} that individuals can adopt to become more effective in various aspects of life. \emph{\textbf{The habits are seen as foundational principles}}.
\item Clear's \emph{\textbf{Atomic Habits}}: Clear focuses on the idea of making \emph{\textbf{small, incremental changes}} or ``\emph{\textbf{atomic habits}}" to achieve significant and lasting improvements over time. The emphasis is on \emph{\textbf{the compounding effect of tiny changes}}.
\end{itemize}

\item \textbf{Number of Habits:}
\begin{itemize}
\item Covey's \emph{\textbf{The 7 Habits}}: Covey outlines \emph{\textbf{seven habits}} that cover a broad spectrum of \emph{personal and professional development}, including \emph{proactivity, prioritization, and synergy}.
\item Clear's \emph{\textbf{Atomic Habits}}: Clear focuses on \emph{\textbf{the idea of habits in general}}, providing \emph{a comprehensive guide} to \emph{understanding and changing habits}. \emph{It doesn't specify a fixed number of habits}.
\end{itemize}

\item \textbf{Character Development vs. Behavior Change:}
\begin{itemize}
\item Covey's \emph{\textbf{The 7 Habits}}: Covey's habits are deeply rooted in \emph{\textbf{character development} and \textbf{principles}}. They involve \emph{\textbf{a shift in mindset and values}}.
\item Clear's \emph{\textbf{Atomic Habits}}: Clear's approach is \emph{more \textbf{behavior-focused}}, emphasizing the role of small habits in shaping one's identity and achieving goals.
\end{itemize}

\item \textbf{Proactive vs. Reactive}:
\begin{itemize}
\item Covey's \emph{\textbf{The 7 Habits}}: Covey introduces the concept of \emph{being \textbf{proactive}, taking responsibility} for one's choices and actions.
\item Clear's \emph{\textbf{Atomic Habits}}: Clear's book also addresses \emph{proactivity} but through \emph{the lens of \textbf{initiating and sustaining} \textbf{positive habits}}.
\end{itemize}


\item \textbf{Time Management}:
\begin{itemize}
\item Covey's \emph{\textbf{The 7 Habits}}: Covey's habits include \emph{principles related to \textbf{time management}}, such as \emph{prioritization and focusing on what matters most}.
\item Clear's \emph{\textbf{Atomic Habits}}: Clear's book \emph{doesn't explicitly focus on time management} but \emph{emphasizes the power of habits in shaping outcomes over time}.
\end{itemize}

\item \textbf{Focus on Identity}:
\begin{itemize}
\item Covey's \emph{\textbf{The 7 Habits}}: Covey's habits involve \emph{a shift in identity and character}, encouraging individuals to align their actions with their values.
\item Clear's \emph{\textbf{Atomic Habits}}: Clear discusses the concept of identity and how habits are linked to one's self-perception.
\end{itemize}

\item \textbf{Practical Strategies}:
\begin{itemize}
\item Covey's \emph{\textbf{The 7 Habits}}: Covey provides practical strategies for personal and professional development, including the use of the time management matrix and the concept of "sharpening the saw."
\item Clear's \emph{\textbf{Atomic Habits}}: Clear offers practical strategies for building and breaking habits, including the implementation of cues, cravings, responses, and rewards.
\end{itemize}
\end{itemize}

\item In summary, while both books address \textit{the topic of habits and personal development}, Covey's \emph{The 7 Habits} focuses on a \emph{\textbf{comprehensive framework} for \textbf{character development}}, while Clear's \emph{Atomic Habits} provides \emph{a \textbf{practical guide} to \textbf{making small changes} for significant results through \textbf{the power of habits}}.
\end{itemize}

\newpage
\section{Paradigms and Principles}
\subsection{Inside-Out}
If we wanted to change the situation, we first had to \emph{\textbf{change ourselves}}. And to change ourselves effectively, we first had to change our \emph{\textbf{perceptions}}.
\subsubsection{Peronality and Character Ethics}
\begin{itemize}
\item In stark contrast, almost all the literature in the first 150 years or so focused on what could be called \emph{\textbf{the character ethic}} as the foundation of success -- things like \emph{integrity}, \emph{humility}, \emph{fidelity}, \emph{temperance}, \emph{courage}, \emph{justice}, \emph{patience}, \emph{industry}, \emph{simplicity}, \emph{modesty}, and \emph{the Golden Rule}. Benjamin Franklin's autobiography is representative of that literature. It is, basically, the story of one man's effort to integrate certain principles and habits deep within his nature.

\item \emph{\textbf{The character ethic} taught that there are basic \textbf{principles} of effective living, and that people can only experience true success and enduring happiness as they learn and integrate these principles into their basic character}.

\item The elements of \emph{\textbf{Personality Ethic}} are sometimes essential for success.
\begin{itemize}
\item \emph{personality growth}
\item \emph{communication skills}
\item eduation in \emph{the field of influcence strategies}
\item \emph{positive thinking}
\end{itemize} 
\end{itemize}
\subsubsection{Primary and Secondary Greatness}
\begin{itemize}
\item The elements of \emph{\textbf{Personality Ethic}} are \emph{\textbf{secondary}}, not primary traits. If my \emph{\textbf{character}} is fundamentally flawed, marked by duplicity and insincerity, then, in the long run, I cannot be successful -- My duplicity will breed \emph{distrust}, and everything i do -- even using so-called good human relations techniques -- will be preceived as \emph{manipulative}. 
\item In \emph{\textbf{one-shot} or \textbf{short-term}} human interactions, you can use Personality Ethics to win over good impressions through charm and skill.
\item If there is little or no trust, there is no foundation for permanent success. Only \emph{\textbf{basic goodness}} gives life to techinque.
\item You always reap what you sow; there is no shortcut.
\item If there isn't \emph{\textbf{deep integrity}} and \emph{\textbf{fundamental character strength}}, the challenges of life will cause true motives to surface and human relationship failure will replace short-term success.
\item Many people with secondary greatness -- that is, social recognition for their talents -- lack primary greatness or goodness in their character.
\end{itemize}
\subsubsection{The Power of A Paradigm}
\begin{itemize}
\item \emph{The Seven Habits of Highly Effective People} embody many of the fundamental principles of human effectiveness. These habits are basic; they are primary. They represent \emph{the \textbf{internalization} of \textbf{correct principles} upon which enduring happiness and success are based}.

\item The word \emph{\textbf{paradigm}} is more commonly used today to mean a model, theory, perception, assumption, or frame of reference. In a more general sense, it means the way we \emph{\textbf{see}} the world in terms of \emph{preceiving}, \emph{understanding}, \emph{interpreting}.

\item A simple way to understand \emph{paradigms} is to see them as \emph{\textbf{maps}}. \emph{\textbf{Maps is not the territory}}. A map is simply an explanation of certain aspects of the territory. It is a theory, an explanation, or model of something else.

\item Each of us has many, many maps in our head, which can be divided into two main categories: 
\begin{itemize}
\item maps of the way things \emph{are}, or \emph{\textbf{realities}}, 
\item and maps of the way things \emph{should be}, or \emph{\textbf{values}}.
\end{itemize} 

\item We seldom question their accuracy; we're usually even unaware that we have them. We simply assume that the way we see things is the way
they really are or the way they should be. 

\item And our \emph{\textbf{attitudes}} and \emph{\textbf{behaviors}} grow out of those assumptions. The way we see things is the source of the way we think and the way we act.


\item The fundamental problem has nothing to do with your \emph{behavior} or your \emph{attitude}. It has everything to do with \emph{having a wrong map}.

\item The paradigms are the source of our attitudes and behaviors. And \emph{\textbf{conditioning effects}} impact our perceptions and paradigms. 

\item The basic \emph{\textbf{flaws}} of \emph{the personality ethic} is oo try to change \emph{outward attitudes and behaviors} does very little good in the long run if we fail to \emph{examine \textbf{the basic paradigms}} from which those attitudes and behaviors flow.
\end{itemize}
\subsubsection{The Power of A Paradigm Shift}
\begin{itemize}
\item \emph{\textbf{Paradigm Shift}}: almost every significant breakthrough in the field of scientific endeavor is first a break with \emph{tradition}, with \emph{old ways of thinking}, with \emph{old paradigms}.

\item Not all Paradigm Shifts are in positive directions. As we have observed, the shift from \emph{the character ethic} to \emph{the personality ethic} has drawn us away from the very roots that nourish true success and happiness.

\item But whether they shift us in positive or negative directions, paradigm shift create powerful change. Our paradigms, correct or incorrect, are the sources of our attitudes and behaviors, and ultimately our relationships with others.

\item Many people experience a similar fundamental shift in thinking when they face a life-threatening crisis and suddenly see their priorities in a different light, or when they suddenly step into a new role, such as that of husband or wife, parent or grandparent, manager or leader.
\end{itemize}
\subsubsection{Seeing and Being}
\begin{itemize}
\item Not all Paradigm Shifts are \emph{instantaneous}. The power of a Paradigm Shift is the essential power of \emph{\textbf{quantum change}}, whether that shift is an instantaneous or a slow and deliberate process

\item Our Paradigms are the way we ``\emph{see}" the world or circumstances -- not in terms of our visual sense of sight, but in terms of \emph{\textbf{perceiving}, \textbf{understanding}, and \textbf{interpreting}}. 

\item \emph{\textbf{Paradigms}} are inseparable from \emph{\textbf{character}}. \emph{\textbf{Being}} is \emph{\textbf{seeing}} in the \emph{human dimension}.
\end{itemize}
\subsubsection{The Principle-Centered Paradigm}
\begin{itemize}
\item The \emph{\textbf{Character Ethic}} is based on the fundamental idea that \emph{there are \textbf{principles} that govern human \textbf{effectiveness}} -- natural laws in the human dimension that are just as real, just as \emph{unchanging and unarguably ``there"} as laws such as gravity are in the physical dimension.

\item Principles are like lighthouses. They are natural laws that cannot be broken. 

\item \emph{\textbf{Principles} are not \textbf{values}}. Principles are the \emph{territory}. Values are \emph{maps}. When we value correct principles, we have \emph{truth} -- a knowledge of things as they are.

\item Principles are \emph{guidelines} for human conduct that are proven to have \emph{enduring}, \emph{permanent value}. They're \emph{fundamental}. They're essentially unarguable because they are \emph{\textbf{self-evident}}. 

\item The more closely our \emph{\textbf{maps or paradigms}} \emph{are \textbf{aligned} with} these \emph{\textbf{principles}} or natural laws, the more accurate and functional they will be. Correct maps will infinitely impact our \emph{personal and interpersonal effectiveness} far more than any amount of effort expended on changing our attitudes and
behaviors
\end{itemize}
\subsubsection{Principles of Growth and Change}
\begin{itemize}
\item In all of life, there are \emph{\textbf{sequential} stages} of \emph{growth} and \emph{development}. Each step is important and each one takes time. No step
can be skipped.

\item This is true in all phases of life, in all areas of development.
\end{itemize}
\subsubsection{The Way we See the Problem Is the Problem}
\begin{itemize}
\item The way we see the problem is the problem.

\item Can you see how fundamentally the paradigms of the personality ethic affect the very way we see our problems as well as the way we attempt to solve them?
\end{itemize}
\subsubsection{A New Level of Thinking}
\begin{itemize}
\item This new level of thinking is what \emph{Seven Habits of Highly Effective People} is about.

\item It's a \emph{\textbf{principle-centered}, \textbf{character-based}, ``\textbf{Inside-Out}" approach} to \emph{\textbf{personal} and \textbf{interpersonal effectiveness}}.

\item ``\emph{\textbf{Inside-Out}}" means to \emph{start first with \textbf{self}}; even more fundamentally, to start with the most \emph{\textbf{inside part of self}} -- with your \emph{\textbf{paradigms}}, your \emph{\textbf{character}}, and your \emph{\textbf{motives}}.

\item The Inside-Out approach says that \emph{Private Victories} precede \emph{Public Victories}, that making
and keeping promises to \emph{ourselves} precedes making and keeping promises to \emph{others}.

\item Inside-Out is a \emph{\textbf{process}} -- a \emph{continuing process of \textbf{renewal}} based on the natural laws that govern human growth and progress. It's an \emph{\textbf{upward spiral} of growth} that leads to progressively higher forms of \emph{responsible} \textbf{\emph{independence}} and \emph{effective} \emph{\textbf{interdependence}}.
\end{itemize}
\subsection{The Seven Habits -- An Overview}
\subsubsection{``Habits" Defined}
\begin{itemize}
\item For our purposes, we will define a \emph{\textbf{habit}} as the intersection of \emph{knowledge}, \emph{skill}, and \emph{desire}.
\begin{itemize}
\item \emph{\textbf{Knowledge}} is the theoretical paradigm, the what to do and the why. 
\item \emph{\textbf{Skill}} is the how to do. 
\item And \emph{\textbf{desire}} is the motivation, the want to do.
\end{itemize}
 In order to make something a habit in our lives, we have to have all three.
\end{itemize}
\subsubsection{The Maturity Continuum}
\begin{itemize}
\item Our growth from infancy to adulthood is in accordance with natural law. And there are many dimensions to growth. Reaching our full physical maturity, for example, does not necessarily assure us of simultaneous \emph{\textbf{emotional} or \textbf{mental maturity}}. On the other hand, a person's \emph{physical dependence} does not mean that he or she is mentally or emotionally immature.

\item On the \emph{\textbf{maturity continuum}}, 
\begin{itemize}
\item \emph{\textbf{dependence}} is the paradigm of \emph{\textbf{you}} -- you take care of me; you come through for me; you didn't come through; I blame you for the results.
\item \emph{\textbf{Independence}} is the paradigm of \emph{\textbf{I}} -- I can do it; I am responsible; I am self-reliant; I can choose.
\item \emph{\textbf{Interdependence}} is the paradigm of \emph{\textbf{we}} -- we can do it: we can cooperate; we can combine our talents and abilities and create something greater together.
\end{itemize}

\item Dependent people \textit{need} others to get what they want. Independent people can get what they want through \emph{their own effort}. Interdependent people \emph{combine} \emph{their own} efforts with the efforts of others to achieve their greatest success.

\item If I were independent, \emph{\textbf{physically}}, I could pretty well make it \emph{\textbf{on my own}}. 

\emph{\textbf{Mentally}}, I could think \emph{my own thoughts}, I could move from one level of abstraction to another. I could think creatively and analytically and organize and express my thoughts in understandable ways.

\emph{\textbf{Emotionally}}, I would \emph{be validated from within}. I would be inner directed. My sense of \emph{worth} would not be a function of being liked or treated well.

\item True independence of character empowers us to \emph{\textbf{act}} rather than \emph{\textbf{be acted upon}}.

\item Independence is a major achievement in and of itself. But independence is not supreme. 

\item Independent thinking alone is not suited to \emph{\textbf{interdependent reality}}. Independent people who do not have the maturity to think and act interdependently may be good individual producers, but they won't be good leaders or team players. They're not coming from the paradigm of interdependence necessary to succeed in marriage, family, or organizational reality.

\item \emph{\textbf{Life}} is, by nature, highly \emph{\textbf{interdependent}}. 

\item \emph{Interdependence} is a far more mature, more advanced concept. 

If I am physically interdependent, I am self-reliant and capable, but I also realize that you and I working together can accomplish far more than, even at my best, I could accomplish alone. 

If I am emotionally interdependent, I derive a great sense of worth within myself, but I also recognize the need for love, for giving, and for receiving love from others.

If I am intellectually interdependent, I realize that I need the best thinking of other people to join with my own.

\item \emph{\textbf{Interdependence}} is a choice \emph{only \textbf{independent} people can make}. Dependent people cannot choose to become interdependent. They don't have the character to do it; they don't own enough of
themselves.
\end{itemize}
\subsubsection{Effectiveness Defined}
\begin{itemize}
\item A paradigm of effectiveness that is in harmony with a natural law, a principle I call the ``\emph{\textbf{P/PC Balance}}." 
\begin{itemize}
\item \textbf{P} stands for \emph{\textbf{Production}},
\item  and \textbf{PC} stands for \emph{\textbf{Production Capactity}}. 
\end{itemize}


\item True \emph{\textbf{effectiveness}} is a function of two things: 
\begin{itemize}
\item what is \emph{produced} (the \emph{production})
\item  and the producing \emph{asset} or \emph{capacity} to produce (the \emph{production capacity})
\end{itemize}

\item Effectiveness lies in the \emph{balance}. Focusing only on production will leave no assets to produce; focusing only on production capacity will fail to produce enough to maintain yourself or the capacity.
\end{itemize}
\subsubsection{Three Kinds of Assets}
\begin{itemize}
\item Basically, there are three kinds of assets: \emph{\textbf{physical}}, \emph{\textbf{financial}}, and \emph{\textbf{human}}. 
\begin{itemize}
\item \emph{physical assets}: house, car, personal belongings, our body or our environment. 
\item \emph{financial assets}: income, job, experience, skills, education, \emph{our capacity to earn}, 
\item \emph{human assets}: trust, responsibility, marriage, child-parent, other human relationships, 
\end{itemize}

\item In the human area, the P/PC Balance is equally fundamental, but even more important, because
people control physical and financial assets.

\item Examples in marriage, in child education.
\end{itemize}
\subsubsection{Organizational PC}
\begin{itemize}
\item One of the immensely valuable aspects of any correct principle is that it is \emph{valid} and \emph{applicable} in a wide variety of circumstances. 

\item When people fail to respect \emph{the P/PC Balance} in their use of \emph{\textbf{physical}} assets in organizations, they decrease organizational effectiveness and often leave others with dying geese.

\item \emph{The P/PC Balance} is particularly important as it applies to the \emph{\textbf{human}} assets of an organization -- the \emph{customers} and the \emph{employees}.

\item There are organizations that talk a lot about the customer and then completely neglect the people that deal with the customer -- the employees. The \emph{PC principle} is to always treat your employees exactly as you want them to treat your best customers.

\item Effectiveness lies in the balance. 

Excessive focus on P results in ruined health, worn-out machines, depleted bank accounts, and broken relationships. 

Too much focus on PC is like a person who runs for three or four hours a day, bragging about the extra 10 years of life it creates, unaware he's spending them running. Or a person endlessly going to school, never producing, living on other people's golden eggs -- the eternal student syndrome.

\item To maintain \emph{the P/PC Balance} is often a difficult judgment call. But I suggest it is the very essence of effectiveness. It balances short term with long term. 

\item The P/PC Balance is the very essence of effectiveness. It's validated in every arena of life. 
\end{itemize}
\subsubsection{How to Use This Book and What You Can Expect}
\begin{itemize}
\item Before we begin work on the Seven Habits of Highly Effective People, I would like to suggest two
Paradigm Shifts that will greatly increase the value you will receive from this material:
\begin{itemize}
\item First, I would recommend that you not ``see" this material as a book.  You may choose to read it completely through once for a sense of the whole. But the material is designed to be a companion in the continual process of change and growth. It is organized
\emph{incrementally} and with suggestions for application at the end of each habit so that you can study and
\emph{focus on any particular habit} as you are ready.

As you progress to deeper levels of understanding and implementation, you can \emph{go back} time and again to the principles contained in each habit and work to expand your knowledge, skill, and desire.

\item Second, I would suggest that you shift your paradigm of your own involvement in this material
from the role of learner to that of teacher. Take an Inside-Out approach, and read with the purpose in
mind of \emph{\textbf{sharing}} or \emph{\textbf{discussing}} what you learn with someone else within 48 hours after you learn it.
\end{itemize}

\item If you decide to open your ``gate of change" to really understand and live the principles embodied in
\emph{the Seven Habits}, I feel comfortable in assuring you several positive things will happen.
\begin{itemize}
\item First, your \emph{\textbf{growth}} with be \emph{\textbf{evolutionary}}, but \emph{\textbf{the net effect}} will be \emph{\textbf{revolutionary}}. 

The net effect of opening the "gate of change" to the first three habits -- the habits of \emph{\textbf{Private Victory}} -- will be significantly increased \emph{\textbf{self-confidence}}. 

As you live your values, your sense of \emph{\textbf{identity}}, \emph{\textbf{integrity}}, \emph{\textbf{control}}, and \emph{\textbf{inner-directedness}} will infuse you with both exhilaration and peace. You will \emph{\textbf{define yourself from within}}, rather than by \emph{people's opinions} or by comparisons to others. ``Wrong" and ``right" will have little to do with being found out.

Ironically, you'll find that as you care less about what others think of you; you will \emph{care more about what \textbf{others} think of \textbf{themselves} and their \textbf{worlds}}, including their relationship with you. You'll no longer \emph{build} your emotional life \emph{on} other people's weaknesses. In addition, you'll find it easier and more desirable to \emph{\textbf{change}} because there is something -- \emph{\textbf{some core deep within}} -- that is essentially \emph{\textbf{changeless}}.

\item As you open yourself to the next three habits -- the habits of \emph{\textbf{Public Victory}} -- you will discover and unleash both the desire and the resources to \emph{heal and rebuild important relationships} that have \emph{deteriorated}, or even broken. \emph{\textbf{Good relationships} will \textbf{improve}} -- become deeper, more solid, more creative, and more adventuresome.

\item The seventh habit, if deeply \emph{internalized}, will \emph{\textbf{renew}} the first six and will make you truly independent and capable of effective interdependence. Through it, you can charge your own batteries.
\end{itemize}
\end{itemize}
\section{Private Victory}
\subsection{Habit 1: Be Proactive -- Principles of Personal Vision}
\emph{\textbf{Self-awareness}} enables us to stand apart and examine even the way we ``see" ourselves -- our paradigm, the most fundamental paradigm of effectiveness. It affects not only our attitudes and behaviors, but also how we see other people. It becomes our map of the basic nature of mankind.
\subsubsection{The Social Mirror}
\begin{itemize}
\item If the only vision we have of ourselves comes from \emph{\textbf{the social mirror}} -- from \emph{the current social paradigm} and from the opinions, perceptions, and paradigms of \emph{the people around us} -- our view of ourselves is like the reflection in the crazy mirror room at the carnival.

\item These visions are \emph{\textbf{disjointed}} and \emph{\textbf{out of proportion}}. They are often more projections than reflections, projecting the concerns and character weaknesses of \emph{people giving the input} \emph{rather than} accurately \emph{reflecting what we are}.

\item While we have acknowledged the tremendous power of \emph{\textbf{conditioning}} in our lives, to say that we are determined by it, that we have no control over that influence, creates quite a different map.
\end{itemize}
\subsubsection{Between Stimulus and Response}
\begin{itemize}
\item The deterministic paradigm comes primarily from the study of animals. Our unique human endowments lift us above the animal world. The extent to which we exercise and develop these endowments empowers us to fulfill our uniquely human potential. \emph{\textbf{Between stimulus and response} is our greatest \textbf{power}} -- \emph{the \textbf{freedom} to choose}.

\item Within the freedom to choose are those \emph{\textbf{endowments}} that make us uniquely human. 
\begin{itemize}
\item In addition to \emph{\textbf{self-awareness}},
\item we have \emph{\textbf{imagination}} -- \emph{the ability to create} in our minds \emph{beyond our present reality}.
\item We have \emph{\textbf{conscience}} -- \emph{a deep inner awareness of right and wrong}, of the principles that govern our behavior, and a sense of the degree to which our thoughts and actions are in harmony with them.
\item And we have \emph{\textbf{independent will}} -- \emph{the ability to \textbf{act}} based on our self-awareness, free of all other influences.
\end{itemize}


 
\end{itemize}
\subsubsection{``Proactivity" Defined}
\begin{itemize}
\item In discovering the basic principle of the nature of man, Frankl described an accurate self-map from which he began to develop the first and most basic habit of a highly effective person in any environment, the habit of \emph{\textbf{Proactivity}}.

\item \emph{\textbf{Proactivity}} means more than merely \textbf{\emph{taking initiative}}. It means that as human beings, we \emph{\textbf{are responsible} for our \textbf{own lives}}. Our \emph{\textbf{behavior}} is a function of our \emph{\textbf{decisions}}, \textbf{not} our \emph{\textbf{conditions}}. We can subordinate \emph{feelings} to \emph{values}. We have the \emph{initiative} and the \emph{responsibility} to make things happen.

\item \emph{Highly \textbf{proactive} people \textbf{recognize} that \textbf{responsibility}}. They do not blame circumstances, conditions, or conditioning for their behavior. Their behavior is a product of their own conscious choice, based on values, rather than a product of their conditions, based on feeling.

\item Because we are, by nature, \emph{proactive}, if our lives are a function of \emph{\textbf{conditioning}} and \emph{\textbf{conditions}}, it is because we have, by \emph{conscious decision} or \emph{by default}, \emph{\textbf{chosen}} to \emph{empower those things to control us}.

\item In making such a choice, we become \emph{\textbf{reactive}}. 

Reactive people are often affected by their physical environment. 

Reactive people are also affected by their \emph{social environment}, by the ``social weather." Reactive people build their emotional lives around the behavior of others, \emph{empowering the \textbf{weaknesses of other} people to \textbf{control} them}.

\item The ability to \emph{\textbf{subordinate} an \textbf{impulse} to a \textbf{value}} is the \emph{\textbf{essence}} of \emph{the proactive person}. Reactive
people are driven by feelings, by circumstances, by conditions, by their environment. Proactive people are driven by \emph{\textbf{values}} -- carefully thought about, selected and internalized values.

I admit this is very hard to accept emotionally, especially if we have had years and years of
explaining our misery in the name of circumstance or someone else's behavior. But until a person can
say deeply and honestly, ``I am what I am today because of the choices I made yesterday," that person
cannot say, ``\emph{I choose otherwise}."

\item Our character, our basic identity, does not have to be hurt at all. In fact, our most difficult experiences become the crucibles that forge our character and develop \emph{the \textbf{internal powers}}, the \emph{\textbf{freedom}} to \emph{handle difficult circumstances} in the future and to inspire others to do so as well.

\item Viktor Frankl suggests that there are \emph{\textbf{three central values}} in life -- 
\begin{itemize}
\item the \textbf{\emph{experiential}}, or that which happens to us; 
\item the \textbf{\emph{creative}}, or that which we bring into existence; 
\item and the \textbf{\emph{attitudinal}}, or our \emph{response} in difficult circumstances such as terminal illness
\end{itemize}
What matters most is \emph{\textbf{how} we \textbf{respond} to what we \textbf{experience} in life}.

\item \emph{\textbf{Difficult circumstances} often create \textbf{Paradigm Shifts}}, whole new frames of reference by which people see the world and themselves and others in it, and what life is asking of them. Their larger perspective reflects the attitudinal values that lift and inspire us all.
\end{itemize}
\subsubsection{Taking the Initiative}
\begin{itemize}
\item \emph{Our basic nature is to \textbf{act}, and \textbf{not} \textbf{be acted upon}}. As well as \emph{enabling} us to \emph{\textbf{choose our response}} to particular circumstances, this empowers us to \emph{\textbf{create circumstances}}.

\item Taking initiative does not mean being \emph{pushy}, \emph{obnoxious}, or \emph{aggressive}. It does mean \emph{\textbf{recognizing our responsibility}} to make things happen.

\item Many people \emph{\textbf{wait for something to happen}} or \emph{\textbf{someone to take care of them}}. But people who end up with the good jobs are the \textbf{proactive} ones who are \emph{\textbf{solutions} to problems}, \emph{not problems themselves}, who \emph{\textbf{seize} the initiative} to do whatever is \emph{\textbf{necessary}}, \emph{\textbf{consistent} with correct \textbf{principles}}, to \emph{get the job \textbf{done}}.

\item \emph{Holding} people to the \emph{responsible} course is not demeaning; it is \emph{\textbf{affirming}}. By respecting the proactive nature of other people, we provide them with at least one clear, undistorted reflection from the social mirror.
\end{itemize}
\subsubsection{Act or Be Acted Upon}
\begin{itemize}
\item It takes initiative to create t\emph{he P/PC Balance} of effectiveness in your life. It takes initiative to develop \emph{the Seven Habits}. As you study the other six habits, you will see that each depends on the development of your proactive muscles. Each puts the responsibility on you to act. 

If you wait to be acted upon, you \emph{\textbf{will be acted upon}}. And \emph{growth} and \emph{opportunity consequences} attend either road.

\item ``What is our response? What are we going to do? How can we exercise initiative in this situation?"

\item The \textbf{difference} between \emph{positive thinking} and \emph{\textbf{proactivity}}: \emph{\textbf{Proactivity did face reality}}. We faced the reality of the current circumstance and of future projections. But we also faced the reality that \emph{we had the \textbf{power} to \textbf{choose a positive response}} to those circumstances and projections. Not facing reality would have been to \emph{accept} the idea that what's happening in our environment had to determine us.
\end{itemize}
\subsubsection{Listening to Our Language}
\begin{itemize}
\item Because our attitudes and behaviors flow out of our paradigms, if we use our self-awareness to examine them, we can often see in them the nature of our underlying maps. Our language, for example, is a very real indicator of the degree to which we see ourselves as proactive people.

\item The \emph{\textbf{language}} of reactive people \emph{absolves them of responsibility}. That language comes from a basic \emph{paradigm} of \emph{\textbf{determinism}}. And the whole spirit of it is \emph{the transfer of responsibility}.

\item A serious problem with reactive language is that it becomes a \emph{\textbf{self-fulfilling prophecy}}. People become \emph{reinforced} in the paradigm that they are \emph{determined}, and they produce evidence to support the belief. They feel increasingly victimized and out of control, not in charge of their life or their destiny.
They blame outside forces -- other people, circumstances, even the stars -- for their own situation.

\item Proactive people \emph{\textbf{subordinate feelings to values}}. Love, the feeling, can be recaptured.
\end{itemize}
\subsubsection{Circle of Concern or Circle of Influence}
\begin{itemize}
\item Another excellent way to become more self-aware regarding our own degree of proactivity is to look at \emph{where} we focus our time and energy.
\begin{itemize}
\item \emph{\textbf{Circle of Concern}}: a wide range of \emph{concerns}; separate from things in which we have no particular mental or emotional involvement.

\item \emph{\textbf{Circle of Influence}}: within the Circle of Concerns, there are things that we have real \emph{control}.
\end{itemize}

Within the Circle of Concerns but \emph{\textbf{outside}} Circle of Influence, there are some things over which we have \emph{no real control} and \emph{others that we can do something about}. 

\item By determining which of these two circles is the focus of most of our time and energy, we can discover much about the degree of our proactivity.
\begin{itemize}
\item \emph{\textbf{Proactive}} people \emph{\textbf{focus}} their efforts in \emph{\textbf{the Circle of Influence}}. 
\item \emph{Reactive} people, on the other hand, \emph{focus} their efforts in \emph{the Circle of Concern}. They focus on the weakness of other people, the problems in the environment, and circumstances \emph{over which they have \textbf{no control}}. 

The negative energy generated by that focus, combined with neglect in areas they could do something about, causes their \emph{Circle of Influence} to \emph{\textbf{shrink}}.
\end{itemize}

\item As long as we are working in our \emph{Circle of Concern}, we empower the things within it to control us. We aren't taking the proactive initiative necessary to effect positive change.

\item It was only when we went to \emph{\textbf{work in our Circle of Influence}}, when we focused on our \emph{own paradigms}, that we began to create a \emph{\textbf{positive energy}} that changed ourselves and eventually influenced our son as well. By \emph{working} on ourselves instead of \emph{worrying about} conditions, we were able to
\emph{\textbf{influence} the conditions}.

\item Because of position, wealth, role, or relationships, there are some circumstances in which a person's Circle of Influence is \emph{\textbf{larger}} than his or her Circle of Concern. This situation reflects on a \emph{self-inflicted emotional myopia} -- another \emph{reactive selfish life-style} focused in \emph{the Circle of Concern}.
\end{itemize}
\subsubsection{Direct, Indirect, and No Control}
\begin{itemize}
\item The problems we face fall in one of three areas: 
\begin{itemize}
\item \emph{\textbf{direct control}} (problems involving our own behavior);
\item \emph{\textbf{indirect control}} (problems involving other people's behavior);
\item or \emph{\textbf{no contro}}l (problems we can do nothing about, such as our past or situational realities). 
\end{itemize}

\item \begin{itemize}
\item \emph{\textbf{Direct control} problems} are solved by working on our \textbf{\emph{habits}} since they are within our Circle of Influence. See ``\emph{Private Victories}".

\item \emph{\textbf{Indirect control} problems} are solved by changing our \emph{\textbf{methods of influence}}. These are the ``\emph{Public Victories}"

\item \emph{\textbf{No control} problems} involve taking the responsibility to smile, to \emph{genuinely and peacefully \textbf{accept} these problems} and \emph{learn to \textbf{live with them}}, even though we don't like them.
\end{itemize}

\item \emph{Changing our habits}, \emph{changing our methods of influence} and changing the way we see our no control problems are all \emph{within our Circle of Influence}.
\end{itemize}
\subsubsection{Expanding Circle of Influence}
\begin{itemize}
\item It is inspiring to realize that \emph{in \textbf{choosing our response} to circumstance, we powerfully \textbf{affect} our circumstance}. 

\item This man focused on his \emph{Circle of Influence}. He was treated like a gofer, also. But he would do more than what was expected. He anticipated the president's need. He read with empathy the president's underlying concern, so when he presented information, he also gave his analysis and his recommendations based on that analysis.

\item There are some people who interpret ``proactive" to mean pushy, aggressive, or insensitive; but that isn't the case at all. Proactive people aren't pushy. They're \emph{smart}, they're \emph{value driven}, they \emph{read reality}, and they \emph{know what's needed}.
\end{itemize}
\subsubsection{The ``Have's" and the ``Be's"}
\begin{itemize}
\item One way to determine which circle our concern is in is to distinguish between the \emph{have's} and the \emph{be's}.
\begin{itemize}
\item \emph{The Circle of Concern} is filled with the \emph{have's}.
\item \emph{The Circle of Influence} is filled with the \emph{be's}.
\end{itemize}

\item Anytime we think the problem is ``out there," that thought is the problem. We empower what's out there to \emph{control} us. The change paradigm is ``outside-in" -- what's out there has to change before we can change.

\item The \emph{\textbf{proactive}} approach is to change from the \emph{\textbf{Inside-Out}}: \emph{\textbf{to be} different}, and by being different, to \emph{\textbf{effect}} \emph{positive change} in what's out there -- I can be more resourceful, I can be more diligent, I can be more creative, I can be more cooperative.

\item It is so much easier to blame other people, conditioning, or conditions for our own stagnant situation. But we are responsible -- ``response-able" -- to control our lives and to powerfully influence our circumstances by working on be, on what we are.

\item If I really want to improve my situation, I can work on the one thing over which I have control -- myself. 

\item As proactive people, we can carry our own physical or social weather with us. We can be happy and accept those things that at present we can't control, while we focus our efforts on the things that we can.
\end{itemize}
\subsubsection{The Other End of the Stick}
\begin{itemize}
\item Before we totally shift our life focus to our Circle of Influence, we need to consider two things in our \emph{Circle of Concern} that merit deeper thought -- \emph{\textbf{consequences}} and \emph{\textbf{mistakes}}.

\item While we are \emph{free to choose} our \emph{\textbf{actions}}, we are \emph{\textbf{not free} to choose the \textbf{consequences} of those actions}. \emph{\textbf{Consequences} are governed by natural \textbf{law}}. 

\item Our \emph{behavior} is governed by \emph{\textbf{principles}}. Living in \emph{\textbf{harmony}} with them brings \emph{\textbf{positive consequences}}; \emph{\textbf{violating}} them brings \emph{\textbf{negative consequences}}. We are free to choose our response in any situation, but in doing so, we choose the attendant consequence. ``\emph{When we pick up one end of the stick, we pick up the other.}"

\item Sometimes our choices have brought consequences we would rather have lived without. If we had the choice to make over again, we would make it differently. We call these choices \emph{\textbf{mistakes}}, and they are the second thing that merits our deeper thought.

\item For those filled with \emph{\textbf{regret}}, perhaps the most needful exercise of \emph{proactivity} is to realize that \emph{\textbf{past mistakes}} are also out there in \emph{\textbf{the Circle of Concern}}. We can't recall them, we can't undo them, we can't control the consequences that came as a result.

\item The proactive approach to a mistake is to \emph{\textbf{acknowledge} it \textbf{instantly}, \textbf{correct} it, and \textbf{learn} from it}.

\item But not to acknowledge a mistake, not to correct it and learn from it, \emph{is a mistake of \textbf{a different order}}. It usually puts a person on a self-deceiving, self-justifying path, often involving rationalization (rational lies) to self and to others. This \emph{\textbf{second} mistake}, this \emph{cover-up}, \emph{empowers the first}, giving it disproportionate importance, and causes far \emph{deeper injury to self}.
\end{itemize}
\subsubsection{Making and Keeping Commitments}
\begin{itemize}
\item At the very heart of our Circle of Influence is our ability to \emph{\textbf{make} and \textbf{keep} \textbf{commitments} and \textbf{promises}}. The commitments we make to ourselves and to others, and our \emph{\textbf{integrity}} to those commitments, is \emph{the essence and clearest manifestation of our proactivity}.

\item It is also the essence of our growth. Through our human endowments of \emph{\textbf{self-awareness}} and \emph{\textbf{conscience}}, we become conscious of areas of weakness, areas for improvement, areas of talent that could be developed, areas that need to be changed or eliminated from our lives. 

\item Then, as we recognize and use our \emph{\textbf{imagination}} and \emph{\textbf{independent will}} to act on that awareness -- making promises, setting goals,
and being true to them -- we \emph{build the strength of character}, the being, that makes possible every other positive thing in our lives.

\item It is here that we find two ways to put ourselves in control of our lives immediately. 
\begin{itemize}
\item We can make a promise -- and \textbf{keep} it. 
\item Or we can set a goal -- and \textbf{work to achieve} it.
\end{itemize}

\item As we make and keep commitments, even small commitments, we begin to establish an inner integrity that gives us the awareness of self-control and the courage and strength to accept more of the responsibility for our own lives. By making and keeping promises to ourselves and others, little by little, our honor becomes greater than our moods.

\item The power to \emph{make and keep commitments to ourselves} is the \emph{essence} of developing the basic habits of effectiveness. 
\end{itemize}
\subsubsection{Proactivity: the Thirty-Day Test}
\begin{itemize}
\item I would challenge you to test the principle of proactivity for 30 days. Simply try it and see what
happens. 

\item For 30 days work only in your \emph{Circle of Influence}. 
\begin{itemize}
\item \textbf{Make small commitments and keep them}.
\item \textbf{Be a light, not a judge}. Be a model, not a critic.  Look at the weaknesses of others with \emph{\textbf{compassion}}, not \emph{accusation}. The issue is your own \emph{chosen response} to the situation and what you \emph{should be doing}.
\item Be part of the \textbf{solution}, not part of the problem.
\end{itemize}
 
 \item We are responsible for our own effectiveness, for our own happiness, and ultimately, I would say, for most of our circumstances.
 
 \item Knowing that we are responsible -- "response-able" -- is \emph{\textbf{fundamental to effectiveness}} and to every other habit of effectiveness we will discuss.
\end{itemize}
\subsection{Habit 2: Begin with the End in Mind -- Principles of Personal Leadership}
\subsubsection{What It Means}
\begin{itemize}
\item \emph{\textbf{To Begin with the End in Mind}} means to start with a clear understanding of your \emph{\textbf{destination}}. It means to know where you're going so that you better understand \emph{where you are now} and so that the \emph{steps} you take are always in \emph{the right direction}.

\item How different our lives are when we really know what is deeply important to us, and, keeping that picture in mind, we manage ourselves each day to be and to do what really matters most.

\item We may be very busy, we may be very \emph{efficient}, but we will also be truly \emph{\textbf{effective}} only when we \emph{\textbf{Begin with the End in Mind}}.
\end{itemize}
\subsubsection{All Things are Created Twice}
\begin{itemize}
\item ``\emph{Begin with the End in Mind}" is based on the \emph{\textbf{principle}} that \emph{\textbf{all things are created twice}}.
\begin{itemize}
\item  There's a \emph{\textbf{mental}} or \emph{first} creation,
\item  and a \emph{\textbf{physical}} or \emph{second} creation to all things.
\end{itemize}


\item The carpenter's rule is ``\emph{measure twice, cut once}." You have to make sure that the \emph{\textbf{blueprint}}, the first
creation, is really what you want, that you've thought everything through. Then you put it into bricks and mortar. Each day you go to the construction shed and pull out the blueprint to get marching orders for the day. You Begin with the End in Mind.

\item The same is true with parenting.

\item To the extent to which we understand \emph{the principle of two creations} and accept the \emph{responsibility} for both, we \emph{\textbf{act within} and \textbf{enlarge} the borders of our \textbf{Circle of Influence}}. 

To the extent to which we do not operate in harmony with this principle and take charge of the first creation, we \emph{diminish} it.
\end{itemize}
\subsubsection{By Design Or Default}
\begin{itemize}
\item Not all first creations are \emph{by \textbf{conscious} \textbf{design}}.

\item In our personal lives, if we do not develop our own \emph{\textbf{self-awareness}} and become \emph{\textbf{responsible}} for \emph{first creations}, we \emph{empower other people and circumstances} outside our Circle of Influence to shape much of our lives \textbf{\emph{by default}}. 

\item Whether we are aware of it or not, whether we are in control of it or not, there is a first creation to every part of our lives. 

We are \emph{either} \emph{the \textbf{second} creation of our \textbf{own} proactive \textbf{design}}, or we are \emph{the second creation of other people's \textbf{agendas}}, of \emph{circumstances}, or of \emph{past habits}.

\item Habit 1 says, ``\emph{You are the creator}." Habit 2 \emph{is} the first creation.
\end{itemize}
\subsubsection{Leadership And Management -- the Two Creations}
\begin{itemize}
\item Habit 2 is based on \emph{\textbf{principles} of \textbf{personal leadership}}.
\begin{itemize}
\item \emph{\textbf{Leadership}} is the \emph{\textbf{first}} creation.
\item \emph{\textbf{Management}} is the \emph{second} creation, which we'll discuss in the chapter on Habit 3. But leadership has to come first.
\end{itemize}

\item ``\emph{\textbf{Management is doing things right; leadership is doing the right things}}."
\begin{itemize}
\item \emph{\textbf{Management}} is a \emph{bottom-line focus}. 
\item \emph{\textbf{Leadership}} deals with the \emph{top line}.
\end{itemize} 
Management is efficiency in climbing the ladder of success; leadership determines whether the ladder is leaning against the right wall.

\item We are more in need of a \emph{vision} or \emph{designation} and a \emph{compass} (\emph{a set of \textbf{principles} or \textbf{directions}}) and less in need of a \emph{road map}. 

\item \emph{\textbf{Effectiveness}} -- often even survival -- does not depend solely on how much \emph{effort} we expend, but on whether or not the effort we expend is in the \emph{right jungle}. And the metamorphosis taking place in most every industry and profession demands \emph{\textbf{leadership first and management second}}.

\item Efficient management without effective leadership is, as one individual phrased it, ``like straightening deck chairs on the Titanic." 
\emph{\textbf{No management success can compensate for failure in leadership}}. But leadership is hard because we're often caught in a \emph{\textbf{management paradigm}}.

In \emph{\textbf{management paradigm}}, one often thinks of control, efficiency, and rules instead of direction, purpose, and family feeling.

\item And leadership is even more lacking in our personal lives. We're into managing with efficiency, setting and achieving goals before we have even clarified our values.
\end{itemize}
\subsubsection{Rescripting: Become Your Own First Creator}
\begin{itemize}
\item The two additional unique human endowments that enable us to expand our proactivity and to exercise personal leadership in our lives are \emph{\textbf{imagination}} and \emph{\textbf{conscience}}.

\item Through \emph{\textbf{imagination}}, we can \emph{\textbf{visualize}} the \emph{uncreated} worlds of \emph{\textbf{potential}} that lie within us.

Through \emph{\textbf{conscience}}, we can come in contact with \emph{universal laws or principles} with our own singular talents and avenues of contribution, and with the \emph{\textbf{personal guidelines}} within which we can most effectively develop them. 

\item Because we already live with many scripts that have been handed to us, the process of writing our own script is actually more a process of ``\emph{\textbf{rescripting}}," or \emph{Paradigm Shifting} -- of changing some of the ·basic paradigms that we already have. 

\item In developing our own \emph{\textbf{self-awareness}} many of us discover \emph{ineffective} scripts, \emph{deeply embedded habits} that are totally \emph{unworthy} of us, totally \emph{incongruent} with the things we \emph{really value} in life. 

Habit 2 says we don't have to live with those scripts. 

We are response-able to use our \emph{\textbf{imagination}} and \emph{\textbf{creativity}} to write new ones that are more effective, more congruent with our deepest values and with
the correct principles that give our values meaning.

\item \emph{To Begin with the End in Mind} means to approach \emph{\textbf{my role} as a parent}, as well as my \emph{other roles} in life, with my \emph{values} and \emph{directions} clear.

It also means to begin each day with those values firmly in mind. I don't have to \emph{react} to the emotion, the circumstance. I can be truly proactive, value driven, because my values are clear.
\end{itemize}
\subsubsection{A Personal Mission Statement}
\begin{itemize}
\item The most effective way I know to \emph{Begin with the End in Mind} is to develop a \emph{\textbf{personal mission statement}} or \emph{philosophy} or \emph{creed}. 

It focuses on what you \emph{\textbf{want to be}} (character) and \emph{\textbf{to do}} (contributions and achievements) and on the \emph{\textbf{values}} or \emph{\textbf{principles}} upon which being and doing are based.

\item A \emph{personal mission statement} based on correct principles becomes \emph{\textbf{personal constitution}}, the basis for making major, life-directing decisions, the basis for making daily decisions in the midst of the circumstances and emotions that affect our lives. It empowers individuals with the same \emph{\textbf{timeless strength}} in the midst of change.

\item People can't live with change if there's not a \emph{\textbf{changeless core}} inside them. The \emph{\textbf{key}} to the \emph{\textbf{ability to change}} is a \emph{\textbf{changeless} sense of who you are}, \emph{what you are about} and \emph{what you value}.

\item With a mission statement, we can \emph{\textbf{flow with changes}}. We don't need \emph{\textbf{prejudgments}} or \emph{\textbf{prejudices}}. We don't need to figure out everything else in life, to \emph{stereotype} and \emph{categorize} everything and everybody in order to accommodate reality.

\item Our personal environment is also \emph{\textbf{changing}} at an \emph{ever-increasing pace}. Such \emph{rapid change} burns out a large number of people who feel they can \emph{hardly handle} it, can \emph{hardly cope with} life. They become \emph{\textbf{reactive}} and essentially \emph{\textbf{give up}}, hoping that the things that happen to them will be good.

\item Once you have that sense of \emph{\textbf{mission}}, you have the essence of your \emph{own proactivity}. You have the \emph{vision} and the \emph{values} which direct your life. You have the basic direction from which you set your long- and short-term goals. You have the power of a written constitution based on correct principles, against which every decision concerning the most effective use of your time, your talents, and your energies can be effectively measured.
\end{itemize}
\subsubsection{At The Center}
\begin{itemize}
\item In order to write a \emph{personal mission statement}, we must begin at the very \emph{\textbf{center}} of our \emph{\textbf{Circle of Influence}}, that center comprised of our most basic \emph{Our \textbf{paradigms}}, the lens through which we see the world.

\item It is here that we deal with our \emph{\textbf{vision}} and our \emph{\textbf{values}}. 

\item It is also here that our focused efforts achieve the greatest results. As we work within the very \emph{center} of our \emph{Circle of Influence}, we \emph{\textbf{expand}} it.

\item Whatever is at the center of our life will be the source of our \emph{\textbf{security}}, \emph{\textbf{guidance}}, \emph{\textbf{wisdom}}, and \emph{\textbf{power}}.
\begin{itemize}
\item \emph{\textbf{Security}} represents your \emph{\textbf{sense} of \textbf{worth}}, your \emph{\textbf{identity}}, your \emph{\textbf{emotional anchorage}}, your \emph{\textbf{self-esteem}}, your \emph{basic \textbf{personal strength}} or lack of it.
\item \emph{\textbf{Guidance}} means your \emph{\textbf{source} of \textbf{direction in life}}. Encompassed by your map, your \emph{internal frame of reference} that interprets for you what is happening out there, are \emph{standards} or \emph{principles} or \emph{implicit criteria} that govern moment-by-moment decision-making and doing.
\item \emph{\textbf{Wisdom}} is your \emph{\textbf{perspective on life}}, your \emph{sense of \textbf{balance}}, your \emph{understanding} of \emph{\textbf{how}} the various parts and principles \emph{\textbf{apply}} and \emph{\textbf{relate to}} each other. It embraces \emph{judgment}, \emph{discernment}, \emph{comprehension}. It is a gestalt or oneness, an integrated wholeness.
\item \emph{\textbf{Power}} is the \emph{faculty} or \emph{\textbf{capacity} to \textbf{act}}, the \emph{\textbf{strength}} and \emph{\textbf{potency to accomplish}} something. It is the \emph{\textbf{vital energy}} to make choices and decisions. It also includes the \emph{capacity to overcome} deeply embedded habits and to cultivate higher, more effective ones.
\end{itemize}

\item These four factors -- \emph{security, guidance, wisdom, and power} -- are \emph{\textbf{interdependent}}. 

Security and clear guidance bring true wisdom, and wisdom becomes the spark or catalyst to release and direct power.

\item Your \emph{\textbf{security}} lies somewhere on the \emph{\textbf{continuum}} between extreme insecurity on one end, wherein your life is buffeted by all the fickle forces that play upon it, and a deep sense of \emph{high intrinsic worth} and \emph{personal security} on the other end. 

\item Your \emph{\textbf{guidance}} ranges on the \emph{continuum} from dependence on \emph{the social mirror} or other unstable, fluctuating sources to \emph{strong inner direction}. 

\item Your \emph{\textbf{wisdom}} falls somewhere between a totally inaccurate map where everything is \emph{distorted} and nothing seems to fit, and a complete and accurate \emph{\textbf{map of life}} wherein all the parts and \emph{principles} are properly related to each other.

\item Your \emph{\textbf{power}} lies somewhere between immobilization or being a \emph{puppet} pulled by someone else's strings to high \emph{\textbf{proactivity}}, \emph{the power to act} according to \emph{your own values} instead of \emph{being acted upon} by other people and circumstances.
\end{itemize}
\subsubsection{Alterative Centers}
\begin{itemize}
\item Each of us has a center, though we usually don't recognize it as such.

\item Let's briefly examine several centers or core paradigms people typically have:
\begin{itemize}
\item \emph{\textbf{Spouse Centeredness}}.
\begin{itemize}
\item \emph{\textbf{security}}: 
\begin{itemize}
\item Your feelings of security are based on the way your spouse treats you.
\item You are highly vulnerable to the moods and feelings of your spouse.
\item There is deep disappointment resulting in conflict when your spouse disagrees with you or does not meet your expectations.
\item Anything that may impinge on the relationship is perceived as a threat. 
\end{itemize}
  
\item \emph{\textbf{guidance}}:
\begin{itemize}
\item Your direction comes from your own needs and wants and from those of your spouse.
\item Your decision-making criterion is limited to what you think is best for your marriage or your mate, or to the preference and opinions of your spouse.
\end{itemize}
  
\item \emph{\textbf{wisdom}}:
\begin{itemize}
\item Your life perspective surrounds things which may positively or negatively influence your spouse or your relationship.
\end{itemize}

\item \emph{\textbf{power}}:
\begin{itemize}
\item Your power to act is limited by weakness in yourself and your spouse.
\end{itemize}
\end{itemize}

\item \emph{\textbf{Family Centeredness}}.
\begin{itemize}
\item \emph{\textbf{security}}: 
\begin{itemize}
\item Your security is founded on family acceptance and fulfilling family expectations.
\item Your sense of personal security is as volatile as the family.
\item Your feeling of family worth are based on the family reputation.
\end{itemize}
  
\item \emph{\textbf{guidance}}:
\begin{itemize}
\item Family scripting is your source of correct attitudes and behaviors.
\item Your decision-making criterion is what is good for the family, or what family members want.
\end{itemize}
  
\item \emph{\textbf{wisdom}}:
\begin{itemize}
\item Your interpret all of life in terms of your family, creating a partial understanding and family nasrcissism.
\end{itemize}

\item \emph{\textbf{power}}:
\begin{itemize}
\item Your actions are limited by family models and traditions.
\end{itemize}
\end{itemize}

\item \emph{\textbf{Money Centeredness}}.
\begin{itemize}
\item \emph{\textbf{security}}: 
\begin{itemize}
\item Your personal worth is determined by your net worth.
\item You are vulnerable to anything that threatens your economic security.
\end{itemize}
  
\item \emph{\textbf{guidance}}:
\begin{itemize}
\item Profit is your decision-making criterion.
\end{itemize}
  
\item \emph{\textbf{wisdom}}:
\begin{itemize}
\item Money-making is the lens through which life is seen and understood, creating imbalanced judgment.
\end{itemize}

\item \emph{\textbf{power}}:
\begin{itemize}
\item You are restricted to what you can accomplish with your money and your limited vision.
\end{itemize}
\end{itemize}

\item \emph{\textbf{Work Centeredness}}.
\begin{itemize}
\item \emph{\textbf{security}}: 
\begin{itemize}
\item You tend to define yourself by your occupational role.
\item You are only comfortable when you are working.
\end{itemize}
  
\item \emph{\textbf{guidance}}:
\begin{itemize}
\item You make your decisions based on the needs and expectations of your work.
\end{itemize}
  
\item \emph{\textbf{wisdom}}:
\begin{itemize}
\item You tend to be limited to your work role.
\end{itemize}

\item \emph{\textbf{power}}:
\begin{itemize}
\item Your actions are limited by work role models, organizational constraints, occupational opportunities, your boss's perceptions, and your possible inability at some point in your life to do that particular work.
\end{itemize}
\end{itemize}

\item \emph{\textbf{Possession Centeredness}}.
\begin{itemize}
\item \emph{\textbf{security}}: 
\begin{itemize}
\item Your security is based on your reputation, your social status, or the tangible things you possess.
\item You tend to compare what you have to what others have.
\end{itemize}
  
\item \emph{\textbf{guidance}}:
\begin{itemize}
\item You make  your decisions baed on what will protect, increase, or better display your possessions.
\end{itemize}
  
\item \emph{\textbf{wisdom}}:
\begin{itemize}
\item You see the world in terms of comparative economic and social relationships
\end{itemize}

\item \emph{\textbf{power}}:
\begin{itemize}
\item You function within the limits of what you can buy or the social prominence you can achieve.
\end{itemize}
\end{itemize}

\item \emph{\textbf{Pleasure Centeredness}}.

\item \emph{\textbf{Friend Centeredness}}.
\begin{itemize}
\item \emph{\textbf{security}}: 
\begin{itemize}
\item Your security is a function of the social mirror.
\item You are highly dependent on the opinions of others.
\end{itemize}
  
\item \emph{\textbf{guidance}}:
\begin{itemize}
\item Your decision criterion is ``What will they think?"
\item You are easily embarrassed.
\end{itemize}
  
\item \emph{\textbf{wisdom}}:
\begin{itemize}
\item You see the world through a social lens.
\end{itemize}

\item \emph{\textbf{power}}:
\begin{itemize}
\item You are limited by your social comfort zone.
\item Your actions are as fickle as opinion.
\end{itemize}
\end{itemize}

\item \emph{\textbf{Enemy Centeredness}}.

\item \emph{\textbf{Church Centeredness}}.

\item \emph{\textbf{Self-Centeredness}}.
\begin{itemize}
\item \emph{\textbf{security}}: 
\begin{itemize}
\item Your security is constantly changing and shifting.
\end{itemize}
  
\item \emph{\textbf{guidance}}:
\begin{itemize}
\item Your judgement criteria are: ``If it feels good", ``What I want", ``What I need", ``What 's in it for me?"
\end{itemize}
  
\item \emph{\textbf{wisdom}}:
\begin{itemize}
\item Your view the world by how decisions, events or circumstances will affect you.
\end{itemize}

\item \emph{\textbf{power}}:
\begin{itemize}
\item Your ability to act is limited to your own resources, without the benefits of interdependency.
\end{itemize}
\end{itemize}
\end{itemize}
\end{itemize}
\subsubsection{Identify Your Center}
\begin{itemize}
\item Perhaps the best way to identify your own center is to look closely at \emph{your \textbf{life-support factors}}. If you can identify with one or more of the descriptions above, you can trace it back to the center from which it flows, a \emph{center} which may be \emph{\textbf{limiting} your \textbf{personal effectiveness}}.

\item More often than not, a person's center is some \emph{\textbf{combination}} of these and/or other \emph{centers}. 

\item Most people are very much a function of \emph{a variety of influences} that play upon their lives. Depending on
\emph{external or internal conditions}, \emph{one particular center} may be \emph{activated} until the underlying needs are satisfied. Then another center becomes the compelling force.

\item The \emph{\textbf{ideal}}, of course, is to \emph{create} \emph{one clear center} from which you \emph{consistently} derive a high degree of security, guidance, wisdom, and power, empowering your proactivity and giving congruency and harmony to every part of your life.
\end{itemize}
\subsubsection{A Principle Center}
\begin{itemize}
\item By \textbf{centering} our lives on \emph{\textbf{correct}} \emph{\textbf{principles}}, we create a solid foundation for development of the four life-support factors. 

\item \emph{\textbf{Correct principles do not change}}. We can depend on them.

\item \emph{Principles} don't \emph{\textbf{react}} to anything. They don't \emph{\textbf{depend}} on the behavior of \emph{others}, the \emph{environment}, or the \emph{current fad} for their \emph{\textbf{validity}}. Principles don't \emph{die}.

\item Principles are \emph{\textbf{deep}, \textbf{fundamental truths}}, classic truths, generic common denominators.

\item Even in the midst of people or circumstances that seem to ignore the principles, we can be \emph{\textbf{secure} in the knowledge} that \emph{principles are bigger than people or circumstances}.

\item Admittedly, we're not omniscient. Our \emph{knowledge} and \emph{understanding} of correct principles is \emph{\textbf{limited}}.

\item We are limited, but \emph{we can \textbf{push back} the borders of our limitations}. 

\item The \emph{\textbf{wisdom}} and \emph{\textbf{guidance}} that accompany \emph{principle-centered living} come from \emph{correct maps}, from the way things really are, have been, and will be.  \emph{\textbf{Correct maps}} enable us to clearly see where we want to go and how to get there. We can make our decisions using the \emph{correct data} that will make their implementation possible and meaningful.

\item The personal \emph{\textbf{power}} that comes from principle-centered living is the power of a \emph{\textbf{self-aware}}, \emph{\textbf{knowledgeable}}, \emph{\textbf{proactive}} individual, \emph{unrestricted} by the attitudes, behaviors, and actions of \emph{others} or by many of the \emph{circumstances} and \emph{environmental influences} that \emph{limit} other people.

\item The only real limitation of power is the natural consequences of the principles themselves. We are free to choose our actions, based on our knowledge of correct principles, but we are not free to choose the consequences of those actions.

\item By centering our lives on \emph{timeless, unchanging principles}, we create a \emph{\textbf{fundamental paradigm} of \textbf{effective living}}. It is the \emph{center} that puts all \emph{other centers} in perspective.

\item Remember that your \emph{\textbf{paradigm}} is the \emph{source} from which your \emph{\textbf{attitudes}} and \emph{\textbf{behaviors}} flow.

\item \emph{\textbf{Principle Centeredness}}.
\begin{itemize}
\item \emph{\textbf{security}}: 
\begin{itemize}
\item Your security is based on \emph{\textbf{correct principles}} that \emph{\textbf{do not change}}, regardless of external conditions or circumstances.
\item You know that true principles can r\emph{epeatedly be validated} in your own life, through your own experiences.
\item As a measurement of self-improvement, correct principles function with \emph{exactness}, \emph{consistency}, \emph{beauty} and \emph{strength}.
\item Correct principles help you understand your own \emph{development}, endowing you with the confidence to learn more, thereby increasing your knowledge and understanding.
\item Your source of security provides you with an \emph{immovable}, \emph{unchanging}, \emph{unfailing} \textbf{\emph{core}} enabling you to see change as an exciting adventure and opportunity to make significant contributions.
\end{itemize}
  
\item \emph{\textbf{guidance}}:
\begin{itemize}
\item You are guided by a \emph{\textbf{compass}} which enables you to see where you want to go and how you will get there.
\item You use \emph{accurate data} which makes your decisions both implementable and meaningful. 
\item You stand apart from life's situations, and circumstances and look at \emph{\textbf{the balanced whole}}. 
\item Your decisions and actions reflect both \emph{short and long-term considerations} and implications.
\item In every situation, you \emph{consciously}, \emph{proactively} determine the best alternative, basing decisions on conscience educated by principles.
\end{itemize}
  
\item \emph{\textbf{wisdom}}:
\begin{itemize}
\item Your judgment encompasses a broad spectrum of \emph{long-term consequences} and reflects a \emph{wise balance} and quiet assurance.
\item You see things differently and thus you think and act differently from the largely \emph{reactive} world.
\item You \emph{\textbf{view}} the world through a \emph{\textbf{fundamental paradigm}} for effective, provident living.
\item You see the world in terms of what you can do for the world and its people.
\item You adopt a \emph{\textbf{proactive lifestyle}}, seeking to serve and build others.
\item You interpret all of life's experiences in terms of \emph{opportunities for learning and contribution}.
\end{itemize}

\item \emph{\textbf{power}}:
\begin{itemize}
\item Your power is limited only by your understanding and observance of natural law and correct principles and by the natural consequences of the principles themselves.
\item You become a \emph{self-aware, knowledgeable, proactive individual}, largely unrestricted by the attitudes, behaviors, or actions of others.
\item Your ability to act \emph{reaches far beyond your own resources} and encourages highly developed levels of \emph{interdependency}.
\item Your decisions and actions are not driven by your current financial or circumstantial limitations.
\item You experience an \emph{\textbf{interdependent freedom}}.
\end{itemize}
\end{itemize}
\end{itemize}
\subsubsection{Writing and Using a Personal Mission Statement}
\begin{itemize}
\item As we go deeply within ourselves, as we understand and realign our basic paradigms to bring them in harmony with correct principles, we create both an effective, empowering center and a clear lens through which we can see the world. We can then focus that lens on how we, as unique individuals, relate to that world.

\item We \emph{\textbf{detect}} rather than \emph{invent} our \emph{missions in life}. 

Each of us has an internal monitor or sense, a \emph{\textbf{conscience}}, that gives us an \emph{awareness} of our own uniqueness and the singular contributions that we can make.

\item In seeking to give verbal expression to that uniqueness, we are again reminded of the \emph{fundamental importance} of \emph{\textbf{proactivity}} and of working \emph{\textbf{within}} our \emph{\textbf{Circle of Influence}}. 

\item Man should not ask what the \emph{\textbf{meaning}} of his life is, but rather must recognize that \emph{it is \textbf{he} who is asked}. In a word, each
man is questioned by life; and he can only answer to life by \emph{\textbf{answering} for his own life}; to life he can only respond by being \emph{\textbf{responsible}}.

\item \emph{\textbf{Personal responsibility}}, or \emph{\textbf{proactivity}}, is fundamental to the first creation. Habit 1 says ``You are the programmer." Habit 2, then, says, ``Write the program."

\item A \textbf{mission statement} is not something you write overnight. It takes \emph{deep introspection, careful analysis, thoughtful expression}, and often \emph{\textbf{many rewrites}} to produce it in final form.

Even then, you will want to \emph{\textbf{review}} it \emph{\textbf{regularly}} and \emph{make minor changes} as the years bring additional insights or \textit{changing circumstances}.

\item I find the \emph{process} is as important as the \emph{product}. Writing or reviewing a mission statement changes you because it forces you to think through your priorities deeply, carefully, and to align your behavior with your beliefs. 
\end{itemize}
\subsubsection{Using Your Whole Brain}
\begin{itemize}
\item Our \emph{\textbf{self-awareness}} empowers us to examine our own thoughts. This is particularly helpful in creating a personal mission statement because the two unique human endowments that enable us to practice Habit 2 -- \emph{\textbf{imagination}} and \emph{\textbf{conscience}} -- are primarily functions of the right side of the brain.

Understanding how to tap into that \emph{\textbf{right brain capacity}} greatly increases our \emph{first-creation ability}.
\end{itemize}
\subsubsection{Two Ways to Tap the Right Brain}
\begin{itemize}
\item The quality of our first creation is significantly impacted by our ability to use our creative right brain.

\item The two ways to tap the right brain:
\begin{itemize}
\item \emph{\textbf{Expand Perspective}}
\item \emph{\textbf{Visualization}} and \emph{\textbf{Affirmation}}
\end{itemize}
\end{itemize}
\subsubsection{Expand Perspective}
\begin{itemize}
\item Sometimes we are knocked out of our left-brain environment and thought patterns and into the right brain by an \emph{\textbf{unplanned experience}}. 

\item If you're \emph{\textbf{proactive}}, you \emph{don't have to wait} for circumstances or other people to create perspective-expanding experiences. You can \emph{\textbf{consciously} \textbf{create your own}}.

\item Expand your mind. \emph{Visualize} in rich detail. Involve as many emotions and feelings as possible. Involve as many of the senses as you can.

\item Things are suddenly placed in a \emph{different perspective}. \emph{\textbf{Values} quickly \textbf{surface}} that before weren't even recognized.
\end{itemize}
\subsubsection{Visualization and Affirmation}
\begin{itemize}
\item \emph{\textbf{Personal leadership}} is not a singular experience. It doesn't begin and end with the writing of a \emph{personal mission statement}. 

\item It is, rather, the \emph{\textbf{ongoing process}} of keeping your \emph{vision} and \emph{values} before you and \emph{aligning} your life to be congruent with those most important things.

\item And in that effort, your powerful right-brain capacity can be a great help to you on a daily basis as you work to \emph{\textbf{integrate}} your personal mission statement \emph{into your life}. It's another application of ``\emph{Begin with the End in Mind}."

\item A good \textbf{\emph{affirmation}} has five basic ingredients:
\begin{itemize}
\item  it's \emph{\textbf{personal}},
\item  it's \emph{\textbf{positive}},
\item  it's \emph{\textbf{present tense}},
\item  it's \emph{\textbf{visual}}, 
\item and it's \emph{\textbf{emotional}}. 
\end{itemize}

\item The more clearly and vividly I can imagine the detail, the more deeply I will experience it, the less I will see it as a spectator.

\item \emph{Affirmation} and \emph{visualization} are forms of \emph{\textbf{programming}}, and we must be certain that we do not submit ourselves to any programming that is not \emph{\textbf{in harmony with}} our basic \emph{\textbf{center}} or that comes from \emph{sources} centered on money-making, self interest, or anything other than \emph{correct principles}.

\item  I believe the higher use of \emph{\textbf{imagination}} is \emph{in harmony with} the use of \emph{\textbf{conscience}} to \emph{transcend self} and create a \emph{life of contribution} based on unique purpose and on the \emph{principles} that govern \emph{\textbf{interdependent reality}}.
\end{itemize}
\subsubsection{Identifying Roles and Goals}
\begin{itemize}
\item We each have a number of \emph{\textbf{different roles}} in our lives -- different areas or capacities in which we have \emph{\textbf{responsibility}}. And each of these roles is important.

\item One of the major problems that arises when people work to become more effective in life is that they don't \emph{\textbf{think broadly}} enough. 

They lose the sense of \emph{proportion}, the \emph{balance}, the natural \emph{ecology} necessary to effective living.

\item You may find that \emph{\textbf{your mission statement}} will be much more \emph{\textbf{balanced}}, much easier to work with, if you \emph{\textbf{break it down} into the \textbf{specific role} areas of your life} and the \emph{\textbf{goals}} you want to accomplish in \emph{each}
area. 

\item Writing your mission in terms of the important \emph{roles} in your life gives you \emph{\textbf{balance}} and \emph{\textbf{harmony}}. It keeps each role clearly before you. You can \emph{\textbf{review}} your roles \emph{frequently} to make sure that you don't get totally absorbed by one role to the exclusion of others that are equally or even more important in your life.

\item After you identify your various roles, then you can think about \emph{\textbf{the long term goals}} you want to accomplish in each of those roles. 

\item If these goals are the extension of a mission statement based on correct principles, they will be vitally different from the goals people normally set. They will be in
harmony with correct principles, with natural laws, which gives you greater power to achieve them. They are your goals. They reflect your deepest
values, your unique talent, your sense of mission. And they grow out of your chosen roles in life.

\item \emph{\textbf{An effective goal}} \emph{focuses} primarily on \emph{\textbf{results}} rather than \emph{activity}. It identifies where you want to
be, and, in the process, helps you determine where you are. It gives meaning and purpose to all you do.

\item \textit{\textbf{Roles} and \textbf{goals}} give \emph{\textbf{structure}} and \emph{\textbf{organized direction}} to your \emph{personal mission}.
\end{itemize}
\subsubsection{Organizational Mission Statements}
\begin{itemize}
\item \emph{\textbf{Mission statements}} are also vital to successful organizations. 

\item One of the \emph{\textbf{fundamental problems}} in organizations, including families, is that people are \emph{\textbf{not committed to}} the determinations of other people for their lives. They simply \emph{don't buy into them}.

\item Many times as I work with organizations, I find people whose goals are totally different from the goals of the enterprise. I commonly find reward systems completely out of alignment with stated value systems.

\item \emph{\textbf{Without involvement, there is no commitment}}. 

\item That's why creating an organizational mission statement \emph{takes time}, \emph{patience}, \emph{involvement}, \emph{skill}, and \emph{empathy}. Again, it's not a quick fix. It takes time and sincerity, correct principles, and the courage and integrity to align systems, structure, and management style to the shared vision and values. But it's based on correct principles and it works.

\item An organizational mission statement -- one that truly reflects the deep \emph{shared} vision and values of \emph{everyone within that organization} -- creates a great unity and tremendous commitment.
\end{itemize}
\subsection{Habit 3: Put First Things First -- Principles of Personal Management}
 \begin{enumerate}
\item Habit 1 says, ``\emph{You're the creator. You are \textbf{in charge}.}" It's based on the four unique \emph{human endowments} of \emph{\textbf{imagination}}, \emph{\textbf{conscience}}, \emph{\textbf{independent will}}, and particularly, \emph{\textbf{self-awareness}}. 
\item Habit 2 is the \emph{first} or \emph{\textbf{mental creation}}.  It's based on \emph{\textbf{imagination}} and \emph{\textbf{conscience}}.  It's the deep contact with our basic paradigms and values and the vision of
what we can become.
\item Habit 3, then, is the second creation -- \emph{\textbf{the physical creation}}. It's the exercise of \emph{\textbf{independent will}} toward becoming \emph{principle-centered}. 
\end{enumerate}

My own maxim of personal effectiveness is this: \emph{\textbf{Manage from the left}; \textbf{lead from the right}}.
\subsubsection{The Power of Independent Will}
\begin{itemize}
\item In addition to \emph{self-awareness, imagination, and conscience}, it is the fourth human endowment -- \emph{\textbf{independent will}} -- that really makes effective self-management possible. It is the ability to make decisions and choices and to act in accordance with them.

\item The degree to which we have developed our \emph{independent will} in our everyday lives is measured by our \emph{\textbf{personal integrity}}. \emph{Integrity} is, fundamentally, the \emph{value} we place \emph{on ourselves}. It's our ability to \emph{make and keep commitments} to ourselves.

\item \emph{\textbf{Effective management} is putting \textbf{first} things first}. While leadership decides what ``first things" are, it is management that puts them first, day-by-day, moment-by-moment. Management is \emph{discipline}, carrying it out.
\end{itemize}
\subsubsection{Four Generations of Time Management}
\begin{itemize}
\item The best thinking in the area of \emph{time management} can be captured in a single phrase: \emph{\textbf{Organize} and \textbf{execute around priorities}}. 

\item In the area of \textbf{time management}, there are four generations of efforts. Each generation builds on the one before it.
\begin{enumerate}
\item The first wave or generation could be characterized by \emph{\textbf{notes}} and \emph{\textbf{checklists}}
\item The second generation could be characterized by \emph{\textbf{calendars}} and \emph{appointment books}.\
\item The third generation reflects the current time-management field. It adds to those preceding generations the important idea of \emph{\textbf{prioritization}}, of clarifying \emph{\textbf{values}}, and of comparing the relative worth of activities based on their \emph{\textbf{relationship}} to those values. In addition, it focuses on setting \emph{\textbf{goals}} -- specific long-, intermediate-, and short-term targets toward which time and energy would be directed in harmony with values. It also includes the concept of \emph{\textbf{daily planning}}, of making a specific plan to accomplish those goals and activities determined to be of greatest worth.
\item The essential focus of the fourth generation of management can be captured in \emph{the \textbf{Time Management Matrix}} diagrammed on the next page. The challenge is not to manage time, but to \emph{manage ourselves}. \emph{Satisfaction} is a function of \emph{\textbf{expectation}} as well as \emph{\textbf{realization}}. And expectation (and satisfaction) lie in our \emph{\textbf{Circle of Influence}}.
\end{enumerate}
\end{itemize}
\subsubsection{Quadrant II}
\begin{itemize}
\item The two factors that define an activity are \emph{\textbf{urgent}} and \emph{\textbf{important}}.
\begin{itemize}
\item  \emph{\textbf{Urgent}} means it requires \emph{immediate attention}.  Urgent matters are usually \emph{visible}. They press on us; they \emph{insist} on \emph{\textbf{action}}. 
\item \emph{\textbf{Importance}}, on the other hand, has to do with \emph{\textbf{results}}. If something is important, it \emph{contributes} to your mission, your values, your \emph{high priority goals}.
\end{itemize}

\item We \emph{\textbf{react}} to \emph{\textbf{urgent} matters}.

\emph{\textbf{Important} matters} that are \emph{\textbf{not urgent}} require more \emph{initiative}, more \emph{\textbf{proactivity}}. 

\item Look for a moment at the four \emph{quadrants} in the \emph{\textbf{Time Management Matrix}}.
\begin{enumerate}
\item \textbf{Quadrant I} is \emph{\textbf{both} \textbf{urgent} and \textbf{important}}. It deals with significant results that require immediate attention. We usually call the activities in Quadrant I ``\textit{\textbf{crises}}" or ``\emph{\textbf{problems}}." Quadrant I \emph{\textbf{consumes}} many people. They are \emph{crisis managers}, \emph{problem-minded people}, the \emph{deadline-driven} producers.

As long as you \emph{focus} on Quadrant I, it keeps getting bigger and bigger until it \emph{\textbf{dominates}} you.  

\item \textbf{Quadrant II} is \emph{\textbf{not urgent} but \textbf{important}}. Quadrant II is the \emph{\textbf{heart}} of effective personal management. It consists of all \emph{\textbf{prevention}}, \emph{\textbf{planning}}, \emph{\textbf{relationship building}} and \emph{capacity-building}  activities.

\item \textbf{Quadrant III} is \emph{\textbf{urgent} but \textbf{not important}}. These are \emph{interruptions}, \emph{meetings}, \emph{mails}, \emph{calls} etc. The reality is that the \emph{urgency} of these matters is often based on \emph{the priorities and expectations of \textbf{others}}.
\item \textbf{Quadrant IV} is \emph{\textbf{neither} urgent nor important}. Some people are literally beaten up by the problems all day every day. The only \emph{\textbf{relief}} they have is in escaping to the not important, not urgent activities of Quadrant IV. 
\end{enumerate}

\item \emph{\textbf{Effective}} people \emph{\textbf{stay out of Quadrants III and IV}} because, urgent or not, they \emph{\textbf{aren't important}}.

They also \emph{\textbf{shrink Quadrant I }} down to size by spending more time in Quadrant II.

\item In Quadrant II, It deals with things like \emph{building relationships}, writing a \emph{personal mission statement}, \emph{long-range planning}, \emph{exercising}, \emph{preventive maintenance}, \emph{preparation} -- all those things we know we \textbf{need to do}, but somehow seldom get around to doing, because they aren't urgent.

\item \emph{\textbf{Effective} people are \textbf{not problem-minded}; they're \textbf{opportunity-minded}}. They feed \emph{opportunities} and \emph{starve problems}. They think \emph{\textbf{preventively}}. 

They keep \emph{P} and \emph{PC} in \emph{balance} by focusing on the \emph{important, but not the urgent}, high-leverage \emph{\textbf{capacity-building activities}} of Quadrant II.

\item What one thing could you do in your personal and professional life that, if you did on \emph{a regular basis}, would make a \emph{tremendous positive difference} in your life? Quadrant II activities have that kind of impact. Our \emph{effectiveness} takes the \emph{quantum leaps} when we do them.
\end{itemize}
\subsubsection{What It Takes to say ``No"}
\begin{itemize}
\item The only place to \textbf{get time} for Quadrant II in the beginning is from \emph{\textbf{Quadrants III and IV}}. You can't ignore the urgent and important activities of Quadrant I, although it will shrink in size as you spend more time with prevention and preparation in Quadrant II. But the initial time for Quadrant II has come out of III and IV.

\item You have to be \emph{\textbf{proactive}} to work on Quadrant II because Quadrant I and III work on you. To say ``yes" to important Quadrant II priorities, you have to learn to \emph{\textbf{say}} ``\emph{\textbf{no}}" to other activities, \emph{sometimes apparently urgent things}.

\item You have to decide what your \emph{\textbf{highest priorities}} are and have the \emph{\textbf{courage}} -- \emph{pleasantly, smiling, nonapologetically} -- to \emph{say ``\textbf{no}" to other things}. And the way you do that is by \emph{\textbf{having abigger ``yes" burning inside}}.

Only when you have the \emph{self-awareness} to examine your program -- and the \emph{imagination} and \emph{conscience} to create a new, unique, principle-centered program to which you can say ``yes" -- only then will you have sufficient independent will power to say ``no," with a genuine smile, to the unimportant.


\item The enemy of the ``\emph{\textbf{best}}" is often the ``good."  Even when the urgent is good, the good can keep you from your best, keep you from your unique contributions, if you let it.

\item The essence of \emph{effective \textbf{time} and \textbf{life management}} is to \emph{\textbf{organize} and \textbf{execute around balanced priorities}}. Then I ask this question: if you were to fault yourself in one of three areas, which would it be: 
\begin{enumerate}
\item  the inability to prioritize;
\item the inability or desire to organize around those priorities; 
\item or the lack of discipline to execute around them, to stay with your priorities and organization?
\end{enumerate} 
Most people say their main fault is a lack of discipline. On deeper thought, I believe that is not the case. The basic problem is that \emph{\textbf{their priorities} have not become \textbf{deeply planted in their hearts and minds}}. They haven't really internalized Habit 2.

\item Without a \emph{\textbf{principle center}} and a \emph{\textbf{personal mission statement}}, they don't have the necessary foundation to \emph{\textbf{sustain}} their efforts. 

\item A \emph{\textbf{Quadrant II focus}} is a paradigm that grows out of a \emph{\textbf{principle center}}. Your independent will alone \emph{cannot} effectively \emph{discipline} you against your center.

\item In the words of the architectural maxim, \emph{\textbf{form follows function}}. Likewise, \emph{\textbf{management follows leadership}}. The way you spend your time is a result of the way you see your time and the way you really see your priorities. 
\end{itemize}
\subsubsection{Moving into Quadrant II}
\begin{itemize}
\item If Quadrant II activities are clearly the heart of effective personal management -- the ``first things" we need to put first -- then how do we organize and execute around those things?

\item
 \begin{enumerate}
\item The first generation of time management does not even recognize the concept of priority. 
\item Second-generation managers assume a little more control. They plan and schedule in advance and generally are seen as more responsible because they "show up" when they're supposed to.

But again, the activities they schedule have no priority or recognized correlation to deeper values and goals. They have few significant achievements and tend to be schedule-oriented.

\item Third-generation managers take a significant step forward. They clarify their values and set goals. They plan each day and prioritize their activities. But this third generation has some critical \emph{\textbf{limitations}}. First, it limits \emph{\textbf{vision}}, in the form of ``daily planning". The very language  ``daily planning" focuses on the \emph{urgent}.

In addition, the third generation makes no provision for managing \emph{\textbf{roles}} in a \emph{\textbf{balanced}} way. And its efficiency, time-management
focus tends to \emph{strain relationships} rather than build them.

Even the third generation, with its vast array of planners and materials, focuses primarily on helping people prioritize and plan their Quadrant I and III activities.
\end{enumerate}
\end{itemize}
\subsubsection{The Quadrant II Tool}
\begin{itemize}
\item The objective of Quadrant II management is to manage our lives effectively 
\begin{enumerate}
\item from a center of sound principles, 
\item for a knowledge of our personal mission, 
\item with a focus on the important as well as the urgent,
\item and within the framework of maintaining a balance between increasing our Production and increasing our Production Capability
\end{enumerate}

\item A Quadrant II organizer will need to meet six important criteria.
\begin{itemize}
\item \emph{\textbf{Coherence}}: Coherence suggests that there is harmony, unity, and \emph{integrity} between your \emph{vision} and \emph{mission}, your \emph{roles} and \emph{goals}, your \emph{priorities} and \emph{plans}, and your desires and discipline.
\item \emph{\textbf{Balance}}: Your tool should help you to keep balance in your life, to identify your various roles and keep them right in front of you, so that you \emph{don't neglect important areas} such as your health, your family, professional preparation, or personal development.
\item \emph{\textbf{Quadrant II Focus}}: You need a tool that encourages you, motivates you, actually helps you spend the time you need in Quadrant II, so that you're dealing with \emph{\textbf{prevention}} \emph{rather than} \emph{\textbf{prioritizing}} \emph{\textbf{crises}}.

Organizing on a \emph{\textbf{weekly basis}} provides much greater balance and context than daily planning. The key is \emph{not to prioritize what's on your schedule}, but to \emph{\textbf{schedule your priorities}}. 

\item \emph{\textbf{A ``People" Dimension}}: You also need a tool that \emph{deals with \textbf{people}}, not just schedules. While you can think in terms of efficiency in dealing with time, a principle-centered person thinks in terms of \emph{effectiveness} in dealing with people. Your tool needs to reflect that value, to facilitate implementation rather than create guilt when a schedule is not followed.
\item \emph{\textbf{Flexibility}}: Your planning tool should be your servant, never your master. 
\item \emph{\textbf{Portability}}: Your tool should also be portable, so that you can carry it with you most of the time.
\end{itemize}

\end{itemize}
\subsubsection{Becoming A Quadrant II Self-Manager}
\begin{itemize}
\item Quadrant II organizing involves four key activities:
\begin{enumerate}
\item \textbf{\emph{Identifying Roles}}: The first task is to write down your \emph{\textbf{key roles}}. 

You don't need to worry about defining the roles in a way that you will live with for the rest of your life -- just consider the week and write down the areas you see yourself spending time in during the next seven days.

\item \textbf{\emph{Selecting Goals}}: The next step is to think of two or three important results you feel you should accomplish in each role during the next seven days. These would be recorded as \emph{\textbf{goals}}.

At least some of these goals should reflect Quadrant II activities. Ideally, these short-term goals
would be tied to the longer-term goals you have identified in conjunction with your personal mission
statement.

\item \textbf{\emph{Scheduling}}: Now you look at the week ahead \emph{with your goals in mind} and schedule time to achieve them.

Having identified roles and set goals, you can \emph{\textbf{translate} each goal to a specific day} of the week, either as a \emph{\textbf{priority item}} or, even better, as a \emph{specific appointment}. 

You can also check your annual or monthly calendar for any appointments you may have previously made and evaluate their importance in the context of your goals, transferring those you decide to keep to your schedule and making plans to \emph{\textbf{reschedule}} or \emph{\textbf{cancel}} others.

\item \textbf{\emph{Daily Adapting}}: With Quadrant II weekly organizing, daily planning becomes more a function of \emph{daily adapting}, or prioritizing activities and responding to unanticipated events, relationships, and experiences in a meaningful way.

Taking a few minutes each morning to review your schedule can put you in touch with the \emph{value-based decisions} you made as you organized the week as well as unanticipated factors that may have come up.  

As you overview the day, you can see that your \emph{roles} and \emph{goals} provide a \emph{natural prioritization} that grows out of your innate sense of balance. 
\end{enumerate}
\end{itemize}
\subsubsection{Living It}
\begin{itemize}
\item Living it is primarily a function of our independent will, our \emph{self-discipline}, our \emph{integrity}, and \emph{commitment} -- not to short-term goals and schedules or to the impulse of the moment, but to the \emph{correct principles} and our own deepest values, which give meaning and context to our goals, our schedules, and our lives.

\item The popularity of reacting to the urgent but unimportant priorities of other people in Quadrant III or the pleasure of escaping to Quadrant IV will threaten to overpower the important Quadrant II activities you have planned. 

\item Your \emph{principle center}, your \emph{self-awareness}, and your \emph{conscience} can provide a high degree of \emph{intrinsic security}, \emph{guidance}, and \emph{wisdom} to empower you to use your independent will and maintain \emph{\textbf{integrity}} to the \emph{\textbf{truly important}}.

\item Again, you simply can't think efficiency \emph{with people}. You think \emph{\textbf{effectiveness} with \textbf{people}} and \emph{\textbf{efficiency} with \textbf{things}}.
\end{itemize}
\subsubsection{Advances of the Fourth Generation}
\begin{itemize}
\item The efficiency paradigm of the third generation of management is out of harmony with the principle that \emph{\textbf{people are more important than things}}.
\item The fourth-generation tool recognizes that the first person you need to consider in terms of \emph{effectiveness} rather than \emph{efficiency} is \emph{\textbf{yourself}}.
\item The fourth generation of self-management is more advanced than the third in five important ways.The fourth generation of self-management is more advanced than the third in five important ways.
\begin{enumerate}
\item First, it's \emph{\textbf{principle-centered}}.
\item Second, it's \emph{\textbf{conscience-directed}}.
\item Third, it defines your unique \emph{\textbf{mission}}, including values and long-term goals. 
\item Fourth, it helps you \emph{\textbf{balance}} your life by \emph{\textbf{identifying roles}}, and by \emph{\textbf{setting goals}} and \emph{scheduling}
activities \emph{in each key role} every week.
\item And fifth, it gives \emph{greater \textbf{context}} through weekly organizing, rising above the limiting perspective of a single day and putting you in touch with your deepest values through review of your key roles.
\end{enumerate}
\end{itemize}
\subsubsection{Delegation: Increasing P and PC}
\begin{itemize}
\item We accomplish all that we do through \textbf{delegation} -- either to time or to other people. If we \emph{\textbf{delegate}} to \emph{time}, we think \emph{efficiency}. If we \emph{delegate to \textbf{other people}}, we think \emph{\textbf{effectiveness}}.

\item Effectively delegating to others is perhaps the single most powerful high-leverage activity there is.

\item \emph{\textbf{Transferring responsibility}} to other skilled and trained people enables you to give your energies \emph{to other high-leverage activities}. \emph{\textbf{Delegation means growth}}, both for individuals and for organizations.

\item Because delegation involves other people, it is a \emph{Public Victory} and could well be included in Habit 4. But because we are focusing here on \emph{principles of personal management}, and \emph{\textbf{the ability to delegate to others}} is the main difference between the role of \emph{\textbf{manager}} and \emph{independent producer}, I am approaching delegation from the standpoint of your personal managerial skills.

\item \emph{\textbf{Management}} is essentially moving the fulcrum over, and the \emph{key} to \emph{\textbf{effective management}} is \textbf{\emph{delegation}}.
\end{itemize}
\subsubsection{Gofer Delegation}
\begin{itemize}
\item There are basically two kinds of delegation: ``\emph{gofer delegation}" and ``\emph{stewardship delegation}."
\begin{itemize}
\item \emph{\textbf{Gofer delegation}} means ``Go for this, go for that, do this, do that, and tell me when it's done."  If they are given a position of supervision or management, they still think like producers. Because they are focused on \emph{methods}, they become responsible for the \emph{results}.
\end{itemize}
\end{itemize}
\subsubsection{Stewardship Delegation}
\begin{itemize}
\item \begin{itemize}
\item \emph{\textbf{Stewardship delegation}} is focused on \emph{\textbf{results}} instead of \emph{\textbf{methods}}. It gives people a choice of method and \emph{makes them responsible for results}. It takes more time in the beginning, but it's time well invested. You can move the fulcrum over, you can increase your leverage, through stewardship delegation.
\end{itemize}

\item Stewardship delegation involves clear, up-front mutual understanding and commitment regarding \emph{expectations} in five areas.
\begin{itemize}
\item \emph{\textbf{Desired Results}}: Create a clear, mutual understanding of what needs to be accomplished, focusing on what, not how; results, not methods.
\item \emph{\textbf{Guidelines}}: Identify the parameters within which the individual should operate. These should be as few as possible to avoid methods delegation, but should include any formidable restrictions. 

If you know the failure paths of the job, identify them. Be honest and open

\item \emph{\textbf{Resources}}: Identify the human, financial, technical, or organizational resources the person can draw on to accomplish the desired results.
\item \emph{\textbf{Accountability}}: Set up the standards of performance that will be used in evaluating the results and the specific times when reporting and evaluation will take place.
\item \emph{\textbf{Consequences}}: Specify what will happen, both good and bad, as a result of the evaluation.
\end{itemize}

\item \emph{\textbf{Trust}} is the highest form of \emph{\textbf{human motivation}}. It brings out the very best in people. But it takes time and \emph{patience}, and it doesn't preclude the necessity to \emph{train} and \emph{develop} people so that their competency can rise to the level of that trust.

\item I am convinced that if \emph{stewardship delegation} is done \emph{correctly}, both parties will benefit and ultimately much more work will get done in much less time.
\end{itemize}
\subsubsection{The Quadrant II Paradigm}
\begin{itemize}
\item The key to \emph{effective management} of \emph{\textbf{self}}, or of others \emph{through \textbf{delegation}}, is not in any technique or tool or extrinsic factor. It is intrinsic -- in the Quadrant II paradigm that empowers you to see through \emph{the lens of \textbf{importance} rather than \textbf{urgency}}.
\end{itemize}
\section{Public Victory}
Before moving into the area of Public Victory, we should remember that \emph{effective interdependence can only be built on a foundation of \textbf{true independence}}. \emph{\textbf{Private Victory} precedes \textbf{Public Victory}}.
\subsection{Paradigms of Interdependence}
\subsubsection{The Emotional Bank Account}
\begin{itemize}
\item An \emph{\textbf{Emotional Bank Account}} is a metaphor that describes the amount of \emph{trust} that's been \emph{built up in a relationship}. It's the feeling of \emph{safeness} you have with another human being.

\item If I make \textbf{\emph{deposits}} into an \emph{Emotional Bank Account} with you through \emph{courtesy, kindness, honesty}, and \emph{keeping my commitments} to you, I build up a \emph{\textbf{reserve}}. 

Your trust toward me becomes higher, and I can call upon that trust many times if I need to. I can even make mistakes and that trust level, that emotional reserve, will \emph{\textbf{compensate}} for it. 

\item Remember that quick fix is a mirage. Building and repairing relationships takes time. 
\end{itemize}
\subsubsection{Six Major Deposits}
\begin{itemize}
\item Let me suggest six major deposits that build the Emotional Bank Account:
\begin{itemize}
\item \emph{\textbf{Understanding the Individual}}: \emph{\textbf{Seeking to understand}} another person is probably one of the most important deposits you can make, and it is the key to every other deposit. You simply don't know what constitutes a deposit to another person until you understand that individual.

To make a deposit, what is important to another person must be as important to you as the other person is to you. 

\item \emph{\textbf{Attending to the Little Things}}: The little kindnesses and courtesies are so important. Small discourtesies, little unkindnesses, little forms of disrespect make large withdrawals. \emph{In \textbf{relationships}, \textbf{the little things} are the \textbf{big} things}.

People are very tender, very sensitive inside.

\item \emph{\textbf{Keeping Commitments}}: \emph{Keeping a commitment or a promise} is a \emph{\textbf{major deposit}}; \emph{\textbf{breaking} one} is a \emph{major \textbf{withdrawal}}. 

\item \emph{\textbf{Clarifying Expectations}}: The deposit is to \emph{make the \textbf{expectations} clear} and \emph{\textbf{explicit}} in the beginning.


\item  \emph{\textbf{Showing Personal Integrity}}: \emph{Personal integrity} generates trust and is the basis of many different kinds of deposits.

\item \emph{\textbf{Apologizing Sincerely When You Make a Withdrawal}}: When we make withdrawals from the Emotional Bank Account, we need to apologize and we need to do it sincerely. Great deposits come in the sincere words.
\end{itemize}
\end{itemize}
\subsubsection{Attending to the Little Things}
\begin{itemize}
\item The little kindnesses and courtesies are so important. Small discourtesies, little unkindnesses, little forms of disrespect make large withdrawals.

\item  \emph{In \textbf{relationships}, \textbf{the little things} are the \textbf{big} things}.

\item People are very tender, very sensitive inside.
\end{itemize}
\subsubsection{Keeping Commitments}
\begin{itemize}
\item \emph{Keeping a commitment or a promise} is a \emph{\textbf{major deposit}}; \emph{\textbf{breaking} one} is a \emph{major \textbf{withdrawal}}. 

\item In fact, there's probably not a more \emph{massive withdrawal} than to make a promise that's important to someone and then not to come through. The next time a promise is made, they \emph{won't believe} it.
\end{itemize}
\subsubsection{Clarifying Expectations}
\begin{itemize}
\item Many \emph{\textbf{expectations}} are \emph{\textbf{implicit}}. They haven't been explicitly stated or announced, but people
nevertheless bring them to a particular situation. Although these expectations have not been discussed, or sometimes even recognized by the person who has them, \emph{fulfilling them} makes \emph{great deposits} in the \emph{relationship} and \emph{violating} them makes \emph{withdrawals}.

\item That's why it's so important whenever you come into a \emph{\textbf{new situation}} to \emph{get all the expectations out on the table}. People will begin to \emph{judge} each other through those expectations. 

\item We create many \emph{\textbf{negative situations}} by simply assuming that \emph{our expectations} are \emph{\textbf{self-evident}} and that they are clearly understood and shared by other people.

\item The deposit is to \emph{make the \textbf{expectations} clear} and \emph{\textbf{explicit}} in the beginning.

\end{itemize}
\subsubsection{Showing Personal Integrity}
\begin{itemize}
\item \emph{Personal integrity} generates trust and is the basis of many different kinds of deposits.

\item \emph{\textbf{Integrity}} includes but goes beyond \emph{\textbf{honesty}}. Honesty is \emph{\textbf{telling the truth}} -- in other words, conforming our words to reality. \emph{Integrity} is \emph{\textbf{conforming reality to our words}} -- in other words, keeping promises and fulfilling expectations. This requires an integrated character, a oneness, primarily with self but also with life.

\item One of the most important ways to \emph{manifest integrity} is to \emph{be \textbf{loyal} to those who are \textbf{not present}}. In doing so, we build the trust of those who are present. When you defend those who are absent, you retain the trust of those present.

\item  Integrity in an interdependent reality is simply this: you treat everyone by the same set of principles. As you do, people will come to trust you.
\end{itemize}
\subsubsection{Apologizing Sincerely When You Make a Withdrawal}
\begin{itemize}
\item When we make withdrawals from the Emotional Bank Account, we need to \emph{\textbf{apologize}} and we need to do it \emph{\textbf{sincerely}}. Great deposits come in the sincere words.

\item It takes a great deal of \emph{\textbf{character strength}} to \emph{apologize} quickly out of one's heart rather than out of pity. A person must possess himself and have a deep sense of \emph{security in fundamental principles and values} in order to genuinely apologize.

People with little internal security can't do it. It makes them too \emph{vulnerable}. 

\item Sincere apologies make deposits; \emph{\textbf{repeated apologies}} interpreted as \emph{insincere} make \emph{withdrawals}. And the quality of the relationship reflects it.

\item It is one thing to make a \emph{mistake}, and quite another thing \emph{\textbf{not to admit}} it. 
\end{itemize}
\subsubsection{The Laws of Love and the Laws of Life}
\begin{itemize}
\item When we make deposits of \emph{\textbf{unconditional love}}, when we live the primary laws of love, we encourage others to live the primary laws of life.

\item In other words, when we truly love others without condition, without strings, we help them feel \emph{\textbf{secure} and safe and \textbf{validated} and affirmed in their \textbf{essential worth}}, \emph{identity}, and \emph{integrity}.

\item Their natural growth process is encouraged. We make it easier for them to live \emph{\textbf{the laws of life}} -- \emph{\textbf{cooperation}, \textbf{contribution}, \textbf{self-discipline}, \textbf{integrity}} -- and to \emph{discover} and live \emph{true to the highest and best within them}.

\item When we violate the primary \emph{laws of love} -- when we attach strings and conditions to that gift -- we actually encourage others to \emph{violate} the primary \emph{laws of life}.

We put them in a \emph{reactive}, \emph{defensive} position where they feel they have to prove ``I matter as a person, independent of you."
\end{itemize}
\subsubsection{P Problems are PC Opportunities}
\begin{itemize}
\item This experience also taught me another powerful paradigm of \emph{\textbf{interdependence}}. 

\item The very \emph{\textbf{problem}} created the \emph{\textbf{opportunity}} to build a \emph{deep relationship} that empowered us to work together as a \emph{strong} \emph{complementary} team.

\item By recognizing that \emph{\textbf{the P/PC Balance}} is necessary to \emph{\textbf{effectiveness}} in an \emph{interdependent reality}, we can \emph{value} our \emph{problems} as \emph{opportunities} to \emph{increase PC}.
\end{itemize}

\subsection{Habit 4: Think Win/Win -- Principles of Interpersonal Leadership}
\subsubsection{Six Paradigms of Human Interaction}
\begin{itemize}
\item Win-win is not a technique; it's a total philosophy of human interaction. 

\item There are six paradigms of human interaction:
\begin{itemize}
\item \textbf{Win/Win}. Win-win is a frame of mind and heart that constantly seeks mutual benefit in all human interactions. Win-win means that agreements or solutions are mutually beneficial, mutually satisfying. 

Win/Win sees life as a \emph{\textbf{cooperative}}, not a competitive arena. 

Win/Win is a belief in the \emph{Third Alternative}. It's not your way or my way; it's a better way, a higher way.

\item \textbf{Win/Lose}. In leadership style, Win/Lose is the \emph{\textbf{authoritarian}} approach. Most people have been deeply scripted in the Win/Lose mentality since birth.

It's based on \emph{\textbf{power} and \textbf{position}} rather than on \emph{principle}.

Certainly there is a place for Win/Lose thinking in truly competitive and \emph{low-trust situations}. But most of life is not a \emph{competition}.

\item \textbf{Lose/Win}. Lose/Win is \emph{\textbf{worse}} than Win/Lose because it has \emph{\textbf{no standards}} -- \emph{no demands, no expectations, no vision}. 

People who think Lose/Win are usually quick to please or appease. They seek strength from \emph{\textbf{popularity}} or \emph{\textbf{acceptance}}. They have \emph{\textbf{little courage}} to express their own feelings and convictions and are easily \emph{intimidated} by the ego strength of others.

In negotiation, \emph{Lose/Win is seen as \textbf{capitulation}} -- \emph{\textbf{giving in} or \textbf{giving up}}. In leadership style, it's \emph{\textbf{permissiveness}} or \emph{\textbf{indulgence}}. Lose/Win means being \emph{\textbf{a nice guy}}, even if ``nice guys finish last".

\item \textbf{Lose/Lose}. When two \emph{Win/Lose} people get together -- that is, when two determined, stubborn, ego-invested individuals interact -- the result will be \emph{Lose/Lose}. \emph{\textbf{Both} will lose}.

\emph{Lose/Lose} is also the philosophy of the \emph{\textbf{highly dependent}} person \emph{without inner direction} who is \emph{miserable} and thinks everyone else should be, too. ``If nobody ever wins, perhaps being a loser isn't so bad".

\item \textbf{Win}. People with the \emph{win mentality} don't necessarily want someone else to lose. That's irrelevant. 

When there is \emph{\textbf{no} sense of contest or \textbf{competition}}, win is probably the most common approach in everyday \emph{negotiation}. A person with the win mentality thinks in terms of securing his own ends -- and leaving it to others to secure theirs.

\item \textbf{Win/Win or No Deal}.
\end{itemize}

\item The best choice, then, depends on \emph{\textbf{reality}}. The challenge is to read that reality accurately and not to translate win/lose or other scripting into every situation.

\item  Most situations, in fact, are part of an \emph{\textbf{interdependent reality}}, and then win/win is really the only viable alternative of the five.
\begin{itemize}
\item Win/lose is not viable because, although I appear to win in a confrontation with you, your feelings, your attitudes toward me and our relationship have been affected.
\item If we come up with a lose/win, you may appear to get what you want for the moment. You may carry battle scars with me into any future negotiations.
\item if I focus on my own win and don't even consider your point of view, there's no basis for any kind of productive relationship.
\end{itemize}
\end{itemize}
\subsubsection{Win/Win or No Deal}
\begin{itemize}
\item If these individuals had not come up with a \emph{\textbf{synergistic} solution} -- one that was \emph{agreeable to both} -- they could have gone for an even higher expression of win/win, \emph{\textbf{Win/Win or No Deal}}.

\item \emph{\textbf{No deal}} basically means that if we can't find a solution that would \emph{\textbf{benefit us both}}, we \emph{\textbf{agree to disagree} agreeably} -- no deal. No expectations have been created, no performance contracts established. 

\item Anything less than win-win in an interdependent reality is a \emph{poor second best} that will have impact in the long-term relationship. The cost of the impact needs to be carefully considered.



\item \emph{Win-Win or No Deal} provides tremendous \emph{emotional freedom} in the \emph{\textbf{family relationship}}. If family members can't agree on a video that everyone will enjoy, they can simply decide to do something else -- no deal -- rather than having some enjoy the evening at the expense of others.

\item The \emph{Win-Win or No Deal} approach is most \emph{realistic} at the \emph{beginning} of a \emph{\textbf{business relationship} or enterprise}. In a continuing business relationship, no deal may not be a viable option, which can create serious problems, especially for family businesses or businesses that are begun initially on the basis of friendship.

\item Without no deal, many such businesses simply \emph{deteriorate} and either fail or have to be turned over to professional managers. 
\end{itemize}
\subsubsection{Five Dimensions of Win/Win}
\begin{itemize}
\item Think Win/Win is the \emph{\textbf{habit}} of \emph{\textbf{interpersonal leadership}}.

\item Effective interpersonal leadership requires the \emph{\textbf{vision}}, the \emph{\textbf{proactive initiative}}, and the \emph{security, guidance, wisdom, and power} that come from \textit{\textbf{principle-centered personal leadership}}.

\item The principle of win-win is fundamental to success in all our interactions, and it embraces \emph{\textbf{five interdependent dimensions}} of life. It begins with \emph{\textbf{character}} and moves toward \emph{\textbf{relationships}}, out of which flow \emph{\textbf{agreements}}. It is nurtured in an environment where \emph{structure and \textbf{systems}} are based on win/win. And it involves \emph{\textbf{process}}; we cannot achieve win/win ends with win/lose or lose/win means.
\end{itemize}
\subsubsection{Charater}
\begin{itemize}
\item \emph{\textbf{Character}} is the \emph{foundation} of win/win, and everything else builds on that foundation. There are \emph{three} character traits essential to the win-win paradigm.
\begin{itemize}
\item \textbf{\emph{Integrity}}: We've already defined integrity as the value we place on ourselves. Habits 1, 2, and 3 help us develop and maintain integrity. 

As we clearly identify our values and proactively organize and execute around those values on a daily basis, we develop \emph{self-awareness} and \emph{independent will} by making and \emph{keeping} meaningful \emph{promises} and \emph{commitments}.

There's no way to go for a win in our own lives if we don't even know, in a deep sense, \emph{\textbf{what constitutes a win}} -- what is, in fact, harmonious with our \emph{innermost values}. 

\item \textbf{\emph{Maturity}}: Maturity is the \emph{\textbf{balance}} between \emph{\textbf{courage}} and \emph{\textbf{consideration}}. If a person can express his feelings and convictions with \emph{courage balanced with consideration} for the feelings and convictions of another person, he is \emph{mature}, particularly if the issue is very important to both parties.

It is a deep embodiment of \emph{the P/PC Balance}. Whether it's called the \emph{\textbf{ego strength/empathy balance}}, the \emph{\textbf{self confidence/respect for others balance}}, the \emph{\textbf{concern for people/concern for tasks balance}}, the quality sought for is the balance of what I call courage and consideration.

The basic task of \emph{leadership} is to increase the standard of living and the quality of life for all stakeholders.

To achieve that balance between courage and consideration, is the essence of real maturity and is \emph{fundamental} to win/win.

\item \textbf{\emph{Abundance Mentality}}: The third character trait essential to win/win is the \emph{Abundance Mentality}, the paradigm that \emph{\textbf{there is plenty out there for everybody}}. \emph{The \textbf{Scarcity Mentality}} is the \emph{\textbf{zero-sum paradigm}} of life.

People with a Scarcity Mentality have a very difficult time sharing recognition and credit, power or profit -- even with those who help in the production. It's almost as if something is being taken from them when someone else receives special recognition or windfall gain or has remarkable success or achievement. 

Often, people with a \emph{Scarcity Mentality} harbor secret \emph{hopes that others might suffer misfortune} -- not terrible misfortune, but acceptable misfortune that would keep them ``in their place."  They're always \emph{\textbf{comparing}}, always \emph{\textbf{competing}}. They give their energies to possessing things or other people in order to increase their sense of worth.

\emph{The Abundance Mentality}, on the other hand, flows out of a \emph{deep inner sense} of \emph{\textbf{personal worth}} and \emph{\textbf{security}}. It is the paradigm that there is plenty out there and enough to \emph{spare for everybody}. It results in \emph{\textbf{sharing}} \emph{of prestige, of recognition, of profits, of decision making}. It opens \emph{possibilities}, \emph{options}, \emph{alternatives}, and \emph{creativity}. 

The Abundance Mentality takes the personal joy, satisfaction, and fulfillment of Habits 1, 2, and 3 and turns it \emph{outward}, \emph{appreciating the uniqueness}, the \emph{inner direction}, the \emph{proactive nature} of others.

\emph{\textbf{Public Victory}} does not mean victory over other people. It means success in \emph{\textbf{effective interaction}} that brings \emph{\textbf{mutually beneficial}} results to everyone involved. Public Victory means working together, communicating together, making things happen together that even the same people couldn't make happen by working independently. And \emph{Public Victory} is an \emph{\textbf{outgrowth}} of the \emph{Abundance Mentality paradigm}.
\end{itemize}

\item A character rich in \emph{integrity}, \emph{maturity}, and the \emph{Abundance Mentality} has a genuineness that goes far beyond technique, or lack of it, in human interaction.
\end{itemize}
\subsubsection{Relationships}
\begin{itemize}
\item From the foundation of character, we build and maintain win-win relationships. The \emph{\textbf{trust}}, \emph{the Emotional Bank Account}, is \emph{the essence of win-win}. Without trust, the best we can do is compromise; without trust, we lack the credibility for open, mutual learning and communication and real creativity.

\item Enough deposits have been made so that you know and I know that we deeply respect each other. 

\item A relationship where \emph{bank accounts are high} and \emph{both parties} are deeply committed to win/win is the ideal springboard for \emph{tremendous \textbf{synergy}} (Habit 6). 

\item Dealing with \emph{win/lose} is the real \emph{test} of \emph{win/win}. Rarely is win/win easily achieved in any circumstance. Deep issues and fundamental differences have to be dealt with. But it is much easier when both parties are aware of and committed to it and where there is a high Emotional Bank Account in the relationship.

\item When you're dealing with a person who is coming from a paradigm of win/lose, the \emph{relationship} is still the key. The place to focus is on your \emph{Circle of Influence}. 

You make deposits into the Emotional Bank Account through genuine courtesy, respect, and appreciation for that person and for the other point of view. You stay longer in the communication process. You \emph{listen} more, you listen in greater depth. You express yourself with greater \emph{courage}. You aren't reactive. You go deeper inside yourself for \emph{strength} of character to be \emph{proactive}. You keep hammering it out until the other person begins to realize that you genuinely want the resolution to be a real win for both of you. That very process is a tremendous deposit in the Emotional Bank Account.

\item It goes beyond \emph{transactional leadership} into \emph{\textbf{transformational leadership}}, transforming the individuals involved as well as the relationship.

\item Remember that no deal is always an option. Or you may occasionally choose to go for the low form of win-win -- compromise.

\item It's important to realize that not all decisions need to be win-win, even when the Emotional Bank Account is high.

\item  An agreement means very little in letter without the \emph{character} and \emph{relationship base} to sustain it in spirit. So we need to approach win/win from a genuine desire to invest in the relationships that make it possible.
\end{itemize}
\subsubsection{Agreements}
\begin{itemize}
\item From relationships flow the \emph{\textbf{agreements}} that give definition and direction to win/win. They are sometimes called \emph{\textbf{performance agreements}} or \emph{\textbf{partnership agreements}}, or shifting the \emph{paradigm of productive interaction} from \emph{vertical} to \emph{\textbf{horizontal}}, from \emph{hovering supervision} to \emph{\textbf{self-supervision}}, from \emph{positioning} to \emph{\textbf{being partners}} in success.

\item Win/Win Agreements cover a wide scope of \emph{interdependent interaction}. 

\item In Win/Win agreement, the following five elements are made very explicit:
\begin{itemize}
\item \emph{\textbf{Desired results}} (not methods) identify what is to be done and when.
\item \emph{\textbf{Guidelines}} specify the parameters (principles, policies, etc.) within which results are to be accomplished.
\item \emph{\textbf{Resources}} identify the human, financial, technical, or organizational support available to help accomplish the results.
\item \emph{\textbf{Accountability}} sets up the standards of performance and the time of evaluation.
\item \emph{\textbf{Consequences}} specify -- good and bad, natural and logical -- what does and will happen as a result of the evaluation.
\end{itemize}
These five elements give Win-Win Agreements a life of their own. A clear \emph{mutual understanding} and \emph{agreement} \emph{\textbf{up front}} in these areas creates a standard against which people can measure their own success.
\end{itemize}
\subsubsection{Win/Win Performance Agreements}
\begin{itemize}
\item Creating Win-Win Performance Agreements requires vital \emph{Paradigm Shifts}. The focus is on \emph{\textbf{results}}; not \emph{methods}. Most of us tend to \emph{supervise methods}. But \emph{\textbf{Win-Win Agreements} focus on \textbf{results}}, releasing tremendous individual human potential and \emph{creating greater synergy}, \emph{building PC in the process} instead of \emph{focusing exclusively on P}.

\item With win-win accountability, people evaluate themselves. 

\item Developing such a Win-Win Agreement is the central activity of \emph{\textbf{management}}. With an agreement in place, employees can manage themselves within the framework of that agreement. 

\item There are basically four kinds of consequences (rewards and penalties) that management or parents can control -- \emph{\textbf{financial}, \textbf{psychic}, \textbf{opportunity}, and \textbf{responsibility}}. 
\begin{itemize}
\item \emph{\textbf{Financial consequences}} include such things as income, stock options, allowances, or penalties. 
\item \emph{\textbf{Psychic}} or \emph{\textbf{psychological consequences}} include recognition, approval, respect, credibility, or the loss of them. Unless people are in a survival mode, psychic compensation is often more motivating than financial compensation. 
\item \emph{\textbf{Opportunity}} includes training, development, perks, and other benefits. 
\item \emph{\textbf{Responsibility}} has to do with scope and authority, either of which can be enlarged or diminished.
\end{itemize}
\end{itemize}
\subsubsection{Systems}
\begin{itemize}
\item Win/win can only survive in an organization when \emph{the \textbf{systems support} it}. If you talk win/win but reward win/lose, you've got a losing program on your hands.

\item You basically get what you reward. If you want to achieve the goals and reflect the values in your mission statement, then you need to \emph{\textbf{align} the \textbf{reward system} with these \textbf{goals} and \textbf{values}}.

\item \emph{Competition has its place in the marketplace} or against last year's performance -- perhaps even against another office or individual where there is no particular interdependence, no need to cooperate.

But \emph{cooperation in the workplace} is as important to free enterprise as competition in the marketplace. \emph{The spirit of win/win cannot survive in an \textbf{environment of competition and contests}}.

\item For win/win to work, the systems have to support it. The training system, the planning system, the communication system, the budgeting system, the information system, the compensation system -- all have to be based on the principle of win/win.

\item So often the problem is in the \textbf{\emph{system}}, not in the people. If you put good people in \emph{bad systems}, you get \emph{bad results}. You have to water the flowers you want to grow.

\item As people really learn to Think Win/Win, they can set up the systems to \emph{create and reinforce} it. They can transform \emph{\textbf{unnecessarily competitive situations}} to \emph{\textbf{cooperative}} ones and can powerfully impact their effectiveness by building both P and PC.
\end{itemize}
\subsubsection{Processes}
\begin{itemize}
\item There's no way to achieve win-win ends with win/lose or lose/win means. So the question becomes how to arrive at a win/win solution.

\item ``principled" approach versus the ``positional" approach to bargaining in their tremendously useful and insightful book.

\item The \emph{essence} of \emph{\textbf{principled negotiation}} is
\begin{itemize}
\item to \emph{\textbf{separate the person from the problem}},
\item to \emph{\textbf{focus on interests and not on positions}}, 
\item to \emph{\textbf{invent options for mutual gain}}, 
\item and to \emph{\textbf{insist on objective criteria}}
\end{itemize}
 --  some external standard or principle that both parties can buy into.
 
 \item I suggest that they become involved in the following four-step process:
 \begin{itemize}
 \item First, see the problem \emph{\textbf{from the other point of view}}. Really seek to understand and give expression to the needs and concerns of the other party as well as or better than they can themselves. 
 \item Second, \emph{identify \textbf{the key issues}} and concerns (not positions) involved. 
 \item Third, determine what \emph{results} would constitute a \emph{\textbf{fully acceptable solution}}.
 \item And fourth, identify possible \emph{\textbf{new options}} to achieve those results.
 \end{itemize}
 
 \item You can only achieve \emph{Win/Win solutions} with \emph{Win/Win processes} -- \emph{the \textbf{end} and the \textbf{means} are the \textbf{same}}.
\end{itemize}
\subsection{Habit 5: Seek First to Understand, Then to Be Understood -- Principles of Empathic Communication}
We have such a tendency to rush in, to fix things up with good advice. But we often fail to take the time to diagnose, to really, deeply understand the problem first.

If I were to summarize in one sentence the single most important principle I have learned in the field of interpersonal relations, it would be this: \emph{\textbf{Seek First to Understand, Then to Be Understood}}. This principle is the key to \emph{\textbf{effective interpersonal communication}}.
\subsubsection{Charater and Communication}
\begin{itemize}
\item Communication is the most important skill in life. 

\item But what about \emph{\textbf{listening}}? 

\item If you want to \emph{\textbf{interact} effectively} with me, to \emph{\textbf{influence}} me -- your spouse, your child, your neighbor, your boss, your coworker, your friend -- you first need to understand me. And you can't do that with technique alone. If I sense you're using some technique, I sense duplicity, manipulation.

\item The real key to your \emph{influence} with me is your example, your actual \emph{\textbf{conduct}}. Your example flows naturally \emph{out of your character}, of the kind of person you truly are -- not what others say you are or what you may want me to think you are.

\item Your \emph{\textbf{character}} is constantly \emph{radiating}, \emph{communicating}. From it, in the long run, I come to instinctively trust or distrust you and your efforts with me.

\item But unless I open up with you, unless you understand me and my unique situation and feelings, you won't know how to advise or counsel me.

\item Unless \emph{you're influenced by} my \emph{\textbf{uniqueness}}, I'm not going to be influenced by your advice. So if you want to be really effective in the habit of interpersonal communication, you cannot do it with technique alone. You have to build the skills of empathic listening on a base of character that inspires openness and trust. And you have to build the Emotional Bank Accounts that create a commerce between hearts.
\end{itemize}
\subsubsection{Empathic Listening}
\begin{itemize}
\item ``\emph{\textbf{Seek first to understand}}" involves a very deep shift in paradigm. We typically \emph{seek first to be understood}. 

\item Most people do not listen with the intent to understand; \emph{they listen with \textbf{the intent to reply}}. They're either \emph{speaking} or \emph{preparing to speak}. They're \emph{filtering} everything through \emph{their own paradigms}, reading their \emph{\textbf{autobiography}} into other people's lives.

\item We're filled with \emph{our own rightness}, \emph{our own autobiography}. We want to be understood. Our conversations become \emph{\textbf{collective monologues}}, and we never really understand what's going on inside another human being.

\item When another person speaks, we're usually ``\emph{listening}" at one of four levels. 
\begin{itemize}
\item We may be \emph{\textbf{ignoring}} another person, not really listening at all. 
\item We may practice \emph{\textbf{pretending}}. 
\item We may practice \emph{\textbf{selective listening}}, hearing only certain parts.
\item Or we may even practice \emph{\textbf{attentive listening}}, paying attention and focusing energy on the words that are being said.
\end{itemize} But very few of us ever practice the fifth level, the highest form of listening, \emph{\textbf{empathic listening}}.

\item When I say empathic listening, I am not referring to the techniques of ``\emph{active}" listening or ``\emph{reflective}" listening, which basically involve \emph{mimicking} what another person says. That kind of listening is skill-based, truncated from character and relationship, and often \emph{\textbf{insults}} those ``listened" to in such a way. 

\item It is also essentially \emph{\textbf{autobiographical}}. If you practice those techniques, you may not project your autobiography in the actual interaction, but your \emph{\textbf{motive in listening}} is \emph{autobiographical}.

\item You listen with \emph{reflective} skills, but you listen with \emph{intent to reply, to control, to manipulate}.

\item When I say empathic listening, I mean \emph{listening with \textbf{intent to understand}}. I mean seeking \emph{first to understand}, to really understand. It's an entirely different paradigm.

\item \emph{\textbf{Empathic} (from empathy) \textbf{listening}} gets \emph{inside another person's frame of reference}. 

\item \emph{\textbf{Empathy} is \textbf{not sympathy}}. \emph{\textbf{Sympathy}} is a form of \emph{agreement}, a form of \emph{judgment}. It makes them \emph{dependent}. The essence of empathic listening is not that you agree with someone; it's that you \emph{fully, deeply, \textbf{understand}} that person, emotionally as well as intellectually.

\item \emph{Empathic listening} involves much more than \emph{\textbf{registering}}, \emph{\textbf{reflecting}}, or even \emph{\textbf{understanding}} the words that are said.

Communications experts estimate, in fact, that only 10 percent of our communication is represented by the words we say. Another 30 percent is represented by our sounds, and 60 percent by our \emph{\textbf{body language}}. In empathic listening, you listen with your \emph{ears}, but you also, and more importantly, listen with your \emph{eyes} and with your \emph{heart}.

\item \emph{Empathic listening} is so powerful because it gives you \emph{\textbf{accurate data}} to work with. You're dealing with the reality inside another person's head and heart. You're \emph{listening to understand}. You're focused on receiving the deep communication of another human soul.

\item In addition, \emph{\textbf{empathic listening}} is the key to \emph{\textbf{making deposits} in Emotional Bank Accounts}, because nothing you do is a deposit unless the other person \emph{\textbf{perceives}} it as such. 

It's deeply \emph{\textbf{therapeutic}} and \emph{healing} because it gives a person ``psychological air".

\item This is one of the greatest insights in the field of human motivations: \emph{\textbf{Satisfied needs do not motivate}}. It's only the \emph{unsatisfied} need that motivates. Next to physical survival, the greatest need of a human being is \emph{\textbf{psychological survival}} -- to be understood, to be affirmed, to be validated, to be appreciated.

\item Seeking first to understand, \emph{\textbf{diagnosing before you prescribe}}, is hard. It's so much easier in the short run to hand someone a pair of glasses that have fit you so well these many years.

\item But in the long run, it severely depletes both P and PC. You can't achieve \emph{maximum interdependent production} from an inaccurate understanding of where other people are coming from. And you can't have \emph{interpersonal PC} -- high Emotional Bank Accounts -- if the people you relate with don't really feel understood.

\item Empathic listening is also \emph{\textbf{risky}}. It takes a great deal of security to go into a deep listening experience because you \emph{open yourself up to be influenced}. You become \emph{\textbf{vulnerable}}.

It's a paradox, in a sense, because in order to have \emph{influence}, you have \emph{to be influenced}. 
\end{itemize}
\subsubsection{Diagnose Before You Prescribe}
\begin{itemize}
\item Although it's risky and hard, \emph{seek first to understand}, or \emph{diagnose before you prescribe}, is a correct principle manifesting many areas of life. It's the mark of all true professionals.

\item If you don't have \emph{confidence} in the diagnosis, you won't have confidence in the prescription.

\item \emph{Diagnosing before you prescribe} is also fundamental to law. 

\item \emph{Seek first to understand} is a correct principle evident in all areas of life. It's a \emph{generic, common-denominator principle}, but it has its greatest power in the area of \emph{interpersonal relations}.
\end{itemize}
\subsubsection{Four Autobiographical Responses}
\begin{itemize}
\item Because we listen \emph{\textbf{autobiographically}}, we tend to respond in one of four ways. 
\begin{itemize}
\item We \emph{\textbf{evaluate}} -- we either agree or disagree; 
\item we \emph{\textbf{probe}} -- we ask questions from our own frame of reference;
\item we \emph{\textbf{advise}} -- we give counsel based on our own experience; 
\item or we \emph{\textbf{interpret}} -- we try to figure people out, to explain their motives, their behavior, based on our own motives and behavior.
\end{itemize}
But how do they affect our ability to really understand?

\item We are so deeply \emph{\textbf{scripted}} in these responses that we don't even realize when we use them.

\item To many, seek first to understand becomes the most exciting, the most immediately applicable, of all the Seven Habits.

\item Can you see how \emph{limited} we are when we try to understand another person on the basis of \emph{words alone}, especially when we're looking at that person through our own glasses?

\item You will never be able to truly \emph{step inside another person}, to see the world as he sees it, until you develop the \emph{\textbf{pure desire}}, \emph{the \textbf{strength} of \textbf{personal character}}, and \emph{the \textbf{positive} \textbf{Emotional Bank Account}}, as well as \emph{\textbf{the empathic listening skills}} to do it.

\item The \emph{skills}, the tip of the iceberg of empathic listening, involve four developmental stages
\begin{itemize}
\item The first and least effective is to \emph{\textbf{mimic content}}. Mimicking content is easy. You just listen to the words that come out of someone's mouth and you
repeat them. You're hardly even using your brain at all.
\item The second stage of empathic listening is to \emph{\textbf{rephrase} the content}. It's a little more effective, but it's still limited to the verbal communication.
\item The third stage brings your right brain into operation. You \emph{\textbf{reflect feeling}}.
\item The fourth stage includes \emph{\textbf{both}} the second and the third. You \emph{rephrase the content} and \emph{reflect the feeling}.
\end{itemize}

\item Now, what happens when you use fourth stage empathic listening skills is really incredible. As you authentically seek to understand, as you rephrase content and reflect feeling, you give him psychological air. You also help him work through his own thoughts and feelings. As he grows in his confidence of your sincere desire to really listen and understand, the barrier between what's going on inside him and what's actually being communicated to you disappears. It opens a soul-to-soul flow. He's not thinking and feeling one thing and communicating another. He begins to trust you with his innermost tender feelings and thoughts.

\item And we'll never get to the problem if we're so caught up in our own autobiography, our own paradigms, that we don't take off our glasses long enough to see the world from another point of view.

\item By seeking first to understand, this father has just turned a \emph{\textbf{transactional opportunity}} into a \emph{\textbf{transformational opportunity}}.

\item As long as the response is \emph{logical}, the father can effectively ask questions and give counsel. But the moment the response becomes \emph{emotional}, he needs to go back to \emph{empathic listening}.

\item There are times \emph{when \textbf{transformation requires no outside counsel}}. Often when people are really given the chance to open up, they unravel their own problems and the solutions become clear to them in the process.

\item At other times, they really need additional perspective and help. The key is to genuinely seek the welfare of the individual, to listen with empathy, to let the person get to the problem and the solution at his own pace and time.

\item In fact, sometimes words may just get in your way. That's one very important reason why technique alone will not work. That kind of understanding transcends technique. Isolated technique only gets in the way.

\item The skills will not be effective unless they come from a \emph{\textbf{sincere} desire to understand}. People resent any attempt to manipulate them.

\item \emph{\textbf{Affirming your motive}} is a huge deposit.

\item But if you're not sincere, I wouldn't even try it. It may create an openness and a vulnerability that will later turn to your harm when a person discovers that you really didn't care, you really didn't want to listen, and he's left open, exposed, and hurt.

\item \emph{Empathic listening \textbf{takes time}}, but it doesn't take anywhere near as much time as it takes to back up and correct misunderstandings when you're already miles down the road, to redo, to live with unexpressed and unsolved problems, to deal with the results of not giving people psychological air.

\item A \emph{discerning empathic listener} can read what's happening down deep fast, and can show such acceptance, such understanding, that other people feel safe to open up layer after layer until they get to that soft inner core where the problem really lies.

\item People want to be understood.
\end{itemize}
\subsubsection{Understanding and Perception}
\begin{itemize}
\item As you learn to listen deeply to other people, you will \emph{discover tremendous \textbf{differences}} in \emph{perception}. You will also begin to \emph{\textbf{appreciate}} the impact that these differences can have as people try to work together in interdependent situations.

\item Now, with all our differences, we're trying to work together -- in a marriage, in a job, in a community service project -- to manage resources and accomplish results. 

\item So how do we do it? The answer is Habit 5. It's \emph{the \textbf{first} step} in \emph{the process of \textbf{win-win}}. Even if (and especially when) the other person is not coming from that paradigm, \emph{\textbf{seek first to understand}}.
\end{itemize}
\subsubsection{Then Seek to Be Understood}
\begin{itemize}
\item \emph{Seek First to Understand, \textbf{Then to be Understood}}. 

\item \emph{\textbf{Knowing how to be understood}} is \emph{the other half of Habit 5}, and is equally critical in reaching win-win solutions.

\item \emph{Seeking to understand} requires \emph{consideration}; \emph{\textbf{seeking to be understood}} takes \emph{\textbf{courage}}. Win/win requires a high
degree of both.

\item \begin{itemize}
\item \emph{\textbf{Ethos}} is your \emph{\textbf{personal credibility}}, the faith people have in your integrity and competency. It's the
trust that you inspire, your Emotional Bank Account.
\item \emph{\textbf{Pathos}} is the empathic side -- it's the feeling. It means that you are in alignment with the emotional trust of another person's communication. 
\item \emph{\textbf{Logos}} is the logic, the reasoning part of the presentation.
\end{itemize}

\item Notice the sequence: ethos, pathos, logos -- your \emph{\textbf{character}}, and your \emph{\textbf{relationships}}, and then \emph{the \textbf{logic} of your presentation}.

\item When you can \emph{present your own ideas \textbf{clearly}}, specifically, \emph{\textbf{visually}}, and most important, \emph{\textbf{contextually}} -- in the context of a deep understanding of their paradigms and concerns -- you \emph{significantly increase the credibility} of your ideas.

\end{itemize}
\subsubsection{One on One}
\begin{itemize}
\item Habit 5 is powerful because it is right in the middle of your \emph{\textbf{Circle of Influence}}. Many factors in \emph{interdependent} situations are in your Circle of Concern -- \emph{problems}, \emph{disagreements}, \emph{circumstances}, other people's \emph{behavior}.

\item But you can always \emph{seek first to understand}. That's something that's \emph{\textbf{within your control}}.

And as you do that, as you focus on your \emph{Circle of Influence}, you really, deeply understand other people. You have accurate information to work with, you get to the heart of matters quickly, you build \emph{Emotional Bank Accounts}, and you give people the \emph{psychological air} they need so you can work together effectively.

\item It's the \emph{Inside-Out approach}. Because you really listen, you become \emph{\textbf{influenceable}}. And being influenceable is the key to influencing
others. Your circle begins to expand. You increase your ability to influence many of the things in your Circle of Concern.

\item The next time you communicate with anyone, you can put aside your own autobiography and genuinely seek to understand. Even when people don't want to open up about their problems, you can be empathic. 

\item You can sense their hearts, you can sense the hurt, and you can respond, ``You seem down today." They may say nothing. That's all right. You've shown understanding and respect.

\item Don't push; be \emph{\textbf{patient}}; be \emph{\textbf{respectful}}. People don't have to open up verbally before you can empathize. You can empathize all the time with their behavior. You can be \emph{discerning}, \emph{sensitive}, and \emph{aware} and you can live outside your autobiography when that is needed.

\item We often share our different perceptions of the situation, and we \emph{role-play} more effective approaches to difficult interpersonal family problems.

\item The time you invest to deeply understand the people you love brings tremendous dividends in open communication. 

\item Seek first to understand. Before the problems come up, before you try to evaluate and prescribe, before you try to present your own ideas -- \emph{seek to understand}. It's a powerful habit of effective interdependence.
\end{itemize}
\subsection{Habit 6: Synergize -- Principles of Creative Cooperation}
When properly understood, \emph{\textbf{synergy}} is the highest activity in all life -- the true test and manifestation of all the other habits put together.

The highest forms of synergy focus 
\begin{itemize}
\item the four unique human endowments, 
\item the motive of win-win, 
\item and the skills of empathic communication 
\end{itemize} on the toughest challenges we face in life. 
What results is almost miraculous. We create new alternatives -- something that wasn't there before.

\emph{\textbf{Synergy}} is the essence of \emph{\textbf{Principle-Centered Leadership}}. It is the essence of principle-centered parenting. It \emph{catalyzes}, \emph{unifies}, and \emph{unleashes the greatest powers within people}. All the habits we have covered prepare us to create the miracle of synergy.

What is \emph{\textbf{synergy}}? Simply defined, it means that \emph{the \textbf{whole} is greater than the sum of its parts}. It means that the relationship which the parts have to each other is a part in and of itself. It is not only a part, but the most \emph{catalytic}, the most \emph{empowering}, the most \emph{unifying}, and the most \emph{exciting} part.

Synergy is everywhere in nature. The whole is greater than the sum of its parts. One plus one equals three or more.

Family life provides many opportunities to observe synergy and to practice it. 

The \emph{\textbf{essence}} of \emph{synergy} is to \emph{\textbf{value differences}} -- to respect them, to build on strengths, to compensate for weaknesses.
\subsubsection{Synergistic Communication}
\begin{itemize}
\item When you \emph{communicate \textbf{synergistically}}, you are simply opening your mind and heart and expressions to new possibilities, new alternatives, new options.

You're not sure when you engage in synergistic communication how things will work out or what the end will look like, but you do have an inward sense of excitement and security and adventure, believing that it will be significantly better than it was before. And that is the end that you have in mind.

\item Many people have not really experienced even a moderate degree of synergy in their family life or in other interactions. They've been trained and \emph{scripted} into \emph{\textbf{defensive} and \textbf{protective} \textbf{communications}} or into believing that life or other people can't be trusted. As a result, they are never really open to Habit 6 and to these principles.

This represents one of the great tragedies and wastes in life, because so much potential remains untapped -- completely undeveloped and unused.

\item These things can be produced regularly, consistently, almost daily in people's lives. But it requires enormous \emph{\textbf{personal security}} and \emph{\textbf{openness}} and a spirit of adventure.

\item Almost all \emph{\textbf{creative endeavors}} are somewhat unpredictable. They often seem ambiguous, hit-or-miss, trial and error. 
\end{itemize}
\subsubsection{Synergy in Classroom}
\begin{itemize}
\item There are times when neither the teacher nor the student know for sure what's going to happen. In the beginning, there's a safe environment that enables people to be really open and to learn and to listen to each other's ideas. Then comes brainstorming where the spirit of evaluation is subordinated to the spirit of creativity, imagining, and intellectual networking. Then an absolutely unusual phenomenon begins to take place. The entire class is transformed with the excitement of a new thrust, a new idea, a new direction that's hard to define, yet it's almost palpable to the people involved.
\item Synergy is almost as if a group collectively agrees to subordinate old scripts and to write a new one.
\end{itemize}
\subsubsection{Synergy in Business}
\begin{itemize}
\item The synergistic process that led to the creation of our mission statement engraved it in all the hearts and minds of everyone there, and it has served us well as a frame of reference of what we are about, as well as what we are not about.

\item Once people have experienced \emph{real synergy}, they are never quite the same again. They know the possibility of having other such mind-expanding adventures in the future.

\item Often attempts are made to \emph{recreate} a particular \emph{synergistic experience}, but this seldom can be done. However, the essential purpose behind creative work can be recaptured. We seek not to imitate past creative synergistic experiences, rather we seek new ones around new and different and sometimes higher purposes.
\end{itemize}
\subsubsection{Synergy and Communication}
\begin{itemize}
\item \emph{\textbf{Synergy} is exciting}. \emph{\textbf{Creativity} is exciting}. It's phenomenal \emph{what \textbf{openness} and \textbf{communication} can produce}. The possibilities of truly significant gain, of significant improvement are so real that it's worth the risk such openness entails.

\item The following diagram illustrates how closely trust is related to different \emph{\textbf{levels of communication}}.
 \begin{itemize}
\item The lowest level of communication coming out of \emph{\textbf{low-trust situations}} would be characterized by \emph{\textbf{defensiveness}}, \emph{\textbf{protectiveness}}, and often \emph{\textbf{legalistic} language}, which covers all the bases and spells out qualifiers and the escape clauses in the event things go sour. 

Such communication produces only \emph{win/lose or lose/lose}. It isn't effective -- there's no P/PC Balance -- and it creates further reasons to defend and protect.

\item The middle position is \emph{\textbf{respectful communication}}. This is the level where fairly \emph{mature people} interact. They have \emph{respect} for each other, but they want to \emph{\textbf{avoid} the possibility of ugly \textbf{confrontations}}, so they \emph{communicate \textbf{politely} but \textbf{not empathically}}. 

Respectful communication works in independent situations and even in interdependent situations, but the creative possibilities are \emph{\textbf{not opened up}}. 

\item \emph{\textbf{Synergy}} means that $1 + 1$ may equal $8$, $16$, or even $1,600$. The synergistic position of high trust produces solutions better than any originally proposed, and all parties know it.
\end{itemize}
\end{itemize}

\subsubsection{Fishing for the Third Alternative}
\begin{itemize}
\item Whatever compromise they finally agree on, it could be rehearsed over the years as evidence of insensitivity, neglect, or a bad priority decision on either part. It could be a source of contention for years and could even polarize the family. 

Many marriages that once were beautiful and soft and spontaneous and loving have deteriorated to the level of a hostility through a series of incidents just like this.

\item Because they have a high \emph{Emotional Bank Account}, they have \emph{trust and open communication} in their marriage. Because they \emph{Think Win/Win}, they believe in a \emph{\textbf{Third Alternative}}, a solution that is \emph{mutually beneficial} and is better than what either of them originally proposed. Because they \emph{listen empathically} and \emph{seek first to understand}, they create within themselves and between them a comprehensive picture of the values and the concerns that need to be taken into account in making a decision.
\end{itemize}

\subsubsection{Negative Synergy}
\begin{itemize}
\item \emph{\textbf{Seeking the Third Alternative}} is a major \emph{\textbf{Paradigm Shift}} from the \emph{dichotomous}, \emph{\textbf{either/or mentality}}. But look at the difference in results.

\item The problem is that \emph{highly dependent people} are trying to succeed in an \emph{interdependent reality}. They're either dependent on \emph{borrowing strength from \textbf{position power}} and they go for \emph{\textbf{win/lose}} or they're dependent on being \emph{\textbf{popular}} with others and they go for \emph{\textbf{lose/win}}. They may talk win/win technique, but they don't really want to listen; they want to \emph{manipulate}. And synergy can't thrive in that environment.
\end{itemize}
\subsubsection{Valuing the Differences}
\begin{itemize}
\item \emph{\textbf{Valuing the differences}} is the essence of synergy -- the \emph{mental}, the \emph{emotional}, the \emph{psychological} differences between people. And the key to valuing those differences is to realize that all people see the world, not as it is, but as they are.

\item The person who is truly effective has the \emph{\textbf{humility}} and \emph{\textbf{reverence}} to \emph{\textbf{recognize}} his own \emph{\textbf{perceptual limitations}} and to appreciate the rich resources available through \emph{interaction with the hearts and minds of other human beings}. That person \emph{values the differences} because those differences add to his knowledge, to his understanding of reality. When we're left to our own experiences, we constantly suffer from a shortage of data.

\item And unless we value the differences in our perceptions, unless we value each other and give credence to the possibility that we're both right, that life is not always a dichotomous either/or, that there are almost always Third Alternatives, we will never be able to transcend the limits of that conditioning.
\end{itemize}
\subsubsection{All Nature is Synergistic}
\begin{itemize}
\item Ecology is a word which basically describes the synergism in nature -- everything is related to everything else. It's in the relationship that creative powers are maximized, just as the real power in these Seven Habits is in their relationship to each other, not just in the individual habits themselves.

\item The relationship of the parts is also the power in creating a \emph{\textbf{synergistic culture}} inside a family or an organization. 

\item Synergy works; it's a correct principle.
\item Although you cannot control the paradigms of others in an interdependent interaction or the synergistic process itself, a great deal of synergy is within your Circle of Influence.

\item Your own \emph{\textbf{internal synergy}} is completely within the circle. You can respect both sides of your own nature -- the \emph{\textbf{analytical}} side and the \emph{\textbf{creative}} side. You can value the difference between them and use that difference to catalyze creativity.

\item You can exercise the courage in interdependent situations to be open, to express your ideas, your feelings, and your experiences in a way that will encourage other people to be open also.
\end{itemize}
\section{Renewal}
\subsection{Habit 7: Sharpen the Saw -- Principles of Balanced Self-Renewal}
\subsubsection{Four Dimensions of Renewal}
\begin{itemize}
\item Habit 7 is \emph{\textbf{personal PC}}. It's preserving and enhancing the greatest asset you have -- you. It's renewing \emph{the four dimensions of your nature} -- \emph{\textbf{physical}}, \emph{\textbf{spiritual}}, \emph{\textbf{mental}}, and \textbf{\emph{social/emotional}}.

\item ``\emph{\textbf{Sharpen the Saw}}" basically means expressing all four motivations. It means \emph{\textbf{exercising}} all four dimensions of our nature, \emph{regularly and consistently,} in \emph{wise} and \emph{balanced} ways.

\item To do this, we must be \emph{\textbf{proactive}}. \emph{Taking time to sharpen the saw} is a definite \emph{Quadrant II activity}, and Quadrant II must be acted on. Quadrant I, because of its urgency, acts on us; it presses upon us constantly. \emph{\textbf{Personal PC}} must be pressed upon until it becomes second nature, until it becomes a kind of healthy addiction. Because it's at the center of our \emph{Circle of Influence}, no one else can do it for us. We must do it for ourselves.

\item This is the single most powerful \emph{investment} we can ever make in life -- \emph{\textbf{investment in ourselves}}, in the only instrument we have with which to deal with life and to contribute. 
\end{itemize}
\subsubsection{The Physical Dimension}
\begin{itemize}
\item The \emph{\textbf{physical dimension}} involves \emph{\textbf{caring} effectively for our physical \textbf{body}} -- eating the right kinds of \emph{foods}, getting sufficient \emph{rest} and relaxation, and \emph{exercising} on a regular basis.

\item \emph{\textbf{Exercise} is one of those Quadrant II, high-leverage activities} that most of us don't do consistently because it isn't urgent.

\item A good exercise program is one that you can do in your own home and one that will build your body in three areas: \emph{\textbf{endurance}}, \emph{\textbf{flexibility}}, and \emph{\textbf{strength}}.
\begin{itemize}
\item \emph{\textbf{Endurance}} comes from \emph{\textbf{aerobic exercise}}, from cardiovascular efficiency -- the ability of your heart to pump blood through your body.

\item \emph{\textbf{Flexibility}} comes through \emph{\textbf{stretching}}. Most experts recommend warming up before and cooling down/stretching after aerobic exercise. Before, it helps loosen and warm the muscles to prepare for more vigorous exercise.

\item \emph{\textbf{Strength}} comes from \emph{\textbf{muscle resistance exercises}} -- like simple calisthenics, push-ups, and sit-ups, and from working with weights. 
\end{itemize}

\item The essence of renewing the physical dimension is to \emph{sharpen the saw}, to exercise our bodies on a \emph{regular basis} in a way that will preserve and enhance our capacity to work and adapt and enjoy.
\end{itemize}
\subsubsection{The Spiritual Dimension}
\begin{itemize}
\item Renewing the spiritual dimension provides \emph{\textbf{leadership} to your life}. It's highly related to Habit 2.

\item The \emph{\textbf{spiritual dimension}} is your \emph{\textbf{core}}, your \emph{\textbf{center}}, your \emph{\textbf{commitment} to your \textbf{value system}}. It's a very private area of life and a supremely important one. It draws upon the sources that inspire and uplift you and tie you to the timeless truths of all humanity. And people do it very, very differently.

\item I find renewal in daily prayerful meditation on the scriptures because they represent my value system. As I read and meditate, I feel renewed, strengthened, centered, and recommitted to serve.

\item Spiritual renewal takes an investment of time. But it's a Quadrant II activity we don't really have time to neglect.

\item The idea is that when we take time to draw on the leadership center of our lives, what life is ultimately all about, it spreads like an umbrella over everything else. It \emph{renews} us, it \emph{refreshes} us, particularly if we \emph{recommit} to it.
\end{itemize}
\subsubsection{The Mental Dimension}
\begin{itemize}
\item Most of our \emph{\textbf{mental development}} and study discipline comes through \emph{\textbf{formal education}}. But as soon as we leave the external discipline of school, many of us let our minds atrophy.

\item Wisdom in watching television requires the \emph{effective \textbf{self-management}} of Habit 3, which enables you to \emph{\textbf{discriminate}} and to \emph{\textbf{select}} the \emph{informing}, \emph{inspiring}, and \emph{entertaining programs} which best serve and express your purpose and values.

\item \emph{\textbf{Education}} -- continuing education, continually honing and expanding the mind -- is \emph{vital mental renewal}. Sometimes that involves the \emph{external discipline} of the classroom or systematized study programs; more often it does not. \emph{Proactive} people can figure out many, many ways to educate themselves.

\item It is extremely valuable to \emph{\textbf{train}} the mind to stand apart and examine its own program. Training, without such education, narrows and closes
the mind so that the assumptions underlying the training are never examined. That's why it is so valuable to read broadly and to expose yourself to great minds.

\item There's no better way to inform and expand your mind on a regular basis than to get into the habit of \emph{\textbf{reading}} good literature. That's another high-leverage \emph{Quadrant II activity}. You can get into the best minds that are now or that have ever been in the world. I highly recommend starting with a goal
of a book a month then a book every two weeks, then a book a week. ``The person who doesn't read is no better off than the person who can't read."

\item \emph{\textbf{Writing}} is another powerful way to \emph{sharpen the mental saw}. Keeping a journal of our thoughts, experiences, insights, and learnings promotes \emph{mental clarity}, exactness, and context. Writing good letters -- communicating on the deeper level of thoughts, feelings, and ideas rather than on the shallow, superficial level of events -- also affects our ability to \emph{think clearly}, to \emph{reason accurately}, and to be \emph{understood effectively}.

\item \emph{\textbf{Organizing and planning}} represent other forms of mental renewal associated with Habits 2 and 3. It's \emph{beginning with the end in mind} and being able mentally to organize to accomplish that end. It's exercising the \emph{visualizing}, \emph{imagining} power of your mind to see the end from the beginning and to see the entire journey, at least in principles, if not in steps.

\item There's no other way you could spend an hour that would begin to compare with the Daily Private Victory in terms of value and results. It will affect every decision, every relationship. It will greatly improve the quality, the effectiveness, of every other hour of the day, including the depth and restfulness of your sleep. It will build the \emph{long-term physical, spiritual, and mental strength} to enable you to handle difficult challenges in life.
\end{itemize}
\subsubsection{The Social/Emotional Dimension}
\begin{itemize}
\item While the physical, spiritual, and mental dimensions are closely related to Habits 1, 2, and 3 -- centered on the principles of personal vision, leadership, and management -- \emph{the \textbf{social/emotional dimension}} focuses on Habits 4, 5, and 6 -- centered on \emph{\textbf{the principles of interpersonal leadership}}, \emph{\textbf{empathic communication}}, and \textbf{creative cooperation}.

\item The social and the emotional dimensions of our lives are tied together because our \emph{\textbf{emotional life}} is primarily, but not exclusively, developed out of and manifested in \emph{our \textbf{relationships} with others}.

\item Renewing our social/emotional dimension does not take time in the same sense that renewing the other dimensions does. We can do it in our normal \emph{\textbf{everyday interactions}} with other people. But it definitely requires \emph{\textbf{exercise}}. 

\item But we see things differently; we're looking through different glasses.

\item If our \emph{personal security} comes from \emph{sources within ourselves}, then we have the \emph{strength} to practice the habits of \emph{Public Victory}. If we are emotionally insecure, even though we may be intellectually very advanced, practicing Habits 4, 5, and 6 with people who think differently on jugular issues of life can be terribly threatening.

\item Where does \emph{\textbf{intrinsic security}} come from? It doesn't come from the scripts they've handed us. It doesn't come from our \emph{circumstances} or \emph{our position}.

\item It comes from \emph{within}. It comes from \emph{accurate paradigms} and \emph{correct principles} deep in our own mind and heart. It comes from \emph{Inside-Out congruence}, from living a life of \emph{integrity} in which \emph{our daily \textbf{habits} reflect our deepest \textbf{values}}.

\item \emph{\textbf{Peace of mind}} comes when your life is \emph{in harmony with} true \emph{principles} and \emph{values} and in no other way.

\item There is also \emph{the intrinsic security} that comes as a result of \emph{\textbf{effective interdependent living}}. There is security in knowing that \emph{win/win solutions do exist}, that life is not always ``\emph{either/or}," that there are almost always \emph{\textbf{mutually beneficial Third Alternatives}}. 

\item There is intrinsic security that comes from \emph{\textbf{service}}, from helping other people in a meaningful way. One important source is \emph{\textbf{your work}}, when you see yourself in a \emph{\textbf{contributive}} and \emph{\textbf{creative}} mode, really making a difference. 
\end{itemize}
\subsubsection{Scripting Others}
\begin{itemize}
\item Most people are a function of \emph{the social mirror}, \emph{\textbf{scripted} by the opinions}, the \emph{perceptions}, the \emph{paradigms} of the people \emph{around them}. As \emph{interdependent} people, you and I come from a paradigm which includes the realization that \emph{\textbf{we are a part of that social mirror}}.

\item We can choose to \emph{\textbf{reflect back to}} others a clear, undistorted vision of themselves. We can \emph{\textbf{affirm} their proactive nature} and treat them as responsible people. We can help \emph{\textbf{script}} them as principle-centered, value-based, independent, worthwhile individuals. And, with the \emph{Abundance Mentality}, we realize that giving a \emph{\textbf{positive reflection}} to others in no way diminishes us. It increases us because it \emph{increases the opportunities for effective interaction} with other proactive people.

\item At some time in your life, you probably had someone believe in you when you didn't believe in yourself. He or she scripted you. Did that make a difference in your life.

\item What if you were a positive scripter, an affirmer, of other people? When they're being directed by the social mirror to take the lower path, you \emph{\textbf{inspire}} them toward a higher path because you believe in them. You listen to them and empathize with them. You don't absolve them of responsibility; you \emph{\textbf{encourage}} them to be \emph{proactive}.

\item We can refuse to label them -- we can "see" them in new fresh ways each time we're with them. We can help them become independent, fulfilled people capable of deeply satisfying, enriching, and productive relationships with others.
\end{itemize}
\subsubsection{Balance in Renewal}
\begin{itemize}
\item  The \emph{self-renewal process} must include \emph{\textbf{balanced renewal}} in all four dimensions of our nature: the \emph{physical}, the \emph{spiritual}, the \emph{mental}, and the \emph{social/emotional}.

\item Although renewal in each dimension is important, it only becomes optimally effective as we deal with all four dimensions in a wise and balanced way. To neglect any one area negatively impacts the rest.

\item I have found this to be true in organizations as well as in individual lives. 

\end{itemize}
\subsubsection{Synergy in Renewal}
\begin{itemize}
\item \emph{Balanced renewal is optimally synergetic}. The things you do to sharpen the saw in any one dimension have positive impact in other dimensions because they are so highly interrelated. 

Your physical health affects your mental health; your spiritual strength affects your social/emotional strength. As you improve in one dimension, you increase your ability in other dimensions as well.

\item The more proactive you are (Habit 1), the more effectively you can exercise personal leadership (Habit 2) and management (Habit 3) in your life. The more effectively you manage your life (Habit 3), the more Quadrant II renewing activities you can do (Habit 7). The more you seek first to understand (Habit 5), the more effectively you can go for synergetic win-win solutions (Habits 4 and 6). The more you improve in any of the habits that lead to independence (Habits 1, 2, and 3), the more effective you will be in interdependent situations (Habits 4, 5, and 6). And renewal (Habit 7) is the process of renewing all the habits.

\item \emph{The Daily Private Victory} -- \emph{a minimum of one hour a day in \textbf{renewal}} of the physical, spiritual, and mental dimensions -- is \emph{the key to the development of the Seven Habits} and it's completely within your Circle of Influence. It is the Quadrant II focus time necessary to integrate these habits into your life, to become principle-centered.

\item It's also the foundation for the \emph{Daily Public Victory}. It's the source of intrinsic security you need to sharpen the saw in the social/emotional dimension.
\end{itemize}
\subsubsection{The Upward Spiral}
\begin{itemize}
\item \emph{Renewal} is the principle -- and the process -- that empowers us to move on an \emph{\textbf{upward spiral} of growth and change}, of \emph{continuous improvement.}

\item \emph{\textbf{Conscience}} is the endowment that \emph{senses} our \emph{congruence} or disparity with \emph{correct principles} and \emph{lifts} us toward them -- when it's \emph{in shape}.

\item Education of the conscience is vital to \emph{the truly proactive, highly effective person}. Training and educating the conscience, however, requires even greater \emph{concentration}, more \emph{balanced discipline}, more consistently honest living. It requires regular feasting on inspiring literature, thinking noble thoughts and, above all, living in harmony with its still small voice.

\item And there is \emph{no shortcut} in developing them. The Law of the Harvest governs; \emph{\textbf{we will always reap what we sow}} -- no more, no less. 

\item We must show diligence in the process of renewal by educating and \emph{obeying our conscience}. 

\item Moving along the upward spiral requires us to \emph{\textbf{learn}, \textbf{commit}, and \textbf{do}} on increasingly higher planes. We deceive ourselves if we think that any one of these is sufficient. To keep progressing, we must learn, commit, and do -- learn, commit, and do -- and learn, commit, and do again.
\end{itemize}
\subsection{Inside-Out Again}
\subsubsection{Intergenerational Living}
\begin{itemize}
\item There is transcendent power in a strong intergenerational family. An effectively interdependent family of children, parents, grandparents, aunts, uncles, and cousins can be a powerful force in helping people have a sense of who they are and where they came from and what they stand for.
\end{itemize}
\subsubsection{Becoming a Transition Person}
\begin{itemize}
\item Among other things, I believe that giving ``wings" to our children and to others means empowering them with the freedom to rise above negative scripting that had been passed down to us. 

\item Instead of \emph{\textbf{transferring} those scripts} to the next generation, we can \emph{\textbf{change}} them. And we can do it in a way that will build relationships in the process

\item A tendency that's run through your family for generations can stop with you. You're a \emph{\textbf{transition person}} -- a \emph{link} between the past and the future. And \emph{your own change can \textbf{affect} many, many lives downstream}.
\end{itemize}
\newpage
\bibliographystyle{plainnat}
\bibliography{reference.bib}
\end{document}
