\documentclass[11pt]{article}
\usepackage[scaled=0.92]{helvet}
\usepackage{geometry}
\geometry{letterpaper,tmargin=1in,bmargin=1in,lmargin=1in,rmargin=1in}
\usepackage[parfill]{parskip} % Activate to begin paragraphs with an empty line rather than an indent %\usepackage{graphicx}
\usepackage{amsmath,amssymb, mathrsfs,  mathtools, dsfont}
\usepackage{tabularx}
\usepackage{tikz-cd}
\usepackage[font=footnotesize,labelfont=bf]{caption}
\usepackage{graphicx}
\usepackage{xcolor}
%\usepackage[linkbordercolor ={1 1 1} ]{hyperref}
%\usepackage[sf]{titlesec}
\usepackage{natbib}
%\usepackage{tikz-cd}

\usepackage{../../Tianpei_Report}

%\usepackage{appendix}
%\usepackage{algorithm}
%\usepackage{algorithmic}

%\renewcommand{\algorithmicrequire}{\textbf{Input:}}
%\renewcommand{\algorithmicensure}{\textbf{Output:}}



\begin{document}
\title{Book Reading Summary: The 5 Elements of Effective Thinking}
\author{ Tianpei Xie}
\date{Dec. 10th., 2023}
\maketitle
\tableofcontents
\newpage
\section{Introduction -- Elements of Effective Thinking, Learning and Creating}
\begin{itemize}
\item The surprising fact is that just a few learnable strategies of thinking can make you more effective in the classroom, the boardroom, and the living room. You can personally \emph{\textbf{choose}} to become more successful by adopting five learnable habits, which, in this book, we not only explain in detail but also make concrete and practical. Here in this section we briefly introduce those important habits to come.

\item \emph{\textbf{Understand deeply}}:
\begin{itemize}
\item Don’t face complex issues head-on; first understand \emph{\textbf{simple} ideas \textbf{deeply}}. 
\item Clear the clutter and expose what is really \emph{\textbf{important}}. 
\item Be brutally honest about \emph{what you \textbf{know} and \textbf{don’t know}}. 
\item Then see \emph{what’s \textbf{missing}}, identify the \emph{\textbf{gaps}}, and fill them in. 
\item Let go of \emph{\textbf{bias}}, \emph{\textbf{prejudice}}, and \emph{\textbf{preconceived notions}}. 
\item There are \emph{\textbf{degrees} to understanding} (it’s not just a yes-or-no proposition) and you can always heighten yours. 
\item \emph{\textbf{Rock-solid understanding}} is the foundation for success.
\end{itemize}

\item \emph{\textbf{Make mistakes}}:
\begin{itemize}
\item \emph{\textbf{Fail}} to succeed. 
\item \emph{\textbf{Intentionally}} get it wrong to \emph{inevitably} get it even \emph{more right}. 
\item \emph{\textbf{Mistakes}} are great teachers -- they \emph{highlight \textbf{unforeseen opportunities}} and holes in your understanding. 
\item They also show you which way to \emph{\textbf{turn next}}, and they ignite your \emph{\textbf{imagination}}.
\end{itemize}

\item \emph{\textbf{Raise questions}}:
\begin{itemize}
\item Constantly create \emph{\textbf{questions}} to \emph{\textbf{clarify}} and \emph{\textbf{extend}} your understanding. 
\item What’s the \emph{\textbf{real}} question? Working on the wrong questions can waste a lifetime.
\item Ideas are in the air -- the \emph{\textbf{right questions}} will bring them out and help you see \emph{\textbf{connections}} that otherwise would have been invisible.
\end{itemize}

\item \emph{\textbf{Follow the flow of ideas}}:
\begin{itemize}
\item Look back to see \emph{where \textbf{ideas} \textbf{came from}} and then \emph{look ahead} to discover \emph{where those \textbf{ideas} may \textbf{lead}}. 
\item \emph{A new idea} is a \emph{\textbf{beginning}}, not an end. 
\item Ideas are \emph{rare} -- \emph{\textbf{milk}} them. 
\item Following the \emph{\textbf{consequences}} of \emph{\textbf{small ideas}} can result in \emph{big payoffs}.
\end{itemize}

\item \emph{\textbf{Change}}:
\begin{itemize}
\item The unchanging element is change -- by mastering the first four elements, you can change the way you think and learn. You can always improve, grow, and extract more out of your education, yourself, and the way you live your life. 
\item Change is the universal constant that allows you to get the most out of living and learning.
\end{itemize}
\end{itemize}

\section{Ground Your Thinking -- Understand Deeply}
\begin{itemize}
\item Understanding is not a \emph{yes-or-no proposition}; it’s not an on-or-off switch. 

\item When you learn anything, \emph{go for \textbf{depth}} and \emph{make it \textbf{rock solid}}. 

\item You can understand anything better than you currently do. Setting a \emph{\textbf{higher standard}} for yourself for \emph{what you mean by \textbf{understanding}} can revolutionize how you perceive the world. The following steps illustrate why a deep understanding is essential to a solid foundation for future thinking and learning.
\end{itemize}

\subsection{Understand simple things deeply}
\begin{itemize}
\item \emph{\textbf{The most fundamental ideas}} in any subject can be understood with \emph{\textbf{ever-increasing depth}}. 

\item Successful students continue to improve their mastery of the concepts from previous chapters and courses as they move toward the more advanced material on the horizon; successful people regularly focus on the core purpose of their profession or life. 

\item \emph{True experts \textbf{continually deepen} their \textbf{mastery} of the \textbf{basics}}.

\item \emph{\textbf{What is deep understanding?}} 

How can you realize when you don’t know something deeply? In everything you do, \emph{\textbf{refine} your skills and knowledge} about \emph{\textbf{fundamental concepts}} and \emph{\textbf{simple cases}}. \emph{Once is never enough}. As you \emph{\textbf{revisit fundamentals}}, you will find \emph{\textbf{new insights}}. It may appear that returning to basics is a step backward and requires additional time and effort; however, by \emph{building on \textbf{firm foundations}} you will soon see your true abilities soar higher and faster.

\item 
\begin{exercise} [Master the Basics]
Consider a skill you want to improve or a subject area that you wish to understand better.  Spend five minutes writing down \textbf{specific components of the skill} or \textbf{subject area} that are \textbf{basic} to that theme. Your list will be a freeflowing stream of consciousness. 

Now \textbf{pick one} of the items on your list, and spend thirty minutes \textbf{actively improving your mastery} of it.  See how working deeply on the basics makes it possible for you to hone your skill or deepen your knowledge at the higher levels you are trying to attain. Apply this exercise to other things you think you know or would like to know.
\end{exercise}

\item \emph{\textbf{A commonsense approach leads to the core}}. Many of the most complicated, subtle, and profound ideas arise from looking unmercifully clearly at simple, everyday experiences. 

To \emph{learn} any subject well and to \emph{create} ideas beyond those that have existed before, \emph{r\textbf{eturn to the basics repeatedly}}. When you look back after learning a complicated subject, the basics seem far simpler; however, \emph{those \textbf{simple basics are a moving target}}. As you learn more, the fundamentals become at once \emph{simpler} but also \emph{subtler, deeper, more nuanced, and more meaningful}. 

\item 
\begin{exercise}[Ask: What do you know?]
Do you or don’t you truly know the basics? Consider a subject you think you know or a subject you are trying to master.  Open up a blank document on your computer. Without referring to any outside sources, write a \textbf{detailed outline of the fundamentals} of the subject.

Can you write a \textbf{coherent}, \textbf{accurate}, and \textbf{comprehensive} \textbf{description} of the foundations of the subject, or does your knowledge have gaps? Do you struggle to think of \textbf{core examples}? Do you fail to see the \textbf{overall big picture} that puts the pieces together? Now compare your effort to external sources (texts, Internet, experts, your boss).

When you \textbf{discover weaknesses} in your own \textbf{understanding of the basics}, \textbf{take action}. \textbf{Methodically learn} the fundamentals. Thoroughly understand any \textbf{gap} you fill in as well as its \textbf{surrounding territory}. Make these new insights part of your base knowledge and connect them with the parts that you already understood. 

Repeat this exercise regularly as you learn more advanced aspects of the subject (and save your earlier attempts so that you can look back and see how far you’ve traveled). Every return to the basics will deepen your understanding of the entire subject.
\end{exercise}

\item When faced with a \emph{\textbf{difficult challenge}} --\emph{\textbf{don’t do it!}}

Great scientists, creative thinkers, and problem solvers \emph{\textbf{do not solve hard problems} \textbf{head-on}}. When they are faced with a daunting question, they immediately and prudently \emph{admit defeat}. 

They realize that there is no sense in wasting energy vainly grappling with complexity when, instead, they can \emph{productively grapple with \textbf{simpler cases}} that will \emph{teach them how to deal with the complexity to come}.

``\emph{If you can’t solve a problem, then \textbf{there is an easier problem you can’t solve}: find it.}" -- George Polya.

\item When the going gets tough, creative problem solvers \emph{\textbf{create an easier, simpler problem}} that they can solve. They resolve that easier issue thoroughly and then \emph{\textbf{study that simple scenario} with laser focus}. Those insights often point the way to a resolution of the original difficult problem.

\item Apply this mind-set to your work: when faced with a \emph{difficult issue or challenge}, \emph{do something else}. \emph{\textbf{Focus}} entirely on \emph{\textbf{solving a subproblem}} that you know you can successfully resolve. Be completely confident that the extraordinarily thorough work that you invest on the subproblem will later be the guide that allows you to navigate through the complexities of the larger issue. 

But don’t jump to that more complex step while you’re at work on the subissue. 

\item \begin{exercise}[Sweat the small stuff]
Consider some complex issue in your studies or life. Instead of tackling it in its entirety, find one \textbf{small element} of it and \textbf{solve} that part \textbf{completely}. Understand the subissue and its solution \textbf{backwards} and \textbf{forwards}. Understand all its \textbf{connections} and \textbf{implications}. Consider this small piece from many points of view and in great detail. Choose a \textbf{subproblem} small enough that you can
give it this level of attention. Only later should you consider how your efforts could help solve the larger issue.
\end{exercise}
\end{itemize}

\subsection{Clear the clutter -- seak the essential}
\begin{itemize}
\item \emph{\textbf{Uncover the essence}}. 

When faced with an issue that is complicated and multifaceted, attempt to \emph{\textbf{isolate} the \textbf{essential ingredients}}. The essence is n\emph{ot the whole issue}. There is a further step of understanding how the other features of the situation fit together; however, \emph{clearly \textbf{identifying} and \textbf{isolating} essential principles} can guide you through the morass. The strategy of
clearing the clutter and seeking the essential involves two steps:
\begin{enumerate}
\item \textbf{\emph{Identify}} and \emph{\textbf{ignore}} all \emph{\textbf{distracting features}} to \emph{isolate} the \emph{\textbf{essential core}}.
\item  \emph{\textbf{Analyze}} that \emph{\textbf{central issue}} and \emph{apply} those insights to the \emph{\textbf{larger whole}}.
\end{enumerate}

\item Many real questions are surrounded and obscured by history, context, and adornments. Within that cloud of vaguely related, interacting influences, you need to \emph{\textbf{pluck out} the \textbf{central themes}}.  Often you may be surprised that after you pare down a complex issue to its essentials, the essentials are much clearer and \emph{easier to face}. 

Ignoring things is difficult. Often the peripheral clutter is blinking and clanging and trying madly to draw your attention away from what is really going on. By \emph{\textbf{systematically} \textbf{ignoring} one \textbf{distraction} after another}, you can turn your \emph{attention} to more \emph{central} (often initially invisible) \emph{themes}. 

After you \emph{\textbf{clear the clutter}}, what remains will \emph{clarify understanding} and open the door to \emph{creating new ideas}. Remember, you may not be able to see everything, but you can certainly \emph{\textbf{ignore most things}}.


\item \begin{exercise}[Uncover one essential]
Consider a subject you wish to understand, and \textbf{clear the clutter} until you have \textbf{isolated} \textbf{one essential ingredient}. Each complicated issue has \textbf{several possible core ideas}. You are not seeking ``the” essential idea; you are seeking just one -- consider a subject and pare it down to one theme. 

In fact, you might perform this exercise on yourself. What do you view as essential elements of you? Isolating those elements can give a great deal of \textbf{focus to life decisions}.
\end{exercise}

\item Once you have isolated the essential, you have armed yourself with a \emph{\textbf{solid center}} upon which to build and embellish. \emph{The core is not the whole issue}, but it is a lodestar that can guide you through turbulent storms and complications. 

What’s \emph{\textbf{core}}? What’s fluff? Find what’s \emph{\textbf{at the center}} and \emph{work out from there}. You can confidently \emph{\textbf{center yourself}}.
\end{itemize}

\subsection{See what's there}
\begin{itemize}
\item Whenever you ``\emph{see}" an issue or ``\emph{understand}" a concept, be \emph{conscious of the lens} through which you’re viewing the subject. You should assume you’re \emph{\textbf{introducing bias}}. The challenge remains to identify and \emph{let go of that bias or the assumptions} you bring, and actively work to see and understand the subject anew.

Whether it be physical characteristics of what you see, emotional aspects of what you feel, or conceptual underpinnings of what you understand, acknowledging and then \emph{letting go of \textbf{bias} and \textbf{prejudice}} can \emph{lead you to \textbf{see what’s truly there}} and (often more importantly) to \emph{\textbf{discover what’s missing}}.

\item To better understand your world, \emph{consciously \textbf{acknowledge what you actually see}} -- no matter how mundane or obvious -- rather than guess at what you think you are supposed to see. Saying what you actually see forces you to become \emph{\textbf{conscious}} of \emph{what is there and also what is missing}. 

\emph{If you see it, then say it; if you don’t see it, then don’t claim to see it.}

\item \begin{exercise}[Say it like you see it]
Homework assignments, tests, and job-related assessments ask you what you know. Unfortunately, partial credit or social pressure often encourages you to pretend to know a bit more than you actually do. 

So in the privacy of your own room look at assignments or possible test questions and write down the weaknesses as well as the strengths of what you know and don’t know. Deliberately avoid glossing over any gaps or vagueness. Instead boldly assert what is tepid or missing in your understanding. 

Now take the action of filling in the gaps. Identifying and admitting your own uncertainties is an enormous step toward solid understanding.
\end{exercise}

\item If you are writing an essay, \emph{read literally what you have written} -- not what you intended to communicate.  Pretend you don’t know the argument you are making and read your actual words.  What’s confusing and what’s missing? If you think you know an idea but can’t express it clearly, then this process has identified a \emph{gap or vagueness in your understanding}. 

After you admit and address those weaknesses, your exposition will be clearer and more directed to the actual audience. When delivering an address or making a presentation, apply this same process of deliberately listening to the actual words you are speaking rather than what you imagine you are saying.

\item \emph{\textbf{What everybody believes is not always what’s actually true.}}

Commonly held opinions are frequently just plain false. Often we are \emph{persuaded by authority and repetition} rather than by \emph{evidence and reality}. This tendency to \emph{accept what surrounds us} makes it difficult to separate what we really know from what we just believe we know. 

\item Individuals tend to accept ideas if people they know or respect state or believe those ideas. You need to be very clear about the \emph{\textbf{foundations}} of \emph{\textbf{your opinions}}. If you believe something only because another person -- even a professor -- told you it was so, then you should not view your understanding as rock solid. 

\item \emph{\textbf{How do you know?}}

\emph{\textbf{Becoming aware} of the \textbf{basis} of your \textbf{opinions} or \textbf{beliefs}} is an \emph{important step} toward a better understanding of yourself and your world. Regularly consider your opinions, beliefs, and knowledge, and subject them to the ``How do I know?" test.

What is the evidence that your understanding is based upon? Become aware of \emph{the \textbf{sources of your opinions}}. If your \emph{\textbf{sources} are shaky}, then you might want to be more \emph{open-minded} to the possibility that your opinion or knowledge might be \emph{incorrect}. \emph{Regularly} \emph{find \textbf{cases}} in which you need to \emph{\textbf{rethink your views}}.

\item \emph{Opening our minds} to \emph{\textbf{counterintuitive ideas}} can be the \emph{\textbf{key}} to discovering novel solutions and building deeper understanding, but how can we take advantage of those opportunities? Certainly we are not intentionally closed-minded. 

So how can we break free of our unintended closed-mindedness and see the world with less bias? First, we can simply \emph{try out \textbf{alternative ideas} \textbf{hypothetically} and \textbf{temporarily}}. 

I’ll \emph{\textbf{pretend}} my opinions are \emph{the opposite of what I normally believe} (even though I know it’s nonsense), and see where those new beliefs take me. This strategy allows you to explore ideas without having to overcome deeply ingrained moral
or institutional prejudices. 

Even \emph{following} ideas that you know are \emph{\textbf{wrong}} can be illuminating. Because in following the consequences of those ``wrong" ideas, you might be led to better understand why your original belief is \emph{indeed correct}, or you might be led to \emph{new and unexpected insights} that run \emph{counter} to \emph{your original beliefs}.

\item \begin{exercise}[Try on alternatives and size up the fit]
Take some opinion that you hold that other people (those who clearly are wrong) do not hold. Every other hour \textbf{accept your own current opinion} and think about its \textbf{implications}, and on the alternate hours \textbf{accept the alternative opinion} and see where that \textbf{leads}. Try \textbf{not to be judgmental}. Don’t resist the alternative views. You are not committing to any change. 

This exercise has the goal of \textbf{understanding alternatives} more \textbf{realistically}. As a result, you might change an opinion, but more likely you will simply have a better understanding of why the alternative views make sense to others. If an hour is too long a time period, try the challenge in fifteen-minute intervals.
\end{exercise}
\end{itemize}

\subsection{See what's missing}
\begin{itemize}
\item Forcing yourself to see what’s actually in front of you rather than what you believe you should see is a difficult task. 

\item However, an even greater challenge is to \emph{\textbf{see what’s missing}}. One of the most profound ways to see the world more clearly is to \emph{\textbf{look deliberately for the gaps}} -- \emph{\textbf{the negative space}}, as it is called in the art world; that is, the space surrounding the objects or issues of interest. 

In our daily and intellectual experiences there are \emph{gaps} of many sorts. If you’re studying some body of material, ask yourself to \emph{identify those concepts that you \textbf{truly do not fully understand}}. Those concepts may, in fact, be ideas that you were supposed to have mastered in an earlier class or at an earlier point in your life.

Don’t despair. \emph{\textbf{Honestly admitting} those \textbf{gaps}} in \emph{knowledge and understanding} is the first \emph{important step} in \emph{attempting} to fill them. 

\item A harder question is this: How can you see what’s truly invisible?

\item \emph{\textbf{Add the adjective and uncover the gaps}}. By just describing what was there, he was led to see the invisible.

\item \begin{exercise}[See the invisible]
Select your own object, issue, or topic of study and attach an adjective or descriptive phrase (such as ``the First" before ``World War") that points out some reality of the situation, ideally some feature that is limiting or \textbf{taken for granted}. 

Then consider whether your \textbf{phrase} \textbf{suggests new possibilities or opportunities}. It might be helpful to think of this exercise as a word-association game. For example, if you are a student, you could consider a word such as ``semester" and then list the first few \textbf{adjectives} that come to mind -- for example ``busy," ``boring," ``tiring," ``exciting," and the like. Use your newfound
adjectives to create interesting and provocative insights that might otherwise have gone unnoticed.
\end{exercise}


\end{itemize}

\subsection{Final thoughts: Deeper is better}
\begin{itemize}
\item Understanding simple things deeply means \emph{\textbf{mastering} the \textbf{fundamental} principles, ideas, and methods} that then create a \emph{\textbf{solid foundation}} on which you can build. 
\begin{itemize}
\item Seeking the \emph{essential} creates the \emph{core} or \emph{skeleton} that supports your understanding. 
\item Seeing what’s actually there \emph{without prejudice }lets you develop a less biased understanding of your world.
\item And seeing what’s missing helps you to identify the \emph{limits} of your knowledge, to reveal \emph{new possibilities}, and to create new solutions to complex problems. 
\end{itemize} 

From the physical world to society, academics, personal relations, business, abstract ideas, and even sports, a deep examination of the simple and familiar is a \emph{potent \textbf{first step}} for learning, thinking, creating, and problem solving.

\item \emph{Earth} is that which is under where we stand.

\item  Among the \emph{\textbf{goals}} of this book are to describe \emph{how you can \textbf{construct original ideas}}, to show \emph{how you can \textbf{solve old problems}}, and to reveal \emph{how you can \textbf{create new worlds}}. 

\item Here we are advocating a process that
\begin{itemize}
\item starts with your \emph{\textbf{most comfortable surroundings}}, your \emph{most familiar territory}, \emph{the basics} that you know best, 
\item and encourages you to \emph{\textbf{search deeply}} for features that you \emph{don’t ordinarily perceive}.
\end{itemize}

\item \emph{The \textbf{familiar} is full of \textbf{unseen depth} and wonder}. \emph{Clear away the distractions}, \emph{see what’s actually there}, and \emph{make the invisible visible}.
\end{itemize}

\section{Igniting Insights through Mistakes -- Fail to Succeed}
\begin{itemize}
\item
\end{itemize}

\subsection{Welcome accidental missteps -- let your errors be your guide}
\begin{itemize}
\item
\end{itemize}

\subsection{Finding the right question to the wrong answer}
\begin{itemize}
\item
\end{itemize}

\subsection{Failing by intent}
\begin{itemize}
\item
\end{itemize}

\subsection{Final thoughts: A modified mind-set}
\begin{itemize}
\item
\end{itemize}

\section{Creating Questions out of Thin Air -- Be Your Own Socrates}
\subsection{How answers can lead to questions}
\begin{itemize}
\item
\end{itemize}

\subsection{Creating questions enlivens your curiosity}
\begin{itemize}
\item
\end{itemize}

\subsection{What's the real question?}
\begin{itemize}
\item
\end{itemize}

\subsection{Final thoughts: The art of creating questions and active listening}
\begin{itemize}
\item
\end{itemize}

\section{Seeing the Flow of Ideas -- Look Back, Look Forward}
\begin{itemize}
\item
\end{itemize}

\subsection{Understanding current ideas through the flow of ideas}
\begin{itemize}
\item
\end{itemize}

\subsection{Creating new ideas from old ones}
\begin{itemize}
\item
\end{itemize}

\subsection{Final thoughts: ``Under construction" is the norm}
\begin{itemize}
\item
\end{itemize}

\section{Engaging Change -- Transform Yourself}
\begin{itemize}
\item
\end{itemize}

\subsection{You can do it}
\begin{itemize}
\item
\end{itemize}

\subsection{Final thoughts: Becoming the quintessential you}
\begin{itemize}
\item
\end{itemize}


\section{A Way to Provoke Effective Thinking -- A Brief Review}
\begin{itemize}
\item
\end{itemize}

\newpage
\bibliographystyle{plainnat}
\bibliography{reference.bib}
\end{document}
