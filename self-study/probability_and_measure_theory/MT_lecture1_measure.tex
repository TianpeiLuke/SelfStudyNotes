\documentclass[11pt]{article}
\usepackage[scaled=0.92]{helvet}
\usepackage{geometry}
\geometry{letterpaper,tmargin=1in,bmargin=1in,lmargin=1in,rmargin=1in}
\usepackage[parfill]{parskip} % Activate to begin paragraphs with an empty line rather than an indent %\usepackage{graphicx}
\usepackage{amsmath,amssymb, mathrsfs, dsfont}
\usepackage{tabularx}
\usepackage[font=footnotesize,labelfont=bf]{caption}
\usepackage{graphicx}
\usepackage{xcolor}
%\usepackage[linkbordercolor ={1 1 1} ]{hyperref}
%\usepackage[sf]{titlesec}
\usepackage{natbib}
\usepackage{../../Tianpei_Report}

%\usepackage{appendix}
%\usepackage{algorithm}
%\usepackage{algorithmic}

%\renewcommand{\algorithmicrequire}{\textbf{Input:}}
%\renewcommand{\algorithmicensure}{\textbf{Output:}}



\begin{document}
\title{Lecture 1: Development of Measures}
\author{ Tianpei Xie}
\date{ Jul. 13th., 2015 }
\maketitle
\tableofcontents
\newpage
\section{Elementary Measure and Jordan Measure}
\subsection{Measure and its motivations}
\begin{itemize}
\item \begin{remark} 
The basic \textbf{motivation}: an extension of  \emph{measure} $m(E)$ in $\bR^{1}, \bR^{2}, \bR^{3}$ as \emph{length, area and volume} of a geometric body $E$. 

\citep{tao2011introduction} A set of intuitive axioms for a measure function $m$ defined on power set $2^{\bR}$: 
\begin{enumerate}
\item The \emph{\textbf{unit length}} of interval: $E= (0,1]$, then $m((0,1]) = 1$;
\item If $E$ is \emph{\textbf{congruent}} to $F$: (There exists a proper translation, rotation or reflection from $E$ to $F$), then $m(E) = m(F)$;
\item The \emph{\textbf{countably additive}}: for a countable union of disjoint sets, $\bigcup_{k=1}^{\infty}E_{k}$, the measure 
\begin{align*}
m\paren{\bigcup_{k=1}^{\infty}E_{k}} &= \bigcup_{k=1}^{\infty}m\paren{E_{k}}
\end{align*}  

Note that for \emph{\textbf{uncountable union}}, the intuition falls. For example, $f: \mb{x} \mapsto 2\mb{x}$ is \emph{one-to-one correspondence}, but when it applies to measure it means that a unit length interval can be dissembled and reassembled as a length of any $k\ge 1$. 

It is seen that even with finite partition, the dissemble-reassemble procedure could generates bizarre results: see "Banach-Tarski paradox". \citep{tao2011introduction} The idea is that the pieces used there is quite "uncommon" as the interval or cubes used in general work. 
\end{enumerate}
\end{remark}

\item \begin{remark}
Unfortunately, \underline{\emph{\textbf{these three axioms are inconsistent}}}: \emph{\textbf{no proper definition of measure function $m$ could satisfies all these three axioms for any subset in $\bR$}}. The measure theory should be built on a collection of ``ordinary" subsets, which motivates the introduction of \emph{\textbf{$\sigma$-algebra}}. 
\end{remark}


\item The measure for elementary sets: just as $E= (a,b), [a,b], (a,b], [a,b)$, boxes $E^{n}$, or the set $E$ is partitioned as the \emph{finite} union \emph{disjoint} boxes $E_{1},\cdots, E_{k}$. Here $m(E) = (b-a)^{n}$ for a box $E^{n}$, and $m(\bigcup_{i=1}^{k}E_{i})= \sum_{i=1}^{k}m(E_{i})$.
\end{itemize}
\subsection{Elementary Measure}
\begin{itemize}
\item \begin{definition} 
The \underline{\emph{\textbf{elementary measure}}} $m$ is an \emph{\textbf{algebra}} $\srA_{0}$ on set $\bR^{n}$ (,which is \emph{closed} under \emph{\textbf{finite union}} and \emph{\textbf{complements}}). Here $\srA_{0}$ is the \emph{\textbf{minimal algebra}} generated by a collection of all \emph{\textbf{boxes}} $\bigotimes_{i=1}^{n}(a_{i},b_{i}] \subset \bR^{n}$. Define $m: \srA_{0} \rightarrow \bR_{+}$ should satisfy
\begin{enumerate}
\item \emph{\textbf{Non-negative}}: $m(E) \ge 0$, for all $E\in \srA_{0}$;
\item $m(\emptyset) = 0$;
\item $m((0,1]^{n})= 1$;
\item \emph{\textbf{Translation-invariant}}: $m\paren{\mb{x}+ E} = m(E)$ for any $\mb{x}\in \bR^{n}$; 
\item \emph{\textbf{Finitely additive}}: For \emph{a finite collection} of \emph{disjoint} sets $\set{E_{i}: 1\le i\le k} \subset \srA_{0}$, 
$$ m\paren{\bigcup_{i=1}^{k}E_{i}}= \sum_{i=1}^{k}m(E_{i})$$.
\end{enumerate}
\end{definition}

\item \begin{remark}
From the property above, the following properties hold
\begin{enumerate}
\item \textit{\textbf{Monotonicity property}}: If $E \subseteq F$, then 
\begin{align*}
m(E) \le m(F),
\end{align*}

\item \emph{\textbf{Finitely sub-additive}}: For a finite collection of sets $\set{E_{i}: 1\le i\le k} \subset \srA_{0}$, 
$$ m\paren{\bigcup_{i=1}^{k}E_{i}} \le  \sum_{i=1}^{k}m(E_{i}).$$
\end{enumerate}
\end{remark}

\item \begin{remark}
A \emph{\textbf{box}} in $\bR^d$ is a Cartesian product $B := I_1 \xdotx{\times} I_d$ of $d$ intervals $I_1 \xdotx{,} I_d$ (not necessarily of the same length), thus for instance \emph{an \textbf{interval}} is a one-dimensional box. The \emph{\textbf{volume}} $\abs{B}$ of such a box $B$ is defined as $\abs{B} := \abs{I_1} \xdotx{\times}  \abs{I_d}$. \emph{An \textbf{elementary set}} is any \emph{subset} of $\bR^d$ which is the \emph{union} of a\emph{ finite number of boxes}.
\end{remark}

\item \begin{remark}
The collection of all \emph{elementary sets} forms a \emph{\textbf{boolean algebra}}. That is, if $E, F \subset \bR^d$ are elementary sets, then the \emph{union} $E \cup F$, the \emph{intersection} $E \cap F$, and \emph{the set theoretic difference} $E \setminus F := \{x \in E : x \not\in F \}$, and \emph{the symmetric difference} $E \Delta F := ( E \setminus F ) \cup (F \setminus E)$ are also elementary.
\end{remark}

\item \begin{exercise} (\textbf{Uniqueness of elementary measure}). \citep{tao2011introduction}\\
Let $d \le 1$. Let $\widetilde{m} : E(\bR^d) \rightarrow \bR_{+}$ be a map from the collection $E(\bR^d)$ of elementary subsets of $\bR^d$ to the nonnegative reals that obeys the \emph{non-negativity}, \emph{finite additivity}, and \emph{translation invariance properties}. Show that
there exists a constant $c \in \bR_{+}$ such that $\widetilde{m}(E) = c\,m(E)$ for all elementary sets $E$. In particular, if we impose the additional normalisation $\widetilde{m}(E)([0; 1)^d) = 1$, then $\widetilde{m} \equiv m$. (Set $c := \widetilde{m}([0, 1)^d) $, and then compute $\widetilde{m}(E)([0, \frac{1}{n})^d)$ for any positive integer $n$.)
\end{exercise}
\end{itemize}
 
 
 \subsection{Jordan Measure}
 \begin{itemize}
 \item  \begin{definition}
 A generalized measure of set $E$ can be induced by \emph{elementary measure} of subset that \emph{\textbf{inscribed}} $F$ or \emph{\textbf{circumscribed}} $G$ of it: $F\subseteq E \subseteq G$. 
 \begin{itemize}
\item The \underline{\emph{\textbf{outer Jordan measure}}} is defined as 
\begin{align*}
m^{*, J}(E) &= \inf\limits_{G\in \srA_{0}, G\supseteq E}m(G)
\end{align*}

\item The \underline{\emph{\textbf{inner Jordan measure}}} is defined as 
\begin{align*}
m_{*, J}(E) &= \sup\limits_{F\in \srA_{0}, F\subseteq E}m(F)
\end{align*}

\item If $m^{*, J}(E)  = m_{*, J}(E)$, then $E$ is \underline{\emph{\textbf{Jordan measureable}}} and denote $m(E) \equiv m^{*, J}(E)  = m_{*, J}(E). $
\end{itemize} 
\end{definition}

\item \begin{remark}
Jordan measurable sets are those sets which are ``\emph{almost elementary}" with respect to Jordan outer measure.
\end{remark}

\item The Jordan measure has following properties: 
\begin{enumerate}
\item \emph{Non-negative}: $m(E) \ge 0$, for all $E\subset \bR^{n}$, $E$ is Jordan measureable;
\item \emph{Translation-invariant}: $m\paren{\mb{x}+ E} = m(E)$ for any $\mb{x}\in \bR^{n}$; 
\item \emph{Finitely additive}: For a finite collection of \emph{disjoint} sets $\set{E_{i}: 1\le i\le k} \subset \bR^{n}$ and Jordan measureable,  
$$ m\paren{\bigcup_{i=1}^{k}E_{i}}= \sum_{i=1}^{k}m(E_{i}).$$
\item \emph{Finitely sub-additive}: For a finite collection of Jordan measureable sets $\set{E_{i}: 1\le i\le k}$, 
$$ m\paren{\bigcup_{i=1}^{k}E_{i}} \le  \sum_{i=1}^{k}m(E_{i}).$$
\item \emph{Monotonicity}:  If $E \subseteq F$, then $m(E) \le m(F)$.
\item \emph{Boolean closure}: if $E, F \subset \bR^d$ are Jordan measurable sets, then the \emph{union} $E \cup F$, the \emph{intersection} $E \cap F$, and \emph{the set theoretic difference} $E \setminus F := \{x \in E : x \not\in F \}$, and \emph{the symmetric difference} $E \Delta F := ( E \setminus F ) \cup (F \setminus E)$ are also Jordan measurable.
\end{enumerate}

\item \begin{proposition} (\textbf{Characterisation of Jordan measurability}). \citep{tao2011introduction}
Let $E \subseteq \bR^d$ be \textbf{bounded}. Show that the following are \textbf{equivalent}:
\begin{enumerate}
\item $E$ is Jordan measurable.
\item For every $\epsilon > 0$, there exist elementary sets $A \subseteq E \subseteq B$ such that $m(B \Delta A) \le  \epsilon$.
\item For every $\epsilon > 0$, there exists an elementary set $A$ such that $m^{*, J}(A \Delta E) \le \epsilon$.
\end{enumerate}
\end{proposition}

\item Example of Jordan measureable set:
\begin{itemize}
\item Every \emph{\textbf{elementary set}} $E$ is \emph{Jordan measurable}.

\item Every \emph{\textbf{compact convex polytope}} in $\bR^d$ is \emph{Jordan measurable}.

\item All \emph{\textbf{open and closed Euclidean balls}} $B(x; r) := \{y \in \bR^d : \norm{y - x}{2} < r\}$, $\overline{B(x; r)} := \{y \in
\bR^d: \norm{y - x}{2} \le r\}$ in $\bR^d$ are \emph{Jordan measurable}, with Jordan measure $c_d r^d$ for some constant $c_d > 0$ depending only on $d$.

\item The \emph{\textbf{graph of continuous function}} $f: B\rightarrow \bR$ for $B$ compact in $\bR^{n}$, $G = \{(\mb{x}, f(\mb{x})), \mb{x}\in B\} \subset \bR^{n+1}$ is \emph{Jordan measurable}, with $m(G) = 0$.

\item The \emph{\textbf{epigraph of continuous function}} $f: B\rightarrow \bR$ as defined above is the set $\{ (\mb{x}, t): 0\le t\le f(\mb{x}), \mb{x}\in B \} \subset \bR^{n+1}$ is  \emph{Jordan measurable}. 
\end{itemize}

\item \begin{exercise} (\textbf{Uniqueness of Jordan measure}). \citep{tao2011introduction}\\
Let $d \le 1$. Let $\widetilde{m} : \mathcal{J}(\bR^d) \rightarrow \bR_{+}$ be a map from the collection $\mathcal{J}(\bR^d)$ of elementary subsets of $\bR^d$ to the nonnegative reals that obeys the \emph{non-negativity}, \emph{finite additivity}, and \emph{translation invariance properties}. Show that
there exists a constant $c \in \bR_{+}$ such that $\widetilde{m}(E) = c\,m(E)$ for all elementary sets $E$. In particular, if we impose the additional normalisation $\widetilde{m}(E)([0; 1)^d) = 1$, then $\widetilde{m} \equiv m$.
\end{exercise}

\item \begin{exercise}
Show that \textbf{the bullet-riddled square} $[0, 1]^2 \cap \bQ^2$, and set of \textbf{bullets} $[0, 1]^2 \setminus \bQ^2$, both have\emph{ Jordan inner measure zero}
and \emph{Jordan outer measure one}. In particular, \textbf{both sets are not Jordan measurable}.
\end{exercise}

\item \begin{remark}
Informally, any set with a lot of ``\emph{holes}", or a very ``\emph{fractal}" boundary, is unlikely to be Jordan measurable. In order to measure such sets we will need to develop \emph{Lebesgue measure}.
\end{remark}

\item \begin{exercise} (\textbf{Area interpretation of the Riemann integral}).  \citep{tao2011introduction}\\ 
Let $[a, b]$ be an interval, and let $f: [a, b] \rightarrow \bR$ be a \textbf{bounded} function. Show that $f$ is \textbf{Riemann integrable} \textbf{if and only if} the sets $E_{+}:=
\{(x, t) : x \in [a, b]; 0 \le t \le f(x)\}$ and $E_{-} := \{ (x; t) : x \in [a, b];  f(x) \le t \le 0\}$ are both \textbf{Jordan measurable} in $\bR^2$, in which case one has
\begin{align*}
\int_{a}^{b} f(x) dx &= m^2(E_{+}) - m^2(E_{-})
\end{align*} where $m^2$ denotes two-dimensional \textbf{Jordan measure}. 
\end{exercise}
\end{itemize}

\section{Lebesgue Measure}
\subsection{Lebesgue outer measure}
\begin{itemize}
\item \begin{remark} 
\emph{\textbf{The countable union}} of \emph{disjoint Jordan measureable sets} may not be \emph{Jordan measureable}. 

\begin{exercise}
Show that the countable union $\bigcup_{n=1}^{\infty}E_n$ or countable intersection $\bigcap_{n=1}^{\infty}E_n$ of Jordan measurable sets $E_1, E_2, \ldots \subset \bR$
need not be Jordan measurable, even when bounded.
\end{exercise}


Also, for $E = \set{\mb{x}_{1}, \ldots, \mb{x}_{n}} \subset \bR^{n}$, the Jordan outer measure $m^{*, J}$ could be very large. For example, $m^{*, J}\paren{\bQ\cap [-R, R]} = 2\,R$ as $[-R, R]$ is the closure of them. 
\end{remark} 
 
\item \begin{definition}
Define the \underline{\emph{\textbf{Lebesgue outer measure}}} \citep{tao2011introduction}
 \begin{align}
 m^{*}(E) &= \inf\limits_{E \subseteq \cup_{k=1}^{\infty}G_{k}, \atop {\forall G_{k}\in \srA_{0}}}\sum_{k=1}^{\infty}m(G_{k})
 \label{eqn: L_outer_measure}
 \end{align} That is, if $E$ has a \emph{\textbf{a countable covering of elementary sets}} $\set{G_k} \subset \srA_{0}$, then \emph{the Lebesgue outer measure} is the \underline{\emph{\textbf{infimum of the countable sum}}} of \emph{the elementary measures} of these sets. Here \emph{\textbf{the countable sum}} is defined as the supremum over $k\ge 1$ of the $k$-summation 
\begin{align*}
\sum_{n=1}^{\infty}a_{n} &= \sup_{k\ge 1}\sum_{n=1}^{k}a_{n}
\end{align*}
\end{definition}
 
\item \begin{remark} 
Compare to \emph{the Lebesgue outer measure} with \emph{the Jordan outer measure} below,
\begin{align*}
m^{*, J}(E) &=  \inf\limits_{E \subseteq \cup_{k=1}^{n}G_{k}, \atop {\forall G_{k}\in \srA_{0}}}\sum_{k=1}^{n}m(G_{k}),
\end{align*} we see that the Jordan outer measure is \emph{the infimal cost} required to \emph{cover} $E$ by a \emph{\textbf{finite union of boxes}}, while the Lebesgue outer measure is that for \emph{\textbf{a countable infinite union of boxes}}. When the countable sum is infinite, the Lebesgue outer measure is also infinite. 

Moreover, we can show that  $m^{*}(E) \le m^{*,J}(E)$. This is because  we can always \emph{\textbf{pad out}} a finite union of boxes into \emph{an infinite union} by adding \emph{an infinite number of \textbf{empty boxes}}. %we can find $E \subseteq \cup_{k=1}^{\infty}\widehat{G}_{k} \subset \cup_{k=1}^{m}G_{k}$ for any  given $\set{G_{k}: 1\le k \le m}$, $m \in \bN$. 
 \end{remark} 
 
 
\item  \begin{remark} 
Note that the similar defined ``\emph{\textbf{Lebesgue inner measure}}" \emph{does not improve over the Jordan inner measure}, due to the \emph{subadditivity} of the measure. 
\end{remark}  
 
 
\item \begin{proposition} 
 The Lebesgue outer measure $m^{*}: 2^{\bR^{n}} \rightarrow \bR_{+}$ satisfies the following three properties:
\begin{enumerate}
\item \textbf{Empty-set}: $m^{*}(\emptyset) = 0$;
\item \textbf{Monotonicity}: If $E\subset F$, then $m^{*}(E) \le m^{*}(F)$;
\item \textbf{Countably subadditivity}:  For any countable union of sets $\set{E_{i}}_{i\ge 1}$ in $\srA$ 
\begin{align*}
 m^{*}\paren{\bigcup_{i=1}^{\infty}E_{i}} \le  \sum_{i=1}^{\infty}m^{*}(E_{i}).
\end{align*}
\end{enumerate} 

Conversely, any set function $m^{*}: \srA \rightarrow \bR$ on the $\sigma$-algebra $\srA$ on $X$ that satisfies the three axioms above is called \textbf{an outer measure}. \citep{rudin1987real, royden1988real, folland2013real}   
\end{proposition}

%\begin{remark}
%Note that \emph{\textbf{the finite subadditivity}} is deduced from the axiom 3  and axiom 1.
%\end{remark}  
\item \begin{lemma} (\textbf{Finite additivity for separated sets}).\\
 Let $E; F \subset \bR^{d}$ be such that $\text{dist}(E; F) > 0$, where
 \begin{align*}
 \text{dist}(E;F) &= \inf\set{\norm{\mb{x} - \mb{y}}{2}\;|\; \mb{x}\in E, \mb{y}\in F }
 \end{align*} is the distance between two sets $E,F$. 
Then $m^{*}(E\cup F) = m^{*}(E) + m^{*}(F)$. 
\end{lemma}
\begin{proof}
It suffice to prove that $m^{*}(E\cup F) \ge m^{*}(E) + m^{*}(F)$ and the other direction is the subadditivity. Suppose $m^{*}(E\cup F)< \infty$. (It is trivial to have infinite outer measure.) 

For any $\epsilon>0$, we can cover the $E\cup F$ by countably infinite boxes $B_{1}, \cdots,$ such that 
\begin{align*}
\sum_{n=1}^{\infty}\abs{B_{n}} \le m^{*}(E\cup F) + \epsilon. 
\end{align*}
Suppose that each of these boxes intersects at most one of $E$ and $F$. Note that for those boxes that intersect both $E$ and $F$, we can partition them into smaller pieces with diameter $r< \text{dist}(E;F)$. This guarantee that each box only intersect one set. 

We divide these boxes into two parts $B'_{1}, \cdots,$ and $B''_{1}, \cdots, $, which only intersects $E$ and $F$, respectively. Clearly, the first subfamily covers $E$ and the second covers $F$.

By definition of Lebesgue outer measure, 
\begin{align*}
 m^{*}(E) \le \sum_{n=1}^{\infty}\abs{B'_{n}}
\end{align*}
and 
\begin{align*}
 m^{*}(F) \le \sum_{n=1}^{\infty}\abs{B''_{n}}
\end{align*}
Summing up these two terms, we have
\begin{align*}
m^{*}(E)+ m^{*}(F) &\le \sum_{n=1}^{\infty}\abs{B_{n}} \le m^{*}(E\cup F) + \epsilon
\end{align*}
so 
\begin{align*}
m^{*}(E)+ m^{*}(F) &\le m^{*}(E\cup F), 
\end{align*}
which completes the proof. \qed
\end{proof}

\item \begin{lemma} \label{lem: outer_elementary_measure} (\textbf{Outer measure of elementary sets}) \\
Let $E$ be an \textbf{elementary set}. Then the Lebesgue outer measure $m^{*}(E)$ of $E$ is equal to the elementary measure $m(E)$ of $E$: $m^{*}(E) = m(E)$.
\end{lemma}
\begin{proof} %It is essential to apply the separable property of the $\bR^{n}$: find a dense countable subset of $\bR^{n}$ to form a countable union of boxes. 
Note that $m^{*}(E) \le m^{*,J}(E) = m(E)$, so it suffice to prove $m(E) \le m^{*}(E)$. 

\begin{enumerate}
\item If $E$ is closed, since elementary set $E$ is bounded, then by Heine-Borel theorem, $E$ is compact. Then for any $\epsilon>0$, a countable family of boxes $B_{1}, \cdots,$ will cover $E$ and 
\begin{align*}
E &\subset \bigcup_{k=1}^{\infty}B_{k}\\
\sum_{k=1}^{\infty}\abs{B_{k}} &\le m^{*}(E)+ \epsilon.
\end{align*}
This family of boxes need not to be open, while we can add one more $\epsilon>0$ so that the above inequality holds for the family of open boxes $B'_{1}, \cdots$, where $B_{k}\subset B'_{1}$ and $\abs{B'_{k}}\le \abs{B_{k}} + \epsilon/2^{n}$, so that 
\begin{align*}
E &\subset \bigcup_{k=1}^{\infty}B'_{k}\\
\sum_{k=1}^{\infty}\abs{B'_{k}} &\le \sum_{k=1}^{\infty}\abs{B_{k}}+ \sum_{k=1}^{\infty}\frac{\epsilon}{2^{n}}  \le m^{*}(E)+ 2\epsilon.
\end{align*} 
 
Then by compactness, there are finite subcover $\set{B'_{1}, \ldots, B'_{n}}$ of $E$; i.e. 
\begin{align*}
E &\subset \bigcup_{k=1}^{n}B'_{k}\\
m(E) &\le \sum_{k=1}^{n}\abs{B'_{k}},
\end{align*} where the last inequality holds due to the subadditivity elementary measure. 
Note that
\begin{align*}
\sum_{k=1}^{n}\abs{B'_{k}}\le  \sum_{k=1}^{\infty}\abs{B'_{k}}&\le m^{*}(E)+ 2\epsilon,
\end{align*}
so
\begin{align*}
m(E) &\le m^{*}(E)+ 2\epsilon,
\end{align*}
for all $\epsilon>0$. It suffice to show that $m(E) \le m^{*}(E)$.

\item If $E$ is not closed, we can partition $E$ as a finite collection of disjoint boxes $Q_{1}\cup\cdots Q_{m}$, which need not to be closed. Then for any $\epsilon>0$, for any $1\le j\le m$, so that there exists closed box $Q'_{j} \subset Q_{j}$ such that $\abs{Q'_{j}} \ge \abs{Q_{j}}- \epsilon/m$.

Then $E$ contains a finite collection of closed boxes $Q'_{1}, \cdots, Q'_{m}$, so 
\begin{align*}
m(\bigcup_{j=1}^{m}Q'_{j})=\sum_{j=1}^{m}\abs{Q'_{j}} &\ge \sum_{j=1}^{m}\abs{Q_{j}}- \sum_{j=1}^{m}\frac{\epsilon}{m}\\
&= m(E)- \epsilon,
\end{align*}
for $\forall \epsilon>0$.

By monotonicity of outer measure, we see that
\begin{align*}
m^{*}(E)&\ge m^{*}(\bigcup_{j=1}^{m}Q'_{j})\\
&\ge m(\bigcup_{j=1}^{m}Q'_{j})\\
&\ge m(E)- \epsilon, \; \forall \epsilon>0, 
\end{align*} so $m^{*}(E) \ge m(E)$. 
\end{enumerate}
This completes the whole proof. \qed.
\end{proof}

\item \begin{proposition} (\textbf{Lebesgue outer measure vs. Jordan outer / inner measure})\\
For any subset $E \subseteq \bR^{n}$, we have the following relation between the Lebesgue outer measure and the Jordan outer and inner measure.
\begin{align*}
m_{*, J}(E) \le m^{*}(E) \le m^{*, J}(E)
\end{align*}
\end{proposition}
\begin{proof}
Note that we have already shown the upper bound before.  Suppose that for some elementary set $F \subseteq E$, $m(F)$ attained the Jordan inner measure of $E$, i.e. $m_{*, J}(E) = m(F)$.  Lemma \ref{lem: outer_elementary_measure} shows that the outer measure of all elementary sets are the elmentary measures. So $m(F) = m^{*}(F) = m_{*, J}(E)$. By monotonicity, $m^{*}(F) \le m^{*}(E)$ thus proved the lower bound. \qed
\end{proof} 

\item \begin{lemma}
A collection of sets are \textbf{almost disjoint}, if their \textbf{interiors} are disjoint. Show that for $E= \bigcup_{k=1}^{\infty}B_{k}$, where $B_{1},\cdots$ are countable collection of almost disjoint boxes, then 
\begin{align*}
m^{*}(E) &= \sum_{k=1}^{\infty}\abs{B_{k}}
\end{align*}
\end{lemma}
\begin{proof}
Due to the subadditivity, $m^{*}(E) \le  \sum_{k=1}^{\infty}\abs{B_{k}}$. We need to show that $m^{*}(E) \ge  \sum_{k=1}^{\infty}\abs{B_{k}}$.

Note that since $B_{1},\cdots$ are almost disjoint, their volumes does not change by taking the interior, so $m(\bigcup_{k=1}^{n}B_{k}) = \sum_{k=1}^{n}\abs{B_{k}}$ for any $n$. 

It is seen that finite union of subcollections $\bigcup_{k=1}^{n}B_{k} \subset E$ for any $n\ge 1$, it follows by monotonicity that 
\begin{align*}
m^{*}(E) \ge  \sum_{k=1}^{n}\abs{B_{k}}.
\end{align*} 
Take both size for $n\rightarrow \infty$, we have the desired result. \qed.
\end{proof}

\item \begin{lemma} \label{lem: open_disjoint}
Let $E\subset \bR^{d}$ be an \textbf{open set}. Then $E$ can be expressed as \textbf{the countable union of almost disjoint boxes} (and, in fact, as the countable union of almost disjoint closed cubes).
\end{lemma}
\begin{proof}
we use the \emph{dyadic mesh}, which is a discretized structure in $\bR^{d}$. Define a \emph{closed dyadic cube} to be a cube $Q$ of the form
\begin{align*}
Q_{n} &= \brac{\frac{i_{1}}{2^{n}}, \frac{i_{1}+1}{2^{n}}} \times \cdots\times \brac{\frac{i_{d}}{2^{n}}, \frac{i_{d}+1}{2^{n}}},
\end{align*} for $n \in \bN, i_{1}, \cdots, i_{d} \in \bZ$. It has side length $\frac{1}{2^{n}}$. These cubes for $i_{1}, \cdots, i_{d} \in \bZ$ are almost disjoint and covers $\bR^{d}$.

Also given $Q_{n}$, $\exists Q_{n-1}$, such that $Q_{n} \subset Q_{n-1}$ and $Q_{n-1}$ can be partitioned into $2^{d}$ $Q_{n}$'s. As a consequence of these facts, we also obtain the important \emph{dyadic nesting property}: given any two closed dyadic cubes (possibly of different side-length), either they are \emph{almost disjoint}, or one of them is \emph{contained in the other} (, since the grid points are integers. ).

For any $E$ open, $\mb{x}\in E$, then there exists an open neighborhood $U \ni \mb{x}$ and $U\subset E$. Note that there also exists a closed dyadic cube $Q_{n}$ such that $\mb{x}\in Q_{n} \subset U\subset E$ for some $n$. Let $\cQ$ be the collection of all closed dyadic cubes that are contained in $E$, so $\bigcup_{Q\in \cQ}Q \subseteq E$. It is also clear  that $\bigcup_{Q\in \cQ}Q \supseteq E$, since $\cQ$ is a closed cover of $E$, thus $\bigcup_{Q\in \cQ}Q  = E$.

Note that $\cQ$ should be countable as the collection of all $Q$ is countable. To make sure they are almost disjoint, we use the nested property. Note that $\cQ$ is endowed with the partial order relation as proper inclusion. Since any simply-ordered  subcollection of $\cQ$ has an upper bound, then $\cQ$ has maximal elements. Let $\cQ'$ denote those cubes in $\cQ$ that are maximal.  By definition of maximal and the dyadic nested property, the elements in $\cQ'$ are almost disjoint and $\bigcup_{Q\in \cQ'}Q  = E$. As $\cQ'$ is at most countable, we have proved the claim. \qed
\end{proof}

\item \begin{lemma}  \label{lem: outer_regular_outer_measure} (\textbf{Outer regularity})  \citep{tao2011introduction} \\
Let $E\subset \bR^{d}$ be arbitrary set.  Then one has
\begin{align*}
m^{*}(E) &= \inf\limits_{E\subset U, \;U\text{open}}m^{*}(U). 
\end{align*}
\end{lemma}
\begin{proof}
For monotonicity, $m^{*}(E) \le \inf\limits_{E\subset U, \;U\text{open}}m^{*}(U)$, so we only need to show $m^{*}(E) \ge \inf\limits_{E\subset U, \;U\text{open}}m^{*}(U)$. Assume that $m^{*}(E)< \infty$. 

By definition of outer measure of $E$, there exists a countable collection of boxes $B_{1},\cdots$ that covers $E$ and 
\begin{align*}
\sum_{k=1}^{\infty}\abs{B_{k}} &\le m^{*}(E) + \epsilon
\end{align*}
for any $\epsilon>0$. Note that we can enlarge these boxes by open boxes $B'_{1},\cdots$ such that $B_{k}\subseteq B'_{k}$ and $\abs{B'_{k}}\le \abs{B_{k}}+ \epsilon/2^{k}$. Note that $\bigcup_{k=1}^{\infty}B'_{k} \supset E$ and it is open cover, but 
\begin{align*}
m^{*}\paren{\bigcup_{k=1}^{\infty}B'_{k}}&\le \sum_{k=1}^{\infty}\abs{B'_{k}} \\
&\le m^{*}(E) + 2\epsilon
\end{align*}
Thus
\begin{align*}
\inf\limits_{E\subset U, \;U\text{open}}m^{*}(U) &\le m^{*}(E) + 2\epsilon
\end{align*} for any $\epsilon>0$, which proves the claim. \qed
\end{proof}

\item \begin{remark}
Lemma \ref{lem: outer_regular_outer_measure} shows that under \emph{\textbf{the Euclidean topology}} of $\bR^{d}$, \emph{the Lebesgue outer measure} is \underline{\emph{\textbf{regular}}}; i.e.,  let $E\subset \bR^{d}$ be arbitrary set.  Then one has
\begin{enumerate}
\item \underline{\emph{\textbf{outer regular}}} 
\begin{align}
m^{*}(E) &= \inf\limits_{E\subset U, \;U\text{ \textbf{open}}}m^{*}(U). \label{expr: outer_regular}
\end{align} and
\item \underline{\emph{\textbf{inner regular}}}
\begin{align}
m^{*}(E) &= \sup\limits_{E\supset C, \;C\text{ \textbf{compact}}}m^{*}(C). \label{expr: inner_regular}
\end{align}
\end{enumerate}
\end{remark}

\item \begin{exercise} 
Give an example to show that the reverse statement
\begin{align*}
m^{*}(E) &= \sup\limits_{E\supset U, \;U\text{ open}}m^{*}(U)
\end{align*} is \textbf{false}.
 \end{exercise} Note see the \emph{\textbf{example}} section.
\end{itemize}
\subsection{Lebesgue measure}
\begin{itemize}
\item \begin{definition}
A set $E\subset \bR^{d}$ is \underline{\emph{\textbf{Lebesgue measureable}}} if and only if for any $\epsilon>0$, there exists \textbf{open set} $U$ that \textbf{contains} $E$ such that  $m^{*}(U \setminus  E)< \epsilon$. 
\end{definition}
We refer $m(E)= m^{*}(E)$ as the \underline{\emph{\textbf{Lebesgue measure}}} of $E$, for $E$ is \emph{\textbf{Lebesgue measureable}}. 

\item \begin{lemma} (\textbf{Existence of Lebesgue measurable sets}).\\
The following sets are \textbf{Lebesgue measureable sets}
\begin{enumerate}
\item Any \textbf{open sets} or \textbf{closed set} in $\bR^{d}$; The empty set $\emptyset$ is both open and closed, so it is Lebesgue measurable.
\item Any sets with \textbf{Lebesgue outer measure zero}; (called \textbf{null set})
\item If $E\subset \bR^{d}$ is Lebesgue measureable, then the \textbf{complement} $E^{c}$ is also Lebesgue measureable.
\item Any \textbf{countable union} of Lebesgue measureable sets, $\bigcup_{n=1}^{\infty}E_{n}$ is Lebesgue measureable.
\item  Any \textbf{countable intersection} of Lebesgue measureable sets, $\bigcap_{n=1}^{\infty}E_{n}$ is Lebesgue measureable.
\end{enumerate}
\end{lemma}
\begin{proof}
\begin{itemize}
\item \emph{The proof of 1. the open set being Lebesgue measurable} follows the definition where $U =E$. \emph{2. The null set is also Lebesgue measurable} by definition.   
\item We prove that \emph{any closed set is Lebesgue measureable}. 
For any closed set $E$, we need to show that, for any $\epsilon$, there exists open $U\supset E$ such that 
\begin{align*}
m^{*}(U \setminus  E) &\le \epsilon. 
\end{align*}

By outer regularity, for any $\epsilon$, there exists open $U\supset E$ such that 
\begin{align*}
m^{*}(U) &\le m^{*}(E)+ \epsilon. 
\end{align*}

Note that $U \setminus  E$ is open, so it can be decomposed into a countable collection of closed dyadic cubes $Q_{1},\cdots,$ that are \emph{almost disjoint}.  Thus
\begin{align*}
U&= E\cup \paren{\bigcup_{k=1}^{\infty}Q_{k}};\\
\text{and }m^{*}(U \setminus  E) &\le \sum_{k=1}^{\infty}\abs{Q_{k}}
\end{align*}
Since $E$ and $Q_{k}'$s are almost disjoint, 
\begin{align*}
m^{*}(U) &=  m^{*}(E) + \sum_{k=1}^{\infty}\abs{Q_{k}} \\
&\le m^{*}(E)+ \epsilon \quad (\text{by construction of }U),
\end{align*}
and $m^{*}(E)\le \infty$, so $ \sum_{k=1}^{\infty}\abs{Q_{k}} \le \epsilon \Rightarrow m^{*}(U  \setminus   E) \le \epsilon$, which complete our proof.

\item We prove \emph{4. any countable union of Lebesgue measureable sets is Lebesgue measureable.} Let $\epsilon  > 0$ be arbitrary. By hypothesis, each $E_n$ is contained in an open set $U_n$ whose difference $U_n \setminus  E_n$ has Lebesgue outer measure at most $\epsilon/2^n$. By \emph{countable subadditivity}, this implies that $\bigcup_{n=1}^{\infty}E_{n}$ is contained in $\bigcup_{n=1}^{\infty}U_{n}$, and the difference $(\bigcup_{n=1}^{\infty}U_{n}) \setminus  (\bigcup_{n=1}^{\infty}E_{n})$ has Lebesgue outer measure at most $\epsilon$. The set $\bigcup_{n=1}^{\infty}U_{n}$, being \emph{a union of open sets}, is itself \emph{open}, and the claim follows.

\item Now we prove \emph{3. the complement of a Lebesgue measureable set is Lebesgue measureable}.   If $E$ is Lebesgue measurable, then for every $n$ we can find an open set $U_n$ containing $E$ such that
\begin{align*}
m^{*}(U_n \setminus  E) \le \frac{1}{n}.
\end{align*} Letting $F_n$ be the \emph{complement} of $U_n$, we conclude that the complement $\bR^d \setminus  E$ of $E$ contains all of the $F_n$, and that $m^{*}((\bR^d \setminus  E) \setminus  F_n) \le \frac{1}{n}$. If we let $F := \bigcup_{n=1}^{\infty} F_n$, then $\bR^d \setminus  E$ contains $F$, and from \emph{monotonicity} $m^{*}((\bR^d \setminus  E) \setminus  F ) = 0$, thus $\bR^d \setminus  E$ is the union of $F$ and a set of \emph{Lebesgue outer measure zero}. But $F$ is
in turn \emph{the union of countably many closed sets} $F_n$. The claim now follows from statement 1., 2., 4.

\item The statement 5. is the result of statement 3, 4. and \emph{de Morgan’s laws}. \qed
\end{itemize}
\end{proof}

\item \begin{remark} 
Based on above Lemma, the \emph{collection of all Lebesgue measureable set} in $\bR^{d}$ form a \emph{\textbf{$\sigma$-algebra}}, called \emph{\textbf{\underline{Borel $\sigma$-algebra}}} $\cB^{d}$.
\end{remark}

\item \begin{remark} (\emph{\textbf{Lebesgue Measure vs. Jordan Measureable}})\\
Now we look at the Lebesgue measure $m(E)$ of a \emph{Lebesgue measurable set} $E$, which is defined to equal its \emph{Lebesgue outer measure} $m^{*}(E)$. If $E$ is \emph{Jordan measurable}, we see from last section that \emph{the Lebesgue measure and the Jordan measure of $E$ coincide}, thus \textbf{\emph{Lebesgue measure extends Jordan measure}}. This justifies the use of the notation $m(E)$ to denote both \emph{Lebesgue measure} of \emph{a Lebesgue measurable set}, and \emph{Jordan measure} of \emph{a Jordan measurable set} (as well as \emph{elementary measure} of \emph{an elementary set})
\end{remark}


\item \begin{remark} 
Note that by outer regularity \ref{expr: outer_regular}, there always exists $U$ open containing $E$ such that $m^{*}(U)\le m^{*}(E)+\epsilon$. However, \emph{\textbf{outer measure does not preserve the set difference}}, i.e., $m^{*}(U \setminus  E)\ge m^{*}(U)- m^{*}(E)$. 
\end{remark}

\item  \begin{lemma} 
The \textbf{Lebesgue measure}  $m: \srA \rightarrow \bR_{+}$, where $\srA$ is $\sigma$-algebra \textbf{containing} Borel sets,  satisfies the following properties:
\begin{enumerate}
\item $m(\emptyset) = 0$;
\item \underline{\textbf{Countably additivity}}:  For any countable union of disjoint sets $\set{E_{i}}_{i\ge 1}$ in $\srA$ 
\begin{align*}
 m\paren{\bigcup_{i=1}^{\infty}E_{i}} =  \sum_{i=1}^{\infty}m(E_{i}).
\end{align*} This also infers \textbf{the Finitely-additivity}.
%\item \emph{Finitely-additivity} (derived from above): For any finite union of disjoint sets $\set{E_{i}}_{1\le i \le k}$ in $\srA$ 
%\begin{align*}
% m\paren{\bigcup_{i=1}^{k}E_{i}} =  \sum_{i=1}^{k}m(E_{i}).\\
%\end{align*}
\end{enumerate} 
\end{lemma}
\begin{proof} We just need to show number 2.
\begin{itemize}
\item Case 1: \emph{all of $E_{i}$ are \textbf{compact}}:  Note that for two sets are disjoint, one is closed and one is compact, then \emph{their distance} must above zero. Thus the outer measure is finitely additiveable. $m\paren{\bigcup_{i=1}^{n}E_{i}}= \sum_{i=1}^{n}m\paren{E_{i}}$. 

By monotonicity, $m\paren{\bigcup_{i=1}^{\infty}E_{i}} \ge m\paren{\bigcup_{i=1}^{n}E_{i}}= \sum_{i=1}^{n}m\paren{E_{i}}$. Then we take $n\rightarrow \infty$ on both sides, $m\paren{\bigcup_{i=1}^{\infty}E_{i}} \ge \sum_{i=1}^{\infty}m\paren{E_{i}}$. 

And by countable subadditivity, $m\paren{\bigcup_{i=1}^{\infty}E_{i}} \le \sum_{i=1}^{\infty}m\paren{E_{i}}$. Then it completes the proof. 

\item  Case 2: \emph{$E_{i}$ are \textbf{bounded}}:  Since $E_{i}$ are measureable set, by inner regularity, we see that for every $E_{n}$,  any $\epsilon>0$, there exists a closed (bounded as a subset, thus \emph{\textbf{compact}}) set $K_{n}\subset E_{n}$ and 
\begin{align*}
m(E_{n}) &\le m(K_{n})+ \frac{\epsilon}{2^{n}}.
\end{align*} 
Hence 
\begin{align*}
m(\bigcup_{n=1}^{\infty}E_{n}) &\le \sum_{n=1}^{\infty}m(E_{n}) \le \sum_{n=1}^{\infty}m(K_{n})+ \epsilon
\end{align*}
and for compact $K_{n}$, we have
\begin{align*}
\sum_{n=1}^{\infty}m(K_{n})&= m\paren{\bigcup_{n=1}^{\infty}K_{n}}.
\end{align*} 
By monotonicity, 
\begin{align*}
m\paren{\bigcup_{n=1}^{\infty}K_{n}} &\le m\paren{\bigcup_{n=1}^{\infty}E_{n}},
\end{align*}
so 
\begin{align*}
 \sum_{n=1}^{\infty}m(E_{n}) &\le m\paren{\bigcup_{n=1}^{\infty}E_{n}},
\end{align*}
which complete our proof. 

\item Case 3: \emph{general measureable $E_{i}$}:  The basic idea is to \emph{\textbf{decompose}} $E_{i}$ as \emph{\textbf{countable disjoint union}} of \emph{\textbf{bounded}} measureable sets. 

First, decompose $\bR^{d}$ as the countable disjoint union $\bR^{d} = \bigcup_{m=1}^{\infty}A_{m}$ of bounded measurable sets $A_{m}$. This is due to the separable property of $\bR^{d}$. Therefore each $E_{n}$ can be decomposed as a countable disjoint union of bounded measureable sets $E_{n}\cap A_{m}, m=1,\cdots,$ and 
\begin{align*}
m\paren{E_{n}} &= \sum_{m=1}^{\infty}m\paren{E_{n}\cap A_{m}}.
\end{align*}
and also $\bigcup_{n=1}^{\infty}E_{n}= \bigcup_{n=1}^{\infty}\bigcup_{m=1}^{\infty}E_{n}\cap A_{m}$ with countable disjoint union of bounded measureable sets. Therefore, 
\begin{align*}
m\paren{\bigcup_{n=1}^{\infty}E_{n}} &= \sum_{n=1}^{\infty}\sum_{m=1}^{\infty}m\paren{E_{n}\cap A_{m}}\\
&= \sum_{n=1}^{\infty}m\paren{E_{n}} ,
\end{align*} which completes our proof. \qed
\end{itemize}
\end{proof}

\item \begin{remark}
Due to \emph{the countably additivity property}, \emph{\textbf{Lebesgue measure}} obeys significantly \emph{\textbf{better}} properties than \emph{\textbf{Lebesgue outer measure}}.
\end{remark}

\item \begin{proposition} (\textbf{Criteria for measurability})  \citep{tao2011introduction} \\
The followings are equivalent:
\begin{enumerate}
\item $E$ is Lebesgue measureable.
\item (\textbf{Outer approximation} by \textbf{open}) For every $\epsilon>0$, one can contain $E$ in an open set $U$ with $m^{*}(U \setminus  E)\le \epsilon $.
\item (\textbf{Almost open}) For every $\epsilon>0$, one can find an open set $U$ such that  $m^{*}(U\Delta E)\le \epsilon $, where $U\Delta E = (U  \setminus  E)\cup (E  \setminus  U) = (U\cup E) \setminus  (U\cap E)$ is the symmetric difference. (In other words, $E$ differs from an open set by a set of outer measure at most $\epsilon$.)
\item (\textbf{Inner approximation} by \textbf{closed}) For every $\epsilon>0$, one can find a closed set $F$ contained in $E$ with $m^{*}(E  \setminus  F)\le \epsilon $.
\item (\textbf{Almost closed}) For every $\epsilon>0$, one can find a closed set $F$ such that $m^{*}(E\Delta F)\le \epsilon $. (In other words, $E$ differs from a closed set by a set of outer measure at most $\epsilon$.)
\item (\textbf{Almost measurable}) For every $\epsilon>0$, one can find a Lebesgue measurable set $E_{\epsilon}$ such that $m^{*}(E\Delta E_{\epsilon})\le \epsilon $. (In other words, E differs from a measurable set $E_{\epsilon}$ by a set of outer measure at most $\epsilon$.)
\end{enumerate}
\end{proposition}
\begin{proof}
\begin{itemize}
\item (1) $\Rightarrow$ (2) is the definition;  
\item (2) $\Rightarrow$ (3): given that every $\epsilon>0$, one can contain $E$ in an open set $U$ with $m^{*}(U  \setminus  E)\le \epsilon/2 $, we want to show that $E$ is almost open as cited above. 

Note that $U\supset E$, so $E  \setminus  U = \emptyset$ and since $E  \setminus  U$ and $U  \setminus  E$ are disjoint, $m^{*}(U\Delta E)= m^{*}((U  \setminus  E)\cup (U  \setminus  V) )= m^{*}(E  \setminus  U)+ m^{*}(U  \setminus  E) = m^{*}(U  \setminus  E)\le \epsilon$, which completes the proof. 

\item (3) $\Rightarrow$ (4): For every $\epsilon>0$, one can find an open set $U$ such that  $m^{*}(U\Delta E)\le \epsilon/2 $. We need to show that for every $\epsilon>0$, one can find a closed set $F$ contained in $E$ with $m^{*}(E-F)\le \epsilon $. 

Note that $m^{*}(U\Delta E)= m^{*}(E \setminus  U)+ m^{*}(U  \setminus  E) \le \epsilon$, where $U$ is open. Decompose $E = (E\cap U)\cup (E  \setminus  U)$ and $m^{*}(E) = m^{*}(E\cap U) + m^{*}(E  \setminus  U)$. If $m^{*}(E\cap U)=0$, then $m^{*}(E) = m^{*}(E  \setminus  U)\le \epsilon$,  then $F=\emptyset$ and $m^{*}(E \setminus  F)= m^{*}(E)\le \epsilon$.

Suppose $m^{*}(E\cap U)>0$ and $m^{*}(E  \setminus  U) \le \epsilon/2$ with $\epsilon/2\le m^{*}(U \setminus  E) \le \epsilon$. The open set $U$ can be decomposed by a countable collection of almost disjoint closed dyadic cubes $Q_{1},\cdots$, as $U = \bigcup_{k=1}^{\infty}Q_{k}$. Choose a subcollection of $Q'_{1}, \cdots$ that intersects $E$ and $Q''_{1},\cdots$ are the rest of cubes that included in $U-E$, which result in
\begin{align*}
U = \bigcup_{k=1}^{\infty}Q_{k} &= \bigcup_{k=1}^{\infty}Q'_{k} + \bigcup_{k=1}^{\infty}Q''_{k}
\end{align*}
where $E\cap U\subset \bigcup_{k=1}^{\infty}Q'_{k}$ and $\bigcup_{k=1}^{\infty}Q''_{k}\subset U-E$.% Among $Q'_{1},\cdots$, we can find a finite collection $Q'_{k_{1}},\cdots Q'_{k_{n}}$ such that $F=\bigcup_{i=1}^{n}Q'_{k_{i}} \subset E\cap U$ so that $F$ is closed and
%\begin{align*}
%E- U &\subset E- F  \subset E\\
%m^{*}(E-U) &\le m^{*}(E-F) \le m^{*}(E)
%\end{align*}

Note that 
\begin{align*}
%\sum_{k=1}^{\infty}\abs{Q''_{k}}& \le  m^{*}(U-E) \le \epsilon/2\\
m^{*}(E\cap U) &\le \sum_{k=1}^{\infty}\abs{Q'_{k}}\\
\epsilon/2\ge m^{*}(E  \setminus  U)& \ge m^{*}(E) - \sum_{k=1}^{\infty}\abs{Q'_{k}}\\
m^{*}(E)\ge \sum_{k=1}^{\infty}\abs{Q'_{k}}& \ge m^{*}(E) -\epsilon/2.
\end{align*}
For $m^{*}(E)<\infty$, we see that $\sum_{k=1}^{\infty}\abs{Q'_{k}}<\infty$, so for given $\epsilon>0$, there exists $N\in \bN$ such that for all $m\ge N$
\begin{align*}
\sum_{k=m}^{\infty}\abs{Q'_{k}} < \epsilon/2;
\end{align*}
So we can choose a collection of $Q'_{1},\cdots, Q'_{m}$, $m\ge N$ such that $\bigcup_{k=1}^{m}Q'_{k} \subset E\cap U$. Note that it is possible since $m^{*}(E)\le m^{*}(E\cap U)+ \epsilon $ and $m^{*}(E)\ge \sum_{k=1}^{\infty}\abs{Q'_{k}} \ge m^{*}(E) -\epsilon/2$, thus $m^{*}(E\cap U) \ge m^{*}(\bigcup_{k=1}^{m}Q'_{k})$ for large $m$. 

We define $F\equiv \bigcup_{k=1}^{m}Q'_{k}$, and it is a closed set. Also $ \bigcup_{k=1}^{m}Q'_{k}= F \subset E\cap U \subset \bigcup_{k=1}^{\infty}Q'_{k}$. Then 
\begin{align*}
E-F &\subseteq E- \bigcup_{k=1}^{m}Q'_{k} \\
&= (E\cap U) \cup (E \setminus  U)- \bigcup_{k=1}^{m}Q'_{k}\\
&=\paren{E\cap U-  \bigcup_{k=1}^{m}Q'_{k}}\cup (E \setminus  U)\\
&\subseteq \paren{\bigcup_{k=m}^{\infty}Q'_{k}} \cup (E  \setminus  U)\\
m^{*}(E-F)&\le \sum_{k=m}^{\infty}\abs{Q'_{k}} + m^{*}(E  \setminus  U)\\
&\le \epsilon/2+ \epsilon/2 = \epsilon, 
\end{align*}which completes the proof. 

\item (4) $\Rightarrow$ (5) is similar to (2) $\Rightarrow$ (3): We just see that $F\subset E$, so $F \setminus  E = \emptyset$. So 
$m^{*}(E\Delta F)= m^{*}((E  \setminus  F)\cup (F \setminus  E) )= m^{*}(E  \setminus  F)+ m^{*}(F \setminus  E) = m^{*}(E  \setminus  F)\le \epsilon$, which completes the proof. 

\item (5) $\Rightarrow$ (6). It is trivial since any closed set $F = E_{\epsilon}$ is mesureable as required. 

\item (6) $\Rightarrow$ (1): Given  every $\epsilon>0$, one can find a Lebesgue measurable set $E_{\epsilon}$ such that $m^{*}(E\Delta E_{\epsilon}) =m^{*}(E- E_{\epsilon}) + m^{*}(E_{\epsilon}  \setminus  E) \le \epsilon/3 $. First, since $E_{\epsilon}$ is Lebesgue measurable set, there exists open set $U_{\epsilon}\supset E_{\epsilon}$ such that $m^{*}(U_{\epsilon}  \setminus  E_{\epsilon})\le \epsilon/3$. We need to modify $U_{\epsilon}$ to be $U\supset E$ and $m^{*}(U  \setminus  E)\le \epsilon$.

Second, consider for any open set $U\supset E$, we decompose $U \setminus  E$ as 
\begin{align}
U  \setminus  E&= ((U \setminus E)\cap U_{\epsilon})\cup ((U \setminus E)\cap (U \setminus U_{\epsilon}))\nonumber\\
&= ((U \setminus E)\cap (U_{\epsilon}  \setminus E_{\epsilon}))\cup ((U \setminus E)\cap E_{\epsilon}) \cup ((U \setminus E)\cap (U \setminus U_{\epsilon}))\nonumber\\
m^{*}(U \setminus E)&\le m^{*}((U \setminus E)\cap (U_{\epsilon}  \setminus E_{\epsilon}))+ m^{*}((U \setminus E)\cap E_{\epsilon}) + m^{*}((U \setminus E)\cap (U \setminus U_{\epsilon}))  \label{eqn: mearability_lemma_1}
\end{align}
Note that 
\begin{align}
m^{*}((U \setminus E)\cap (U_{\epsilon}  \setminus E_{\epsilon}))&\le m^{*}(U_{\epsilon}  \setminus E_{\epsilon}) \le \epsilon/3 \nonumber\\
m^{*}((U \setminus E)\cap (U \setminus U_{\epsilon}))&\le m^{*}(U \setminus U_{\epsilon})\nonumber\\
m^{*}((U \setminus E)\cap E_{\epsilon})&\le  m^{*}(E_{\epsilon}  \setminus  E)\le \epsilon/3  \label{eqn: mearability_lemma_2}
\end{align}
Then our goal is to find $U\supset E$ such that $m^{*}(U \setminus U_{\epsilon})< \epsilon/3$.

We can decompose $U_{\epsilon}$ into countable union of almost disjoint cubes $Q_{1},\cdots$, as $U_{\epsilon} = \bigcup_{k=1}^{\infty}Q_{k}$ and let $Q'_{1},\cdots,$ are those cubes that meet $E$, so $ E_{\epsilon} \subset \bigcup_{k=1}^{\infty}Q'_{k}$. We can enlarge each $Q'_{k}$ as open set $B_{k}$ so that $m^{*}(B_{k} \setminus  Q'_{k}) \le \frac{1}{6}\epsilon/2^{k}$. Then $ E_{\epsilon} \subset \bigcup_{k=1}^{\infty}B_{k}$ and $E\subset \paren{\bigcup_{k=1}^{\infty}B_{k}}\cup (E \setminus E_{\epsilon}) \subset \paren{\bigcup_{k=1}^{\infty}B_{k}}\cup V$, where open  set  $V\supset (E \setminus E_{\epsilon})$ with $m^{*}(V)\le \epsilon$.

Finally, let $U= \paren{\bigcup_{k=1}^{\infty}B_{k}}\cup V\supset E$ be the open set we need. Hence
\begin{align}
U \setminus U_{\epsilon} &= \paren{\bigcup_{k=1}^{\infty}B_{k}}\cup V  \setminus  \paren{\bigcup_{k=1}^{\infty}Q'_{k}}\cup\paren{U_{\epsilon} \setminus  \bigcup_{k=1}^{\infty}Q'_{k}}\nonumber\\
&\subset \brac{ \paren{\bigcup_{k=1}^{\infty}B_{k}}  \setminus  \paren{\bigcup_{k=1}^{\infty}Q'_{k}}}\cup V^{*}\nonumber\\
&\subset \paren{\bigcup_{k=1}^{\infty}\brac{B_{k} \setminus  Q'_{k}}}\cup V^{*};\nonumber\\
m^{*}(U \setminus U_{\epsilon})&\le \sum_{k=1}^{\infty}m^{*}(B_{k} \setminus  Q'_{k})+ m^{*}(V^{*})\nonumber\\
&\le \sum_{k=1}^{\infty}\frac{1}{6}\frac{\epsilon}{2^{k}} + \frac{1}{6}\epsilon \nonumber\\
&= \frac{1}{3}\epsilon \label{eqn: mearability_lemma_3}
\end{align}
where $V^{*}= V\cup \paren{U_{\epsilon} \setminus  \bigcup_{k=1}^{\infty}Q'_{k}}$ is a null set with outer measure 
\begin{align*}
m^{*}(V^{*})&\le m^{*}(V)+ m^{*}( U_{\epsilon} \setminus  E_{\epsilon})\\
&\le \frac{1}{12}\epsilon + \frac{1}{12}\epsilon = \frac{1}{6}\epsilon
\end{align*}
Note that $\paren{U_{\epsilon} \setminus  \bigcup_{k=1}^{\infty}Q'_{k}} \subset U_{\epsilon} \setminus  E_{\epsilon}$ and $m^{*}(U_{\epsilon} \setminus  E_{\epsilon}) \le \epsilon/12$, so $\paren{U_{\epsilon} \setminus  \bigcup_{k=1}^{\infty}Q'_{k}}$ is a null set. 

Substituting \eqref{eqn: mearability_lemma_3}, and \eqref{eqn: mearability_lemma_2} into \eqref{eqn: mearability_lemma_1}, we have $m^{*}(U \setminus E) \le \epsilon$, which completes our proof.  \qed
\end{itemize}
\end{proof}

\item \begin{proposition}
The Lebesgue measure satisfying the following property
\begin{enumerate}
\item (\textbf{Upward monotone convergence}) Let $E_{1}\subseteq E_{2} \cdots $ be countable \textbf{non-decreasing} nested sets, we have 
$m\paren{\bigcup_{k=1}^{\infty}E_{k}} = \lim\limits_{k\rightarrow \infty}m\paren{E_{k}}$.

\item (\textbf{Downward monotone convergence}) Let $E_{1}\supseteq E_{2} \cdots $ be countable \textbf{non-increasing} nested sets, and \textbf{if at least one $E_{k}$ has finite measure} $m(E_{k})<\infty$, we have $m\paren{\bigcap_{k=1}^{\infty}E_{k}} = \lim\limits_{k\rightarrow \infty}m\paren{E_{k}}$.\\
\end{enumerate}
\end{proposition}

\item \begin{proposition} (\textbf{Carath\'{e}odory criterion}):   \citep{tao2011introduction}\\
Let $E\subset \bR^{d}$, the followings are equivalent:
\begin{enumerate}
\item $E$ is \textbf{Lebesgue measurable};
\item For every \textbf{elementary set} $A\subset \bR^{d}$, one has $m(A) = m^{*}(A \setminus  E) + m^{*}(A\cap E)$.
\item For every \textbf{box} $B$, one has $\abs{B} = m^{*}(B \setminus   E) + m^{*}(B\cap E).$
\end{enumerate}  
\end{proposition}
\begin{proof}
(1) $\Rightarrow$ (2). We see that both $A$ and $E$ are Lebesgue measureable, so does $A \setminus E$ and $A\cap E$. Then since $A= (A \setminus E)\cup (A\cap E)$ for two disjoint set, then by countable additivity, 
\begin{align*}
m(A) = m\paren{ (A \setminus E)\cup (A\cap E)} &= m(A \setminus E) + m(A\cap E)= m^{*}(A \setminus E) + m^{*}(A\cap E).
\end{align*}

(2) $\Rightarrow$ (3). Trivial, as the box $B$ is an elementary set. 


(2) $\Rightarrow$ (1). To show $E$ is measureable, we see to show that for any $\epsilon>0$, there exists an open subset $U\supset E$ such that $m^{*}(U \setminus E)\le \epsilon$. Suppose $m^{*}(E) <\infty $. By definition of outer measure,  for any $\epsilon>0$, there exists a countable collection of elementary sets $A_{1},\cdots$ so that $E\subset \bigcup_{k=1}^{\infty}A_{k}$ and $\sum_{k=1}^{\infty}m(A_{k}) \le m^{*}\paren{E} + \epsilon/2$. Then since elementary set are measurable, there exists a countable collection of open sets $U_{1},\cdots$ so that $A_{k}\subset U_{k}$, $m^{*}(U_{k} \setminus A_{k})\le \epsilon/2^{k+1}$. 

Let $U= \bigcup_{k=1}^{\infty}U_{k}$ open and  $E\subset \bigcup_{k=1}^{\infty}A_{k}\subset U$. Consider $U \setminus E \supset \bigcup_{k=1}^{\infty}A_{k} \setminus  E$, as 
\begin{align*}
U \setminus E &= \paren{U \setminus \bigcup_{k=1}^{\infty}A_{k}}\cup \paren{\bigcup_{k=1}^{\infty}A_{k} \setminus  E}\\
&= \paren{\bigcup_{k=1}^{\infty}U_{k}\cap \bigcap_{k=1}^{\infty}A_{k}^{c}}\cup \paren{\bigcup_{k=1}^{\infty}(A_{k}\cap E^{c})}\\
&\subset    \bigcup_{k=1}^{\infty}\paren{U_{k}\cap A_{k}^{c}}\cup \paren{\bigcup_{k=1}^{\infty}(A_{k}\cap E^{c})}\\
m^{*}(U \setminus E) &\le \sum_{k=1}^{\infty}m^{*}(U_{k} \setminus A_{k}) + \sum_{k=1}^{\infty}m^{*}(A_{k}  \setminus  E)\\
&\le \epsilon/2\sum_{k=1}^{\infty}\frac{1}{2^{k}} + \sum_{k=1}^{\infty}m^{*}(A_{k}  \setminus  E)\\
&= \epsilon/2 + \sum_{k=1}^{\infty}m^{*}(A_{k}  \setminus  E)\\
&\le \epsilon/2 + \epsilon/2 = \epsilon
\end{align*}
The last inequality comes from 
\begin{align*}
\sum_{k=1}^{\infty}m^{*}(A_{k}  \setminus  E)&= \sum_{k=1}^{\infty}m(A_{k})  \setminus  \sum_{k=1}^{\infty}m^{*}(A_{k}\cap E) \quad \text{ (by additivity assumption)} \\
&\le m^{*}\paren{E} + \epsilon/2  \setminus   \sum_{k=1}^{\infty}m^{*}(A_{k}\cap E)\\
&\le m^{*}\paren{E} + \epsilon/2  \setminus  m^{*}(\bigcup_{k=1}^{\infty}A_{k}\cap E)\\
&= m^{*}\paren{E} + \epsilon/2  \setminus  m^{*}\paren{E} = \epsilon/2 \quad (\text{since }E\subset \bigcup_{k=1}^{\infty}A_{k})
\end{align*}

(3) $\Rightarrow$ (1)  Trivial, as the box $B$ is an elementary set. \qed
\end{proof}

\item \begin{proposition}(\textbf{Inner regularity}). \\
Let $E \subset \bR^d$ be Lebesgue measurable, then
\begin{align*}
m(E)& = \sup\limits_{K\subset E, \; K \text{ is compact}}m(K)
\end{align*}
\end{proposition}
\begin{proof}
For $E$ is Lebesgue measureable, for any $\epsilon>0$, we can find a \textbf{\emph{closed}} subset $K\subset E$, such that $m^{*}(E \setminus  K)\le \epsilon$. If $m(E)=\infty$, it is then clear that $m(K)=\infty$. 

Suppose $m(E)<\infty$, we only need to show that $K$ is bounded, then for any $\epsilon>0$, there exists compact (i.e. closed and bounded set) $K$ such that  $m^{*}(E \setminus  K)\le \epsilon$, so $m(E)\le m^{*}(K)+ \epsilon$. Then $m(E)= \sup\limits_{K\subset E, \; K \text{ is compact}}m(K)$.

Clearly if $E$ is bounded, $K$ is bounded. If else, $E = E'\cup S$, where $E'$ is bounded with $S$ unbounded but $m(S)=0$. Then choose $K\subset E'$, then $K$ is bounded. That is, if $E$ is finite measureable, then $K$ is as required. This completes our proof. \qed
\end{proof}

\item \begin{remark}
The \emph{\textbf{inner}} and \emph{\textbf{outer regularity}} properties of measure can be used to define the concept of a \underline{\emph{\textbf{Radon measure}}}.
\end{remark}

\item \begin{proposition} \citep{tao2011introduction}\\
Any Lebesgue measureable set $E$ can be seen as
\begin{enumerate}
\item $G \setminus  N$, where $G$ is a \underline{$G_{\delta}$ set} (i.e. $\cap_{n=1}^{\infty}U_n$ for \textbf{open sets} $U_n$) and $N$ is a null set; or
\item $F\cup N$ where $F$ is a \underline{$F_{\sigma}$ set} (i.e.  $\cup_{n=1}^{\infty}F_n$ for \textbf{closed sets} $F_n$) and $N$ is a null set.
\end{enumerate}
\end{proposition}
 
\item \begin{proposition} (\textbf{Translation invariance}). \\
If $E \subseteq \bR^d$ is Lebesgue measurable, show that $E + x$ is Lebesgue measurable for any $x \in \bR^d$, and that $m(E + x) = m(E)$.
\end{proposition}



\item \begin{remark} (\emph{\textbf{Change of Variables}}). \\
If $E \subseteq \bR^d$ is Lebesgue measurable, and $T: \bR^d \rightarrow \bR^d$ is a \emph{linear transformation}, $T(E)$ is \emph{Lebesgue measurable}, and that $m(T(E)) = \abs{\det{T}}m(E)$. We caution that if $T: \bR^d \rightarrow \bR^{d'}$ is a linear map to a space $\bR^{d'}$ of strictly \emph{\textbf{smaller dimension}} than $\bR^d$, then $T(E)$ \emph{need not be Lebesgue measurable};
\end{remark}

\item \begin{proposition} (\textbf{Uniqueness of Lebesgue measure}).\\
Lebesgue measure $E \mapsto m(E)$ is \textbf{the only map} from Lebesgue measurable sets to $[0, +\infty]$ that obeys the following \textbf{axioms}:
\begin{enumerate}
\item (\textbf{Empty set})  $m(\emptyset) = 0$.
\item \underline{\textbf{Countably additivity}}:  For any countable union of disjoint Lebesgue measurable sets $\set{E_{i}}_{i\ge 1}$ in $\srA$ 
\begin{align*}
 m\paren{\bigcup_{i=1}^{\infty}E_{i}} =  \sum_{i=1}^{\infty}m(E_{i}).
\end{align*}
\item \underline{(\textbf{Translation invariance})} If $E$ is Lebesgue measurable and $x \in \bR^d$, then $m(E + x) = m(E)$.
\item \underline{(\textbf{Normalisation})}  $m([0, 1]^d) = 1$.
\end{enumerate}
\end{proposition}


\item \begin{exercise}
\emph{Lebesgue measure} can be viewed as a \underline{\textbf{metric completion}} of \textbf{elementary measure}. \citep{tao2011introduction};
\begin{enumerate}
\item Let $2^A := \set{E : E \subseteq A}$ be the power set of $A$. We say that two sets $E, F \in 2^A$ are equivalent if $E \Delta F$ is a null set. Show that this is a \textbf{equivalence relation}.
\item Let $2^A / \sim$ be the set of equivalence classes $[E] := \set{F \in 2^A : E \sim F}$ of $2^A$ with respect to the above equivalence relation. Define a \textbf{distance} $d: 2^A /\sim \times 2^A/\sim \rightarrow \bR_{+}$ between two equivalence classes $[E], [E']$ by defining $d([E], [E']) := m^{*}(E \Delta E')$. 

Show that this distance is well-defined (in the sense that $m(E \Delta E') = m(F \Delta F')$ whenever $[E] = [F]$ and $[E'] = [F']$) and gives $2^A /\sim$ the structure of \textbf{a complete metric space}.
\item Let $\cE \subset 2^A$ be the \textbf{elementary subsets} of $A$, and let $\cL \subset 2^A$ be the \textbf{Lebesgue measurable subsets} of $A$. Show that $\cL/\sim$
is the \textbf{closure} of $E/\sim$ with respect to the \textbf{metric} defined above.

In particular,  $\cL/\sim$ is a \textbf{complete metric space} that contains $E/\sim$ as a \textbf{dense} subset; in other words, $\cL/\sim$ is a \textbf{metric completion} of $E/\sim$.
\end{enumerate}
\end{exercise}

\item \begin{remark} (\emph{\textbf{Lebesgue Measurable Sets Do Not Cover All Subsets}})\\
There exists a subset $E \subset [0, 1]$ which is not Lebesgue measurable. (by axiom of choice). Thus the Lebesgue measure is not defined on the whole power set of $[0,1]$.

Consider \emph{\textbf{the quotient group}} $\bR / \bQ = \set{x + \bQ: x \in \bR}$ and let $E := \set{x_C \in C\cap [0,1]: C \in \bR / \bQ}$ be the collection of all the coset representatives $x_C \in C \cap [0,1]$. Note that each coset $C$ of $\bR / \bQ$ is \emph{\textbf{dense}} in $\bR$ so $C \cap [0,1] \neq \emptyset$. We can show that \emph{\textbf{$E$ is not Lebesgue measurable}}.


On the other hand, one can construct \emph{\textbf{finitely additive translation invariant extensions} }of Lebesgue measure to the power set of $\bR$ by the Hahn-Banach theorem.
\end{remark}

\item \begin{example} \emph{\textbf{Projections of measurable sets need not be measurable}}: \\
Let $\pi : \bR^2 \rightarrow \bR$ be the coordinate projection $\pi(x; y) := x$. Then there exists a measurable subset $E$ of $R^{2}$ such that $m(E)$ is not measurable.
\end{example}

\item \begin{remark}
Recall from the beginning that there is no hope to have \emph{countably additivity}, \emph{translation invariance} and \emph{normalisation} for all subsets in $\bR^{d}$. But we can see that there is a large collection of Lebesgue measurable sets that fit all three desireable properties. We will see that the rest are all with \emph{\textbf{zero measures}}. 
\end{remark}
\end{itemize}


\newpage
\section{The Development of Measure Theory}
\subsection{Definition Summary}
\begin{enumerate}
\item Begin with the \emph{\textbf{minimal algebra}} $\srA_{0}$ generated by \emph{a collection of all boxes} $\bigotimes_{i=1}^{n}(a_{i},b_{i}] \subset \bR^{n}$, \underline{\emph{\textbf{the elementary measure}}} is \emph{a generalization of volumes} in $\bR^{n}$: 
\begin{enumerate}
\item \emph{\textbf{Non-negative}}: $m(E) \ge 0$, for all $E\in \srA_{0}$;
\item $m(\emptyset) = 0$;
\item $m((0,1]^{n})= 1$;
\item \emph{\textbf{Translation-invariant}}: $m\paren{\mb{x}+ E} = m(E)$ for any $\mb{x}\in \bR^{n}$; 
\item \emph{\textbf{Finitely additive}}: For a finite collection of \emph{disjoint} sets $\set{E_{i}: 1\le i\le k} \subset \srA_{0}$, 
$$ m\paren{\bigcup_{i=1}^{k}E_{i}}= \sum_{i=1}^{k}m(E_{i}).$$
\end{enumerate}
From the property above, the following properties hold
\begin{enumerate}
\item \textit{\textbf{Monotonicity property}}: If $E \subseteq F$, then 
\begin{align*}
m(E) \le m(F),
\end{align*}

\item \emph{\textbf{Finitely sub-additive}}: For a finite collection of sets $\set{E_{i}: 1\le i\le k} \subset \srA_{0}$, 
$$ m\paren{\bigcup_{i=1}^{k}E_{i}} \le  \sum_{i=1}^{k}m(E_{i}).$$
\end{enumerate}

\begin{remark}
The elementary set has a lot of desireable properties but it is \emph{\textbf{very limited}}. For instance, open balls, and triangles are not counted as elementary set. Similar for many convex polytopes etc.
\end{remark}

\item For \emph{\textbf{arbitrary bounded subset}} $E\subset \bR^{n}$, it is possible that $E\not\in \srA_{0}$. \underline{\emph{\textbf{The Jordan measure}}} is proposed to \emph{generalize} the elementary measure $m$ on $E$, 
 \begin{itemize}
\item The \underline{\emph{\textbf{outer Jordan measure}}} is defined as 
\begin{align*}
m^{*, J}(E) &= \inf\limits_{G\in \srA_{0}, G\supset E}m(G)
\end{align*}

\item The \underline{\emph{\textbf{inner Jordan measure}}} is defined as 
\begin{align*}
m_{*, J}(E) &= \sup\limits_{F\in \srA_{0}, F\subset E}m(G)
\end{align*}

\item If $m^{*, J}(E)  = m_{*, J}(E)$, then $E$ is \emph{\textbf{Jordan measureable}} and denote $m(E) \equiv m^{*, J}(E)  = m_{*, J}(E).$
\end{itemize} 

 \emph{The collection of Jordan measurable sets} form an \emph{\textbf{algebra}} $\srA_{1} \supset \srA_{0}$ on $\bR^{n}$ and \emph{\textbf{the measure function}} extended on $\srA_{1}$ preserve the property as above: 
\begin{enumerate}
\item \emph{Non-negative}: $m(E) \ge 0$, for all $E\subset \bR^{n}$, $E$ is Jordan measureable;
\item \emph{Translation-invariant}: $m\paren{\mb{x}+ E} = m(E)$ for any $\mb{x}\in \bR^{n}$; 
\item \emph{Finitely additive}: For a finite collection of \emph{disjoint} sets $\set{E_{i}: 1\le i\le k} \subset \bR^{n}$ and Jordan measureable,  
$$ m\paren{\bigcup_{i=1}^{k}E_{i}}= \sum_{i=1}^{k}m(E_{i}).$$
\item \emph{Finitely sub-additive}: For a finite collection of Jordan measureable sets $\set{E_{i}: 1\le i\le k}$, 
$$ m\paren{\bigcup_{i=1}^{k}E_{i}} \le  \sum_{i=1}^{k}m(E_{i}).$$
\item \emph{Monotonicity}:  If $E \subseteq F$, then $m(E) \le m(F)$.
\end{enumerate}

\begin{remark}
The algebra $\srA_{1}$ formed by all Jordan measureable sets are much larger than that of elementary set $\srA_{0}$. It can be shown that the Jordan measures are closely related to \emph{\textbf{the Riemann integral}}.  However, like the Riemann integral, the definition of Jordan measure sets still has a lot of \emph{\textbf{limitations}}, esp. when dealing with sets that are \emph{\textbf{countable infinite union}} of Jordan measureable sets. In general, when the set has a lot of ``holes" or very ``fractal", the set is not likely Jordan measurable. Thus, we need to generalize the definition of Jordan measure to cover the limit of sets.
\end{remark}

\item \underline{\emph{\textbf{Lebesgue outer measure}}} is defined as 
\begin{align*}
m^{*}(E) &= \inf\set{ \sum_{i=1}^{\infty}m(B_{i})\Big|\;  E\subset \bigcup_{i=1}^{\infty}B_{i}, B_{i}\in \srA_{0}}.
\end{align*}

The \emph{\textbf{Lesbuge outer measure}} $m^{*}: 2^{X}=\srP(\bR^{n}) \rightarrow \bR_{+}$ satisfies the following three properties:
\begin{enumerate}
\item $m^{*}(\emptyset) = 0$;
\item \emph{Monotonicity}: If $E\subset F$, then $m^{*}(E) \le m^{*}(F)$;
\item \emph{\textbf{Countably subadditivity}}:  For any \emph{countable union of subsets} $\set{E_{i}}_{i\ge 1}$ in $\bR^{n}$ 
\begin{align*}
 m^{*}\paren{\bigcup_{i=1}^{\infty}E_{i}} \le  \sum_{i=1}^{\infty}m^{*}(E_{i}).
\end{align*}
\end{enumerate} 
Note that outer measure does not need to be defined on $\sigma$-algebra. 


\item A set $E\subset \bR^{d}$ is \underline{\emph{\textbf{Lebesgue measureable}}} if and only if for any $\epsilon>0$, there exists \emph{\textbf{open set}} $U$ that contains $E$ such that  $m^{*}(U-E)< \epsilon$. If $E$ is \emph{Lebesgue measureable}, the outer measure of $E$ is called \underline{\emph{\textbf{Lebesgue measure}}}, $m(E)= m^{*}(E)$.

 In other word, \emph{\textbf{the Lebesgue measure}} $m: \srA \rightarrow \bR_{+}$, where $\srA$ is \underline{\emph{\textbf{$\sigma$-algebra}}} containing \emph{Borel sets},  satisfies the following properties:
\begin{enumerate}
\item $m(\emptyset) = 0$;
\item \emph{\textbf{Countably subadditivity}}:  For any countable union of disjoint sets $\set{E_{i}}_{i\ge 1}$ in $\srA$ 
\begin{align*}
 m\paren{\bigcup_{i=1}^{\infty}E_{i}} =  \sum_{i=1}^{\infty}m(E_{i}).
\end{align*}
\item \emph{\textbf{Finitely-additivity} (derived from above)}: For any finite union of disjoint sets $\set{E_{i}}_{1\le i \le k}$ in $\srA$ 
\begin{align*}
 m\paren{\bigcup_{i=1}^{k}E_{i}} =  \sum_{i=1}^{k}m(E_{i}).
\end{align*}
\end{enumerate} 

The \emph{collection of \textbf{all Lebesgue measureable set}} form a \underline{\emph{\textbf{$\sigma$-algebra}}} $\srA\supset \srA_{1}$ (Borel sets in $\bR^{d}$).


\item Finally, we see that \emph{the collection of all Lebesgue measureable sets} is \emph{\textbf{the only one}} that contains all desired properties of \emph{measure}:
 \begin{enumerate}
\item (\textbf{\emph{Empty set}})  $m(\emptyset) = 0$.
\item \underline{\textbf{\emph{Countably additivity}}}:  For any countable union of disjoint Lebesgue measurable sets $\set{E_{i}}_{i\ge 1}$ in $\srA$ 
\begin{align*}
 m\paren{\bigcup_{i=1}^{\infty}E_{i}} =  \sum_{i=1}^{\infty}m(E_{i}).
\end{align*}
\item \underline{(\textbf{\emph{Translation invariance}})} If $E$ is Lebesgue measurable and $x \in \bR^d$, then $m(E + x) = m(E)$.
\item \underline{(\textbf{\emph{Normalisation}})}  $m([0, 1]^d) = 1$.
\end{enumerate}
\end{enumerate}
\newpage
\subsection{Table Summary}
\begin{table}[h!]
\setlength{\abovedisplayskip}{0pt}
\setlength{\belowdisplayskip}{-10pt}
\setlength{\abovedisplayshortskip}{0pt}
\setlength{\belowdisplayshortskip}{0pt}
\footnotesize
\centering
\caption{Comparison between different measures in measure theory}
\label{tab: measure}
%\setlength{\extrarowheight}{1pt}
\renewcommand\tabularxcolumn[1]{m{#1}}
\small
\begin{tabularx}{1\textwidth} { 
  | >{\raggedright\arraybackslash} m{3cm}
  | >{\centering\arraybackslash}X
  | >{\centering\arraybackslash}X
  | >{\centering\arraybackslash}X
  | >{\centering\arraybackslash}X  | }
 \hline
  &  \emph{\textbf{Elementary measure}} & \emph{\textbf{Jordan measure}}   &  \emph{\textbf{Lebesgue outer measure}}   & \emph{\textbf{Lebesgue measure}} \\
  \hline 
\textbf{\emph{compatibility}}    & & $\Leftarrow \checkmark$  & $\Leftarrow \checkmark$  & $\Leftarrow \checkmark$ \\
 \hline
\emph{non-negative} &  $\checkmark$  & $\checkmark$  & $\checkmark$  & $\checkmark$  \\
 \hline
$m(\emptyset) = 0$  &  $\checkmark$  & $\checkmark$  & $\checkmark$  & $\checkmark$  \\
\hline
$m([0,1]^d) = 1$  &  $\checkmark$  & $\checkmark$  & $\checkmark$  & $\checkmark$  \\
\hline \vspace{5pt}
\emph{\textbf{translation-invariant}}  \vspace{-5pt} &  $\checkmark$  & $\checkmark$  & $\checkmark$  & $\checkmark$ \\
\hline \vspace{5pt}
\emph{finitely additive}   \vspace{2pt}&  $\checkmark$  & $\checkmark$  & $\checkmark$  & $\checkmark$   \\
\hline \vspace{5pt}
\emph{monotonicity}   \vspace{2pt}&  $\checkmark$  & $\checkmark$  & $\checkmark$  & $\checkmark$   \\
\hline \vspace{5pt}
\emph{finitely subadditive}   \vspace{2pt}&  $\checkmark$  & $\checkmark$  & $\checkmark$  & $\checkmark$   \\
\hline \vspace{5pt}
\emph{outer regularity}  \vspace{2pt}  & &  &   $\checkmark$ & $\checkmark$ \\
\hline \vspace{5pt}
\emph{inner regularity}  \vspace{2pt}  & &  &  $\checkmark$  & $\checkmark$ \\
\hline \vspace{5pt}
\emph{countably subadditivity}  \vspace{2pt}  & &  & $\checkmark$  & $\checkmark$ \\
\hline \vspace{5pt}
\textbf{\emph{countably additivity}}  \vspace{2pt}  & &  &  & $\checkmark$ \\
\hline \vspace{5pt}
\textbf{\emph{measurable set}}  \vspace{2pt}  & box $I_1 \xdotx{\times} I_d$ & All \emph{elementary sets}; any \emph{\textbf{compact convex }polytope}; any \emph{\textbf{open sets}} and \emph{\textbf{closed sets}}; \emph{\textbf{finite union}} of measurable sets; \emph{\textbf{graph/epigraph} of \textbf{continous function}}; & All \emph{Jordan measurable sets}; \emph{\textbf{countable union}} of measurable sets, e.g. $G_{\delta}$ and $F_{\sigma}$  & forms a \emph{\textbf{$\sigma$-algebra}} that includes \emph{\textbf{all Borel sets}}; sets with Lebesgue outer measure zero (\emph{\textbf{null sets}}). \\
\hline \vspace{5pt}
\textbf{\emph{non-measurable set}}  \vspace{2pt}  & \emph{any subsets} other than box & \emph{countable union} of Jordan measurable sets; \emph{\textbf{bullet-riddled square}} and \emph{\textbf{sets of bullets}}; subsets with a lot of ``\emph{holes}" or ``\emph{fractal}" & same as right & $E = \bR/\bQ \cap [0,1]$ \\
\hline \vspace{5pt}
\emph{\textbf{algebra} for collection of measurable sets}  \vspace{2pt}  & \emph{\textbf{boolean algebra $\srA_0$}} & \emph{boolean algebra} $\srA_1 \supsetneq \srA_0$ &   & \emph{\textbf{$\sigma$-algebra}} $\srA_{2} \supsetneq \srA_1$ \\
\hline \vspace{5pt}
\emph{relation to \textbf{integration}}  \vspace{2pt}  & & \emph{\textbf{Riemann integration}}  &  & \emph{\textbf{Lebesgue integration}} \\
\hline
\end{tabularx}
\end{table}

\newpage
\section{Counterexamples}
\begin{itemize}
\item \begin{example}
For the countable set $\bQ\cap [0,1]$, it has countable open covers
\begin{align*}
U\equiv \bigcup_{k=1}^{\infty}(q_{k}-\epsilon/2^{n+1}, q_{k}+\epsilon/2^{n+1}), \quad q_{k}\in \bQ\cap [0,1],
\end{align*}
for any $\epsilon>0$.

The by countable subadditivity, 
\begin{align*}
m^{*}(U) &\le \sum_{k=1}^{\infty}m((q_{k}-\epsilon/2^{n}, q_{k}+\epsilon/2^{n}))\\
&= \sum_{k=1}^{\infty}\frac{\epsilon}{2^{n}} = \epsilon
\end{align*}

Also it is seen that since $U$ is dense in $[0,1]$, i.e., $\overline{U}\supseteq [0,1]$, therefore
\begin{align*}
m^{*,J}(U)= m^{*,J}(\overline{U}) \ge m^{*,J}([0,1]) = 1.
\end{align*}
It shows that \emph{\textbf{the Jordan outer measure is different from the Lebesgue outer measure}}. 

Also, it is seen that \emph{\textbf{bounded open set}} $U$ is \emph{\textbf{not Jordan measureable}}, but it is Lebesgue measureable. \qed
\end{example}

\item \begin{example}Give an example that satisfies the following 
\begin{align*}
m^{*}(E) &> \sup\limits_{E\supset U, \;U\text{ open}}m^{*}(U). 
\end{align*}
There are \emph{\textbf{Cantor sets}} $C$ that is \emph{\textbf{nowhere dense}} with positive measure. That is $m^{*}(C)>0$ but $C$ contains \emph{\textbf{no interval}} so it is $\sup\limits_{E\supset U, \;U\text{ open}}m^{*}(U)=0$.  

The set of irrational numbers  $[0,1]- \bQ\cap [0,1]$ has outer measures $1$ but contains no interval, so $\sup\limits_{E\supset U, \;U\text{ open}}m^{*}(U)=0$. \qed
\end{example}

\item \begin{example} (\emph{\textbf{Non-Lebesgue-Measurable Set in $[0,1]$}}) \\
Consider \emph{\textbf{the quotient group}} $\bR / \bQ = \set{x + \bQ: x \in \bR}$ and let $E := \set{x_C \in C\cap [0,1]: C \in \bR / \bQ}$ be \emph{the collection of all the coset representatives} $x_C \in C \cap [0,1]$. Note that each coset $C$ of $\bR / \bQ$ is \emph{\textbf{dense}} in $\bR$ so $C \cap [0,1] \neq \emptyset$. We can show that \emph{\textbf{$E$ is not Lebesgue measurable}}.

Let $y$ be any element of $[0, 1]$. Then it must lie in some coset $C$ of $\bR / \bQ$, and thus differs from $x_C$ by some \emph{rational number} in $[-1, 1]$. In other words, we have
\begin{align}
[0, 1] \subseteq  \bigcup_{q \in \bQ\cap [-1,1]}(E + q). \label{eqn: non_measureable_lower_bound}
\end{align} On the other hand, we clearly have
\begin{align}
\bigcup_{q \in \bQ\cap [-1,1]}(E + q) \subseteq  [-1, 2]. \label{eqn: non_measureable_upper_bound}
\end{align} Also, the different translates $E + q$ are \emph{\textbf{disjoint}}, because $E$ contains only one element from each \emph{coset} of $\bQ$.

To see why $E$ is not Lebesgue measurable, suppose \emph{for \textbf{contradiction}} that $E$ was Lebesgue measurable. Then the
translates $E + q$ would also be Lebesgue measurable. By countable additivity, we thus have
\begin{align*}
m\paren{\bigcup_{q \in \bQ\cap [-1,1]}(E + q)} &= \sum_{q \in \bQ\cap [-1,1]} m\paren{E + q}
\end{align*} and thus by translation invariance and \eqref{eqn: non_measureable_lower_bound}, \eqref{eqn: non_measureable_upper_bound} we have
\begin{align*}
1 = m([0, 1])   \le m\paren{\bigcup_{q \in \bQ\cap [-1,1]}(E)} \le m([-1, 2]) = 2-(-1) = 3
\end{align*} On the other hand, the sum $\sum_{q \in \bQ\cap [-1,1]} m\paren{E}$ is either \textbf{\emph{zero}} (if
$m(E) = 0$) or \textbf{\emph{infinite}} (if $m(E) > 0$), leading to the desired \emph{\textbf{contradiction}}. \qed
\end{example}
%
%
%
%\vspace{15pt}
%\item \begin{example}
%The $G_{\delta}$ set is a countable intersection of open sets and $F_{\sigma}$ set is a countable union of closed sets. Then the following are equivalent:
%\begin{enumerate}
%\item  $E$ is Lebesgue measurable;
%\item $E$ is a $G_{\delta}$ set with a null set removed;
%\item $E$ is the union of a $F_{\sigma}$ set and a null set.
%\end{enumerate}
%\end{example}
%\begin{remark}
%From the above exercises, we see that when describing what it means for a set to be Lebesgue measurable, there is a tradeoff between the type of approximation one is willing to bear, and the type of things one can say about the approximation. 
%
%If one is only willing to approximate to within a null set, then one can only say that a measurable set is approximated by \emph{a $G_{\delta}$ or a $F_{\sigma}$ set}, which is a fairly weak amount of structure. If one is willing to add on an epsilon of error (as measured in the Lebesgue outer measure), one can make a measurable set \emph{open}; dually, if one is willing to take away an epsilon of error, one can make a measurable set \emph{closed}. Finally, if one is willing to both add and subtract an epsilon of error, then one can make a measurable set (of finite measure) \emph{elementary}, or even a finite union of dyadic cubes.
%\end{remark}
\end{itemize}
\newpage
\bibliographystyle{plainnat}
\bibliography{reference.bib}
\end{document}