\documentclass[11pt]{article}
\usepackage[scaled=0.92]{helvet}
\usepackage{geometry}
\geometry{letterpaper,tmargin=1in,bmargin=1in,lmargin=1in,rmargin=1in}
\usepackage[parfill]{parskip} % Activate to begin paragraphs with an empty line rather than an indent %\usepackage{graphicx}
\usepackage{amsmath,amssymb, mathrsfs, dsfont}
\usepackage{tabularx}
\usepackage{tikz-cd}
\usepackage[all,cmtip]{xy}
\usepackage[font=footnotesize,labelfont=bf]{caption}
\usepackage{graphicx}
\usepackage{xcolor}
%\usepackage[linkbordercolor ={1 1 1} ]{hyperref}
%\usepackage[sf]{titlesec}
\usepackage{natbib}
%\usepackage{tikz-cd}

\usepackage{../../Tianpei_Report}

%\usepackage{appendix}
%\usepackage{algorithm}
%\usepackage{algorithmic}

%\renewcommand{\algorithmicrequire}{\textbf{Input:}}
%\renewcommand{\algorithmicensure}{\textbf{Output:}}



\begin{document}
\title{Lecture 10: Product Measure}
\author{ Tianpei Xie}
\date{Dec. 23rd., 2022}
\maketitle
\tableofcontents
\newpage
\section{Product $\sigma$-Algebra}
\begin{itemize}
\item \begin{definition} (\emph{\textbf{Product Topology}})\\
Let $X$ and $Y$ be topological spaces. \underline{\emph{\textbf{The product topology}}} on $X \times Y$ is the topology having as basis the collection $\srB$ of all sets of the form $U \times V$, where $U$ is an open subset of $X$ and $V$ is an open subset of $Y$.
\end{definition}

\begin{definition}
Let $\pi_X : X \times Y \rightarrow X$ be defined by the equation
\begin{align*}
\pi_X (x, y) = x;
\end{align*}
$\pi_Y : X \times Y \rightarrow Y$ he defined by the equation
\begin{align*}
\pi_Y (x, y) = y.
\end{align*}
The maps $\pi_X$ and $\pi_Y$ are called \emph{the \textbf{projections} of $X \times Y$ \textbf{onto} its \textbf{first} and \textbf{second} factors}, respectively.
\end{definition} 

\item \begin{definition} (\emph{\textbf{Product $\sigma$-Algebra}})\\
Suppose that $(X, \srB_X)$ and $(Y, \srB_Y)$ are measurable spaces. We can form the \underline{\emph{\textbf{pullback $\sigma$-algebras}}}
\begin{align*}
\pi_{X}^{*}(\srB_X) & := \set{\pi_{X}^{-1}(E): E \in \srB_X} = \set{E \times Y: E \in \srB_X}\\
\pi_{Y}^{*}(\srB_Y) & := \set{\pi_{Y}^{-1}(F): F \in \srB_Y} = \set{X \times F: F \in \srB_Y}
\end{align*}
We then define \underline{\emph{\textbf{the product $\sigma$-algebra $\srB_X \times \srB_Y$}}} to be the \emph{\textbf{$\sigma$-algebra} \textbf{generated by the union of these two pull-back $\sigma$-algebras}}:
\begin{align*}
\srB_X \times \srB_Y &:= \langle  \pi_{X}^{*}(\srB_X)  \cup  \pi_{Y}^{*}(\srB_Y)   \rangle.
\end{align*}
This definition has several equivalent formulations:
\end{definition}

\item \begin{proposition} (\textbf{Equivalent Definition of Product $\sigma$-Algebra}) \citep{tao2011introduction} \\
Let  $(X, \srB_X)$ and $(Y, \srB_Y)$ be measurable spaces.
\begin{enumerate}
\item $\srB_X \times \srB_Y$ is the \textbf{$\sigma$-algebra generated by the sets $E \times F$ with $E \in \srB_X$,  $Y \in \srB_Y$}. In other words, $\srB_X \times \srB_Y$ is
the \textbf{coarsest $\sigma$-algebra} on $X \times Y$ with the property that \textbf{the product} of a $\srB_X$-measurable set and a $\srB_Y$-measurable set is
always $\srB_X \times \srB_Y$ measurable.

\item $\srB_X \times \srB_Y$ is the \textbf{coarsest $\sigma$-algebra} on $X \times Y$  that makes the \textbf{projection maps} $\pi_X, \pi_Y$ both measurable. 
\end{enumerate}
\end{proposition}


\item \begin{proposition} (\textbf{Property of Product $\sigma$-Algebra}) \citep{tao2011introduction} \\
Let  $(X, \srB_X)$ and $(Y, \srB_Y)$ be measurable spaces.
\begin{enumerate}
\item (\textbf{Slices of Meaurable Set}) If $E \in \srB_X \times \srB_Y$, then sets 
\begin{align*}
E_x := \set{y \in Y: (x, y) \in E} \in \srB_Y
\end{align*} for every $x \in X$, and similarly that the sets
\begin{align*}
E^y := \set{x \in X: (x, y) \in E} \in \srB_X
\end{align*} for every $y \in Y$.

\item (\textbf{Slices of Meaurable Function}) If $f : X \times Y \to [0, +\infty]$ is measurable (with respect to $\srB_X \times \srB_Y $),  then the function 
\begin{align*}
f_x: y \to f(x, y)
\end{align*} is $\srB_Y$-measurable for every $x \in X$, and similarly that the function
\begin{align*}
f^{y}: x \to f(x, y)
\end{align*} is $\srB_X$-measurable for every $y \in Y$.

\item The product of two \textbf{trivial $\sigma$-algebras} (on two different spaces $X, Y$) is again \textbf{trivial}.

\item The product of two \textbf{atomic $\sigma$-algebras} is again \textbf{atomic}.

\item The product of two \textbf{finite $\sigma$-algebras} is again \textbf{finite}.

\item The product of two \textbf{Borel $\sigma$-algebras} (on two Euclidean spaces $\bR^d$, $R^{d'}$ with $d, d' \ge 1$) is again the \textbf{Borel $\sigma$-algebra} (on $\bR^d \times \bR^{d'} \equiv \bR^{d+d'}$).

\item The product of two \textbf{Lebesgue $\sigma$-algebras} (on two Euclidean spaces $\bR^d$, $R^{d'}$ with $d, d' \ge 1$) is \textbf{not} the \textbf{Lebesgue
$\sigma$-algebra}. (Hint: argue by \textbf{contradiction} and use slices of measurable set as above proposition.)

\item However, the Lebesgue $\sigma$-algebra on $\bR^{d+d'}$ is the \textbf{completion} of the product of the Lebesgue $\sigma$-algebras of $\bR^d$ and $\bR^{d'}$ with respect to $(d + d')$-dimensional Lebesgue measure.
\end{enumerate}
\end{proposition}

\item \begin{exercise} \citep{tao2011introduction}\\
If $E \in \srB_X \times \srB_Y$, show that the slices $E_x := \set{y \in Y: (x, y) \in E}$ lie in a countably generated $\sigma$-algebra. In other words, show that there exists an at most countable collection $\srA = \srA_E$ of sets (which can depend on $E$) such that $s\set{E_x : x \in X} \subseteq \langle \srA \rangle$. Conclude in particular that the number of \textbf{distinct} slices $E_x$ is at most $c$, the \textbf{cardinality} of the continuum.
\end{exercise}

\item   \begin{exercise} \citep{tao2011introduction}\\
Give an example to show that the product of two \textbf{discrete $\sigma$-algebras} is \textbf{not} necessarily \textbf{discrete}.

On the other hand, show that the product of two \textbf{discrete $\sigma$-algebras} $2^X$, $2^Y$ is again a \textbf{discrete $\sigma$-algebra} if at least one of the domains $X, Y$ is \textbf{at most countably infinite}.
\end{exercise}
\end{itemize}

\section{Product Measure}
\begin{itemize}
\item \begin{definition} (\textbf{\emph{$\sigma$-Finite}}).\\
A measure space $(X, \srB, \mu)$ is \underline{\textbf{\emph{$\sigma$-finite}}} if $X$ can be expressed as the \textbf{countable union} of sets of \textbf{finite} measure, i.e. $X = \bigcup_{n}X_n$, $\mu(X_n) < \infty$ for all $n$.
\end{definition}

\item \begin{example} ($\bR^d$)\\
$\bR^d$ with \textbf{\emph{Lebesgue measure}} is $\sigma$-finite, as $\bR^d$ can be expressed as the union of (for instance) the balls $B(0, n)$ for
$n = 1, 2, 3, \ldots$, each of which has finite measure. 

On the other hand, $\bR^d$ with \textbf{\emph{counting measure is not $\sigma$-finite}}.
\end{example}

\item \begin{proposition}  (\textbf{Existence and Uniqueness of Product Measure}) \citep{tao2011introduction}\\
Let $(X, \srB_X, \mu_X)$ and $(Y, \srB_Y, \mu_Y)$ be \textbf{$\sigma$-finite measure spaces}. Then there exists a \textbf{unique} measure $\mu_X \times \mu_Y: \srB_X \times \srB_Y \to [0, \infty]$ on product $\sigma$-algebra $\srB_X \times \srB_Y$ that obeys 
\begin{align*}
\mu_X \times \mu_Y(E \times F) &= \mu_X(E)\,\mu_Y(F)
\end{align*} whenever $E \in \srB_X$ and $F \in \srB_Y$.
\end{proposition}

\item \begin{remark}
When $X, Y$ are \textbf{not} \textbf{both} $\sigma$-finite, then one can still construct \textbf{at least one product measure}, but it will, in general, \textbf{not be unique}. 
\end{remark}

\item \begin{remark} (\emph{\textbf{Product Measure of Lebesgue Measures}})\\
The \textbf{\emph{product}} $m^d \times m^{d'}$ of the \textbf{\emph{Lebesgue measures}} $m^d, m^{d'}$ on $(\bR^d, \cL[\bR^d])$ and $(\bR^{d'}, \cL[\bR^{d'}]$) respectively will \textbf{\emph{agree}} with Lebesgue measure $m^{d+d'}$ on the product space $\cL[\bR^d] \times  \cL[\bR^{d'}]$, which is a \textbf{\emph{subalgebra}} of $\cL[\bR^{d+d'}]$. After taking the \emph{\textbf{completion}} $\overline{m^d \times m^{d'}}$ of this product measure, one obtains the \emph{full Lebesgue measure} $m^{d+d'}$.
\end{remark}

\item \begin{proposition}\citep{tao2011introduction}\\
Let  $(X, \srB_X)$ and $(Y, \srB_Y)$ be measurable spaces.
\begin{enumerate}
\item The product of two \textbf{Dirac measures} on $(X, \srB_X)$,  $(Y, \srB_Y)$ is a \textbf{Dirac measure} on $(X \times Y, \srB_X \times \srB_Y)$.

\item If $X, Y$ are \textbf{at most countable}, the product of the two \textbf{counting measures} on $(X, \srB_X)$,  $(Y, \srB_Y)$ is the \textbf{counting measure} on $(X \times Y, \srB_X \times \srB_Y)$.
\end{enumerate}
\end{proposition}

\item \begin{proposition} (\textbf{Associativity of Product}). \citep{tao2011introduction}\\
Let $(X, \srB_X, \mu_X)$, $(Y, \srB_Y, \mu_Y)$, $(Z, \srB_Z, \mu_Z)$ be \textbf{$\sigma$-finite sets}. We may identify the Cartesian products
$(X \times Y) \times Z$ and $X \times (Y \times Z)$ with each other in the obvious manner. If we do so, then 
\begin{align*}
 (\srB_X \times \srB_Y) \times \srB_Z = \srB_X \times (\srB_Y \times \srB_Z) 
\end{align*}
and
\begin{align*}
(\mu_X \times \mu_Y ) \times \mu_Z = \mu_X \times (\mu_Y \times \mu_Z).
\end{align*}
\end{proposition}
\end{itemize}

\section{Integration in Product Space}
\subsection{Tonelli's Theorem}
\begin{itemize}
\item \begin{definition} (\textbf{\emph{Monotone Class}})\\
Define a \emph{\textbf{\underline{monotone class}} in $X$} is a collection $\srB$ of subsets of $X$ with the following two \textbf{closure properties}:
\begin{enumerate}
\item If $E_1 \subset E_2 \subset \ldots$ are a \textbf{\emph{countable increasing}} sequence of sets in $\srB$, then $\bigcup_{n=1}^{\infty}E_n \in \srB$.

\item If $E_1 \supset E_2 \supset \ldots$ are a \textbf{\emph{countable decreasing}} sequence of sets in $\srB$, then $\bigcap_{n=1}^{\infty}E_n \in \srB$.
\end{enumerate}
\end{definition}

\item \begin{lemma} (\textbf{Monotone Class Lemma}). \citep{tao2011introduction}\\
Let $\srA$ be a \textbf{Boolean algebra} on $X$. Then $\langle \srA \rangle$ is the \textbf{smallest monotone class} that contains $\srA$.
\end{lemma}

\item \begin{theorem} (\textbf{Tonelli's Theorem, Incomplete Version}).   \citep{tao2011introduction}\\
Let $(X, \srB_X, \mu_X)$ and $(Y, \srB_Y, \mu_Y)$ be \textbf{$\sigma$-finite measure spaces}, and let $f : X \times Y \to [0, +\infty]$ be measurable with respect to $\srB_X \times \srB_Y$. Then:
\begin{enumerate}
\item The functions 
\begin{align*}
x \to \int_Y f(x,y) d\mu_Y(y)\\
y \to \int_X f(x,y) d\mu_X(x)
\end{align*} (which are well-defined) are \textbf{measurable} with respect to $\srB_X$ and $\srB_Y$ respectively.

\item We have
\begin{align}
\int_{X \times Y} f(x, y) \;d\paren{\mu_X \times \mu_Y}(x, y) &= \int_X\paren{\int_Y f(x, y) d\mu_Y(y)} d\mu_X(x) \nonumber\\
&=  \int_Y\paren{\int_Xf(x, y) d\mu_X(x) }d\mu_Y(y)  \label{eqn: tonelli_theorem_1}
\end{align}
\end{enumerate}
\end{theorem}

\item \begin{corollary} (\textbf{Slice of Null Set})\citep{tao2011introduction}\\
Let $(X, \srB_X, \mu_X)$ and $(Y, \srB_Y, \mu_Y)$ be \textbf{$\sigma$-finite measure spaces}, and let $E \in \srB_X \times \srB_Y$ be a \textbf{null} set with respect to $\mu_X \times \mu_Y$. Then for $\mu_X$-almost every  $x \in X$, the set $E_x := \set{y \in Y: (x, y) \in E}$ is a \textbf{$\mu_Y$-null set}; and similarly, for $\mu_Y$-almost every $y \in Y$, the set $E^y := \set{x \in X: (x, y) \in E}$ is a \textbf{$\mu_X$-null set}.
\end{corollary}

With this corollary, we can extend \emph{Tonelli's theorem} to the completion $(X \times Y, \overline{\srB_X \times \srB_Y},  \overline{\mu_X \times \mu_Y})$ of the product space $(X \times Y, \srB_X \times \srB_Y,  \mu_X \times \mu_Y)$.

\item \begin{theorem}  (\textbf{Tonelli's Theorem, Complete Version}).   \citep{tao2011introduction}\\
Let $(X, \srB_X, \mu_X)$ and $(Y, \srB_Y, \mu_Y)$  be \textbf{complete $\sigma$-finite measure spaces}, and let  $f : X \times Y \to [0, +\infty]$ be measurable with respect to $\srB_X \times \srB_Y$. Then:
\begin{enumerate}
\item For $\mu_X$-almost every  $x \in X$,  the function
\begin{align*}
y \to f(x,y)
\end{align*} is \textbf{$\srB_Y$-measurable} and in particular, $\int_Y f(x,y) d\mu_Y(y)$ exists. Furthermore, the ($\mu_X$-almost everywhere defined) map
\begin{align*}
x \to \int_Y f(x,y) d\mu_Y(y)
\end{align*} is \textbf{$\srB_X$-measurable}.

\item  For $\mu_Y$-almost every  $y \in Y$,  the function
\begin{align*}
x \to f(x,y)
\end{align*} is \textbf{$\srB_X$-measurable} and in particular, $\int_X f(x,y) d\mu_X(x)$ exists. Furthermore, the ($\mu_Y$-almost everywhere defined) map
\begin{align*}
y \to \int_X f(x,y) d\mu_X(x)
\end{align*} is \textbf{$\srB_Y$-measurable}.

\item We have
\begin{align}
\int_{X \times Y} f(x, y) \;d\paren{\overline{\mu_X \times \mu_Y}}(x, y) &= \int_X\paren{\int_Y f(x, y) d\mu_Y(y)} d\mu_X(x) \nonumber\\
&=  \int_Y\paren{\int_Xf(x, y) d\mu_X(x) }d\mu_Y(y)  \label{eqn: tonelli_theorem_2}
\end{align}
\end{enumerate}
\end{theorem}

\item Specialising to the case when $f$ is an indicator function $f = \mathds{1}_E$, we conclude
\begin{corollary} (\textbf{Tonelli’s Theorem for Sets}). \citep{tao2011introduction}\\
Let $(X, \srB_X, \mu_X)$ and $(Y, \srB_Y, \mu_Y)$  be \textbf{complete $\sigma$-finite measure spaces}, and let $E \in \overline{\srB_X \times \srB_Y}$. Then:
\begin{enumerate}
\item For $\mu_X$-almost every  $x \in X$,  the set
\begin{align*}
E_x := \set{y \in Y: (x, y) \in E} \in \srB_Y
\end{align*} and the ($\mu_X$-almost everywhere defined) map
\begin{align*}
x \to \mu_Y(E_x)
\end{align*} is \textbf{$\srB_X$-measurable}.

\item  For $\mu_Y$-almost every  $y \in Y$,  the set
\begin{align*}
E^y := \set{x \in X: (x, y) \in E} \in \srB_X
\end{align*} and the ($\mu_Y$-almost everywhere defined) map
\begin{align*}
y \to \mu_X(E^y)
\end{align*} is \textbf{$\srB_Y$-measurable}.

\item We have
\begin{align}
\overline{\mu_X \times \mu_Y}(E) &= \int_X\mu_Y(E_x)\; d\mu_X(x) \nonumber\\
&=  \int_Y \mu_X(E^y)\;d\mu_Y(y)  \label{eqn: tonelli_theorem_3}
\end{align}
\end{enumerate}
\end{corollary}

\item \begin{remark}
Tonelli’s theorem can \emph{\textbf{fail}} if \emph{\textbf{the $\sigma$-finite hypothesis} is \textbf{removed}}, and also that \emph{product measure need not be unique}.
\end{remark}
\end{itemize}
\subsection{Fubini's Theorem}
\begin{itemize}
\item \begin{remark}
\emph{Tonelli's theorem} is for \emph{the \textbf{unsigned integral}}, but it leads to an important analogue for \emph{the absolutely integral}, known as \emph{Fubini's theorem}:
\end{remark}

\item \begin{theorem} (\textbf{Fubini’s Theorem}). \citep{tao2011introduction}\\
Let $(X, \srB_X, \mu_X)$ and $(Y, \srB_Y, \mu_Y)$ be \textbf{complete $\sigma$-finite measure spaces}, and let $f: X \times Y \to \bC$ be \textbf{absolutely integrable} \textbf{with respect to} $\overline{\srB_X \times \srB_Y}$. Then:
\begin{enumerate}
\item For $\mu_X$-almost every  $x \in X$,  the function
\begin{align*}
y \to f(x,y)
\end{align*} is \textbf{absolutely integrable} \textbf{with respect to $\mu_Y$} and in particular, $\int_Y f(x,y) d\mu_Y(y)$ exists. Furthermore, the ($\mu_X$-almost everywhere defined) map
\begin{align*}
x \to \int_Y f(x,y) d\mu_Y(y)
\end{align*} is \textbf{absolutely integrable} \textbf{with respect to $\mu_X$}.

\item  For $\mu_Y$-almost every  $y \in Y$,  the function
\begin{align*}
x \to f(x,y)
\end{align*} is \textbf{absolutely integrable} \textbf{with respect to $\mu_X$} and in particular, $\int_X f(x,y) d\mu_X(x)$ exists. Furthermore, the ($\mu_Y$-almost everywhere defined) map
\begin{align*}
y \to \int_X f(x,y) d\mu_X(x)
\end{align*} is \textbf{absolutely integrable} \textbf{with respect to $\mu_Y$}.

\item We have
\begin{align}
\int_{X \times Y} f(x, y) \;d\paren{\overline{\mu_X \times \mu_Y}}(x, y) &= \int_X\paren{\int_Y f(x, y) d\mu_Y(y)} d\mu_X(x) \nonumber\\
&=  \int_Y\paren{\int_Xf(x, y) d\mu_X(x) }d\mu_Y(y)  \label{eqn: fubini_theorem_1}
\end{align}
\end{enumerate}
\end{theorem}

\item \begin{remark}
Fubini's theorem fails when one drops the hypothesis that $f$ is \emph{absolutely integrable \textbf{with respect to the product space}}.
\end{remark}

\item \begin{remark}
Despite the failure of Tonelli's theorem in the $\sigma$-finite setting, it is possible to \emph{(carefully) extend Fubini's theorem to \textbf{the non-$\sigma$-finite setting}}, as \emph{the absolute integrability hypotheses}, when combined with \emph{Markov's inequality}, can provide a substitute for \emph{the $\sigma$-finite property}.
\end{remark}

\item \begin{remark}
Informally, \emph{Fubini's theorem} allows one to \emph{always \textbf{interchange the order of two integrals}}, as long as the \emph{integrand is absolutely integrable} \emph{\textbf{in the product space} (or its completion)}. In particular, specialising to Lebesgue measure, we have
\begin{align*}
\int_{\bR^{d+d'}}f(x,y) d(x,y) = \int_{\bR^d}\paren{\int_{\bR^{d'}} f(x, y)dy} dx = \int_{\bR^{d'}}\paren{\int_{\bR^{d}} f(x, y)dx} dy
\end{align*}
whenever $f: \bR^{d+d'} \to \bC$ is \emph{absolutely integrable}. In view of this, we often write $dxdy$ (or $dydx$) for $d(x, y)$.
\end{remark}

\item By combining \emph{Fubini's theorem} with \emph{Tonelli's theorem}, we can recast the \emph{absolute integrability hypothesis}:
\begin{corollary} (\textbf{Fubini-Tonelli Theorem}). \citep{tao2011introduction}\\
Let $(X, \srB_X, \mu_X)$ and $(Y, \srB_Y, \mu_Y)$ be \textbf{complete $\sigma$-finite measure spaces}, and let $f: X \times Y \to \bC$ be \textbf{measurable} \textbf{with respect to} $\overline{\srB_X \times \srB_Y}$.  If
\begin{align*}
 \int_X\paren{\int_Y \abs{f(x, y)} d\mu_Y(y)} d\mu_X(x) < \infty
\end{align*} then $f$ is \textbf{absolutely integrable} with respect to $\overline{\srB_X \times \srB_Y}$, and in particular the
conclusions of \textbf{Fubini's theorem} hold. 

Similarly if we use $ \int_Y\paren{\int_X \abs{f(x, y)} d\mu_X(x) }d\mu_Y(y)$ instead of $ \int_X\paren{\int_Y \abs{f(x, y)} d\mu_Y(y)} d\mu_X(x) $.
\end{corollary}

\item \begin{proposition} (\textbf{Area Interpretation of Integral}).  \citep{tao2011introduction}\\
Let $(X, \srB, \mu)$ be a $\sigma$-finite measure space, and let $\bR$ be equipped with Lebesgue measure $m$ and the Borel $\sigma$-algebra $\cB[\bR]$. Then $f : X \to [0, +\infty]$ is \textbf{measurable} \textbf{if and only if} its \textbf{subgraph}
\begin{align*}
\set{(x, t)\in X \times \bR: 0 \le t\le f(x)}
\end{align*} is \textbf{measurable} in $\srB \times \cB[\bR]$, in which case we have
\begin{align*}
\paren{\mu \times m}\set{(x, t)\in X \times \bR: 0 \le t\le f(x)} &= \int_{X} f(x) d\mu(x).
\end{align*} Similarly if we replace $\set{(x, t)\in X \times \bR: 0 \le t\le f(x)}$ by $\set{(x, t)\in X \times \bR: 0 \le t < f(x)}$.
\end{proposition}

\item \begin{proposition} (\textbf{Distribution Formula}).  \citep{tao2011introduction}\\
Let $(X, \srB, \mu)$ be a $\sigma$-finite measure space, and let $f : X \to [0, +\infty]$ be \textbf{measurable}. Then
\begin{align}
 \int_{X} f(x) d\mu(x) &= \int_{[0, \infty]}\mu\set{x\in X: f(x) \ge \lambda} d\lambda \label{eqn: distribution_formula}
\end{align}
(Note that the integrand on the right-hand side is monotone and thus Lebesgue measurable.) Similarly if we replace $\set{x\in X: f(x) \ge \lambda}$ by $\set{x\in X: f(x) > \lambda}$.
\end{proposition}
\end{itemize}
\newpage
\bibliographystyle{plainnat}
\bibliography{reference.bib}
\end{document}