\documentclass[11pt]{article}
\usepackage[scaled=0.92]{helvet}
\usepackage{geometry}
\geometry{letterpaper,tmargin=1in,bmargin=1in,lmargin=1in,rmargin=1in}
\usepackage[parfill]{parskip} % Activate to begin paragraphs with an empty line rather than an indent %\usepackage{graphicx}
\usepackage{amsmath,amssymb, amscd}
\usepackage{mathrsfs, dsfont}
\usepackage[all,cmtip]{xy}
\usepackage{tikz-cd}
%\diagramstyle[labelstyle=\scriptstyle]
\usepackage{tabularx}
\usepackage[font=footnotesize,labelfont=bf]{caption}
\usepackage{graphicx}
\usepackage{xcolor}
%\usepackage[linkbordercolor ={1 1 1} ]{hyperref}
%\usepackage[sf]{titlesec}
\usepackage{natbib}
\usepackage{../../Tianpei_Report}

%\usepackage{appendix}
%\usepackage{algorithm}
%\usepackage{algorithmic}

%\renewcommand{\algorithmicrequire}{\textbf{Input:}}
%\renewcommand{\algorithmicensure}{\textbf{Output:}}



\begin{document}
\title{Lecture 6: Martingale}
\author{ Tianpei Xie}
\date{ Feb.2nd., 2023 }
\maketitle
\tableofcontents
\newpage
\section{Conditional Expectation}
\begin{itemize}
\item \begin{definition} (\emph{\textbf{Conditional Expectation}}) \citep{resnick2013probability}\\
Let $(\Omega, \srF, \cP)$ be a probability space and $\srG \subset \srF$ be a sub-$\sigma$-algebra. Suppose $X\in  L^{1}(\Omega, \srF, \cP)$. There exists a function $\E{}{X| \srG}$, called the \underline{\emph{\textbf{conditional expectation}}} of $X$ \emph{\textbf{with respect to}} $\srG$ such that
\begin{enumerate}
\item $\E{}{X| \srG}$ is \emph{\textbf{$\srG$-measureable}} and \emph{\textbf{integrable}} \emph{with respect to} $\cP$. 

\item $\E{}{X| \srG}$ satisfies \emph{\textbf{the functional equation}}: 
\begin{align*}
\int_{G} X d\cP &= \int_{G} \E{}{X| \srG} d\cP, \quad \forall\, G \in \srG.
\end{align*}
\end{enumerate}
\end{definition}

\item \begin{remark}
To \emph{prove the \textbf{existence}} of such a random variable, 
\begin{enumerate}
\item consider first the case of
\emph{\textbf{nonnegative} X}. Define a measure $\nu$ on $\srG$ by 
\begin{align*}
\nu(G) = \int_{G} X d\cP = \int_{\Omega} X \mathds{1}_{G}\; d\cP.
\end{align*}
This measure is \emph{finite} because $X$ is \emph{integrable}, and it is \emph{\textbf{absolutely continuous}} with respect to $\cP$. By the \emph{Lebesgue-Radon-Nikodym Theorem},  there is a $\srG$-measurable function $f$
such that 
\begin{align*}
\nu(G) = \int_{G} f d\cP.
\end{align*}
This $f$ has properties (1) and (2). 
\item If $X$ is \emph{not necessarily nonnegative}, $\E{}{X_{+}| \srG} - \E{}{X_{-}| \srG}$ clearly has the required properties.
\end{enumerate}
\end{remark}

\item \begin{remark}
As $\srG$ increases, condition (1) becomes \emph{\textbf{weaker}} and condition (2) becomes \emph{\textbf{stronger}}.
\end{remark}

\item \begin{remark}
Let $(\Omega, \srF, \cP)$ be a probability space, with  $\srG \subset \srF$ a sub-$\sigma$-algebra, define
\begin{align*}
\cP[A | \srG] &= \E{}{\mathds{1}_{A}| \srG}
\end{align*} for all $A\in \srF$.
\end{remark}

\item \begin{remark}
By definition, the conditional expectation is a \emph{\textbf{Radon-Nikodym derivative}} of $d\nu|_{\srG} = X d\cP|_{\srG}$ w.r.t. $d\cP|_{\srG}$ within $\srG$.
\begin{align*}
\E{}{X| \srG}&:=  \frac{Xd\cP|_{\srG}}{d\cP|_{\srG}} = X|_{\srG}.
\end{align*} Thus \underline{\emph{$\E{}{X| \srG}$ is the \textbf{projection} of $X$ on \textbf{sub $\sigma$-algebra} $\srG$}}.
\end{remark}

\item \begin{remark} (\emph{\textbf{Conditioning on Random Variables}})\\
By definition, conditioning on random variables $(X_{t}, t\in T)$ on $(\Omega, \srB)$ can be expressed as 
\begin{align*}
\E{}{X| X_{t}, t\in T} &\equiv \E{}{X| \sigma(X_{t}, t\in T)}, 
\end{align*}
where $\sigma(X_{t}, t\in T)$ is the $\sigma$-algebra generated by the cylinder set
\begin{align*}
C_{n}[A]&\equiv \set{\omega: (X_{t}(\omega), 1\le t\le n) \in A  } \in \srB, \quad A\in \cB(\bR^{n}), \forall \, n
\end{align*}
\end{remark}

\item \begin{remark}(\emph{\textbf{$\sigma$-Algebra Generated by Partition of Sample Space}})\\
As above, assume that the sub $\sigma$-algebra $\srG$ is generated by a \emph{\textbf{partition}} $B_1, B_2, \ldots$ of $\Omega$, then for $X \in L^1(\Omega, \srF, \cP)$, 
\begin{align*}
\E{}{X | B_i} &= \int X d\cP(X | B_i) = \int_{B_i} X d\cP / \cP(B_i)
\end{align*} where $\cP(X | B_i)$ is \emph{the conditional probability} defined in previous section. If $\cP(B_i) = 0$, then $\E{}{X | B_i} = 0$.
We claim that 
\begin{enumerate}
\item 
\begin{align*}
\E{}{X | \srG} &= \sum_{i=1}^{\infty}\E{}{X | B_i} \mathds{1}_{B_i}, \quad a.s.
\end{align*}
\item For any $A \in \srF$,
\begin{align*}
\cP(A | \srG) &= \sum_{i=1}^{\infty}\cP(A | B_i) \mathds{1}_{B_i}, \quad a.s.
\end{align*}
\end{enumerate}
\end{remark}



\item \begin{remark}
Both $P[A | \srF]$ and $\E{}{X| \srF}$ are \emph{random variables} from $\Omega \rightarrow \bR$. Formally speaking, 
\begin{align*}
P\brac{(X,Y)\in A| \sigma(X)}_{\omega} &\equiv P\brac{(X(\omega), Y)\in A}\\
&= P\set{\omega': (X(\omega), Y(\omega'))\in A}\\
&\equiv f(X(\omega))\\
&= \rlat{\nu}{\sigma(X)}(A)\\
\E{}{(X,Y)| \sigma(X)}_{\omega} &= \lim\limits_{m(A)\rightarrow 0\atop \omega\in A\in \sigma(X)} \frac{P\set{\omega': (X(\omega), Y(\omega'))\in A}}{m(A)}
\end{align*}
It is the expected value of $X$ for someone who knows for each $E\in \srF$, whether or not $\omega\in E$, which $E$ itself remains unknown.
\end{remark}

\item \begin{proposition} (\textbf{Properties of Conditional Expectation}) \citep{resnick2013probability}\\
Let $(\Omega, \srF, \cP)$ be a probability space and $\srG \subset \srF$ be a sub-$\sigma$-algebra. Suppose $X, Y \in  L^{1}(\Omega, \srF, \cP)$ and $\alpha, \beta \in \bR$.
\begin{enumerate}
\item (\textbf{Linearity}): $\E{}{\alpha X+ \beta Y | \srG} = \alpha \E{}{X| \srG} +\beta  \E{}{Y | \srG}$;
\item (\textbf{Projection}): If $X$ is \textbf{$\srG$-measurable}, then $\E{}{X | \srG} = X$ almost surely.
\item (\textbf{Conditioning on Indiscrete $\sigma$-Algebra}): 
\begin{align*}
\E{}{X | \set{\emptyset, \Omega}} = \E{}{X}.
\end{align*}
\item (\textbf{Monotonicity}):  If $X \ge 0$, then $\E{}{X | \srG} \ge 0$ almost surely. Similarly, if $X \ge Y$, then $\E{}{X | \srG} \ge \E{}{Y | \srG}$  almost surely.
\item (\textbf{Modulus Inequality}): 
\begin{align*}
\abs{\,\E{}{X | \srG}\,} &\le \E{}{\,\abs{X} \, | \srG}.
\end{align*}
\item (\textbf{Monotone Convergence Theorem}): If  $\set{X_n}_{n=1}^{\infty} \subset  L^{1}(\Omega, \srF, \cP)$,  $0 \le X_1 \le X_{2}\le \ldots$ is a \textbf{monotone sequence} of \textbf{non-negative} random variables and $X_n \rightarrow X$ then
\begin{align*}
\lim\limits_{n\rightarrow \infty}\E{}{X_n | \srG} &= \E{}{\lim\limits_{n\rightarrow \infty} X_n \big| \srG} = \E{}{X | \srG}.
\end{align*}
\item (\textbf{Fatou Lemma}): If  $\set{X_n}_{n=1}^{\infty} \subset  L^{1}(\Omega, \srF, \cP)$, and $X_n \ge 0$ for all $n$, then
\begin{align*}
\E{}{\liminf\limits_{n\rightarrow \infty} X_n \big| \srG} &\le \liminf\limits_{n\rightarrow \infty} \E{}{X_n | \srG} 
\end{align*} 
\item  (\textbf{Dominated Convergence Theorem}):  If  $\set{X_n}_{n=1}^{\infty} \subset  L^{1}(\Omega, \srF, \cP)$ and $\abs{X_n} \le Z$, where $Z \in L^{1}(\Omega, \srF, \cP)$ is a random variable, $X_n \rightarrow X$ almost surely,  then
\begin{align*}
\lim\limits_{n\rightarrow \infty}\E{}{X_n | \srG} &= \E{}{\lim\limits_{n\rightarrow \infty} X_n \big| \srG} = \E{}{X | \srG}, \quad a.s.
\end{align*}
\item (\textbf{Product Rule}): If $Y$ is $\srG$-measurable, 
\begin{align*}
\E{}{X\,Y | \srG} &= Y\,\E{}{X | \srG},\;\; \text{ a.s.}
\end{align*}


\item (\textbf{Smoothing}): For $\srF_{1}\subset \srF_{0} \subset \srF$, 
\begin{align*}
\E{}{\E{}{X| \srF_{0}}\,|\srF_{1} } &= \E{}{X| \srF_{1}}\\
\E{}{\E{}{X| \srF_{1}}\,|\srF_{0} } &= \E{}{X| \srF_{1}}.
\end{align*} Note that $ \E{}{X| \srF_{1}}$ is \textbf{smoother} than $ \E{}{X| \srF_{0}}$.
Moreover
\begin{align*}
\E{}{X} = \E{}{X| \set{\emptyset, \Omega}} &= \E{}{\E{}{X| \srF_{0}}\,|\set{\emptyset, \Omega} } = \E{}{\E{}{X| \srF_{0}}}.
\end{align*}


\item (\textbf{The Conditional Jensen's Inequality}). Let $\phi$ be a \textbf{convex} function, $\phi(X) \in L^1(\Omega, \srF, \cP)$. Then almost surely
\begin{align*}
\phi\paren{\E{}{X | \srG}} &\le \E{}{\phi(X) | \srG}
\end{align*}
\end{enumerate}
\end{proposition}
\end{itemize}

\section{Martingale}




\newpage
\bibliographystyle{plainnat}
\bibliography{reference.bib}
\end{document}