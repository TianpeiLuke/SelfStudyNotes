\documentclass[11pt]{article}
\usepackage[scaled=0.92]{helvet}
\usepackage{geometry}
\geometry{letterpaper,tmargin=1in,bmargin=1in,lmargin=1in,rmargin=1in}
\usepackage[parfill]{parskip} % Activate to begin paragraphs with an empty line rather than an indent %\usepackage{graphicx}
\usepackage{amsmath,amssymb, mathrsfs, dsfont}
\usepackage{tabularx}
\usepackage[font=footnotesize,labelfont=bf]{caption}
\usepackage{graphicx}
\usepackage{xcolor}
%\usepackage[linkbordercolor ={1 1 1} ]{hyperref}
%\usepackage[sf]{titlesec}
\usepackage{natbib}
\usepackage{../../Tianpei_Report}

%\usepackage{appendix}
%\usepackage{algorithm}
%\usepackage{algorithmic}

%\renewcommand{\algorithmicrequire}{\textbf{Input:}}
%\renewcommand{\algorithmicensure}{\textbf{Output:}}



\begin{document}
\title{Lecture 4:  Abstract Measures and Integrations}
\author{ Tianpei Xie}
\date{ Jul. 25th., 2015 }
\maketitle
\tableofcontents
\newpage
\section{Recall}
\begin{itemize}
\item \begin{definition} \citep{tao2011introduction}\\
Let $X$ be a set. \emph{A (concrete) \underline{\textbf{Boolean algebra (Boolean field)}}} on $X$ is a \emph{collection of subsets} $\srB$ of $X$ which obeys the following
properties:
\begin{enumerate}
\item (\textbf{\emph{Empty set}}) $\emptyset \in \srB$;
\item (\textbf{\emph{Complements}}) For any $E\in \srB$, then $E^{c}\equiv (X \setminus E) \in \srB$;
\item (\textbf{\emph{Finite unions}}) For any  $E, F \subset \srB$, $E\cup F \in \srB$.
\end{enumerate}
We sometimes say that $E$ is \textbf{\emph{$\srB$-measurable}}, or \textbf{\emph{measurable with respect to $\srB$}}, if $E \in \srB$.
\end{definition}

\item \begin{remark}
Note that \emph{\textbf{the finite difference}} $A-B$, $A\Delta B$ and \emph{\textbf{intersections}} $A\cap B$ are also \emph{\textbf{closed}} under the Boolean algebra. 
\end{remark}

\item \begin{definition}
A \underline{\textbf{\emph{field} (\emph{algebra})}} is a \emph{non-empty collection of subsets} in $X$ that is \emph{\textbf{closed}} under \emph{\textbf{finite union}} and \emph{\textbf{complements}}. 

It is just a \emph{subset (sub-algebra)} of \emph{Boolean field} $(X, \subset, \cup, \cdot^{c})$. 
\end{definition} 

\item \begin{definition}
Given two Boolean algebras $\srB, (\srB)'$ on $X$, we say that $(\srB)'$ is \underline{\emph{\textbf{finer}}} than, a \emph{\textbf{sub-algebra}} of, or a \underline{\emph{\textbf{refinement}}} of $\srB$, or that $\srB$ is \underline{\emph{\textbf{coarser}}} than or a \underline{\emph{\textbf{coarsening}}} of $(\srB)'$, if $\srB \subset (\srB)'$.
\end{definition}

\item \begin{definition}
A Boolean algebra is \emph{\textbf{finite}} if it only consists of \emph{finite many of subsets} (i.e., its \emph{cardinality} is finite)  \citep{tao2011introduction}.  
\end{definition}

\item \begin{definition}
Given a collection of sets $\cF$, then $\langle \cF \rangle_{bool}$ is \underline{\emph{\textbf{the Boolean algebra generated}}} by $\cF$, i.e. the \emph{\textbf{intersection}} of all the Boolean algebras that contain $\cF$. 
\begin{align*}
\langle \cF \rangle_{bool} = \bigwedge_{\srB_{\alpha} \supseteq \cF}\srB_{\alpha}.
\end{align*} 
\end{definition}

\item
\begin{definition}
 Given space $X$, a \underline{\emph{\textbf{$\sigma$-field}} (or, \emph{\textbf{$\sigma$-algebra}})} $\srF$ is a non-empty collection of \emph{subsets} in $X$ such that 
\begin{enumerate}
\item $\emptyset \in \srF$; $X\in \srF$;
\item \emph{\textbf{Complements}}:  For any $B\in \srF$, then $B^{c}\equiv (X-B) \in \srF$;
%\item \emph{Finite union}: for any  $A, B \subset \srF$, 
%\begin{align*}
%A\cup B \in \srF;
%\end{align*} 
\item \underline{\emph{\textbf{Countable union}}}: for any sub-collection $\set{B_{k}}_{k=1}^{\infty} \subset \srF$, 
\begin{align*}
\bigcup_{k=1}^{\infty}B_{k} \in \srF;
\end{align*} 
Also, \emph{\textbf{Countable intersection}}: $\bigcap_{k=1}^{\infty}B_{k} \in \srF,$ \emph{\textbf{de Morgan's law}}.
\end{enumerate} 
We refer to the pair $(X, \srF)$ of a set $X$ together with a $\sigma$-algebra on that set as \emph{\textbf{a measurable space}}.
\end{definition}

\item
\begin{definition}
Denote $\sigma(\cF) := \langle \cF \rangle$ as  \underline{\emph{\textbf{the $\sigma$-algebra generated by $\cF$}}}, given by 
\begin{align*}
\sigma(\cF) = \langle \cF \rangle = \bigwedge_{\srB_{\alpha} \supseteq \cF}\srB_{\alpha}.
\end{align*}
 It is the  \emph{\textbf{coarsest}} $\sigma$-algebra containing $\cF$, for any $\sigma$-algebra that contains $\cF$  . 
\end{definition}

\item \begin{definition} (\emph{\textbf{Borel $\sigma$-algebra}}). \citep{tao2011introduction} \\
 Let $X$ be a \emph{\textbf{metric space}}, or more generally \emph{\textbf{a topological space}}. The \underline{\emph{\textbf{Borel $\sigma$-algebra}}} $\cB[X]$ of $X$ is defined to be \underline{the $\sigma$-algebra \emph{generated by the \textbf{open subsets} of $X$}}.

Elements of $\cB[X]$ will be called \emph{\textbf{Borel measurable}}.
\end{definition}

\item \begin{exercise}
Show that the Borel $\sigma$-algebra $\cB[\bR^d]$ of a Euclidean set is generated by any of the following collections of sets:
\begin{enumerate}
\item The open subsets of $\bR^d$.
\item The closed subsets of $\bR^d$.
\item The compact subsets of $\bR^d$.
\item The open balls of $\bR^d$.
\item The boxes in $\bR^d$.
\item The elementary sets in $\bR^d$.
\end{enumerate} 
(Hint: To show that two families $\cF, \cF'$ of sets generate the same $\sigma$-algebra, it suffices to show that every $\sigma$-algebra that contains $\cF$, contains $\cF'$ also, and conversely.)
\end{exercise}
\end{itemize}

\newpage
\section{Countably Additive Measures and Measure Spaces}
\subsection{Finitely Additive Measure}
\begin{itemize}
\item
\begin{definition}
Let $\srB$ be a \emph{Boolean algebra} on a space $X$. An (unsigned) \underline{\emph{\textbf{finitely additive measure}}} $\mu$ on $\srB$ is a map $\mu : \srB \rightarrow [0,+\infty]$ that obeys the following axioms
\begin{enumerate}
\item $\mu(\emptyset) = 0$;
\item \emph{\textbf{Finite union}}: for any  \emph{\textbf{disjoint sets}} $A, B \in \srB$, 
\begin{align*}
\mu\paren{A\cup B} = \mu(A)+ \mu(B).
\end{align*} 
\end{enumerate}
\end{definition}

\item \begin{proposition} (\textbf{Properties of Finitely Additive Measure}) \citep{tao2011introduction}\\
Let $\mu: \srB \rightarrow [0, +\infty]$be a finitely additive measure on a Boolean $\sigma$-algebra $\srB$. 
\begin{enumerate}
\item (\textbf{Monotonicity}) If $E, F$ are $\srB$-measurable and $E \subseteq F$, then
\begin{align*}
\mu(E) \le \mu(F).
\end{align*}
\item  (\textbf{Finite additivity}) If $k$ is a natural number, and $E_1 \xdotx{,} E_k$ are $\srB$-measurable and \textbf{disjoint}, then 
\begin{align*}
\mu(E_1 \xdotx{\cup} E_k) = \mu(E_1)  \xdotx{+} \mu(E_k).
\end{align*}
\item  (\textbf{Finite subadditivity}) If $k$ is a natural number, and $E_1 \xdotx{,} E_k$ are $\srB$-measurable, then
\begin{align*}
\mu(E_1 \xdotx{\cup} E_k) \le \mu(E_1)  \xdotx{+} \mu(E_k).
\end{align*}
\item \textbf{(Inclusion-exclusion for two sets}) If $E, F$ are $\srB$-measurable, then
\begin{align*}
\mu(E \cup F ) + \mu(E \cap F ) = \mu(E) + \mu(F).
\end{align*}
\end{enumerate}
(Caution: remember that the cancellation law $a+c = b+c \Rightarrow a = b$ does not hold in [0; +1] if c is infinite, and so the use of cancellation
(or subtraction) should be avoided if possible.)
\end{proposition}

\item \begin{example}
See the following examples on finitely additive measures:
\begin{enumerate}
\item \emph{\textbf{Lebesgue measure}} $m$ is a \emph{finitely additive measure} on \emph{\textbf{the Lebesgue $\sigma$-algebra}}, and hence on \emph{all sub-algebras} (such as \emph{the null algebra}, \emph{the Jordan algebra}, or \emph{the elementary algebra}).

\item \emph{\textbf{Jordan measure}} and \emph{\textbf{elementary measure}} are \emph{finitely additive} (adopting the convention that co-Jordan measurable sets have infinite Jordan measure, and co-elementary sets have infinite elementary measure).

\item \emph{\textbf{Lebesgue outer measure}} is \emph{\textbf{not}} \emph{finitely additive} on \emph{\textbf{the discrete algebra}}.

\item \emph{\textbf{Jordan outer measure}} is \emph{\textbf{not}} \emph{finitely additive} on \emph{\textbf{the Lebesgue algebra}}.
\end{enumerate}
\end{example}

\item  \begin{example} (\emph{\textbf{Dirac measure}}). \\
Let $x \in X$ and $\srB$ be an arbitrary \emph{Boolean algebra} on $X$. Then \underline{\emph{\textbf{the Dirac measure}}} $\delta_x$ at $x$, defined by
setting $\delta_x(E) := \ind{x \in E}$, is \emph{\textbf{finitely additive}}.
\end{example}

\item \begin{example} (\emph{\textbf{Zero measure}}). \\
The \emph{\textbf{zero measure}} $0: E \mapsto 0$ is a \emph{finitely additive measure} on any Boolean algebra.
\end{example}

\item \begin{example} (\emph{\textbf{Linear combinations of measures}}). \\
If $\srB$ is a Boolean algebra on $X$, and $\mu, \nu: \srB \rightarrow [0, +\infty]$ are \emph{finitely additive measures} on $\srB$, then $\mu + \nu: E \mapsto \mu(E)+ \nu(E)$ is also a \emph{\textbf{finitely additive measure}}, as is $c\mu: E \mapsto c \times \mu(E)$ for any $c \in [0, +\infty]$. Thus, for instance, the sum of Lebesgue measure and a Dirac measure is also a finitely additive measure on the Lebesgue algebra (or on any of its sub-algebras).

In other word, \underline{\emph{\textbf{the space of all finitely additive measures} on $\srB$ is a \textbf{vector space}}}.
\end{example}

\item \begin{example} (\emph{\textbf{Restriction of a measure}}).\\
If $\srB$ is a Boolean algebra on $X$, $\mu: \srB \rightarrow [0, +\infty]$ is a \emph{finitely additive measure}, and $Y$ is a $\srB$-measurable subset of $X$, then \emph{\textbf{the restriction}} $\mu|_{Y}: \srB|_{Y}  \rightarrow [0, +\infty]$ of $\srB$ to $Y$, defined by setting $\mu|_{Y}(E) := \mu(E)$ whenever $E \in \srB|_{Y}$ (i.e.
if $E \in \srB$ and $E \subseteq Y$), is also a \emph{\textbf{finitely additive measure}}.
\end{example}

\item \begin{example} (\emph{\textbf{Counting measure}}).\\
If $\srB$ is a Boolean algebra on $X$, then the function $\#: \srB \rightarrow [0, +\infty]$ defined by setting $\#(E)$ to be the \emph{\textbf{cardinality}} of $E$ if $E$ is \textit{finite}, and $\#(E) := +\infty$ if $E$ is infinite, is a \emph{\textbf{finitely additive measure}}, known as \underline{\emph{\textbf{counting measure}}}.
\end{example}

\item \begin{proposition} (\textbf{Finitely Additive Measures on Atomic Algebra})\\
Let $\srB$ be a \textbf{finite} Boolean algebra, generated by a finite family $A_1 \xdotx{,} A_k$ of non-empty \textbf{atoms}. For every \textbf{finitely additive measure} $\mu$ on $\srB$ there exists $c_1 \xdotx{,} c_k \in [0, +\infty]$ such that
\begin{align*}
\mu(E) &= \sum_{1 \le j \le k: A_j \subseteq E} c_j.
\end{align*}
Equivalently, if $x_j$ is a point in $A_j$ for each $1 \le j \le k$, then
\begin{align*}
\mu &= \sum_{j=1}^{k}c_j\,\delta_{x_j}. 
\end{align*}
where $c_1 \xdotx{,} c_k$ are \textbf{uniquely} determined by $\mu$.
\end{proposition}
\begin{proof}
Since $\srB$ is the atomic algebra generated by $\set{A_1 \xdotx{,} A_k}$, every measurable subset $E = \cup_{j \in I_{E}}A_{j}$ where $I_E = \set{j:  A_j \subseteq E} \subseteq \set{1 \xdotx{,} k}$ is a finite set. Then due to finite additivity, for any finitely additive measure $\mu$ on $\srB$
\begin{align*}
\mu\paren{E} = \mu\paren{ \bigcup_{1 \le j \le k: A_j \subseteq E}A_{j}} &= \sum_{1 \le j \le k: A_j \subseteq E}\mu(A_{j}) :=  \sum_{1 \le j \le k: A_j \subseteq E}c_j
\end{align*} where $c_j = \mu(A_j)$ for $j=1 \xdotx{,} k$. Given  $A_1 \xdotx{,} A_k$, we have that $c_j$ is uniquely determined by $\mu$. \qed
\end{proof}
\end{itemize}

\subsection{Countably Additive Measure}
\begin{itemize}
\item \begin{definition} 
Let $(X, \srB)$ be a measurable space. \emph{An (unsigned) \underline{\textbf{countably additive measure}}} $\mu$ on $\srB$, or \emph{\textbf{measure}} for short, is a map $\mu: \srB \rightarrow [0, +\infty]$ that obeys the following axioms:
\begin{enumerate}
\item (\emph{\textbf{Empty set}}) $\mu(\emptyset) = 0$.
\item (\emph{\textbf{Countable additivity}}) Whenever $E_1, E_2, \ldots \in \srB$ are a \emph{\textbf{countable sequence}} of \emph{\textbf{disjoint} measurable sets}, then 
\begin{align*}
\mu\paren{\bigcup_{n=1}^{\infty} E_n} &= \sum_{n=1}^{\infty} \mu(E_n).
\end{align*}
\end{enumerate}
A triplet $(X, \srB, \mu)$, where $(X, \srB)$ is a \emph{\textbf{measurable space}} and $\mu: \srB \rightarrow [0, +\infty]$ is a \emph{\textbf{countably additive measure}}, is known as \underline{\emph{\textbf{a measure space}}}.
\end{definition}

\item \begin{remark}
Note the distinction between a \emph{\textbf{measure space}} and a \emph{\textbf{measurable space}}. The latter has the \emph{\textbf{capability}} to be equipped with a \emph{measure}, but the former is \emph{\textbf{actually}} equipped with a \emph{measure}.
\end{remark}

\item \begin{definition} \citep{folland2013real}\\
Let $(X, \srB, \mu)$ be a measure space. 
\begin{itemize}
\item If $\mu(X)< \infty$ (which implies that $\mu(E) < \infty$ for all $E \in \srB$), then $\mu$ is called \emph{\textbf{finite}}. 
\item If $X = \bigcup_{j=1}^{\infty}E_j$ where $E_j \in \srB$ and $\mu(E_j) < \infty$, then $\mu$ is called \emph{\textbf{$\sigma$-finite}}. More generally, if $E = \bigcup_{j=1}^{\infty}E_j$ where $E_j \in \srB$ and $\mu(E_j) < \infty$, then $E$ is said to be \emph{\textbf{$\sigma$-finite}} for $\mu$.
\item If for each $E \in \srB$ with $\mu(E) = \infty$ there exists $F\in \srB$ with $F \subseteq E$ and $0 < \mu(F) < \infty$, then $\mu$ is called \emph{\textbf{semi-finite}}.
\end{itemize}
\end{definition}

\item \begin{example}
The followings are examples for \emph{countably additive measures}:
\begin{enumerate}
\item \emph{\textbf{Lebesgue measure}} is a \emph{countably additive measure} on the \emph{\textbf{Lebesgue $\sigma$-algebra}}, and hence on every sub-$\sigma$-algebra (such as the Borel $\sigma$-algebra)

\item The \emph{\textbf{Dirac measures}} $\delta_x$ are \emph{\textbf{countably additive}} 

\item The \emph{\textbf{counting measure}} $\#$ is \emph{countably additive measure}.

\item The \emph{\textbf{zero measure}} is \textit{countably additive measure}.

\item Any \emph{\textbf{restriction}} of \emph{a countably additive measure} to \emph{\textbf{a measurable subspace}} is again \emph{countably additive}.
\end{enumerate}
\end{example}

\item \begin{example} (\emph{\textbf{Countable combinations of measures}}). \\
Let $(X, \srB)$ be a \emph{measurable space}.
\begin{enumerate}
\item If $\mu$ is a \emph{countably additive measure} on $\srB$, and $c \in [0, +\infty]$, then $c\mu$ is also \emph{countably additive}.

\item If $\mu_1, \mu_2, \ldots$ are a \emph{sequence of countably additive measures} on $\srB$, then \emph{the sum} $\sum_{n=1}^{\infty}\mu_n: E \mapsto \sum_{n=1}^{\infty}\mu_n(E)$ is also a \emph{countably additive measure}.
\end{enumerate}
That is, \underline{\emph{\textbf{the space of all countable additive measures} on $\srB$ is a \textbf{vector space}}}.
\end{example}

\item \begin{remark}
Note that \emph{\textbf{countable additivity measures are necessarily finitely additive}} (by padding out a finite union into a countable union using the empty set), and so countably additive measures inherit all the properties of finitely additive properties, such as monotonicity and finite subadditivity. But one also has additional properties:
\end{remark}

\begin{proposition} 
Let $(X, \srB, \mu)$ be a \textbf{measure space}.
\begin{enumerate}
\item (\textbf{Countable subadditivity}) If $E_1, E_2, \ldots $ are $\srB$-measurable, then 
\begin{align*}
\mu\paren{\bigcup_{n=1}^{\infty} E_n} &\le \sum_{n=1}^{\infty} \mu(E_n).
\end{align*}
\item (\textbf{Upwards monotone convergence}) If $E_1 \subseteq E_2 \subseteq \ldots$ are $\srB$-measurable, then
\begin{align}
\mu\paren{\bigcup_{n=1}^{\infty} E_n} &= \lim\limits_{n\rightarrow \infty}\mu(E_n) = \sup\limits_{n}\mu(E_n). \label{eqn: countable_additive_measure_upward_monotone_convergence}
\end{align}
\item (\textbf{Downwards monotone convergence}) If $E_1 \supseteq E_2 \supseteq \ldots$ are $\srB$-measurable, and \underline{$\mu(E_n) < \infty$ for \textbf{at least one $n$}}, then
\begin{align}
\mu\paren{\bigcap_{n=1}^{\infty} E_n} &= \lim\limits_{n\rightarrow \infty}\mu(E_n) = \inf\limits_{n}\mu(E_n). \label{eqn: countable_additive_measure_downward_monotone_convergence}
\end{align}
\end{enumerate}
\end{proposition}

\item \begin{exercise}
Show that the \textbf{downward monotone convergence} claim can \textbf{fail} if the hypothesis that $\mu(E_n) < \infty$ for at least one $n$ is \textbf{dropped}. 
\end{exercise}

\item \begin{proposition} (\textbf{Dominated convergence for sets}). \citep{tao2011introduction} \\
Let $(X, \srB, \mu)$ be a measure space. Let $E_1, E_2, \ldots $ be a sequence of $\srB$-measurable sets that \textbf{converge} to another set $E$, in the sense that $\mathds{1}_{E_n}$ converges \textbf{pointwise} to $\mathds{1}_{E}$. Then 
\begin{enumerate}
\item $E$ is also $\srB$-measurable.
\item If there exists a $\srB$-measurable set $F$ of \textbf{finite measure} (i.e. $\mu(F) < \infty$) that \textbf{contains all of the $E_n$}, then
\begin{align*}
\lim\limits_{n \rightarrow \infty} \mu(E_n) = \mu(E). 
\end{align*}
(Hint: Apply downward monotonicity to the sets $\bigcup_{n>N}(E_n \Delta E)$.)
\item The previous part of this proposition can \textbf{fail} if the hypothesis that all the $E_n$ are contained in a set of finite measure is \textbf{omitted}.
\end{enumerate}
\end{proposition}
\begin{proof} 
(1) Since $\lim\limits_{n \rightarrow \infty}\mathds{1}_{E_n}(x) = \mathds{1}_{E}(x)$ for every $x$, then for arbitrary $x$, $\forall \epsilon >0$, $\exists\, N \in \bN$ such that for all $ n \ge N$, $\abs{\mathds{1}_{E_n}(x)  - \mathds{1}_{E}(x) } < \epsilon$.   This means that  if $x \in E$ then $\exists\, N \in \bN$ so that $x \in E_n$ for $n \ge N$, i.e.
\begin{align*}
E \subseteq \bigcup_{N =1}^{\infty}\bigcap_{n = N}^{\infty}E_n \in \srB.
\end{align*} So $E$ is  $\srB$-measurable.

(2) First, note that for any $A, B \in \srB$,
\begin{align*}
\abs{\mu\paren{A} - \mu(B)} &= \abs{(\mu(A \setminus B) + \mu(A \cap B) ) - (\mu(B \setminus A) + \mu(A \cap B) )} \\
&= \abs{\mu(A \setminus B)  - (\mu(B \setminus A) }\\
& \le \abs{\mu(A \setminus B) } + \abs{\mu(B \setminus A)}\\
& = \mu(A \setminus B)  + \mu(B \setminus A) = \mu\paren{(A \setminus B)  \cup (B \setminus A)} = \mu(A \Delta B)
\end{align*}
Thus
\begin{align*}
\lim\limits_{n \rightarrow \infty} \abs{\mu(E_n) - \mu(E)} &\le \lim\limits_{n \rightarrow \infty}\paren{ \mu(E_n \Delta E)}\\
&\le  \lim\limits_{N \rightarrow \infty} \mu\paren{\bigcup_{n>N}(E_n \Delta E)}
\end{align*}
As $N$ increases, $\bigcup_{n \ge N}\paren{E_n \Delta E}$ is monotone decreasing. Note that $\bigcup_{n \ge N}\paren{E_n \Delta E} \le F \Delta E$, thus $\mu(\bigcup_{n \ge N}\paren{E_n \Delta E}) \le \mu(F) <\infty$. By downward monotone convergence,
\begin{align*}
 \lim\limits_{N \rightarrow \infty} \mu\paren{\bigcup_{n>N}(E_n \Delta E)} &=  \mu\paren{\bigcap_{N=1}^{\infty}\bigcup_{n\ge N}(E_n\Delta E)}\\
\Rightarrow  \abs{\lim\limits_{n \rightarrow \infty} \mu(E_n) - \mu(E)} &\le \mu\paren{\bigcap_{N=1}^{\infty}\bigcup_{n\ge N}(E_n\Delta E)}
\end{align*}

We claim that $\bigcap_{N=1}^{\infty}\bigcup_{n\ge N}(E_n\Delta E) = \emptyset$, therefore $\mu(\bigcap_{N=1}^{\infty}\bigcup_{n\ge N}(E_n\Delta E)) = 0$.  Note that 
\begin{align*}
\bigcap_{N=1}^{\infty}\bigcup_{n\ge N}(E_n\Delta E) &= \bigcap_{N=1}^{\infty}\bigcup_{n\ge N}((E_n \setminus E) \cup (E\setminus E_n))\\
&= \paren{ \paren{\bigcap_{N=1}^{\infty}\bigcup_{n\ge N}E_n} \setminus E} \bigcup \paren{E \setminus \paren{ \bigcup_{N=1}^{\infty}\bigcap_{n\ge N} E_n} }
\end{align*}
Note that 
\begin{align*}
E = \liminf_{n\rightarrow \infty}E_n = \bigcup_{N=1}^{\infty}\bigcap_{n\ge N} E_n \Rightarrow \paren{E \setminus \paren{ \bigcup_{N=1}^{\infty}\bigcap_{n\ge N} E_n} } = \emptyset
\end{align*}
Also 
\begin{align*}
E = \limsup_{n\rightarrow \infty}E_n = \bigcap_{N=1}^{\infty}\bigcup_{n\ge N} E_n \Rightarrow \paren{ \paren{\bigcap_{N=1}^{\infty}\bigcup_{n\ge N}E_n} \setminus E} = \emptyset
\end{align*}
Therefore $\bigcap_{N=1}^{\infty}\bigcup_{n\ge N}(E_n\Delta E) = \emptyset$. So $\lim\limits_{n \rightarrow \infty}\abs{ \mu(E_n) - \mu(E)} = 0$ thus $\lim\limits_{n \rightarrow \infty} \mu(E_n) = \mu(E)$. \qed
\end{proof}

\item \begin{exercise}  (\textbf{Countably Additive Measures on Countable Set with Discrete $\sigma$-Algebra})\\
Let $X$ be an at most \textbf{countable} set with \textbf{the discrete $\sigma$-algebra}. Show that every measure $\mu$ on this measurable space can be uniquely represented in the form
\begin{align*}
\mu &= \sum_{x \in X}c_x \, \delta_x
\end{align*} for some $c_x \in [0, +\infty]$, thus
\begin{align*}
\mu(E) &= \sum_{x \in E}c_x
\end{align*} for all $E \subseteq X$. (This claim fails in the \textbf{uncountable} case, although showing this is slightly tricky.)
\end{exercise}

\item \begin{definition}(\emph{\textbf{Completeness}}). \citep{tao2011introduction} \\
A \underline{\emph{\textbf{null set}}} of a measure space $(X, \srB, \mu)$ is defined to be a $\srB$-measurable set of \emph{\textbf{measure zero}}. A \emph{\textbf{sub-null}} set is any subset of a null set. 

\emph{A measure space} is said to be \underline{\emph{\textbf{complete}}} if \emph{every sub-null set is a null set}.
\end{definition}

\item \begin{theorem}
The \textbf{Lebesgue measure space} $(\bR^d, \cL[\bR^d], m)$ is \emph{\textbf{complete}}, but \textbf{the Borel measure space} $(\bR^d, \cB[\bR^d], m)$ is \textbf{not}.
\end{theorem}

\item Completion is a convenient property to have in some cases, particularly when dealing with properties that hold almost everywhere.
Fortunately, it is fairly easy to modify any measure space to be complete:

\begin{proposition} (\textbf{Completion}).\\
 Let $(X, \srB, \mu)$  be a measure space. There exists a \textbf{unique refinement} $(X, \overline{\srB}, \overline{\mu})$, known as \textbf{the completion} of $(X, \srB, \mu)$, which is the \textbf{coarsest} refinement of  $(X, \srB, \mu)$ that is \textbf{complete}. Furthermore, $\overline{\srB}$ consists precisely of those sets that differ from a $\srB$-measurable set by \textbf{a $\srB$-subnull set}.
\end{proposition}

\item \begin{remark}
\emph{The Lebesgue measure space} $(\bR^d, \cL[\bR^d], m)$ is the \textbf{completion} of \emph{the Borel measure space} $(\bR^d, \cB[\bR^d], m)$.
\end{remark}

\item \begin{exercise} (\textbf{Approximation by an algebra}). \\
Let $\srA$ be a \textbf{Boolean algebra} on $X$, and let $\mu$ be a measure on the $\sigma$-algebra generated by $\srA$, i.e. $\langle \srA \rangle$.
\begin{enumerate}
\item If $\mu(X) < \infty$, show that for every $E \in \langle \srA \rangle$ and $\epsilon > 0$ there exists $F \in \srA$ such that $\mu(E \Delta F) < \epsilon$.
\item More generally, if $X = \cup_{n=1}^{\infty} A_n$ for some $A_1, A_2, \ldots \in \srA$ with $\mu(A_n) < \infty$ for all $n$, $E \in \langle \srA \rangle$ has finite measure, and $\epsilon > 0$, show that there exists $F \in \srA$ such that $\mu(E \Delta F) < \epsilon$.
\end{enumerate}
\end{exercise}
\end{itemize}

\subsection{Outer Measures and the Carath{\'e}odory Extension Theorem}
\begin{itemize}
\item \begin{remark}
Just like when we constructed the Lebesgue measure, we first constructed the Lebesgue outer measure. We can abstract this process: 
\begin{enumerate}
\item We first define outer measure for \emph{\textbf{all subsets}} in $X$ (not just in $\sigma$-algebra);
\item Then we use the \emph{Carath{\'e}odory Extension Theorem} to constructs a countably additive measure from outer measure.
\end{enumerate}
\end{remark}

\item \begin{definition} (\emph{\textbf{Abstract outer measure}}). \citep{tao2011introduction} \\
Let $X$ be a set. \underline{\emph{An \textbf{abstract outer measure}}} (or \underline{\emph{\textbf{outer measure}}} for short) is a map $\mu^{*}: 2^X \rightarrow [0, +\infty]$ that assigns an \emph{unsigned extended real number} $\mu^{*}(E) \in [0, +\infty]$ to every set $E \subseteq X$ which obeys the following axioms:
\begin{enumerate}
\item (\textbf{\emph{Empty set}}) $\mu^{*}(\emptyset) = 0$.
\item \underline{(\textbf{\emph{Monotonicity}})} If $E \subseteq F$,  then $\mu^{*}(E) \le  \mu^{*}(F)$.
\item  \underline{(\textbf{\emph{Countable subadditivity}})} If $E_1, E_2, \ldots \subseteq X$ is a countable sequence of subsets of X, then 
\begin{align*}
\mu^{*}\paren{\bigcup_{n=1}^{\infty} E_n} &\le \sum_{n=1}^{\infty}\mu^{*}(E_n).
\end{align*}
\end{enumerate}
Outer measures are also known as \underline{\emph{\textbf{exterior measures}}}.
\end{definition}

\item \begin{remark}
\emph{\textbf{Lebesgue outer measure}} $m^{*}$ is \emph{an outer measure}. On the other hand, \emph{\textbf{Jordan outer measure}} $m^{*, J}$ is only \emph{finitely subadditive} rather than \emph{countably subadditive} and thus is \emph{\textbf{not}}, strictly speaking, \emph{an outer measure}.
\end{remark}

\item \begin{remark}
Note that \emph{outer measures} are \emph{\textbf{weaker}} than measures in that they are merely \emph{countably \textbf{subadditive}}, rather than \emph{countably additive}. On the other hand, they are able to \emph{measure \textbf{all subsets} of $X$}, whereas measures can only measure a $\sigma$-algebra of \emph{measurable sets}.
\end{remark}


\item \begin{definition} (\emph{\textbf{Carath{\'e}odory measurability}}).\\
Let $\mu^{*}$ be an \emph{outer measure} on a set $X$. A \emph{set} $E \subseteq X$ is said to be \underline{\emph{\textbf{Carath{\'e}odory measurable}}} \emph{with respect to $\mu^{*}$} (or, \emph{\textbf{$\mu^{*}$-measurable}}) if one has
\begin{align*}
\mu^{*}(A) &= \mu^{*}(A \setminus E) + \mu^{*}(A \cap E)
\end{align*} for every set $A \subseteq X$.
\end{definition}

\item \begin{example} (\emph{\textbf{Null sets are Carath{\'e}odory measurable}}). \\
Suppose that $E$ is a \emph{\textbf{null set}} for \emph{an \textbf{outer measure}} $\mu^{*}$  (i.e. $\mu^{*}(E) = 0$).  Then 
that \emph{$E$ is Carath{\'e}odory measurable with respect to $\mu^{*}$}.
\end{example}

\item \begin{example} (\emph{\textbf{Compatibility with Lebesgue measurability}}). 
A set $E \subseteq \bR^d$ is \emph{Carath\'eodory measurable with respect to Lebesgue outer measurable} if and only if it is \emph{Lebesgue measurable}.
\end{example}

\item \begin{theorem} (\textbf{Carath\'eodory extension theorem}). \citep{tao2011introduction} \\
Let $\mu^{*}: 2^X \rightarrow [0, +\infty]$ be an outer measure on a set X, let $\srB$ be the collection of all subsets of X that are \textbf{Carath\'eodory measurable with respect to $\mu^{*}$}, and let $\mu: \srB \rightarrow [0, +\infty]$ be the \textbf{restriction} of $\mu^{*}$ to $\srB$ (thus $\mu(E) := \mu^{*}(E)$
whenever $E \in \srB$). Then \textbf{$\srB$ is a $\sigma$-algebra}, and \textbf{$\mu$ is a measure}.
\end{theorem}

\item \begin{remark}
The measure $\mu$ constructed by \emph{the Carath\'eodory extension theorem} is automatically \emph{\textbf{complete}}.
\end{remark}

\item \begin{proposition}
Let $\srB$ be a \textbf{Boolean algebra} on a set $X$. Then $\srB$ is a $\sigma$-algebra \textbf{if and only if} it is \textbf{closed} under countable \emph{disjoint}
unions, which means that $\bigcup_{n=1}^{\infty} E_n \in \srB$ whenever $E_1, E_2, E_3, \ldots \in \srB$ are a countable sequence of \textbf{disjoint} sets in $\srB$.
\end{proposition}

\item \begin{definition}  (\emph{\textbf{Pre-measure}}). \\
\underline{\emph{A \textbf{pre-measure}}} on a \emph{\textbf{Boolean algebra}} $\srB_{0}$  is a function $\mu_0 : \srB_0 \rightarrow [0, +\infty]$ that satisfies the conditions:
\begin{enumerate}
\item (\textbf{\emph{Empty Set}}): $\mu_0(\emptyset) = 0$
\item (\textbf{\emph{Countably Additivity}}): IF $E_1, E_2, \ldots \in \srB_0$ are \emph{disjoint sets} such that $\bigcup_{n=1}^{\infty} E_n$ is in $\srB_0$,
 \begin{align*}
\mu_0\paren{\bigcup_{n=1}^{\infty} E_n} &= \sum_{n=1}^{\infty} \mu_0(E_n).
\end{align*} 
\end{enumerate} 
\end{definition}

\item \begin{remark}
A \emph{pre-measure} $\mu_0$ is a \emph{\textbf{finitely additive measure}} that \emph{\textbf{already}} is \emph{countably additive} \emph{\textbf{within}} a Boolean algebra $\srB_0$. 
\end{remark}

\item \begin{remark}
\emph{The countably additivity condition} for pre-measure can be releaxed to be \emph{the countably subadditivity} $\mu_0(\cup_{n=1}^{\infty} E_n) \le \sum_{n=1}^{\infty} \mu_0(E_n)$ without affecting the definition of a pre-measure.
\end{remark}

\item \begin{proposition} \label{prop: outer_measure_premeasure}
Let $\srB \subset 2^X$ and $\mu_0: \srB \rightarrow [0, +\infty]$ be such that $\emptyset, X \in \srB$, and $\mu_0(\emptyset) = 0$. For any $A \subseteq X$, define 
\begin{align*}
\mu^{*}(A) &:= \inf\set{\sum_{j=1}^{\infty}\mu_0(E_j): E_j \in \srB, \text{ and } A \subseteq \bigcup_{j=1}^{\infty}E_j}. 
\end{align*} Then $\mu^{*}$ is an outer measure. 
\end{proposition}

\item \begin{theorem} (\textbf{Hahn-Kolmogorov Theorem}).\\
Every \textbf{pre-measure} $\mu_0 : \srB_0 \rightarrow [0, +\infty]$  on a Boolean algebra $\srB_{0}$ in $X$ can be \textbf{extended} to a \textbf{countably additive measure} $\mu : \srB \rightarrow [0, +\infty]$.
\end{theorem}

\item \begin{remark}
We can construct an \emph{outer measure} $\mu^{*}$ according to Proposition \ref{prop: outer_measure_premeasure}. Let $\srB$ be the \emph{collection} of all sets $E \subseteq X$ that are \textit{Carath\'eodory measurable with respect to $µ^{*}$ ($\mu^{*}$-measurable)}, and let $\mu$ be the \emph{restriction} of $\mu^{*}$  to $\srB$. The tuple $(X, \srB, \mu)$ is what we want in \emph{Hahn-Kolmogorov theorem}. 

\emph{The measure $\mu$} constructed in this way is called \emph{\textbf{ \underline{the Hahn-Kolmogorov extension} of the pre-measure $\mu_0$}}. 
\end{remark}

\item \begin{proposition} (\textbf{Uniqueness of the Hahn-Kolmogorov Extension})\\
Let $\mu_0 : \srB_0 \rightarrow [0, +\infty]$ be a \textbf{pre-measure}, let $\mu : \srB \rightarrow [0, +\infty]$ be the \textbf{Hahn-Kolmogorov extension} of $\mu_0$, and let $\mu' : \srB' \rightarrow  [0, +\infty]$ be \textbf{another} countably additive extension of $\mu_0$. Suppose also that $\mu_0$ is \textbf{$\sigma$-finite}, which means that one can express the whole space $X$ as the countable union of sets $E_1, E_2, \ldots \in \srB_{0}$ for which $\mu_0(E_n) < \infty$ for all $n$. Then $\mu$ and $\mu'$ agree on their common domain of definition. In other words, show that  $\mu(E) = \mu'(E)$ for all $E \in \srB \cap \srB'$.
\end{proposition} (Hint: first show that $\mu'(E) \le \mu^{*}(E)$ for all $E \in \srB'$.)

\item \begin{exercise} The purpose of this exercise is to show that the \textbf{$\sigma$-finite hypothesis} above \textbf{cannot be removed}. Let $\srA$ be the
collection of all subsets in $\bR$ that can be expressed as finite unions of half-open intervals $[a, b)$. Let $\mu_0 : \srA \rightarrow [0, +\infty]$ be the function such
that $\mu_0(E) = +\infty$ for non-empty $E$ and $\mu_0(\emptyset) = 0$.
\begin{enumerate}
\item Show that $\mu_0$ is a pre-measure.
\item Show that $\langle A \rangle$ is the Borel $\sigma$-algebra $\cB[\bR]$.
\item Show that the Hahn-Kolmogorov extension $\mu: \cB[\bR] \rightarrow [0, +\infty]$ of $\mu_0$ assigns an \textbf{infinite measure} to any non-empty Borel set.
\item Show that \textbf{counting measure} $\#$ (or more generally, $c\#$ for any $c \in (0, +\infty]$) is \textbf{another extension} of $\mu_0$ on $\cB[\bR]$.
\end{enumerate}
\end{exercise}
\end{itemize}




\newpage
\bibliographystyle{plainnat}
\bibliography{reference.bib}
\end{document}