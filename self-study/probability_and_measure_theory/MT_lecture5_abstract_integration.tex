\documentclass[11pt]{article}
\usepackage[scaled=0.92]{helvet}
\usepackage{geometry}
\geometry{letterpaper,tmargin=1in,bmargin=1in,lmargin=1in,rmargin=1in}
\usepackage[parfill]{parskip} % Activate to begin paragraphs with an empty line rather than an indent %\usepackage{graphicx}
\usepackage{amsmath,amssymb, mathrsfs, dsfont}
\usepackage{tabularx}
\usepackage[font=footnotesize,labelfont=bf]{caption}
\usepackage{graphicx}
\usepackage{xcolor}
%\usepackage[linkbordercolor ={1 1 1} ]{hyperref}
%\usepackage[sf]{titlesec}
\usepackage{natbib}
\usepackage{../../Tianpei_Report}

%\usepackage{appendix}
%\usepackage{algorithm}
%\usepackage{algorithmic}

%\renewcommand{\algorithmicrequire}{\textbf{Input:}}
%\renewcommand{\algorithmicensure}{\textbf{Output:}}



\begin{document}
\title{Lecture 5:  Abstract Integrations}
\author{ Tianpei Xie}
\date{Nov 15th., 2022}
\maketitle
\tableofcontents
\newpage
\section{Recall}
\begin{itemize}
\item
\begin{definition}
Let $\srB$ be a \emph{Boolean algebra} on a space $X$. An (unsigned) \underline{\emph{\textbf{finitely additive measure}}} $\mu$ on $\srB$ is a map $\mu : \srB \rightarrow [0,+\infty]$ that obeys the following axioms
\begin{enumerate}
\item $\mu(\emptyset) = 0$;
\item \emph{\textbf{Finite union}}: for any  \emph{\textbf{disjoint sets}} $A, B \in \srB$, 
\begin{align*}
\mu\paren{A\cup B} = \mu(A)+ \mu(B).
\end{align*} 
\end{enumerate}
\end{definition}

\item \begin{proposition} (\textbf{Properties of Finitely Additive Measure}) \citep{tao2011introduction}\\
Let $\mu: \srB \rightarrow [0, +\infty]$be a finitely additive measure on a Boolean $\sigma$-algebra $\srB$. 
\begin{enumerate}
\item (\textbf{Monotonicity}) If $E, F$ are $\srB$-measurable and $E \subseteq F$, then
\begin{align*}
\mu(E) \le \mu(F).
\end{align*}
\item  (\textbf{Finite additivity}) If $k$ is a natural number, and $E_1 \xdotx{,} E_k$ are $\srB$-measurable and \textbf{disjoint}, then 
\begin{align*}
\mu(E_1 \xdotx{\cup} E_k) = \mu(E_1)  \xdotx{+} \mu(E_k).
\end{align*}
\item  (\textbf{Finite subadditivity}) If $k$ is a natural number, and $E_1 \xdotx{,} E_k$ are $\srB$-measurable, then
\begin{align*}
\mu(E_1 \xdotx{\cup} E_k) \le \mu(E_1)  \xdotx{+} \mu(E_k).
\end{align*}
\item \textbf{(Inclusion-exclusion for two sets}) If $E, F$ are $\srB$-measurable, then
\begin{align*}
\mu(E \cup F ) + \mu(E \cap F ) = \mu(E) + \mu(F).
\end{align*}
\end{enumerate}
(Caution: remember that the cancellation law $a+c = b+c \Rightarrow a = b$ does not hold in [0; +1] if c is infinite, and so the use of cancellation
(or subtraction) should be avoided if possible.)
\end{proposition}

\item  \begin{example} (\emph{\textbf{Dirac measure}}). \\
Let $x \in X$ and $\srB$ be an arbitrary \emph{Boolean algebra} on $X$. Then \underline{\emph{\textbf{the Dirac measure}}} $\delta_x$ at $x$, defined by
setting $\delta_x(E) := \ind{x \in E}$, is \emph{\textbf{finitely additive}}.
\end{example}

\item \begin{example} (\emph{\textbf{Zero measure}}). \\
The \emph{\textbf{zero measure}} $0: E \mapsto 0$ is a \emph{finitely additive measure} on any Boolean algebra.
\end{example}

\item \begin{example} (\emph{\textbf{Linear combinations of measures}}). \\
If $\srB$ is a Boolean algebra on $X$, and $\mu, \nu: \srB \rightarrow [0, +\infty]$ are \emph{finitely additive measures} on $\srB$, then $\mu + \nu: E \mapsto \mu(E)+ \nu(E)$ is also a \emph{\textbf{finitely additive measure}}, as is $c\mu: E \mapsto c \times \mu(E)$ for any $c \in [0, +\infty]$. Thus, for instance, the sum of Lebesgue measure and a Dirac measure is also a finitely additive measure on the Lebesgue algebra (or on any of its sub-algebras).

In other word, \underline{\emph{\textbf{the space of all finitely additive measures} on $\srB$ is a \textbf{vector space}}}.
\end{example}

\item \begin{example} (\emph{\textbf{Restriction of a measure}}).\\
If $\srB$ is a Boolean algebra on $X$, $\mu: \srB \rightarrow [0, +\infty]$ is a \emph{finitely additive measure}, and $Y$ is a $\srB$-measurable subset of $X$, then \emph{\textbf{the restriction}} $\mu|_{Y}: \srB|_{Y}  \rightarrow [0, +\infty]$ of $\srB$ to $Y$, defined by setting $\mu|_{Y}(E) := \mu(E)$ whenever $E \in \srB|_{Y}$ (i.e.
if $E \in \srB$ and $E \subseteq Y$), is also a \emph{\textbf{finitely additive measure}}.
\end{example}

\item \begin{example} (\emph{\textbf{Counting measure}}).\\
If $\srB$ is a Boolean algebra on $X$, then the function $\#: \srB \rightarrow [0, +\infty]$ defined by setting $\#(E)$ to be the \emph{\textbf{cardinality}} of $E$ if $E$ is \textit{finite}, and $\#(E) := +\infty$ if $E$ is infinite, is a \emph{\textbf{finitely additive measure}}, known as \underline{\emph{\textbf{counting measure}}}.
\end{example}

\item \begin{proposition} (\textbf{Finitely Additive Measures on Atomic Algebra})\\
Let $\srB$ be a \textbf{finite} Boolean algebra, generated by a finite family $A_1 \xdotx{,} A_k$ of non-empty \textbf{atoms}. For every \textbf{finitely additive measure} $\mu$ on $\srB$ there exists $c_1 \xdotx{,} c_k \in [0, +\infty]$ such that
\begin{align*}
\mu(E) &= \sum_{1 \le j \le k: A_j \subseteq E} c_j.
\end{align*}
Equivalently, if $x_j$ is a point in $A_j$ for each $1 \le j \le k$, then
\begin{align*}
\mu &= \sum_{j=1}^{k}c_j\,\delta_{x_j}. 
\end{align*}
where $c_1 \xdotx{,} c_k$ are \textbf{uniquely} determined by $\mu$.
\end{proposition}

\item \begin{definition} 
Let $(X, \srB)$ be a measurable space. \emph{An (unsigned) \underline{\textbf{countably additive measure}}} $\mu$ on $\srB$, or \emph{\textbf{measure}} for short, is a map $\mu: \srB \rightarrow [0, +\infty]$ that obeys the following axioms:
\begin{enumerate}
\item (\emph{\textbf{Empty set}}) $\mu(\emptyset) = 0$.
\item (\emph{\textbf{Countable additivity}}) Whenever $E_1, E_2, \ldots \in \srB$ are a \emph{\textbf{countable sequence}} of \emph{\textbf{disjoint} measurable sets}, then 
\begin{align*}
\mu\paren{\bigcup_{n=1}^{\infty} E_n} &= \sum_{n=1}^{\infty} \mu(E_n).
\end{align*}
\end{enumerate}
A triplet $(X, \srB, \mu)$, where $(X, \srB)$ is a \emph{\textbf{measurable space}} and $\mu: \srB \rightarrow [0, +\infty]$ is a \emph{\textbf{countably additive measure}}, is known as \underline{\emph{\textbf{a measure space}}}.
\end{definition}

\item \begin{remark}
Note the distinction between a \emph{\textbf{measure space}} and a \emph{\textbf{measurable space}}. The latter has the \emph{\textbf{capability}} to be equipped with a \emph{measure}, but the former is \emph{\textbf{actually}} equipped with a \emph{measure}.
\end{remark}

\item \begin{definition} \citep{folland2013real}\\
Let $(X, \srB, \mu)$ be a measure space. 
\begin{itemize}
\item If $\mu(X)< \infty$ (which implies that $\mu(E) < \infty$ for all $E \in \srB$), then $\mu$ is called \emph{\textbf{finite}}. 
\item If $X = \bigcup_{j=1}^{\infty}E_j$ where $E_j \in \srB$ and $\mu(E_j) < \infty$, then $\mu$ is called \emph{\textbf{$\sigma$-finite}}. More generally, if $E = \bigcup_{j=1}^{\infty}E_j$ where $E_j \in \srB$ and $\mu(E_j) < \infty$, then $E$ is said to be \emph{\textbf{$\sigma$-finite}} for $\mu$.
\item If for each $E \in \srB$ with $\mu(E) = \infty$ there exists $F\in \srB$ with $F \subseteq E$ and $0 < \mu(F) < \infty$, then $\mu$ is called \emph{\textbf{semi-finite}}.
\end{itemize}
\end{definition}

\item \begin{proposition} 
Let $(X, \srB, \mu)$ be a \textbf{measure space}.
\begin{enumerate}
\item (\textbf{Countable subadditivity}) If $E_1, E_2, \ldots $ are $\srB$-measurable, then 
\begin{align*}
\mu\paren{\bigcup_{n=1}^{\infty} E_n} &\le \sum_{n=1}^{\infty} \mu(E_n).
\end{align*}
\item (\textbf{Upwards monotone convergence}) If $E_1 \subseteq E_2 \subseteq \ldots$ are $\srB$-measurable, then
\begin{align}
\mu\paren{\bigcup_{n=1}^{\infty} E_n} &= \lim\limits_{n\rightarrow \infty}\mu(E_n) = \sup\limits_{n}\mu(E_n). \label{eqn: countable_additive_measure_upward_monotone_convergence}
\end{align}
\item (\textbf{Downwards monotone convergence}) If $E_1 \supseteq E_2 \supseteq \ldots$ are $\srB$-measurable, and \underline{$\mu(E_n) < \infty$ for \textbf{at least one $n$}}, then
\begin{align}
\mu\paren{\bigcap_{n=1}^{\infty} E_n} &= \lim\limits_{n\rightarrow \infty}\mu(E_n) = \inf\limits_{n}\mu(E_n). \label{eqn: countable_additive_measure_downward_monotone_convergence}
\end{align}
\end{enumerate}
\end{proposition}


\item \begin{proposition} (\textbf{Dominated convergence for sets}). \citep{tao2011introduction} \\
Let $(X, \srB, \mu)$ be a measure space. Let $E_1, E_2, \ldots $ be a sequence of $\srB$-measurable sets that \textbf{converge} to another set $E$, in the sense that $\mathds{1}_{E_n}$ converges \textbf{pointwise} to $\mathds{1}_{E}$. Then 
\begin{enumerate}
\item $E$ is also $\srB$-measurable.
\item If there exists a $\srB$-measurable set $F$ of \textbf{finite measure} (i.e. $\mu(F) < \infty$) that \textbf{contains all of the $E_n$}, then
\begin{align*}
\lim\limits_{n \rightarrow \infty} \mu(E_n) = \mu(E). 
\end{align*}
(Hint: Apply downward monotonicity to the sets $\bigcup_{n>N}(E_n \Delta E)$.)
\item The previous part of this proposition can \textbf{fail} if the hypothesis that all the $E_n$ are contained in a set of finite measure is \textbf{omitted}.
\end{enumerate}
\end{proposition}

\item \begin{exercise}  (\textbf{Countably Additive Measures on Countable Set with Discrete $\sigma$-Algebra})\\
Let $X$ be an at most \textbf{countable} set with \textbf{the discrete $\sigma$-algebra}. Show that every measure $\mu$ on this measurable space can be uniquely represented in the form
\begin{align*}
\mu &= \sum_{x \in X}c_x \, \delta_x
\end{align*} for some $c_x \in [0, +\infty]$, thus
\begin{align*}
\mu(E) &= \sum_{x \in E}c_x
\end{align*} for all $E \subseteq X$. (This claim fails in the \textbf{uncountable} case, although showing this is slightly tricky.)
\end{exercise}

\item \begin{definition}(\emph{\textbf{Completeness}}). \citep{tao2011introduction} \\
A \underline{\emph{\textbf{null set}}} of a measure space $(X, \srB, \mu)$ is defined to be a $\srB$-measurable set of \emph{\textbf{measure zero}}. A \emph{\textbf{sub-null}} set is any subset of a null set. 

\emph{A measure space} is said to be \underline{\emph{\textbf{complete}}} if \emph{every sub-null set is a null set}.
\end{definition}

\item \begin{theorem}
The \textbf{Lebesgue measure space} $(\bR^d, \cL[\bR^d], m)$ is \emph{\textbf{complete}}, but \textbf{the Borel measure space} $(\bR^d, \cB[\bR^d], m)$ is \textbf{not}.
\end{theorem}

\item Completion is a convenient property to have in some cases, particularly when dealing with properties that hold almost everywhere.
Fortunately, it is fairly easy to modify any measure space to be complete:

\begin{proposition} (\emph{\textbf{Completion}}).\\
 Let $(X, \srB, \mu)$  be a measure space. There exists a \textbf{unique refinement} $(X, \overline{\srB}, \overline{\mu})$, known as \textbf{the completion} of $(X, \srB, \mu)$, which is the \textbf{coarsest} refinement of  $(X, \srB, \mu)$ that is \textbf{complete}. Furthermore, $\overline{\srB}$ consists precisely of those sets that differ from a $\srB$-measurable set by \textbf{a $\srB$-subnull set}.
\end{proposition}

\item \begin{definition} (\emph{\textbf{Abstract outer measure}}). \citep{tao2011introduction} \\
Let $X$ be a set. \underline{\emph{An \textbf{abstract outer measure}}} (or \underline{\emph{\textbf{outer measure}}} for short) is a map $\mu^{*}: 2^X \rightarrow [0, +\infty]$ that assigns an \emph{unsigned extended real number} $\mu^{*}(E) \in [0, +\infty]$ to every set $E \subseteq X$ which obeys the following axioms:
\begin{enumerate}
\item (\textbf{\emph{Empty set}}) $\mu^{*}(\emptyset) = 0$.
\item \underline{(\textbf{\emph{Monotonicity}})} If $E \subseteq F$,  then $\mu^{*}(E) \le  \mu^{*}(F)$.
\item  \underline{(\textbf{\emph{Countable subadditivity}})} If $E_1, E_2, \ldots \subseteq X$ is a countable sequence of subsets of X, then 
\begin{align*}
\mu^{*}\paren{\bigcup_{n=1}^{\infty} E_n} &\le \sum_{n=1}^{\infty}\mu^{*}(E_n).
\end{align*}
\end{enumerate}
Outer measures are also known as \underline{\emph{\textbf{exterior measures}}}.
\end{definition}

\item \begin{definition} (\emph{\textbf{Carath{\'e}odory measurability}}).\\
Let $\mu^{*}$ be an \emph{outer measure} on a set $X$. A \emph{set} $E \subseteq X$ is said to be \underline{\emph{\textbf{Carath{\'e}odory measurable}}} \emph{with respect to $\mu^{*}$} (or, \emph{\textbf{$\mu^{*}$-measurable}}) if one has
\begin{align*}
\mu^{*}(A) &= \mu^{*}(A \setminus E) + \mu^{*}(A \cap E)
\end{align*} for every set $A \subseteq X$.
\end{definition}

\item \begin{example} (\emph{\textbf{Null sets are Carath{\'e}odory measurable}}). \\
Suppose that $E$ is a \emph{\textbf{null set}} for \emph{an \textbf{outer measure}} $\mu^{*}$  (i.e. $\mu^{*}(E) = 0$).  Then 
that \emph{$E$ is Carath{\'e}odory measurable with respect to $\mu^{*}$}.
\end{example}

\item \begin{example} (\emph{\textbf{Compatibility with Lebesgue measurability}}). 
A set $E \subseteq \bR^d$ is \emph{Carath\'eodory measurable with respect to Lebesgue outer measurable} if and only if it is \emph{Lebesgue measurable}.
\end{example}

\item \begin{theorem} (\textbf{Carath\'eodory extension theorem}). \citep{tao2011introduction} \\
Let $\mu^{*}: 2^X \rightarrow [0, +\infty]$ be an outer measure on a set X, let $\srB$ be the collection of all subsets of X that are \textbf{Carath\'eodory measurable with respect to $\mu^{*}$}, and let $\mu: \srB \rightarrow [0, +\infty]$ be the \textbf{restriction} of $\mu^{*}$ to $\srB$ (thus $\mu(E) := \mu^{*}(E)$
whenever $E \in \srB$). Then \textbf{$\srB$ is a $\sigma$-algebra}, and \textbf{$\mu$ is a measure}.
\end{theorem}

\item \begin{definition}  (\emph{\textbf{Pre-measure}}). \\
\underline{\emph{A \textbf{pre-measure}}} on a \emph{\textbf{Boolean algebra}} $\srB_{0}$  is a function $\mu_0 : \srB_0 \rightarrow [0, +\infty]$ that satisfies the conditions:
\begin{enumerate}
\item (\textbf{\emph{Empty Set}}): $\mu_0(\emptyset) = 0$
\item (\textbf{\emph{Countably Additivity}}): IF $E_1, E_2, \ldots \in \srB_0$ are \emph{disjoint sets} such that $\bigcup_{n=1}^{\infty} E_n$ is in $\srB_0$,
 \begin{align*}
\mu_0\paren{\bigcup_{n=1}^{\infty} E_n} &= \sum_{n=1}^{\infty} \mu_0(E_n).
\end{align*} 
\end{enumerate} 
\end{definition}

\item \begin{remark}
A \emph{pre-measure} $\mu_0$ is a \emph{\textbf{finitely additive measure}} that \emph{\textbf{already}} is \emph{countably additive} \emph{\textbf{within}} a Boolean algebra $\srB_0$. 
\end{remark}

\item \begin{remark}
\emph{The countably additivity condition} for pre-measure can be releaxed to be \emph{the countably subadditivity} $\mu_0(\cup_{n=1}^{\infty} E_n) \le \sum_{n=1}^{\infty} \mu_0(E_n)$ without affecting the definition of a pre-measure.
\end{remark}

\item \begin{proposition} \label{prop: outer_measure_premeasure}
Let $\srB \subset 2^X$ and $\mu_0: \srB \rightarrow [0, +\infty]$ be such that $\emptyset, X \in \srB$, and $\mu_0(\emptyset) = 0$. For any $A \subseteq X$, define 
\begin{align*}
\mu^{*}(A) &:= \inf\set{\sum_{j=1}^{\infty}\mu_0(E_j): E_j \in \srB, \text{ and } A \subseteq \bigcup_{j=1}^{\infty}E_j}. 
\end{align*} Then $\mu^{*}$ is an outer measure. 
\end{proposition}

\item \begin{theorem} (\textbf{Hahn-Kolmogorov Theorem}).\\
Every \textbf{pre-measure} $\mu_0 : \srB_0 \rightarrow [0, +\infty]$  on a Boolean algebra $\srB_{0}$ in $X$ can be \textbf{extended} to a \textbf{countably additive measure} $\mu : \srB \rightarrow [0, +\infty]$.
\end{theorem}

\item \begin{remark}
We can construct an \emph{outer measure} $\mu^{*}$ according to Proposition \ref{prop: outer_measure_premeasure}. Let $\srB$ be the \emph{collection} of all sets $E \subseteq X$ that are \textit{Carath\'eodory measurable with respect to $\mu^{*}$ ($\mu^{*}$-measurable)}, and let $\mu$ be the \emph{restriction} of $\mu^{*}$  to $\srB$. The tuple $(X, \srB, \mu)$ is what we want in \emph{Hahn-Kolmogorov theorem}. 

\emph{The measure $\mu$} constructed in this way is called \emph{\textbf{ \underline{the Hahn-Kolmogorov extension} of the pre-measure $\mu_0$}}. 
\end{remark}

\item \begin{proposition} (\textbf{Uniqueness of the Hahn-Kolmogorov Extension})\\
Let $\mu_0 : \srB_0 \rightarrow [0, +\infty]$ be a \textbf{pre-measure}, let $\mu : \srB \rightarrow [0, +\infty]$ be the \textbf{Hahn-Kolmogorov extension} of $\mu_0$, and let $\mu' : \srB' \rightarrow  [0, +\infty]$ be \textbf{another} countably additive extension of $\mu_0$. Suppose also that $\mu_0$ is \textbf{$\sigma$-finite}, which means that one can express the whole space $X$ as the countable union of sets $E_1, E_2, \ldots \in \srB_{0}$ for which $\mu_0(E_n) < \infty$ for all $n$. Then $\mu$ and $\mu'$ agree on their common domain of definition. In other words, show that  $\mu(E) = \mu'(E)$ for all $E \in \srB \cap \srB'$.
\end{proposition} 
\end{itemize}
\newpage
\section{Measurable Functions, and Integration on a Measure Space}
\subsection{Measurable Functions}
\begin{itemize}
\item \begin{definition}
Let $(X, \srB)$ be a measurable space, and let $f : X \rightarrow [0, +\infty]$ or $f: X \rightarrow \bC$ be an \emph{\textbf{unsigned}} or \emph{\textbf{complex-valued function}}. We say that $f$ is \underline{\emph{\textbf{measurable}}} if $f^{-1}(U)$ is \underline{\emph{$\srB$-\textbf{measurable}}} for every \emph{\textbf{open subset}} $U$ of $[0, +\infty]$ or $\bC$.
\end{definition}

\item \begin{remark}
The inverse image of a Lebesgue measurable set by a \emph{measurable function} need not remain Lebesgue. measurable. This is due to the definition of above measureable function. The pre-image of $E$ is Lebesgue measureable, if if $E$ has a slightly stronger measurability property than Lebesgue measurability, namely \emph{\textbf{Borel measurability}}.
\end{remark}

\item In general, we have the following
\begin{definition}
For $f: X\rightarrow Y$, and $X\equiv (X, \srF)$, $Y\equiv (Y, \srB)$ are measurable spaces, then $f$ is called \underline{\emph{\textbf{$(\srF, \srB)$ measureable}}} (or $(\srF/\srB)$ measureable or, simply, \emph{measureble}), if $f^{-1}(E) \in \srF$ for every $E\in \srB$.
\end{definition}

\item  \begin{definition} 
Note that if $\set{(Y_{\alpha}, \srB_{\alpha})}$ is a family of measureable spaces, and $\set{f_{\alpha}}$ for $f_{\alpha}: X\rightarrow Y_{\alpha}$, then there is a \emph{\textbf{unique smallest}} $\sigma$-algebra on $X$ so that $\set{f_{\alpha}}$ are all measureable. It is generated by $f_{\alpha}^{-1}(E_{\alpha}), E_{\alpha}\in \srB_{\alpha}$. It is called the \emph{\textbf{$\sigma$-algebra generated by $\set{f_{\alpha}}$}}. 

In particular, $X= \prod_{\alpha}Y_{\alpha}$ has \emph{\textbf{product $\sigma$-algebra}} that is \emph{generated by coordinate functions} $\set{\pi_{\alpha}}$.
\end{definition}

\item \begin{proposition}
Let $(X, \srB)$ be a measurable space.
\begin{enumerate}
\item $f : X \rightarrow [0, +\infty]$ is \textbf{measurable} if and only if the \textbf{level sets} $\set{x \in X : f(x) > \lambda}$ are $\srB$-measurable.
\item The \textbf{indicator function} $\mathds{1}_E$ of a set $E \subseteq X$ is \textbf{measurable} if and only if $E$ itself is $\srB$-\textbf{measurable}.
\item $f : X \rightarrow [0, +\infty]$ or $f: X \rightarrow \bC$ is measurable if and only if $f^{-1}(E)$ is $\srB$-measurable for every \textbf{Borel-measurable} subset $E$ of $[0, +\infty]$ or $\bC$.
\item $f: X \rightarrow \bC$ is measurable if and only if its real and imaginary parts are measurable.
\item $f: X \rightarrow \bR$ is measurable if and only if the \textbf{magnitudes} $f_{+} := \max\{f, 0\}$, $f_{-} := \max\{-f, 0\}$ of its \textbf{positive} and \textbf{negative} parts are \textbf{measurable}.
\item  If $f_n : X \rightarrow [0, +\infty]$ are a sequence of \textbf{measurable} functions that converge \textbf{pointwise} to a limit $f : X \rightarrow [0, +\infty]$, then $f$ is also \textbf{measurable}. The same claim holds if $[0, +\infty]$ is replaced by $\bC$.
\item  If $f : X \rightarrow [0, +\infty]$ is measurable and $\varphi: [0, +\infty] \rightarrow [0, +\infty]$ is \textbf{continuous}, the composite $\varphi \circ f$ is measurable. The same claim holds if $[0, +\infty]$ is replaced by $\bC$.
\item The \textbf{sum} or \textbf{product} of two \textbf{measurable} functions in $[0, +\infty]$ or $\bC$ is still measurable.
\end{enumerate}
\end{proposition}

\item  \begin{definition} 
A function $f: (X,\srF)\rightarrow (Y,\srB)$ is \underline{\emph{\textbf{simple}}} if it only takes \emph{\textbf{finitely many} different values} $s_{1},\cdots, s_{k}\in Y$. 

Then the $\sigma$-algebra $f^{-1}(\srB)$ reduce to $\sigma\paren{\set{f^{-1}(\set{s_{\alpha}})}_{\alpha=1}^{k}}$, the \emph{\textbf{finite $\sigma$-algebra}} generated by \emph{atomic algebra} with atoms $E_{\alpha}\equiv f^{-1}(\set{s_{\alpha}})$. The \emph{\textbf{canonical representation}} of $f$ is 
\begin{align*}
f = \sum_{\alpha=1}^{k}s_{\alpha}\ind{E_{\alpha}},
\end{align*}
which is determined up to a reordering.
\end{definition}


\item \begin{proposition} (\textbf{Measurable Function with respect to Atomic Algebra is Simple})\\
Let $(X, \srB)$ be a measurable space that is \textbf{atomic}, thus $\srB = \srA((A_{\alpha})_{\alpha\in I})$ for some partition $X = \bigcup_{\alpha\in I}A_{\alpha}$ of X into disjoint non-empty atoms. A function $f : X \rightarrow [0,+\infty]$ or $f : X \rightarrow \bC$ is measurable if and only if it is \textbf{constant} on each atom, or equivalently if one has a \textbf{representation of the form}
\begin{align*}
f(x)&= \sum_{\alpha \in I}c_{\alpha}\ind{x\in A_{\alpha}},
\end{align*} for some constants $c_{\alpha} \in [0;+\infty]$ or in $\bC$ as appropriate. Furthermore, the $c_{\alpha}$ are uniquely determined by $f$.
\end{proposition}

\item \begin{theorem} (\textbf{Egorov's theorem}). \citep{tao2011introduction} \\
Let $(X, \srB, \mu)$ be a \textbf{finite measure space} (so $\mu(X) < \infty$), and let $f_n : X \rightarrow \bC$ be a sequence of measurable functions that \textbf{converge pointwise almost everywhere} to a limit $f : X \rightarrow \bC$.  For $\epsilon > 0$, there exists a measurable set $E$ of measure \textbf{at most} $\epsilon$ such that $f_n$ converges \textbf{uniformly} to $f$ outside of $E$. 
\end{theorem}

\item \begin{remark}
Give an example to show that the claim can \textbf{fail} when the measure $\mu$ is not finite.
\end{remark}


%\item An unsigned (measureable) function $f$ can be well-approximated by an unsigned function $g$, which is measureable, bounded with finite measure support, in $L^{1}(\bR^{d})$ sense. 
\end{itemize}
\subsection{Simple Integral of Simple Functions}
\begin{itemize}
\item \begin{definition} (\emph{\textbf{Simple integral}}).\\
 Let $(X, \srB, \mu)$ be a measure space with $\srB$ \emph{\textbf{finite}} (i.e., its \emph{cardinality} is \emph{finite} and there are only \emph{finitely many measurable sets}). $X$ can then be partitioned into a finite number of atoms $A_1, \cdots, A_n$. If $f : X \rightarrow [0, +\infty]$ is measurable, it has \emph{\textbf{a unique representation}} of the form
\begin{align*}
f(x)&= \sum_{\alpha \in I}c_{\alpha}\ind{x\in A_{\alpha}},
\end{align*} 
for some constants $c_{\alpha} \in [0;+\infty]$. We then define the \underline{\emph{\textbf{simple integral}}}
$\text{simp}\int_{X} f d\mu$ of $f$ by the formula
\begin{align*}
\text{simp}\int_{X} f d\mu &\equiv \sum_{\alpha\in I}c_{\alpha}\mu(A_{\alpha})
\end{align*} 
\end{definition}


\item \begin{remark}
Note that the precise decomposition into atoms \emph{does not affect} the definition of the simple integral. 

\begin{proposition}\label{prop: simple_integral_refinement} (\textbf{Simple integral unaffected by refinements}).  \citep{tao2011introduction} \\
Let $(X, \srB, \mu)$ be a measure space, and let $(X, \srB', \mu')$ be a \textbf{refinement} of $(X, \srB, \mu)$, which means that $\srB'$ contains $\srB$ and $\mu': \srB' \rightarrow [0, +\infty]$ \textbf{agrees} with $\mu: \srB \rightarrow [0, +\infty]$ on $\srB$. Suppose that both $\srB, \srB'$ are \textbf{finite}, and let $f: \srB \rightarrow [0, +\infty]$ be measurable. We have
\begin{align*}
\text{simp }\int_{X} f d\mu &= \text{simp }\int_{X} f d\mu'.
\end{align*}
\end{proposition}
\begin{proof}
Since taking simple integrals both w.r.t. $\srB$  and $\srB'$  implies that $f$ is both $\srB$-measurable and $\srB'$-measurable, we see that for the finite values $a_1 \xdotx{,} a_k$  of $f$  we have $f^{-1}(a_i) \in \srB$.
\begin{align*}
\text{simp }\int_{X} f d\mu' = \sum_{i=1}^{k}a_i \mu'(f^{-1}(a_i)) = \sum_{i=1}^{k}a_i \mu'|_{\srB}(f^{-1}(a_i)) = \sum_{i=1}^{k}a_i \mu(f^{-1}(a_i)) = \text{simp }\int_{X} f d\mu. \qed
\end{align*}
\end{proof}
\end{remark}


\item The above proposition allows one to extend the \emph{simple integral} to \emph{simple functions}:
\begin{definition} (\emph{\textbf{Integral of simple functions}}).\\ 
An \underline{\emph{\textbf{(unsigned) simple function}}} $f : X \rightarrow [0,+\infty]$ on a measurable space $(X, \srB)$ is a \emph{measurable function} that takes on \emph{\textbf{finitely many values}} $a_1, \cdots , a_k$. Note that such a function is then automatically \textit{measurable} with respect to \emph{at least one \textbf{finite sub-$\sigma$-algebra}} $\srB'$ of $\srB$, namely \emph{the $\sigma$-algebra $\srB'$ \textbf{generated by the preimages}} $f^{-1}\set{a_1}, \cdots,  f^{-1}\set{a_k}$ of $a_1, \cdots , a_k$.

We then define the \underline{\emph{\textbf{simple integral}}} $\text{simp}\int_{X} f d\mu $ by the formula
\begin{align*}
\text{simp}\int_{X} f d\mu &\equiv \text{simp}\int_{X} f \rlat{d\mu}{\srB'}\\
&= \sum_{i=1}^{k}a_{i}\mu\paren{f^{-1}\set{a_k}}
\end{align*}
where $\rlat{\mu}{\srB'} : \srB' \rightarrow [0,+\infty]$ is the \emph{\textbf{restriction}} of $\mu : \srB \rightarrow [0,+\infty]$ to $\srB'$.
\end{definition}



\item \begin{remark}
Note that there could be \emph{\textbf{multiple finite $\sigma$-algebras}} with respect to which $f$ is \emph{measurable}, but all such
extensions will give the same simple integral.  Indeed, if $f$ were measurable with respect to two separate finite sub-$\sigma$-algebras $\srB'$ and $\srB''$ of $\srB$, then it would also be \emph{measurable} with respect to their \emph{\textbf{common refinement}} $\srB' \lor \srB'' := (\srB' \cup \srB'')$, which is also \emph{finite} and then by Proposition \ref{prop: simple_integral_refinement}, $\int_X f d\mu|_{\srB'}$ and $\int_X f d\mu|_{\srB''}$ are both equal to $\int_X f d\mu|_{\srB' \lor \srB''}$, and hence equal to each other.
\end{remark}

\item \begin{remark}
As with the Lebesgue theory, we say that a property $P(x)$ of an element $x \in X$ of a measure space $(X, \srB, \mu)$ \underline{\emph{\textbf{holds $\mu$-almost everywhere}}} if it \emph{\textbf{holds}} \emph{\textbf{outside}} of a \emph{\textbf{sub-null set}}, i.e. $\mu(\set{P(x) \text{ \emph{does not hold}}}) = 0$.
\end{remark}

\item \begin{proposition}
Let  $(X, \srB, \mu)$ be a measure space, and let $f, g: X \rightarrow [0, +\infty]$  be simple unsigned functions.
\begin{enumerate}
\item (\textbf{Monotonicity}) If $f \le g$  then $\text{simp }\int_X f d\mu \le \text{simp }\int_X g d\mu$.
\item (\textbf{Compatibility with measure}) For every $\srB$-measurable set $E$, we have $\text{simp }\int_X \mathds{1}_{E} d\mu = \mu(E)$.
\item (\textbf{Homogeneity}) For every $c \in [0, +\infty]$,  one has $\text{simp }\int_X (cf) d\mu = c\times \text{simp }\int_X f d\mu  $.
\item (\textbf{Finite additivity}) We have $\text{simp }\int_X (f + g)d\mu = \text{simp }\int_X f d\mu + \text{simp }\int_X g d\mu$.
\item (\textbf{Insensitivity to refinement}) Let $(X, \srB, \mu)$ be a measure space, and let $(X, \srB', \mu')$ be its refinement, which means that $\srB'$ contains $\srB$ and $\mu': \srB' \rightarrow [0, +\infty]$ \textbf{agrees} with $\mu: \srB \rightarrow [0, +\infty]$ on $\srB$. Suppose that both $\srB, \srB'$ are \textbf{finite}, and let $f: \srB \rightarrow [0, +\infty]$ be measurable. We have
\begin{align*}
\text{simp }\int_{X} f d\mu &= \text{simp }\int_{X} f d\mu'.
\end{align*}
\item (\textbf{Almost everywhere equivalence}) If $\mu$-almost everywhere $f =g$, then  $\text{simp }\int_X f d\mu =  \text{simp }\int_X g d\mu$
\item (\textbf{Finiteness}) $\text{simp }\int_X f d\mu < \infty$  if and only if $f$  is \textbf{finite} \textbf{$\mu$-almost everywhere} and is \textbf{supported} on a set of \textbf{finite measure}.
\item (\textbf{Vanishing}) $\text{simp }\int_X f d\mu  = 0$  if and only if $f = 0$ $\mu$-almost everywhere.
\end{enumerate}
\end{proposition}

\item \begin{exercise}(\textbf{Inclusion-exclusion principle}).\\
 Let $(X, \srB, \mu)$ be a measure space, and let $A_1 \xdotx{,} A_n$ be $\srB$-measurable sets of \textbf{finite measure}. Show that
 \begin{align*}
\mu\paren{\bigcup_{i=1}^{n}A_i} &= \sum_{J \subseteq [1:n], J \neq \emptyset}(-1)^{\abs{J}-1}\mu\paren{\bigcap_{i\in J}A_i}
 \end{align*}
(Hint: Compute $\text{simp }\int_{X}(1 - \prod_{i=1}^{n}(1- \mathds{1}_{A_i})) d\mu$ in two different ways.)
\end{exercise}

\item \begin{remark}
The simple integral could also be defined on \emph{finitely additive measure spaces}, rather than \emph{countably additive ones}, and all the above properties would still apply. However, on a finitely additive measure space one would have difficulty extending the integral beyond simple functions.
\end{remark}
\end{itemize}

\subsection{Unsigned Integral}
\begin{itemize}
\item \begin{definition}
Let $(X,\srB, \mu)$ be a measure space, and let $f : X\rightarrow  [0,+\infty]$ be \emph{(unsigned) measurable}. Then we define the \underline{\emph{\textbf{unsigned integral}}} $\int_X f d\mu$ of $f$ by the formula
\begin{align*}
\int_{X}f d\mu &\equiv \sup\limits_{0\le g\le f, \atop g \text{ simple}} \text{simp}\int_{X}g d\mu
\end{align*}
\end{definition}

\item \begin{proposition} (\textbf{Properties of the unsigned integral}). \\
Let $(X,\srB, \mu)$ be a measure space, and let $f, g : X \rightarrow [0, +\infty]$ be measurable.
\begin{enumerate}
\item (\textbf{Almost everywhere equivalence}) If $f = g$ $\mu$-almost everywhere, then $\int_X f d\mu = \int_X g d\mu$
\item (\textbf{Monotonicity}) If $f \le g$ $\mu$-almost everywhere, then $\int_X f d\mu \le \int_X g d\mu$.
\item (\textbf{Homogeneity}) We have $\int_X (cf) d\mu = c\, \int_X f d\mu$ for every $c \in [0, +\infty]$.
\item (\textbf{Superadditivity}) We have $\int_X (f+g) d\mu \ge \int_X f d\mu + \int_X g d\mu$.
\item (\textbf{Compatibility with the simple integral}) If $f$ is \textbf{simple}, then $\int_X f d\mu = \text{simp }\int_X f d\mu$.
\item (\textbf{Markov's inequality}) For any $0 < \lambda < 1$, one has 
\begin{align*}
\mu\paren{\set{x \in X: f(x) \ge \lambda}} \le \frac{1}{\lambda}\int_{X} f d\mu
\end{align*}
In particular, if $\int_X f d\mu < \infty$, then the sets $\{x \in X : f(x) \ge \lambda \}$ have finite measure for each $\lambda > 0$.
\item (\textbf{Finiteness}) If $\int_X f d\mu < \infty$, then $f(x)$ is \textbf{finite} for \textbf{$\mu$-almost every $x$}.
\item (\textbf{Vanishing}) If $\int_X f d\mu = 0$, then $f(x)$ is zero for $\mu$-almost every $x$.
\item (\textbf{Vertical truncation}) We have 
\begin{align*}
\lim\limits_{n \rightarrow \infty} \int_{X} \min\set{f, n} d\mu = \int_{X} f d \mu
\end{align*}
\item  (\textbf{Horizontal truncation}) If $E_1 \subseteq E_2 \subseteq \ldots$ is an \textbf{increasing sequence} of $\srB$-measurable sets, then
\begin{align*}
\lim\limits_{n\rightarrow \infty} \int_X f \mathds{1}_{E_n} d\mu = \int_X f  \mathds{1}_{\cup_{n=1}^{\infty}E_n} d\mu.
\end{align*}
\item (\textbf{Restriction}) If $Y$ is a measurable subset of $X$, then 
\begin{align*}
\int_X f \mathds{1}_{Y} d\mu = \int_Y f|_{Y}  d\mu|_{Y},
\end{align*}
where $f|_{Y}: Y \rightarrow [0, +\infty]$ is the \textbf{restriction} of $f : X \rightarrow [0, +\infty]$ to $Y$, and $\mu|_{Y}$ is the restriction $\mu$ on $Y$. We will often abbreviate
$ \int_Y f|_{Y}  d\mu|_{Y}$ (by slight abuse of notation) as $\int_Y f d\mu$.
\end{enumerate}
\end{proposition}

\item \begin{theorem}
Let $(X,\srB, \mu)$ be a measure space, and let $f, g : X \rightarrow [0, +\infty]$ be measurable. Then
\begin{align*}
\int_X (f+g) d\mu &= \int_X f d\mu + \int_X g d\mu.
\end{align*}
\end{theorem}

\item \begin{proposition} (\textbf{Linearity in $\mu$}).\\
Let $(X,\srB, \mu)$ be a measure space, and let $f : X \rightarrow [0, +\infty]$ be measurable.
\begin{enumerate}
\item $\int_X f d(c\mu) = c \times \int_X f d\mu$ for every $c \in [0, +\infty]$.
\item If $\mu_1, \mu_2, \ldots$ are a sequence of measures on $\srB$, 
\begin{align*}
\int_X f d\paren{\sum_{n=1}^{\infty}\mu_n} =\sum_{n=1}^{\infty}\int_X f d\mu_n.
\end{align*}
\end{enumerate}
\end{proposition}

\item 
\begin{proposition}  (\textbf{Pushforward Measure}). \\
Let $(X,\srB, \mu)$ be a measure space, and let $\varphi: X \rightarrow Y$ be $(\srB, \srC)$ measureable from $(X, \srB)$ to another measurable space
$(Y, \srC)$. Define the \underline{\textbf{pushforward}} $\phi_{*}\mu: \srC \rightarrow [0, +\infty]$ of $\mu$ \textbf{by} $\varphi$ by the formula
\begin{align*}
\varphi_{*}\mu(E) := \mu(\phi^{-1}(E)).
\end{align*}
\begin{enumerate}
\item $\varphi_{*}\mu$ is a \textbf{measure} on $\srC$, so that $(Y, \srC, \phi_{*}\mu)$ is a measure space.
\item  (\textbf{Change of variables formula}). If $f : Y \rightarrow [0, +\infty]$ is $\srC$-measurable, then 
\begin{align*}
\int_Y f d(\phi_{*}\mu) = \int_X (f \circ \phi) d\mu.
\end{align*}
\end{enumerate}
\end{proposition}

\item \begin{corollary} 
Let $T : \bR^d \rightarrow \bR^d$ be an invertible linear transformation, and let $m$ be Lebesgue measure on $\bR^d$. Then $T_{*}m =\frac{1}{\abs{\det{T}}}m$, where $T_{*}m$ is \textbf{the pushforward of $m$}.
\end{corollary}

\item \begin{example} (\emph{\textbf{Sums as integrals}}).
Let $X$ be an arbitrary set (with the \emph{\textbf{discrete $\sigma$-algebra}}), let $\#$ be \emph{\textbf{counting measure}}, and let $f: X \rightarrow [0, +\infty]$ be an arbitrary unsigned function. Then $f$ is \textbf{\emph{measurable}} with
\begin{align*}
\int_X f d\# = \sum_{x \in X} f(x).
\end{align*}
\end{example}
\end{itemize}

\subsection{Absolutely Convergent Integral}
\begin{itemize}
\item \begin{definition} (\emph{\textbf{Absolutely convergent integral}}). \\
Let $(X,\srF, \mu)$ be a measure space. A \emph{measurable function} $f : X \rightarrow \bC$ is said to be \underline{\emph{\textbf{absolutely integrable}}} if the \emph{unsigned integral} 
\begin{align*}
\norm{f}{L^{1}(X)} &\equiv \int_{X}\abs{f}d\mu 
\end{align*}
is \emph{\textbf{finite}}. We refer to this quantity $\norm{f}{L^{1}(X)}$ as \underline{\emph{\textbf{the $L^1(X)$ norm of $f$}}}, and use $L^1(X)$ or $L^1(X, \srF, \mu)$ or $L^1(\mu)$ to denote the space of absolutely integrable functions. If $f$ is \emph{real-valued} and absolutely integrable, we define \underline{\emph{\textbf{the Lebesgue integral}}} $\int_{X}f d\mu$ by the formula
\begin{align*}
\int_{X}f d\mu &= \int_{X}f_{+}d\mu - \int_{X}f_{-}d\mu
\end{align*}
where $f_{+} = \max\set{f, 0}$ and $f_{-} = \max\set{-f, 0}$ are the magnitudes of the positive and negative components of $f$. (note that the two unsigned integrals on the right-hand side are finite, as $f_+, f_{-}$ are pointwise dominated by $\abs{f}$). If f is \emph{complex-valued} and absolutely integrable, we define \emph{\textbf{the Lebesgue integral}} $\int_{X}f(x)d\mu$ by the formula
\begin{align*}
\int_{X}f d\mu &= \int_{X}\Re( f ) d\mu  + i\,\int_{X}\Im(f) d\mu,
\end{align*}
where the two integrals on the right are interpreted as real-valued absolutely integrable Lebesgue integrals.
\end{definition}

\item \begin{remark}
 Sometimes $\int_{X}f d\mu $ is also denoted as $\int_{X}f(x) \mu(dx) $ or $\int_{X}f(x) d\mu(x)$, where $X\subseteq \bR^{d}$ and $\mu(E) = \int_{E}\mu dx$.
\end{remark}

\item \begin{proposition} (\textbf{Properties of absolutly convergent integral})\\
Let $(X,\srB, \mu)$ be a measure space.
\begin{enumerate}
\item $L^1(X, \srB, \mu)$ is a \textbf{complex vector space}.
\item The integration map $f \mapsto \int_X f d\mu$ is a \textbf{complex linear map} from $L^1(X, \srB, \mu)$ to $\bC$.
\item The \textbf{triangle inequality}
\begin{align*}
\norm{f + g}{L^{1}(\mu)} & \le \norm{f}{L^{1}(\mu)} + \norm{g}{L^{1}(\mu)}
\end{align*}
and the \textbf{homogeneity property} 
\begin{align*}
\norm{c\,f}{L^{1}(\mu)}= \abs{c} \norm{f}{L^{1}(\mu)} 
\end{align*} hold for all $f, g \in L^1(X, \srB, \mu)$ and $c \in \bC$.
\item If $f, g \in L^1(X, \srB, \mu)$ are such that $f(x) = g(x)$ for $\mu$-almost every $x \in X$, then $\int_X f d\mu = \int_X g d\mu$.
\item If $f \in L^1(X, \srB, \mu)$, and $(X, \srB', \mu')$ is a \textbf{refinement} of $(X,\srB, \mu)$, then $f \in L^1(X, \srB', \mu')$, and 
\begin{align*}
\int_X f d\mu' = \int_X f d\mu.
\end{align*}
(Hint: it is easy to get one inequality. To get the other inequality, first work in the case when f is both bounded and has finite measure support (i.e. is both vertically and horizontally truncated).)
\item If $f \in L^1(X, \srB, \mu)$, then $\norm{f}{L^{1}(\mu)}  = 0$  if and only if $f$ is zero $\mu$-almost everywhere.
\item If $Y \subseteq X$ is $\srB$-measurable and $f \in L^1(X, \srB, \mu)$, then $f|_{Y} \in L^1(Y, \srB|_{Y}, \mu|_{Y})$ and 
\begin{align*}
\int_Y f|_{Y} \, d\mu|_{Y} = \int_X f \mathds{1}_Y \,d\mu.
\end{align*}
As before, by abuse of notation we write $\int_Y f d\mu$ for $\int_Y f|_{Y}\, d\mu|_{Y}$.
\end{enumerate}
\end{proposition}
\end{itemize}


\subsection{The Convergence Theorems}
\begin{itemize}
\item \begin{proposition} (\textbf{Uniform Convergence on a Finite Measure Space}). \citep{tao2011introduction}\\
Suppose that $(X, \srB, \mu)$ is a \textbf{finite measure space} (so $\mu(X) < \infty$), and $f_n: X \rightarrow [0, +\infty]$ (resp. $f_n: X \rightarrow \bC$) are a sequence of unsigned measurable functions (resp. absolutely integrable functions) that \textbf{converge uniformly} to a limit $f$. Then $\int_X f_n d\mu$ \textbf{converges} to $\int_X f d\mu$.
\end{proposition}
\begin{proof}
Since $f_n \rightarrow f$ uniformly, we have for all $\epsilon >0$, $\exists N$, for $n \ge  N$, $\sup_{x \in X}\abs{f_n(x) - f(x)} < \epsilon$, thus
\begin{align*}
\sup_{x\in X}\paren{\abs{\abs{f_n(x)} - \abs{f(x)}}} &\le \sup_{x \in X}\paren{\abs{f_n(x) - f(x)}} < \epsilon
\end{align*} So $\abs{f_n} \rightarrow \abs{f}$ and $f$ is absolutely integrable if $f_n$ is absolutely integrable
\begin{align*}
\int_{X} \abs{f} d\mu = \int_{X} \abs{f - f_n + f_n} d\mu \le \int_{X} \abs{f - f_n} d\mu + \int_{X} \abs{f_n} d\mu \le \epsilon \mu(X) +  \int_{X} \abs{f_n} < \infty
\end{align*} To prove convergence, see that we can choose $N$ so that $\sup_{x \in X}\abs{f_n(x) - f(x)} < \epsilon / \mu(X)$ since $\mu(X) < \infty$, then 
\begin{align*}
\abs{\int_X f_n d\mu - \int_X f d\mu} = \abs{\int_X (f_n - f) d\mu} &\le \int_X\abs{f_n - f} d\mu \le  \epsilon / \mu(X) \int_X d\mu = \epsilon. \qed
\end{align*}
\end{proof}


\item \begin{theorem}(\textbf{Monotone Convergence Theorem}). \citep{tao2011introduction} \\
Let $(X, \srB, \mu)$ be a measure space, and let $0 \le f_1 \le  f_2 \le \ldots $ be a \textbf{monotone non-decreasing} sequence of \textbf{unsigned} measurable functions on $X$. Then we have
\begin{align*}
\lim\limits_{n\rightarrow \infty}\int_{X}f_{n} d\mu &= \int_{X}\paren{\lim\limits_{n\rightarrow \infty} f_{n}} d\mu 
\end{align*}
\end{theorem}
\begin{proof}
Let $f \equiv \lim\limits_{n\rightarrow \infty} f_{n}= \sup_n f_n$, then $f : X \rightarrow  [0, +\infty]$ is measurable. Since the $f_n$ are non-decreasing to $f$, we see from monotonicity that $\int_{X}f_{n} d\mu $ are non-decreasing and bounded above $\int_{X}f d\mu$, which gives the bound
\begin{align*}
\lim\limits_{n\rightarrow \infty}\int_{X}f_{n} d\mu &\le  \int_{X}f d\mu.
\end{align*}

It remains to establish the reverse inequality
\begin{align*}
 \int_{X}f d\mu&\le \lim\limits_{n\rightarrow \infty}\int_{X}f_{n} d\mu.
\end{align*}
By definition, it is equivalent to show that 
\begin{align*}
\int_{X}g d\mu &\le \lim\limits_{n\rightarrow \infty}\int_{X}f_{n} d\mu   .
\end{align*}
for any $0\le g\le f$ simple function.  By horizontal truncation we may assume without loss of generality that $g$ is also finite everywhere. Thus by canonical representation
\begin{align*}
g&= \sum_{i=1}^{m}c_{i}\ind{E_{i}}
\end{align*} for some $m$, $c_{1},\ldots, c_{m}\in (0,\infty)$ and $E_{1},\ldots, E_{m}$ being $\srF$-measureable. The integral 
\begin{align*}
\int_{X}g d\mu &=  \sum_{i=1}^{m}c_{i}\mu\set{E_{i}}.
\end{align*}

Let $0 < \epsilon < 1$ be arbitrary. Then we have $f(x) = \sup_{n}\set{f_n(x)} > (1 - \epsilon)c_{i}$ for all $x \in  E_{i}$. Thus, if we define the sets
\begin{align*}
E_{i,n}&= \set{x\in E_{i}:  f_{n}(x)>  (1 - \epsilon)c_{i} }
\end{align*} then the $E_{i,n}$ increase to $E_i$ and are measurable. By upwards monotonicity of measure, we conclude that
\begin{align*}
\lim\limits_{n\rightarrow \infty}\mu(E_{i,n}) &= \mu(E_{i}), \; 1\le i\le m.
\end{align*}
On the other hand, observe the pointwise bound
\begin{align*}
f_{n}(x) &\ge   (1 - \epsilon)\sum_{i=1}^{m}c_{i}\ind{E_{i,n}}
\end{align*} hold for any $n$. Integrate both sides, 
\begin{align*}
\int_{X}f_{n} d\mu &\ge (1 - \epsilon)\sum_{i=1}^{m}c_{i}\mu\paren{E_{i,n}}.
\end{align*} and take the limits $n\rightarrow \infty$,
\begin{align*}
\lim\limits_{n\rightarrow \infty}\int_{X}f_{n} d\mu  &\ge  (1 - \epsilon)\sum_{i=1}^{m}c_{i}\lim\limits_{n\rightarrow \infty}\mu\paren{E_{i,n}}\\
&= (1 - \epsilon)\sum_{i=1}^{m}c_{i}\mu(E_{i}).
\end{align*} Finally, send $\epsilon \rightarrow 0$, we have the required formula. \qed
\end{proof}

\item \begin{remark}
Note that in the special case when each $f_n$ is an indicator function $f_n = \ind{E_n}$, this theorem collapses to the upwards monotone convergence property. Conversely, the upwards monotone convergence property will play a key role in the proof of this theorem.
\end{remark}

\item \begin{remark}
 Note that  the result still holds if the monotonicity $f_n \le f_{n+1}$ only holds almost everywhere rather than everywhere.
 \end{remark}
 
\item \begin{corollary}(\textbf{Tonelli's Theorem for Sums and Integrals})\\
Let $(X, \srB, \mu)$ be a measure space, and let $f_1, f_2, \ldots $ be a sequence of \textbf{unsigned} measurable functions on $X$. Then 
\begin{align*}
\sum_{n=1}^{\infty}\int_{X}f_{n} d\mu &= \int_{X}\paren{\sum_{n=1}^{\infty}f_{n}} d\mu 
\end{align*}
\end{corollary}

\item \begin{exercise}
Give an example to show that this corollary can fail if the $f_n$ are assumed to be absolutely integrable rather than unsigned measurable, even if the sum $\sum_{n=1}^{\infty} f_n(x)$ is absolutely convergent for each $x$. (Hint: think about the three escapes to infinity.)
\end{exercise}

\item \begin{lemma} (\textbf{Borel-Cantelli Lemma}). \citep{tao2011introduction, resnick2013probability} \\
Let $(X, \srB, \mu)$ be a measure space, and let $E_1, E_2,  \ldots$ be a sequence of $\srB$-measurable sets such that $\sum_{n=1}^{\infty}\mu(E_n) < \infty$. Then 
\begin{align*}
\mu\set{ \limsup\limits_{n\rightarrow \infty} E_{n} }  = 0.
\end{align*} That is,  almost every $x \in X$ is contained in \textbf{at most finitely many} of the $E_n$ (i.e. $\set{ n \in \bN : x \in E_n}$ is finite for almost every $x \in X$).
\end{lemma}  (Hint: Apply \emph{Tonelli's theorem} to the indicator functions $\mathds{1}_{E_n}$.)
\begin{proof}
Consider the indicator function $f_n = \ind{x \in E_n}$, which is unsigned $\srB$-measurable functions since $E_n$ is $\srB$-measurable. By Tonelli's theorem, 
\begin{align*}
\sum_{n=1}^{\infty}\mu(E_n)  &= \sum_{n=1}^{\infty}\int_{X}\ind{x \in E_n} d\mu \\
&= \int_{X}\paren{\sum_{n=1}^{\infty}\ind{x \in E_n}} d\mu \\
&= \int_{X}\paren{\ind{x \in \bigcup_{n=1}^{\infty}E_n}} d\mu \\
&= \mu\paren{\bigcup_{n=1}^{\infty}E_n}
\end{align*} Thus $\sum_{n=1}^{\infty}\mu(E_n) < \infty$ implies that $\mu\paren{\bigcup_{n=1}^{\infty}E_n} < \infty$. Then by \emph{the downwards monotone convergence} property
\begin{align*}
\mu\paren{ \limsup\limits_{n\rightarrow \infty} E_{n} } &= \mu\paren{\bigcap_{N=1}^{\infty}\bigcup_{n=N}^{\infty}E_n} \\
&= \lim\limits_{N\rightarrow \infty}\mu\paren{\bigcup_{n=N}^{\infty}E_n} \\
&= \lim\limits_{N\rightarrow \infty}\sum_{n=N}^{\infty}\mu(E_n) \\
&\le  \limsup\limits_{N\rightarrow \infty}\sum_{n=N}^{\infty}\mu(E_{n}) = 0,
\end{align*}
since $\sum_{n=1}^{\infty}\mu(E_{n})<\infty$ implies that $\sum_{n=N}^{\infty}\mu(E_{n}) \rightarrow 0$ as $N\rightarrow \infty$.\qed
\end{proof}



\item When one \emph{does not have monotonicity}, one can at least obtain an important inequality, known as \emph{Fatou's lemma}:
\begin{corollary} (\textbf{Fatou's Lemma}). \\
Let $(X, \srB, \mu)$ be a measure space, and let $f_1, f_2, \ldots : X\rightarrow [0,\infty]$ be a sequence of unsigned measurable functions. Then
\begin{align*}
\int_{X}\paren{\liminf\limits_{n\rightarrow \infty} f_{n}} d\mu &\le \liminf\limits_{n\rightarrow \infty}\int_{X}f_{n} d\mu 
\end{align*}
\end{corollary}
\begin{proof}
Write $F_N \equiv \inf_{n\ge N} f_n$ for each $N$. Then the $F_N$ are measurable and non-decreasing, and hence by the monotone convergence theorem
\begin{align*}
\int_{X}\paren{\lim\limits_{N\rightarrow \infty} F_{N}} d\mu &= \lim\limits_{N\rightarrow \infty}\int_{X}F_{N} d\mu \\
\Rightarrow \int_{X}\paren{\liminf\limits_{n\rightarrow \infty} f_{n}} d\mu  &= \lim\limits_{N\rightarrow \infty}\int_{X}\inf_{n\ge N} f_n d\mu\\
&\le \lim\limits_{N\rightarrow \infty} \inf_{n\ge N} \int_{X}f_n d\mu\\
&=   \liminf\limits_{n\rightarrow \infty}\int_{X}f_{n} d\mu 
\end{align*}
The second last inequality holds since
\begin{align*}
\int_{X}\inf_{n\ge N} f_n d\mu &\le \int_{X}f_n d\mu, \;\; \forall\, n\ge N\\
\Rightarrow \int_{X}\inf_{n\ge N} f_n d\mu &\le \inf_{n\ge N} \int_{X}f_n d\mu,  
\end{align*} which completes the proof. \qed
\end{proof}

\begin{remark}
Informally, \emph{Fatou's lemma} tells us that when taking \emph{\textbf{the pointwise limit}} of \textit{\textbf{unsigned functions}} $f_n$, that mass
$\int_{X}f_{n} d\mu$ can be \emph{\textbf{destroyed in the limit}} (as was the case in the three key moving bump examples), \emph{but} it \emph{\textbf{cannot be created in the limit}}. Of course the unsigned hypothesis is necessary here.

While this lemma was stated only for pointwise limits, the same general \emph{\textbf{principle}} (\emph{that mass can be destroyed, but not created, by the process of taking limits}) tends to hold for other ``\emph{weak}" notions of convergence.
\end{remark}

\item Finally, we give the other major way to shut down loss of mass via \emph{escape to infinity}, which is to \emph{dominate} all of the functions involved by
an \emph{absolutely convergent one}. This result is known as \emph{the dominated convergence theorem}:

\begin{theorem} (\textbf{Dominated Convergence Theorem}). \\
Let $(X, \srB, \mu)$ be a measure space, and let $f_1, f_2, \ldots : X\rightarrow \bC$ be a sequence of measurable functions that converge \textbf{pointwise $\mu$-almost everywhere} to a measurable limit $f : X\rightarrow \bC$. Suppose that there is an \textbf{unsigned absolutely integrable} function $G : X\rightarrow [0, +\infty]$ such that $\abs{f_n}$ are pointwise $\mu$-almost everywhere \textbf{bounded} by $G$ for each $n$. Then we have
\begin{align*}
\lim\limits_{n\rightarrow \infty}\int_{X}f_{n}d\mu  &= \int_{X}f d\mu.
\end{align*}
\end{theorem}
\begin{proof}
By modifying $f_n$, $f$ on a null set, we may assume without loss of generality that the $f_n$ converge to $f$ \emph{pointwise everywhere} rather than \emph{$\mu$-almost everywhere}, and similarly we can assume that $\abs{f_n}$ are \emph{bounded} by $G$ \emph{pointwise everywhere} rather than \emph{$\mu$-almost everywhere}.

By taking real and imaginary parts we may assume without loss of generality that $f_n,  f$ are \emph{real}, thus $-G \le  f_n \le G$ \emph{pointwise}. Of course, this implies that $-G \le  f \le G$ \emph{pointwise} also. 

If we apply Fatou's lemma  to the unsigned functions $f_n + G$, we see that
\begin{align*}
\int_{X}\paren{\liminf\limits_{n\rightarrow \infty}f_n + G}d\mu &\le \liminf\limits_{n\rightarrow \infty}\int_{X}\paren{f_{n}+ G}d\mu\\
\Rightarrow \int_{X}\paren{f + G}d\mu &\le \liminf\limits_{n\rightarrow \infty}\int_{X}\paren{f_{n}+ G}d\mu\\
\int_{X}fd\mu &\le \liminf\limits_{n\rightarrow \infty }\int_{X}f_{n}d\mu \quad (\text{since } \int_{X}G d\mu < \infty )
\end{align*}
Similarly, if we apply that lemma to the unsigned functions $G - f_n$, we obtain
\begin{align*}
-\int_{X}fd\mu &\le \liminf\limits_{n\rightarrow \infty }-\int_{X}f_{n}d\mu \quad (\text{since } \int_{X}G d\mu < \infty )\\
\Rightarrow \int_{X}fd\mu &\ge \limsup\limits_{n\rightarrow \infty }\int_{X}f_{n}d\mu 
\end{align*}
It concludes that $ \limsup\limits_{n\rightarrow \infty }\int_{X}f_{n}d\mu$ $=\liminf\limits_{n\rightarrow \infty }\int_{X}f_{n}d\mu$ $=\lim\limits_{n\rightarrow \infty }\int_{X}f_{n}d\mu=\int_{X}fd\mu$.\qed
\end{proof}

\item \begin{remark}
From the moving bump examples we see that this statement \emph{fails} if there is no \emph{absolutely integrable dominating function} $G$. 
\end{remark}

\item \begin{remark}
Note also that when each of the fn is an indicator function $f_n= \mathds{1}_{E_n}$, the dominated convergence theorem collapses to \emph{dominated convergence for sets} in previous chapter.
\end{remark}

\item \begin{proposition} (\textbf{Almost dominated convergence}). \\
Let $(X, \srB, \mu)$ be a measure space, and let $f_1, f_2, \ldots: X \rightarrow \bC$ be a sequence of measurable functions that converge pointwise $\mu$-almost everywhere to a measurable limit $f : X \rightarrow \bC$. Suppose that there is an unsigned absolutely integrable functions $G, g_1, g_2, \ldots X \rightarrow [0, +\infty]$ such that the $\abs{f_n}$ are pointwise $\mu$-almost everywhere bounded by $G + g_n$, and that $\int_X g_n d\mu \rightarrow 0$ as $n \rightarrow \infty$. Then
\begin{align*}
\lim\limits_{n\rightarrow \infty} \int_X f_n d\mu = \int_X f d\mu.
\end{align*}
\end{proposition}

\item \begin{exercise}  (\textbf{Defect version of Fatou's lemma}). \\
Let $(X, \srB, \mu)$ be a measure space, and let $f_1, f_2, \ldots: X \rightarrow [0, +\infty]$ be a sequence of \textbf{unsigned absolutely integrable functions} that converges \textbf{pointwise} to an absolutely integrable limit $f$. Show that
\begin{align*}
\int_X f_n d\mu - \int_X f d\mu - \norm{f - f_n}{L^1(\mu)} \rightarrow 0
\end{align*} as $n \rightarrow \infty$. (Hint: Apply the dominated convergence theorem to $\min(f_n, f)$.) Informally, this result tells us that the gap between the left and right hand sides of Fatou’s lemma can be measured by the quantity $\norm{f - f_n}{L^1(\mu)}$.
\end{exercise}

\item \begin{proposition} 
Let $(X, \srB, \mu)$ be a measure space, and let $g: X \rightarrow [0, +\infty]$ be measurable. Then the function $\mu_g : \srB \rightarrow [0, +\infty]$ defined by the formula
\begin{align*}
\mu_g(E) := \int_X g\,\mathds{1}_{E} d\mu = \int_E g d\mu
\end{align*} is a measure.
\end{proposition}

\item \emph{The monotone convergence theorem} is, in some sense, \emph{a \textbf{defining property} of the unsigned integral}:
\begin{proposition} (\textbf{Characterisation of the unsigned integral}). \\
Let $(X, \srB)$ be a measurable space. $I: f \mapsto I(f)$ be a map from the space $U(X, \srB)$ of \textbf{unsigned measurable functions} $f: X \rightarrow [0, +\infty]$ to $[0, +\infty]$ that obeys the following axioms:
\begin{enumerate}
\item (\textbf{Homogeneity}) For every $f \in U(X, \srB)$ and $c \in [0, +\infty]$, one has $I(c\,f) = c\,I(f)$.
\item (\textbf{Finite additivity}) For every $f, g  \in U(X, \srB)$, one has $I(f + g) = I(f) + I(g)$.
\item (\textbf{Monotone convergence}) If $0 \le f_1 \le f_2 \le \ldots$ are a \textbf{nondecreasing} sequence of unsigned measurable functions, then $I(\lim_{n\rightarrow \infty} f_n) = \lim_{n\rightarrow \infty} I(f_n)$.
\end{enumerate}
Then there exists a \textbf{unique measure} $\mu$ on $(X, \srB)$ such that 
\begin{align*}
I(f) = \int_X f d\mu, \quad \text{ for all }f \in U(X, \srB).
\end{align*}
Furthermore, $\mu$ is given by the formula $\mu(E) := I(\mathds{1}_E)$ for all $\srB$-measurable sets $E$.
\end{proposition}
\end{itemize}


\newpage
\bibliographystyle{plainnat}
\bibliography{reference.bib}
\end{document}