\documentclass[11pt]{article}
\usepackage[scaled=0.92]{helvet}
\usepackage{geometry}
\geometry{letterpaper,tmargin=1in,bmargin=1in,lmargin=1in,rmargin=1in}
\usepackage[parfill]{parskip} % Activate to begin paragraphs with an empty line rather than an indent %\usepackage{graphicx}
\usepackage{amsmath,amssymb, mathrsfs, dsfont}
\usepackage{tabularx}
\usepackage[font=footnotesize,labelfont=bf]{caption}
\usepackage{graphicx}
\usepackage{xcolor}
%\usepackage[linkbordercolor ={1 1 1} ]{hyperref}
%\usepackage[sf]{titlesec}
\usepackage{natbib}
\usepackage{../../Tianpei_Report}
%\usepackage{appendix}
%\usepackage{algorithm}
%\usepackage{algorithmic}

%\renewcommand{\algorithmicrequire}{\textbf{Input:}}
%\renewcommand{\algorithmicensure}{\textbf{Output:}}



\begin{document}
\title{Approximate nearest neighbor search}
\author{ Tianpei Xie}
\date{Sep. 3rd., 2022 }
\maketitle
\tableofcontents
\newpage
\allowdisplaybreaks
\section{Definitions}
\begin{itemize}
\item Given a dataset $\cD$ with $n$ points and $d$ dimensions and a query point $q$ in the same space as the dataset, the \emph{goal} of \emph{\textbf{$c$-ANN search}} (where $c > 1$ is an approximation ratio) is to return points $o \in \cD$ such that $\text{dist}(o, q) \le  c \times \text{dist}(o^{*}, q)$, where $o^{*}$ is the true nearest neighbor of $q$ in $\cD$ and $\text{dist}$ is the distance between the two points. Similarily, \textbf{\emph{$c$-$k$-ANN search}} aims at returning top-$k$ points such that $\text{dist}(o_i, q) \le  c \times \text{dist}(o_i^{*}, q)$, where $1 \le i \le k$. \citep{jafari2021survey}
\end{itemize}



\newpage
\bibliographystyle{plainnat}
\bibliography{book_reference.bib}
\end{document}