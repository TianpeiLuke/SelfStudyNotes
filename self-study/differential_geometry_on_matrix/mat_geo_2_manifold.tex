\documentclass[11pt]{article}
\usepackage[scaled=0.92]{helvet}
\usepackage{geometry}
\geometry{letterpaper,tmargin=1in,bmargin=1in,lmargin=1in,rmargin=1in}
\usepackage[parfill]{parskip} % Activate to begin paragraphs with an empty line rather than an indent %\usepackage{graphicx}
\usepackage{amsmath,amssymb, mathrsfs, dsfont}
\usepackage{tabularx}
\usepackage[font=footnotesize,labelfont=bf]{caption}
\usepackage{graphicx}
\usepackage{xcolor}
%\usepackage[linkbordercolor ={1 1 1} ]{hyperref}
%\usepackage[sf]{titlesec}
\usepackage{natbib}
\usepackage{../../Tianpei_Report}

%\usepackage{appendix}
%\usepackage{algorithm}
%\usepackage{algorithmic}

%\renewcommand{\algorithmicrequire}{\textbf{Input:}}
%\renewcommand{\algorithmicensure}{\textbf{Output:}}



\begin{document}
\title{Self-study: differential geometry for manifolds}
\author{ Tianpei Xie}
\date{ Jun. 1st., 2015 }
\maketitle
\tableofcontents
\newpage
\section{Definitions}
\subsection{General and differential manifold}
\begin{itemize}
\item A \emph{topological manifold}\citep{munkres2000topology} \\


\item A \emph{differential manifold} \citep{guillemin2010differential}\\


\item is referred as the \emph{parameterization} \citep{guillemin2010differential} \\

\item is referred as a \emph{coordinate system} \citep{guillemin2010differential}\\ 

\item The \emph{differential} of a map $f$, $df$
\end{itemize}

\subsection{Stiefel and Grassmann manifold}
\begin{itemize}
\item A \emph{Stiefel manifold} $V(r,n) \subset \bR^{n\times r}$ is space of all $n$-by-$r$ matrices whose rank is $r$ (full column rank). It is a submanifold embedded in a $rn-$dimensional Euclidean space of all $n$-by-$r$ matrices. $\mb{V}\in V(r,n)$ then $\mb{V} = \mb{T}_{n\times n}\,\brac{\begin{array}{c}
\mb{V}_{r}\\
\mb{0}
\end{array}}$.  The other representation of $V(r,n)$ arises from quotient group.  Consider two orthogonal matrices in $O(n)$ are equivalent if their first $r$-columns are identical, then $U_{1} = U_{1} \paren{\begin{array}{cc}
\mb{I}&0\\
0&\mb{Q}
\end{array}}$, where $\mb{Q}\in O(n-r)$. Then $V(r,n) = O(n)/O(n-r)$ as we can add arbitrary $n-r$ more orthogonal basis vectors and make them all orthogonal.\\

\item A \emph{Grassmann manifold} $G(r,n)$ is the space of all $r-$dimensional linear subspace in $\bR^{n}$ and  $G(r,n)$ is the quotient space of $V(r,n)$ by identifying the $p$-dim orthogonal group, i.e. $G(r,n) = V(r,n)/O(p)$. A point in $G(r,n)$ is an equivalence class with respect to the relationship $\mb{W} \sim_{R} \mb{V} \Rightarrow \mb{W} = \mb{V}\,\mb{U}$, where $\mb{U}\in O(p)$, and two matrices are equivalent if their column span is the same subspace.\\
\end{itemize}





\newpage
\section{Theorems and Properties for topological and differential manifold}

\newpage
\section{Theorems and Properties for Stiefel and Grassmann manifold}



\newpage
\bibliographystyle{plainnat}
\bibliography{book_reference.bib}
\end{document}