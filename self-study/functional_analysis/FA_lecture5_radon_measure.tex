\documentclass[11pt]{article}
\usepackage[scaled=0.92]{helvet}
\usepackage{geometry}
\geometry{letterpaper,tmargin=1in,bmargin=1in,lmargin=1in,rmargin=1in}
\usepackage[parfill]{parskip} % Activate to begin paragraphs with an empty line rather than an indent %\usepackage{graphicx}
\usepackage{amsmath,amssymb, mathrsfs,  mathtools, dsfont}
\usepackage{tabularx}
\usepackage[font=footnotesize,labelfont=bf]{caption}
\usepackage{graphicx}
\usepackage{xcolor}
\usepackage{tikz-cd}
%\usepackage[linkbordercolor ={1 1 1} ]{hyperref}
%\usepackage[sf]{titlesec}
\usepackage{natbib}
\usepackage{../../Tianpei_Report}

%\usepackage{appendix}
%\usepackage{algorithm}
%\usepackage{algorithmic}

%\renewcommand{\algorithmicrequire}{\textbf{Input:}}
%\renewcommand{\algorithmicensure}{\textbf{Output:}}



\begin{document}
\title{Lecture 5: Measure Theory on Compact Space}
\author{ Tianpei Xie}
\date{ Dec. 1st., 2022 }
\maketitle
\tableofcontents
\newpage
\section{Subspaces of Continuous Functions}
\begin{itemize}
\item \begin{remark} (\emph{\textbf{Useful Topologies on $Y^X$}})
\begin{enumerate}
\item \underline{\emph{\textbf{Uniform Topology}}}: generated by the \emph{\textbf{basis}}
\begin{align*}
B_{U}(f, \epsilon) &= \set{g \in Y^X: \sup_{x\in X}\bar{d}(f(x), g(x)) < \epsilon }
\end{align*} It corresponds to \emph{\textbf{the uniform convergence}} of $f_n$ to $f$ in $Y^X$. $\cC(X, Y)$ is \emph{\textbf{closed}} in $Y^X$ under the \emph{uniform topology}, following \emph{the Uniform Limit Theorem}.

\item  \underline{\emph{\textbf{Topology of Pointwise Convergence}}}: generated by the \emph{\textbf{basis}}
\begin{align*}
B_{U_1 \xdotx{,} U_n}(x_1 \xdotx{,} x_n, \epsilon)  &= \bigcap_{i=1}^{n}S(x_i, U_i) \\
&= \set{f \in Y^X: f(x_1) \in U_1 \xdotx{,} f(x_n) \in U_n}, \quad   1 \le n < \infty.
\end{align*} It corresponds to \emph{\textbf{the pointwise convergence}} of $f_n$ to $f$ in $Y^X$. $\cC(X, Y)$ is \emph{\textbf{not closed}} in $Y^X$ under the \emph{topology of pointwise convergence}. Note that \emph{the topology of poinwise convergence} is \emph{the \textbf{product topology} of $Y^X$}.

\item \underline{\emph{\textbf{Topology of Compact Convergence}}}: generated by the \emph{\textbf{basis}}
\begin{align*}
B_{C}(f, \epsilon) &= \set{g \in Y^X: \sup_{x \in C}d(f (x), g(x)) < \epsilon },\; C \text{ is compact set}.
\end{align*} It corresponds to \emph{\textbf{the uniform convergence}} of $f_n$ to $f$ in $Y^X$ for $x \in C$. $\cC(X, Y)$ is \emph{\textbf{closed}} in $Y^X$ under the \emph{topology of compact convergence} \emph{\textbf{if $X$ is compactly generated}}.

On $\cC(X)$, the topology of compact convergence is equal to the compact-open topology: 
 \begin{definition} (\emph{\textbf{Compact-Open Topology on Continuous Function Space}})\\
Let $X$ and $Y$ be topological spaces. If $C$ is a \emph{\textbf{compact subspace}} of $X$ and $U$ is an \emph{open} subset of $Y$, define
\begin{align*}
S(C,U) = \set{ f \in \cC(X, Y): f(C) \subseteq U}.
\end{align*}
The sets $S(C, U)$ form a \emph{\textbf{subbasis}} for a \emph{topology} on $\cC(X, Y)$ that is called \underline{\emph{\textbf{the compact-open}}} \underline{\emph{\textbf{topology}}}.
\end{definition}
\end{enumerate} 
We see that the \emph{uniform topology} is the \emph{finest} among them all and the \emph{topology of pointwise convergence} is the \emph{coarest}.
\begin{align*}
\text{\textbf{\emph{(uniform)}}} \supseteq \text{\textbf{\emph{(compact convergence)}}} \supseteq \text{\emph{\textbf{(pointwise convergence)}}}.
\end{align*}
\end{remark}


\item \begin{definition} (\emph{\textbf{Subspace of Continuous Functions}})\\
Let $\cC(X) := \cC(X, \bR)$ be \emph{the space of \textbf{continuous} real-valued functions} on topological space $X$ and $\cB(X): = \cB(X, \bR)$ be \emph{the space of \textbf{bounded} real-valued functions} on $X$.
\begin{enumerate}
\item The intersection of  $\cB(X)$ and $\cC(X)$ is the space of all \emph{\underline{\textbf{bounded continuous}} functions}
\begin{align*}
\cB\cC(X) := \mathcal{BC}(X, \bR) &= \cB(X, \bR) \cap \cC(X, \bR)
\end{align*} Note that $\mathcal{BC}(X) \subseteq \cB(X)$ is a \emph{\textbf{closed subspace}}. 

\item Define the \emph{\textbf{support}} of a function $f$, $\text{supp}(f)$ as  \emph{the \textbf{smallest closed set} outside of which  $f$ vanishes}. The subset $\cC_{c}(X) \subseteq \cC(X)$ is the space of all \emph{continuous functions} with \underline{\emph{\textbf{compact support}}}
\begin{align*}
\cC_{c}(X) &= \set{f \in \cC(X, \bR): \text{supp }(f) \text{\emph{ is compact}}}.
\end{align*} Note that by \emph{Tietze Extension Theorem}, \emph{the locally compact Hausdorff space} $X$ has a rich supply of continuous functions that vanishes outside a compact set.

\item Recall also that $\cC_{0}(X)$ is the space of \emph{continuous functions} on $X$ that \underline{\emph{\textbf{vanishes at infinity}}}, i.e. for all $\epsilon >0$, $\abs{f(x)} < \epsilon$ if $x \in X\setminus C$ for some \emph{\textbf{compact subset}} $C \subseteq X$.
\begin{align*}
\cC_{0}(X) &= \set{f \in \cC(X, \bR): f \text{\emph{ vanishes at infinity}}}.
\end{align*} 
\end{enumerate}
Note that 
\begin{align*}
\cC_{c}(X)  \subseteq \cC_{0}(X) \subseteq \mathcal{BC}(X) \subseteq \cC(X)
\end{align*}
\end{definition}

\item Recall that 
\begin{proposition}
If $X$ is a \textbf{locally compact Hausdorf} space, $\cC(X)$ is a \textbf{closed} subspace of $\bR^{X}$ in the \textbf{topology of compact convergence}.
\end{proposition}

\item \begin{proposition} \citep{folland2013real}\\
If $X$ is a topological space, $\mathcal{BC}(X)$ is a \textbf{closed} subspace of $\cB(X)$ in the \textbf{uniform metric}; in particular, $\mathcal{BC}(X)$ is \textbf{complete}.
\end{proposition}

\item \begin{proposition}\citep{folland2013real}\\
If $X$ is a \textbf{locally compact Hausdorf} space, $\cC_{0}(X)$ is a \textbf{closure} of $\cC_{c}(X)$ in the \textbf{uniform metric}.
\end{proposition}

\item \begin{remark}
Note that $\cC_0(X) = \overline{\cC_c(X)}$ is the \emph{\textbf{completion}} of $\cC_c(X)$ \emph{under uniform metrc}.
\end{remark}
\end{itemize}

\section{Measures on Locally Compact Hausdorff Space}
\subsection{Baire $\sigma$-algebra}
\begin{itemize}
\item \begin{definition}
\emph{\textbf{A $G_{\delta}$ set}} is a set which is a \emph{\textbf{countable intersection}} of \emph{open sets}. 
\end{definition}

\item \begin{proposition} \citep{reed1980methods}\\
Let I be a \textbf{compact Hausdorff} space and let $f \in \cC(X)$.  Then $f^{-1}([a, \infty))$ is a \textbf{compact $G_{\delta}$ set}. 
\end{proposition}

\item \begin{definition} (\textbf{\emph{Baire $\sigma$-algebra}})\\
\underline{\textbf{\emph{The Baire $\sigma$-algebra}}} is the \emph{$\sigma$-algebra} $\srC$ \emph{generated} by \emph{the \underline{\textbf{compact}  $G_{\delta}$} in a \underline{\textbf{compact space}} $X$}. Each \emph{measurable} set in Baire $\sigma$-algebra is called a \underline{\textbf{\emph{Baire set}}}
\end{definition}

\item \begin{definition} (\emph{\textbf{Baire $\sigma$-algebra on Locally Compact Hausdorff Space}})\\
In general, for \emph{a \textbf{locally compact Hausdorff}} $X$, the \underline{\textbf{\emph{Baire $\sigma$-algebra}}} is generated as
\begin{align*}
\sigma\paren{\set{f^{-1}(U):  f \in \cC_c(X), \; U \in \srB(\bR)}}
\end{align*} That is, \emph{the Baire sets} of a \emph{locally compact Hausdorff} space form \underline{\emph{\textbf{the smallest $\sigma$-algebra}}} such that all \emph{compactly supported continuous functions in} \underline{$\cC_c(X)$ are \emph{\textbf{measurable}}}. 
\end{definition}

\item \begin{remark}
Every \emph{Baire set} is \emph{regular Borel measurable} if $X$ is \emph{second-countable} \emph{locally compact Hausdorff}. Baire sets avoid some pathological properties of Borel sets on spaces \emph{\textbf{without a countable basis (second-countable)} for the topology}.
\end{remark}


\item \begin{definition} (\textbf{\emph{Baire Measure}})\\
Given \emph{a measurable space} $(X, \srC)$, where $X$ is a \emph{\textbf{compact space}}, and $\srC$ is the \emph{Baire $\sigma$-algebra} generated by all \emph{compact $G_{\delta}$ sets} in $X$, \underline{\textbf{\emph{the Baire measure}}} is a \emph{nonegative} function $\mu: \srC \rightarrow [0, +\infty)$ that obeys the following axioms:
\begin{enumerate}
\item (\emph{\textbf{Finiteness}}) $\mu(X) < \infty$.
\item (\emph{\textbf{Empty set}}) $\mu(\emptyset) = 0$.
\item (\emph{\textbf{Countable additivity}}) Whenever $E_1, E_2, \ldots \in \srB$ are a \emph{\textbf{countable sequence}} of \emph{\textbf{disjoint} measurable sets}, then 
\begin{align*}
\mu\paren{\bigcup_{n=1}^{\infty} E_n} &= \sum_{n=1}^{\infty} \mu(E_n).
\end{align*}
\end{enumerate}
\end{definition}

\item \begin{remark}
Baire measure is Borel measure. In practice, the use of \emph{Baire measures} on \emph{Baire sets} can often be replaced by the use of \emph{\textbf{regular Borel measures}} on \emph{\textbf{Borel sets}}. 
\end{remark}

\item \begin{definition} (\textbf{\emph{Baire Measurable Function}})\\
The functions $f: X \rightarrow \bR$ (or  $\bC$) \emph{measurable relative to the Baire $\sigma$-algebra} are called \emph{\textbf{Baire measurable functions}}. 
\end{definition}



\item \begin{theorem} (\textbf{Continuous Functions are Absolutely Integrable under Baire Measure})\citep{reed1980methods}\\
If $\mu$ is a \textbf{Baire measure}, then $\cC(X) \subseteq L^{p}(X, \mu)$ for all $1\le p < \infty$  and $\cC(X)$ is \textbf{dense} in $L^{1}(X, \mu)$ or any $L^{p}$ space.
\end{theorem}

\item \begin{remark}
$\cC(X) = L^{\infty}(X, \mu)$ under uniform metric by definition of $L^{\infty}$ norm.
\end{remark}
\end{itemize}

\subsection{Radon Measure}
\begin{itemize}
\item \begin{remark}
Despite the fact that Baire sets are all that are needed, the reader no doubt  wants to \emph{repress} $G_{\delta}$ and consider \emph{all \textbf{Borel} sets}, i.e. the $\sigma$-algebra $\srB$ generated by  all open sets. 
\end{remark}

\item \begin{definition} (\emph{\textbf{Outer Regularity}}) \citep{folland2013real} \\
Let $\mu$ be a \textbf{\emph{Borel}} measure on $X$ and $E$ a \emph{Borel subset} of $X$. The measure $\mu$ is called \underline{\textbf{\emph{outer regular}}} on $E$  if
\begin{align*}
\mu(E) &= \inf\set{\mu(U): U \supseteq E, U \text{ is open}}
\end{align*}
\end{definition}

\item \begin{definition} (\emph{\textbf{Inner Regularity}}) \citep{folland2013real} \\
Let $\mu$ be a \textbf{\emph{Borel}} measure on $X$ and $E$ a \emph{Borel subset} of $X$. The measure $\mu$ is called \underline{\textbf{\emph{inner regular}}} on $E$  if
\begin{align*}
\mu(E) &= \sup\set{\mu(C): C \subseteq E, C \text{ is compact}}
\end{align*}
\end{definition}

\item \begin{definition}
If $\mu$ is \emph{outer} and \emph{inner regular} on \emph{all Borel sets}, $\mu$ is called \underline{\textbf{\emph{regular}}}. 
\end{definition}

\item \begin{remark}
\emph{\textbf{Baire measure}} is equivalent to a \emph{\textbf{regular Borel measure (Randon measure)}} in the context of \emph{\textbf{compact space}} $X$.
\end{remark}

\item \begin{definition} (\emph{\textbf{Radon Measure}}) \citep{folland2013real} \\
A \underline{\textbf{\emph{Radon measure}}} $\mu$ on $X$ is a \emph{Borel measure} that is 
\begin{enumerate}
\item \emph{\textbf{finite}} on \emph{all \textbf{compact} sets}; i.e. for any \emph{\textbf{compact subset}} $K \subseteq X$, 
\begin{align*}
\mu(K) < \infty.
\end{align*}
\item \emph{\textbf{outer regular}} on \emph{all Borel sets}; i.e. for any \emph{\textbf{Borel set}} $E$ 
\begin{align*}
\mu(E) &= \inf\set{\mu(U): U \supseteq E , U \text{\emph{ is open}}}.
\end{align*}
\item  \emph{\textbf{inner regular}} on all \emph{open sets}; i.e. for any \emph{\textbf{open set}} $U$
\begin{align*}
\mu(U) &= \sup\set{\mu(K): K \subseteq U, K \text{\emph{ is compact and Borel}}}.
\end{align*}
\end{enumerate}
\end{definition}

\item \begin{remark}
\emph{\textbf{Baire measure}} is a \emph{\textbf{Radon measure}}. 

\emph{\textbf{Randon measure}} is called \underline{\emph{\textbf{regular Borel measure}}} in \citep{reed1980methods}.
\end{remark}
\end{itemize}

\subsection{Positive Linear Functionals on $\cC_c(X)$}
\begin{itemize}
\item \begin{definition} (\emph{\textbf{Positive Linear Functional}})\\
Let $\cC(X)$ be \emph{the space of \textbf{continuous} functions} on $X$. A \underline{\textbf{\emph{positive linear functional}}} on $\cC(X)$ is a (not necessarily a 
priori continuous) \emph{linear functiona} $I$ with $I(f) > 0$ for all $f$  with $f(x) \ge 0$ pointwise. 
\end{definition}

\item \begin{lemma} (\textbf{Bounded by Unit Ball in Uniform Metric}) \citep{folland2013real}\\
If $I$ is a positive linear functional on $\cC_{c}(X)$, for each compact $C \subseteq X$ there is a constant $\kappa_{C}$ such that $\abs{I(f)} < \kappa_C\norm{f}{u}$ for all $f \in \cC_{c}(X)$ such that $\text{supp}(f) \subset K$.
\end{lemma}


\item \begin{remark}
If $\mu$ is a \emph{Borel measure} on $X$ such that $\mu(C) < \infty$ for every compact subset $C \subseteq X$, then $\cC_c(X) \subseteq L^1(X, \mu)$. Therefore, $f \mapsto \int f d\mu$ is a \emph{\textbf{positive linear functional}} on $\cC_c(X)$.

The following theorem shows that the \underline{\emph{\textbf{every positive linear functionals}}} on $\cC_c(X)$ can be \emph{\textbf{represented}} as the \emph{integral} \emph{with respect to \underline{\textbf{some Radon measure}}} $\mu$.
\end{remark}

\item \begin{theorem} (\textbf{The Riesz-Markov Representation Theorem}). \citep{folland2013real}\\
Let $X$ be a \textbf{locally compact Hausdorff} space, if $I$ is a \underline{\textbf{positive linear functional}} on \underline{$\cC_{c}(X)$}, there is a \underline{\textbf{unique Radon measure}} $\mu$ on $X$ such that
\begin{align*}
I(f) &= \int f d\mu 
\end{align*}
for all $f \in \cC_{c}(X)$. Moreover, $\mu$ satisfies the following conditions:
\begin{enumerate}
\item for all \textbf{open} sets $U \subseteq X$,
\begin{align}
\mu(U) &= \sup\set{I(f): f \in \cC_c(X), \text{supp}(f) \subseteq U, \; 0 \le f \le 1}. \label{eqn: riesz_markov_outer_regular}
\end{align} 
\item for all \textbf{compact} sets $K \subseteq X$
\begin{align}
\mu(K) &= \inf\set{I(f): f \in \cC_c(X), f \ge \mathds{1}_{K}}. \label{eqn: riesz_markov_inn_regular}
\end{align}
\end{enumerate}
\end{theorem}
\begin{proof}
Let us begin by establishing \emph{\textbf{uniqueness}}. If $\mu$ is a \emph{\textbf{Radon measure}} such that $I(f) = \int f d\mu$ for all $f \in \cC_{c}(X)$, and $U \subset X$ is open, then clearly $I(f) \le  \mu(U)$ whenever $\text{supp}(f) \subseteq U$ and $0 \le f \le 1$ (denoted as $f \prec U$).  On the other hand, if $K \subset U$ is \emph{compact}, by \emph{\textbf{Urysohn's lemma}} there is an $f  \in C_{c}(X)$ such that $0 \le f \le 1$  and $\text{supp}(f) \subseteq U$ and $f = 1$ on $K$, whence 
\begin{align*}
\mu(K) \le \int f d\mu = I(f) \le \mu(U).
\end{align*}
Since $\mu$ is \emph{\textbf{inner regular}} on $U$, i.e. $\mu(U) = \sup\limits_{K\subset U, K\text{ compact}}\mu(K)$ it follows that \eqref{eqn: riesz_markov_outer_regular} is satisfied. Thus $\mu$ is \emph{\textbf{determined}} by $I$ according to  \eqref{eqn: riesz_markov_outer_regular} on \emph{\textbf{open sets}}, and hence on \emph{\textbf{all Borel sets}} because of \emph{\textbf{outer regularity}}.


This argument proves the uniqueness of $\mu$ and also suggests how to go about proving \emph{\textbf{existence}}. We begin by defining \emph{a set function} $\mu: 2^{X} \rightarrow \bR_{+}$ as
\begin{align*}
\mu(U) &= \sup\{I(f): f \in \cC_c(X),\; \text{supp}(f) \subseteq U, \; 0 \le f \le 1\}
\end{align*}
for $U$ \emph{\textbf{open}}, and we then define $\mu^{*}(E)$ for an arbitrary $E \subset X$ by
\begin{align*}
\mu^{*}(E) &= \inf\set{\mu(U): U \supset E, U\text{ open}}.
\end{align*} Clearly $\mu(U) \le \mu(V)$ if $U \subseteq V$, and hence $\mu^{*}(U) = \mu(U)$ if $U$ is \emph{open}. 
The outline of the proof is now as follows. 
\begin{enumerate}
\item First we shall establish that
\begin{enumerate}
\item $\mu^{*}$ is an \emph{\textbf{outer measure}}. (i.e. satisfying \emph{monotonicity}, \emph{countable subadditivity})
\item Every \emph{open set} is \emph{\textbf{$\mu^{*}$-measurable}}.
\end{enumerate}
At this point it follows from \emph{\textbf{Carath\'eodory's theorem}} that every \emph{Borel set} is \emph{$\mu^{*}$-measurable} and that $\mu = \mu^{*}|_{\cB(X)}$ is a \emph{\textbf{Borel measure}}. (The notation is \emph{consistent} because $\mu^{*}(U) = \mu(U)$ for $U$ open.) The measure $\mu$ is \emph{\textbf{outer regular}} and satisfies \eqref{eqn: riesz_markov_outer_regular} by definition. 

\item We next \emph{show} that \emph{\textbf{$\mu$ satisfies \eqref{eqn: riesz_markov_inn_regular}}}. This clearly implies that $\mu$ is \emph{\textbf{finite}} on \emph{\textbf{compact sets}}, and \emph{\textbf{inner regularity}} on \emph{\textbf{open sets}} also follows easily. 
Indeed, if $U$ is \emph{open} and for any $\alpha$ such that $\alpha < \mu(U)$, there exists an $f \in \cC_c(X)$ such that $\text{supp}(f) \subseteq U, \; 0 \le f \le 1$ and $I(f) > \alpha$, and let $K = \text{supp}(f)$. If $g \in \cC_c(X)$  and $g \ge \mathds{1}_{K}$, then
$g - f \ge 0$ and hence $I(g) \ge I(f) > \alpha$. But then $\mu(K) > \alpha$ by \eqref{eqn: riesz_markov_inn_regular}, so $\mu(U) = \sup \mu(K)$, i.e. $\mu$ is \emph{\textbf{inner regular}} on $U$. 

\item Finally, we prove that 
\begin{align*}
I(f) &= \int f d\mu
\end{align*} for all $f \in \cC_c(X)$. With this, the proof of the theorem will be complete. 
\end{enumerate}

We start the proof. 
\begin{enumerate}
\item We  \emph{construct} a \emph{Borel measure} $\mu$ and prove \emph{its outer regularity} first. 
\begin{enumerate}
\item $\mu(\emptyset) = 0$ and $\mu$ is monotone as shown above. By definition, for any $E\subset X$,
\begin{align*}
\mu^{*}(E) &:= \inf\set{\sum_{j=1}^{\infty}\mu(U_j): U_j \text{ is open}, \text{ and } E \subseteq \bigcup_{j=1}^{\infty}U_j}. 
\end{align*} The RHS is \emph{an outer measure} by proposition (See \citep{folland2013real}). So it suffice to show that 
\begin{align*}
\mu(\bigcup_{j=1}^{\infty}U_j) &\le \sum_{j=1}^{\infty}\mu(U_j)
\end{align*} for $U_j$ open sets. Let $U := \bigcup_{j=1}^{\infty}U_j$ be an open subset, $f \in \cC_{c}(X)$, $0 \le f \le 1$ and $\text{supp}(f) \subset U$. Denote the \emph{compact set} $K= \text{supp}(f)$. 

Given that $X$ is a \emph{locally compact Hausdorff space} and $K$ is its \emph{compact subset} with \emph{open cover} $\bigcup_{j=1}^{\infty}U_j$, there is a finite sub-cover $K \subset \bigcup_{j=1}^{n}U_j$ for some $n$. Then there exists \emph{a \textbf{partition of unity}} on $K$ subordinate to $\set{U_j}_{j=1}^{n}$ consisting of \emph{compactly supported functions} $g_1 \xdotx{,} g_n \in  \cC_{c}(X)$ with $\text{supp}(g_j) \subseteq U_j$, $0 \le g_j \le 1$, and $\sum_{j=1}^{n}g_j(x) = 1$ for $x \in K$. Thus $f(x) = \sum_{j=1}^{n}f(x)g_j(x)$ for $x \in K$ and $\text{supp}(f g_j) \subseteq U_j$, $0 \le f g_j \le 1$. The linear functional 
\begin{align*}
I(f) &= I\paren{\sum_{j=1}^{n}fg_j} = \sum_{j=1}^{n}I(fg_j) \le  \sum_{j=1}^{n}\mu(U_j) \le \sum_{j=1}^{\infty}\mu(U_j).
\end{align*} Since this is true for all $f \in \cC_{c}(X)$, $0 \le f \le 1$ and $\text{supp}(f) \subset U$, thus $\mu(U) = \sup\{I(f): f \prec U\} \le \sum_{j=1}^{\infty}\mu(U_j)$.

\item To show that every \emph{open set} is $\mu^{*}$-measurable, let $U \subset X$ be any open set and $E \subset X$ be a subset so that $\mu^{*}(E) < \infty$, then we need to show that 
\begin{align*}
\mu^{*}(E) &= \mu^{*}(E \cap U) + \mu^{*}(E \setminus U).
\end{align*} It suffice to show that $\mu^{*}(E)  \ge \mu^{*}(E \cap U) + \mu^{*}(E \setminus U)$ since the other side holds by \emph{subadditivity}.

First suppose that $E$ is \emph{open}.
Then $E \cap U$ is \emph{open}, so given $\epsilon > 0$ we can find $f\in  \cC_{c}(X)$ such that $f \prec E \cap U$ and
\begin{align*}
I(f) > \mu(E \cap U) - \epsilon.
\end{align*}
Also, $E \setminus (\text{supp}(f))$ is \emph{open}, so we can find $g \in \cC_{c}(X)$
such that $g \prec E \setminus (\text{supp}(f))$ and
\begin{align*}
I(g) > \mu(E \setminus (\text{supp}(f))) - \epsilon.
\end{align*} But then $f + g \prec E$, so
\begin{align*}
\mu(E) \ge I(f) + I(g) &> \mu(E \cap U) - \mu(E \setminus (\text{supp}(f))) - 2\epsilon \\
&\ge  \mu(E \cap U) - \mu(E \setminus U)   - 2\epsilon
\end{align*}
Letting $\epsilon \rightarrow 0$, we obtain the desired inequality. 

For \emph{the general case}, if $\mu^{*}(E) < \infty$,
we can find an \emph{open} $V \supset E$ such that $\mu(V) < \mu^{*}(E) + \epsilon$, and hence
\begin{align*}
 \mu^{*}(E) + \epsilon > \mu(V)  &\ge \mu(V \cap U) - \mu(V \setminus U) \\
&\ge  \mu(E \cap U) - \mu(E \setminus U) .
\end{align*}
Letting  $\epsilon \rightarrow 0$, we are done.
\end{enumerate}

\item If $K$ is \emph{compact}, $f\in  \cC_{c}(X)$, and $f \ge \mathds{1}_{K}$, let 
\begin{align*}
U_{\epsilon} := \set{x: f(x) > 1 - \epsilon}.
\end{align*} Then $U_{\epsilon}$ is \emph{open}, and if $g \prec U_{\epsilon}$, we have $(1 - \epsilon)^{-1}f - g \ge 0$ and so $I(g) \le (1-\epsilon)^{-1} I(f)$. Thus
\begin{align*}
\mu(K) \le \mu(U_{\epsilon}) &\le  (1-\epsilon)^{-1} I(f),
\end{align*} and letting $\epsilon \rightarrow 0$ we see that $\mu(K) \le I(f)$. On the other hand, for any \emph{open} $U \supset K$, by \emph{\textbf{Urysohn’s
lemma}}, there exists $f\in  \cC_{c}(X)$ such that $f \ge \mathds{1}_{K}$ and $f \prec U$, whence
\begin{align*}
I(f) \le \mu(U).
\end{align*}
Since $\mu$ is \emph{outer regular} on $K$, \eqref{eqn: riesz_markov_inn_regular} follows.

\item It suffices to show that $I(f) = \int fd\mu$ if $f \in \cC(X, [0,1])$, as $\cC(X)$ is the \emph{linear span} of the latter set.  Given $N \in \bN$, for $1\le j \le N$ let
\begin{align*}
K_j := \set{x: f(x) \ge \frac{j}{N} }
\end{align*} and let $K_0 = \text{supp}(f)$.  Note that $K_{j-1} \supseteq K_{j}$. Also, define $f_1 \xdotx{,} f_N \in \cC_{c}(X)$ by 
\begin{align*}
f_{j}(x)& = 0 &&\text{ if }x\not\in K_{j-1}, \quad \text{i.e. } f_j \in \left[0, \frac{j-1}{N}\right); \\
f_{j}(x)&=f(x) -  \frac{j-1}{N}  &&\text{ if }x\in K_{j-1} \setminus K_j, \quad \text{i.e. } f_j \in \left[\frac{j-1}{N}, \frac{j}{N}\right); \\
f_{j}(x)&=\frac{1}{N}  &&\text{ if }x\in K_{j}, \quad\text{i.e. }  f_j \in \left[\frac{j}{N}, 1\right].
\end{align*} In other words,
\begin{align*}
&f_{j}(x) = \min\set{\max\set{f(x) -  \frac{j-1}{N}, 0}, \frac{1}{N}}. \\
\Rightarrow &\frac{1}{N}\mathds{1}_{K_{j-1}} \ge f_j \ge  \frac{1}{N}\mathds{1}_{K_j} \\
\Rightarrow &\frac{1}{N}\mu\paren{K_{j-1}} \ge \int  f_j d\mu \ge \frac{1}{N}\mu\paren{K_{j}}.
\end{align*} Also, if $U$ is an \emph{open} set containing $K_{j-1}$ we have $Nf_j \prec U$ and so $I(f_j) \le
N^{-1}\mu(U)$. Hence, by \eqref{eqn: riesz_markov_inn_regular} and \emph{outer regularity},
\begin{align*}
\frac{1}{N}\mu\paren{K_{j-1}} \ge I(f_j) \ge \frac{1}{N}\mu\paren{K_{j}}
\end{align*} Moreover, $f = \sum_{j=1}^{N} f_j$, so that
\begin{align*}
\frac{1}{N}\sum_{j=0}^{N-1}\mu\paren{K_{j}} \ge \int  f d\mu \ge \frac{1}{N}\sum_{j=1}^{N}\mu\paren{K_{j}},\\
\text{and }\frac{1}{N}\sum_{j=0}^{N-1}\mu\paren{K_{j}} \ge I(f) \ge \frac{1}{N}\sum_{j=1}^{N}\mu\paren{K_{j}}.
\end{align*} It follows that 
\begin{align*}
\abs{I(f) - \int  f d\mu } &\le \frac{\mu(K_0) - \mu(K_N)}{N} \le \frac{\mu\paren{\text{supp}(f)}}{N}.
\end{align*} As $N \rightarrow \infty$, since $\mu\paren{\text{supp}(f)} < \infty$, $I(f) = \int  f d\mu$. \qed
\end{enumerate}
\end{proof}

\item \begin{remark}
Following \emph{the Riesz-Markov Theorem}
\begin{align*}
\mu(X) &= \sup\set{\int_X f d\mu: f \in \cC_c(X),  \; 0 \le f \le 1}.
\end{align*}
\end{remark}

\item The following theorem is another version of \emph{the Riesz representation theorem}:
 \begin{theorem} (\textbf{The Riesz-Markov Theorem}) \citep{reed1980methods}\\
Let $X$ be a \underline{\textbf{compact Hausdorff}} space. For any \textbf{positive linear functional} $I$ on \underline{$\cC(X)$}, there is a \textbf{unique Baire measure} $\mu$ on $X$ with 
\begin{align*}
I(f) &= \int f d\mu 
\end{align*} 
\end{theorem}

\item \begin{remark} (\emph{\textbf{Radon Measures $\Leftrightarrow$ Positive Linear Functionals on $\cC_c(X)$}})\\
\emph{The Riesz-Markov theorem} relates \emph{\textbf{linear functionals}} on spaces of \emph{\textbf{continuous} functions on a \textbf{locally compact} space} to \emph{\textbf{measures}} in \textbf{\emph{measure theory}}. 
\end{remark}

\item \begin{remark}
\emph{\textbf{Not to be confused}} with \emph{another Riesz representation theorem}, which related \emph{linear functions on Hilbert space} as inner product with some element in Hilbert space
\begin{align*}
I(f) &= \inn{f}{g_{I}}
\end{align*} for some $g_{I} \in \cH$.
\end{remark}

\item \begin{remark} (\emph{\textbf{Duality between $\cC_{0}(X)$ and $\cM(X)$}})\\
\emph{The Riesz representation theorem} establishes the \emph{\textbf{foundation}} of the \underline{\emph{\textbf{the duality}}} between \emph{the space of compactly supported continuous functions} and \emph{the space of all Radon \textbf{measures}} on $X$. 

In particular, for \emph{locally compact Hausdorff} $X$, 
\begin{align*}
\set{\mu: \mu \text{\emph{ is a Radon measure on }}X} \simeq \set{I \in \cC_{0}(X)^{*}: I \text{ is positive}}
\end{align*}
\end{remark}
\end{itemize}

\subsection{Dual Space of $\cC_0(X)$}
\begin{itemize}
\item \begin{theorem} (\textbf{Monotone Convergence Theorem for Nets}) \citep{reed1980methods}\\
Let $\mu$ be a \textbf{regular Borel} measure on a \textbf{compact Hausdorff} space $X$. Let $\set{f_{\alpha}}_{\alpha \in J}$ be an \textbf{increasing net} of continuous functions.  Then
\begin{align*}
f_{\alpha} \rightarrow f  \in L^1(X, \mu), \quad a.e.
\end{align*}  \textbf{if and only if} $\sup_{\alpha}\norm{f_{\alpha}}{1} < \infty$ and in that case
\begin{align*}
\norm{f_\alpha - f}{1} \rightarrow 0.
\end{align*}
\end{theorem}

\item \begin{lemma} \citep{reed1980methods}\\
Let $f,g \in \cC(X)$ with $f, g \ge 0$. Suppose $h \in \cC(X)$ and $0 \le h \le f + g$. 
Then, we can write $h = h_1 + h_2$ with $0\le h_1 \le f$, $0 \le h_2 \le g$,  $h_1, h_2 \in \cC(X)$. 
\end{lemma}

\item \begin{theorem}  (\textbf{Decomposition of Real Linear Functional}) \citep{reed1980methods, folland2013real}\\
Let  $X$ be a \textbf{compact} space, $I \in (\cC(X))^{*}$ be any continuous linear functional on $\cC(X)$. Then $I$ can be written 
\begin{align*}
I &= I_{+} - I_{-}
\end{align*}
with $I_{+}$ and $I_{-}$ \textbf{positive linear functionals}. Moreover, 
\begin{align*}
I_{+} + I_{-}= \norm{I}{}
\end{align*}
and this \textbf{uniquely determines} $I_{+}$ and $I_{-}$ . 
\end{theorem}

\item \begin{definition} (\emph{\textbf{Complex Radon Measure}})\\
A \underline{\emph{\textbf{signed Radon measure}}} is a \emph{\textbf{signed Borel measure}} whose \emph{\textbf{positive}} and \emph{\textbf{negative variations}} are \emph{\textbf{Radon}}, and a \underline{\emph{\textbf{complex Radon measure}}} is a \emph{\textbf{complex Borel measure}} whose \emph{real and imaginary parts} are \textit{signed Radon measures}. 
% a \underline{\emph{\textbf{complex Radon Measure}}} $\nu$, if $\nu$ can be represented as \emph{a \textbf{finite linear complex combinations}} of \emph{\textbf{Radon measures}}, i.e. $\nu = \sum_{i=1}^{n}c_i\, \nu_i$ where $\nu_i$ are  \emph{Radon measures} and $\set{c_i}$ are \emph{complex linear coefficients}.
\end{definition}

\item \begin{remark}
In \citep{reed1980methods}, one defines \emph{\textbf{the complex Baire measure}} as a \emph{finite linear complex combination} of \emph{Baire measures}. 
\end{remark}

\item \begin{definition} (\emph{\textbf{Space of Complex Radon Measures}})\\
On \emph{locally compact Hausdorff space} $X$, We denote \emph{the space of complex Radon measures
on} $X$ by $\cM(X)$. For $\mu \in \cM(X)$ we define
\begin{align*}
\norm{\mu}{} &= \abs{\mu}(X),
\end{align*}
where $ \abs{\mu}$ is the \emph{\textbf{total variation}} of $\mu$.  
\end{definition}

\item \begin{proposition} (\textbf{$\cM(X)$ is Normed Linear Space}) \citep{folland2013real}\\
If $\mu$ is a \textbf{complex Borel measure}, then $\mu$ is \textbf{Radon} if and only if $\abs{\mu}$ is \textbf{Radon}.
Moreover, $\cM(X)$ is a vector space and $\mu \mapsto \norm{\mu}{}$ is a \textbf{norm} on it.
\end{proposition}



\item \begin{theorem} (\textbf{The Riesz-Markov Theorem, Locally Compact Version}) \citep{reed1980methods, folland2013real}\\
Let $X$ be a \textbf{locally compact Hausdorff} space. For any continuous linear functional $I$ on $\cC_{0}(X)$, (the space of \emph{continuous functions} on $X$ that vanishes at infinity), there is a \textbf{\underline{unique regular countably additive complex Borel measure}} $\mu$ on $X$ such that
\begin{align*}
I(f) &= \int_{X} f d\mu, \quad \text{ for all } f \in \cC_0(X).
\end{align*} The \underline{\textbf{norm}} of $I$ as a linear functional is \underline{\textbf{the total variation}} of $\mu$, that is
\begin{align*}
\norm{I}{} &= \abs{\mu}(X).
\end{align*}
Finally, $I$ is \textbf{positive} if and only if the measure $\mu$ is \textbf{non-negative}.
\end{theorem}

\item \begin{remark}
In other word, the map $\mu \mapsto I_{\mu}$, is an \emph{\textbf{isometric isomorphism}} from $\cM(X)$  to $(\cC_0(X))^{*}$, or 
\begin{align*}
\cM(X) \simeq (\cC_0(X))^{*}.
\end{align*}
\end{remark}

\item \begin{corollary}  \citep{reed1980methods, folland2013real}\\
Let $X$ be a \textbf{compact Hausdorff} space. Then the \underline{\textbf{dual space} $\cC(X)^{*}$} is \textbf{isometric isomorphism} to $\cM(X)$. 
\end{corollary}

\item \begin{definition}
Given $\cM(X) \simeq (\cC_{0}(X))^{*}$, we define subspaces of $\cM$:
\begin{align*}
\cM_{+}(X) &= \set{I \in  \cM(X): I \text{\emph{ is a positive linear functional}}},\\
\cM_{+,1}(X) &= \set{I \in  \cM(X): \norm{I}{} = 1}.
\end{align*} Thus $\cM_{+}(X)$ is \emph{identified} with \emph{\textbf{the space of all positive Randon measures on $X$}}.
\end{definition}

\item \begin{remark} (\emph{\textbf{Isometric Embedding of $L^1(\mu)$ into $M(X)$}})\\
Let $\mu$ be a fixed \emph{positive Radon measure} on $X$. If $f \in L^1(\mu)$, \emph{the complex measure}
\begin{align*}
d\nu_f = f d\mu 
\end{align*}
is easily seen to be \emph{\textbf{Radon}}, and $\norm{\nu}{} = \int \abs{f}d\mu = \norm{f}{1}$.
Thus $f \mapsto \nu_f$ is an \emph{\textbf{isometric embedding}} of $L^1(\mu)$ into $M(X)$ whose range consists precisely of those $\nu \in \cM(X)$ such that $\nu \ll \mu$. 
\end{remark}

\item \begin{remark} (\emph{\textbf{Two Perspectives of Measures}})\\
For \emph{regular Borel measure} $\mu$ or in general, \emph{Radon measures} on \emph{\textbf{locally compact}} space $X$, there are two perspectives:
\begin{enumerate}
\item \emph{\textbf{Nonegative set function on the $\sigma$-algebra $\srA$}}: as a \emph{\textbf{measure of the volume}} of a \emph{subset} in $X$;
\item \emph{\textbf{Positive linear functional on $\cC_0(X)$}}: as a \emph{\textbf{integral} of compactly supported continuous functions with respect to \textbf{given measure}}.
\end{enumerate}
In some cases, it is important to think of \emph{\textbf{measures}} not merely as individual  objects but instead as \emph{elements of $(\cC_{0}(X))^{*}$}, so that we can employ \emph{geometric} ideas. 
\end{remark}


\item \begin{remark} (\emph{\textbf{Weak$^{*}$ Topology on $\cM(X)$}})\\
\emph{The weak$^{*}$ topology} on $\cM(X)$, $X$ a \emph{\textbf{compact Hausdorff}} space, is often called \emph{\textbf{the vague topology}}. Note that $\mu_n \stackrel{w^{*}}{\rightarrow} \mu$ if and only if $\int f d\mu_n \rightarrow \int f d\mu$ for all $f \in \cC_0(X)$.

It can be shown that \emph{the linear combinations of point masses} are \emph{\textbf{weak$^{*}$ dense}} in  $\cM(X)$. That is, for given $\mu \in \cM(X)$,  $f_1 \xdotx{,} f_n \in \cC(X)$ and $\epsilon > 0$, that we can find $\alpha_1 \xdotx{,} \alpha_m \in \bC$ and $x_1 \xdotx{,} x_m \in X$ so that 
\begin{align*}
\abs{\mu(f_i) - \sum_{j=1}^{m}\alpha_j f_{i}(x_j) } < \epsilon, \quad \forall\, i=1 \xdotx{,} n,
\end{align*} i.e. $\sum_{j=1}^{m}\alpha_j \delta_{x_j} \rightarrow \mu$ where $\delta_x(f) = f(x)$ is the \emph{\textbf{evaluation map}} and $\delta_x(\cdot) \mapsto \delta_x$ is identified with the \emph{\textbf{point mass}}.
\end{remark}

\item \begin{proposition} (\textbf{Criterion for Weak$^{*}$ (Vague) Convergence on $\cM(X)$}) \citep{folland2013real}\\
Suppose $\mu_1, \mu_2, \ldots \in \cM(\bR)$, and let $F_{n}(x) = \mu_n((-\infty, x])$ and $F(x) = \mu((-\infty, x])$.
\begin{enumerate}
\item  If $\sup_n\norm{\mu_n}{} < \infty$ and $F_n(x) \rightarrow F(x)$ for \textbf{every x} at which $F$ is \textbf{continuous},
then $\mu_n \rightarrow \mu$ \textbf{vaguely}.
\item If $\mu_n \rightarrow \mu$ \textbf{vaguely}, then $\sup_n\norm{\mu_n}{} < \infty$. If, in addition, the $\mu_n$’s are \textbf{positive},
then $F_n(x) \rightarrow F(x)$ at \textbf{every} $x$ at which $F$ is \textbf{continuous}.
\end{enumerate}
\end{proposition}

\item Finally, we tends to the geometrical properties of subspace of $\cM(X)$
\begin{definition} (\emph{\textbf{Convex Cone}})\\
A set $A$ in a vector space $Y$ is called \emph{\textbf{convex}} if $x$ and $y \in A$  and $0 \le t \le 1$ implies $tx + (1 - t)y \in A$. Thus $A$ is \emph{convex} if the \emph{\textbf{line segment}}  between $x$ and $y$ is in $A$ whenever $x$ and $y$ are in $A$. $A$ is called a \emph{\textbf{cone}} if $x \in A$ implies $tx \in A$ for all $t > 0$. If $A$ is \emph{convex} and a \emph{cone}, it is called  a \emph{\textbf{convex cone}}. 
\end{definition}

\item \begin{proposition} (\textbf{Geometry of $\cM_{+}(X)$ and $\cM_{+,1}(X)$}) \citep{reed1980methods}\\
Let $X$ be a \textbf{compact Hausdorff} space. Then $\cM_{+,1}(X)$ is \textbf{convex} and $\cM_{+}(X)$ is a \textbf{convex cone}. 
\end{proposition}
\end{itemize}


\newpage
\bibliographystyle{plainnat}
\bibliography{reference.bib}
\end{document}