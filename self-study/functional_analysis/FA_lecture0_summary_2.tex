\documentclass[11pt]{article}
\usepackage[scaled=0.92]{helvet}
\usepackage{geometry}
\geometry{letterpaper,tmargin=1in,bmargin=1in,lmargin=1in,rmargin=1in}
\usepackage[parfill]{parskip} % Activate to begin paragraphs with an empty line rather than an indent %\usepackage{graphicx}
\usepackage{amsmath,amssymb, mathrsfs,  mathtools, dsfont}
\usepackage{tabularx}
\usepackage{tikz-cd}
\usepackage[font=footnotesize,labelfont=bf]{caption}
\usepackage{graphicx}
\usepackage{xcolor}
%\usepackage[linkbordercolor ={1 1 1} ]{hyperref}
%\usepackage[sf]{titlesec}
\usepackage{natbib}
\usepackage{../../Tianpei_Report}

\begin{document}
\title{Lecture 0: Summary (Part 3)}
\author{ Tianpei Xie}
\date{ Dec. 15th., 2022 }
\maketitle
\tableofcontents
\newpage
\section{Normed Linear Space}
\begin{itemize}
\item \begin{remark}
Note that the definition of a \emph{metric space} is only about the \emph{\textbf{topology}} of the space. In the field of functional analysis,  we are mostly concerned about \emph{\textbf{the vector space}}, i.e. a space that equipped with algebraic operations such as vector addition and  scalar multiplications. In order to make the \emph{\textbf{metric}\textbf{ topological structure}} \emph{\textbf{compatible}} with \emph{\textbf{the algebraic structure} of \textbf{vector space}}, we need to introduce additional function such as the \emph{\textbf{norm}}.
\end{remark}

\item \begin{definition} (\emph{\textbf{Normed Linear Space}})\\
A \underline{\emph{\textbf{normed linear space}}} is a vector space, $V$, over $\bR$ (or $\bC$) and a function, $\norm{\cdot}{}: V \rightarrow \bR$ which satisfies:
\begin{enumerate}
\item (\emph{\textbf{Non-Negativity}}): $\norm{v}{} \ge 0$ for all $v$ in $V$;
\item (\emph{\textbf{Positive Definiteness}}): $\norm{v}{} = 0$ if and only if $v = 0$;
\item (\emph{\textbf{Absolute Homogeneity}}) $\norm{\alpha v}{} = \abs{\alpha}\norm{v}{}$ for all $v$ in $V$ and $\alpha$ in $\bR$ (or $\bC$)
\item (\emph{\textbf{Subadditivity / Triangle Inequality}}) $\norm{v + w}{}\le \norm{v}{} + \norm{w}{}$ for all $v$ and $w$ in $V$
\end{enumerate}
We denote the normed linear space as $(V, \norm{\cdot}{})$.
\end{definition}

\item \begin{remark}
If the function $p: V \rightarrow \bR$ only satisfies the condition 1, 3 and 4 (without \emph{positive definiteness}), it is called a \underline{\emph{\textbf{semi-norm}}}. The 1. \emph{non-negativity} condition can be derived by the 3. \emph{homogeneity} and 4. \emph{subadditivity} conditions. 
\end{remark}

\item \begin{remark}
\emph{A normed linear space} $(V, \norm{\cdot}{})$ is a \emph{\textbf{metric space}} with \emph{induced metric }
\begin{align*}
d(x, y) = \norm{x - y}{}, \quad \text{for all } x, y \in V
\end{align*}
\end{remark}
\end{itemize}

\section{Banach Space}
\subsection{Definition and Examples}
\begin{itemize}
\item \begin{definition}
A normed linear space $(V, \norm{\cdot}{})$  is \underline{\emph{\textbf{complete}}} if it is \emph{complete} as a \emph{metric space} in \emph{the induced metric}.
\end{definition}

\item \begin{definition}
A \emph{\textbf{complete}} \emph{normed linear space} is called a \underline{\emph{\textbf{Banach space}}}.
\end{definition}

\item \begin{example} ($\cC(X)$ and its subspace $\cC_{\bR}(X)$)\\
Let $\cC(X)$ be the set of all \emph{complex-valued \textbf{continuous functions}} on $X$ and $\cC_{\bR}(X) \subseteq \cC(X)$ be the set of all \emph{\textbf{real-valued continuous functions}} on $X$. Also define $\cC^{b}(X)$ as the set of all \emph{complex-valued \textbf{bounded continuous functions}} on $X$. When $X$ is \emph{\textbf{a compact space}}, $\cC^{b}(X) = \cC(X)$.  Define the norm as 
\begin{align*}
\norm{f}{\infty} &= \sup_{x\in X}\abs{f(x)}.
\end{align*} Then for \underline{\emph{\textbf{compact Hausdorff space}} $X$}, \underline{$\cC(X)$ is a \emph{(complex) \textbf{Banach space}}} and $\cC(X)$ is \emph{\textbf{a (real) Banach space}}.
\end{example}


\item \begin{example} (\emph{\textbf{$L^{\infty}(\bR)$ and its subspace $\cB\cC(\bR)$}})\\
Let $L^{\infty}(\bR)$ be the set of (\emph{equivalence classes of}) \emph{complex-valued measurable functions} on $\bR$ such that $\abs{f(x)}\le M$ a.e. with respect to Lebesgue measure for some $M < \infty$ ($f = g$ means $f(x) = g(x)$ a.e.). Let $\norm{f}{\infty}$ be \emph{\textbf{the smallest such $M$}}. \underline{$L^{\infty}(\bR)$ is a \emph{\textbf{Banach space}}} with norm $\norm{\cdot}{\infty}$. 

\underline{\emph{The \textbf{bounded continuous functions}} $\cB\cC(\bR)$ is a \emph{\textbf{subspace}} of $L^{\infty}(\bR)$}  and restricted to $\cB\cC(\bR)$ the $\norm{\cdot}{\infty}$-norm is just the usual \emph{\textbf{supremum norm}} under which $\cB\cC(\bR)$  is \underline{\emph{\textbf{complete}}} (since \emph{the uniform limit of continuous functions is continuous}). Thus, \emph{\textbf{$\cB\cC(\bR)$ is a closed subspace of $L^{\infty}(\bR)$ }}.

Consider the set $\kappa(\bR)$ of \emph{\textbf{continuous functions} with \textbf{compact support}}, that is, the continuous functions that \emph{vanish outside of some closed interval}.  $\kappa(\bR)$  is a \emph{\textbf{normed linear space}} under $\norm{\cdot}{\infty}$; but \emph{is \textbf{not complete}}, The \emph{\textbf{completion}} of $\kappa(\bR)$ is \emph{\textbf{not all} of $\cB\cC(\bR)$}; for example, if $f$ is the function which is identically equal to one, then I \emph{cannot be approximated by a function in $\kappa(\bR)$} since $\norm{f - g}{\infty} \ge 1$ for all $g \in \kappa(\bR)$. The \emph{\textbf{completion}} of $\kappa(\bR)$ is just $\cC_{\infty}(\bR)$, \emph{the continuous functions which \textbf{approach zero} at $\infty$}. 

Some of the most powerful theorems in functional analysis (\emph{Riesz-Markov}, \emph{Stone-Weierstrass}) are generalizations of properties of $\cB\cC(\bR)$. \qed
\end{example}

\item \begin{example} (\emph{\textbf{$L^{p}$ spaces}})\\
Let $(X, \mu)$ be a measure space and $p\ge 1$. We denote by $L^{p}(X, \mu)$ \emph{\textbf{the set of equivalence classes}} of measurable functions which satisfy:
\begin{align*}
\norm{f}{p} &:= \paren{\int_{X}\abs{f(x)}^p d\mu(x)}^{\frac{1}{p}} < \infty
\end{align*} Two functions are \emph{equivalent} if they differ only on \emph{a set of measure zero}.

The following theorem collects many of the standard facts about $L^{p}$ spaces.
\begin{theorem}
Let $1 \le p < \infty$, then
\begin{enumerate}
\item \underline{(\textbf{The Minkowski Inequality})}: If $f, g \in L^{p}(X, \mu)$, then
\begin{align*}
\norm{f +g}{p} &\le \norm{f}{p} + \norm{g}{p}
\end{align*}
\item  \underline{(\textbf{Riesz-Fisher})}:  $L^{p}(X, \mu)$ is \textbf{complete}.
\item \underline{(\textbf{The H\"older Inequality})} Let $p, q$, and $r$ be positive numbers satisfying
$p, q, r \ge 1$ and $p^{-1} +q^{-1} = r^{-1}$. Suppose  $f \in L^{p}(X, \mu)$, $g \in L^{q}(X, \mu)$. Then
$fg \in L^{r}(X, \mu)$ and
\begin{align*}
\norm{fg}{r} &\le \norm{f}{p}\,\norm{g}{q}
\end{align*}
\end{enumerate}
\end{theorem}

\begin{remark}
\emph{The Minkowski inequality} shows that $L^{p}(X, \mu)$ is a vector space and $\norm{\cdot}{p}$ satisfies the triangle inequality. This together with \emph{Riesz-Fisher theorem} shows that \emph{\textbf{$L^{p}(X, \mu)$ is a Banach space}}.
\end{remark}
\end{example}


\item \begin{example}(\emph{\textbf{Sequence Spaces}})\\
There is a nice class of spaces which is easy to describe and which we will often use to illustrate various concepts.
In the following definitions,
\begin{align*}
a &:= (a_n)_{n=1}^{\infty}
\end{align*} always denotes a sequence of complex numbers.
\begin{align*}
\ell^{\infty} &:= \set{a: \norm{a}{\infty}:= \sup_{n}\abs{a_n} < \infty }\\
c_{0} &:= \set{a: \lim\limits_{n\rightarrow \infty}a_n = 0}\\
\ell^{p} &:= \set{a: \norm{a}{p}:= \paren{\sum_{n=1}^{\infty}\abs{a_n}^{p}}^{\frac{1}{p}} < \infty }\\
s &:= \set{a:  \lim\limits_{n\rightarrow \infty}n^{p}a_{n}= 0 \text{ \emph{for all positive integers} }p}\\
f &:=  \set{a: a_{n} = 0\text{ \emph{for all but a finite number of} }n}
\end{align*} It is clear that as sets $f \subseteq s \subseteq \ell^{p} \subseteq c_{0} \subseteq \ell^{\infty}$.

\underline{\emph{\textbf{The spaces $\ell^{\infty}$ and $c_{0}$ are Banach spaces with the $\norm{\cdot}{\infty}$ norm}}}; \emph{\textbf{\underline{$\ell^{p}$ is a Banach space} with the $\norm{\cdot}{p}$ norm}} (note that $\ell^{p}= L^{p}(\bR, \mu)$ where $\mu$ is the measure with \emph{mass one at each positive integer} and \emph{zero} everywhere else). It will turn out that \emph{\textbf{$s$ is a Frechet space}}. 

One of the reasons that these spaces are easy to handle is that \emph{\underline{$f$ is \textbf{dense} in $\ell^{p}$} (in $\norm{\cdot}{p}$; $p < \infty$} and \emph{\textbf{\underline{$f$ is dense in $c_{0}$} (in the $\norm{\cdot}{\infty}$ norm)}}. Actually, the set of elements of $f$ with only \emph{rational entries} is also \textbf{dense} in $\ell^{p}$ and $c_{0}$. Since this set is \emph{\textbf{countable}}, $\ell^{p}$ and $c_{0}$ are \emph{\textbf{separable}}. \emph{\textbf{$\ell^{\infty}$ is not separable}}.
\end{example}



\item \begin{example} (\emph{\textbf{Hilbert Space}})\\
All \emph{\textbf{Hilbert spaces}} $(\cH, \inn{\cdot}{\cdot})$ are \textbf{\emph{Banach spaces}} with \emph{induced norm} as
\begin{align*}
\norm{x}{} &= \paren{\inn{x}{x}}^{\frac{1}{2}}.
\end{align*} 
\end{example}
\end{itemize}

\subsection{Isomorphism and Equivalence of Norms}
\begin{itemize}
\item \begin{definition} (\emph{\textbf{Absolutely Summable}})\\
A sequence of elements $(x_n)_{n=1}^{\infty}$ in a \emph{normed linear space} $X$ is called \emph{\textbf{absolutely summable}} $\sum_{n=1}^{\infty}\norm{x_n}{} < \infty$. It is called \emph{\textbf{summable}} if $\sum_{n=1}^{N}x_n$ \emph{converges} as $N \rightarrow\infty$ to an $x \in X$.
\end{definition}

\item \begin{proposition} (\textbf{Criterion of Completeness for Normed Linear Space}) \citep{reed1980methods} \\
A normed linear space is \textbf{complete} if and only if every \textbf{absolutely summable} sequence is \textbf{summable}.
\end{proposition}

\item \begin{definition} (\emph{\textbf{Isomorphism between Normed Linear Spaces}})\\
A \textit{\textbf{bounded linear operator}} from a normed linear space $X$ to a normed linear space $Y$ is called an \underline{\emph{\textbf{isomorphism}}} if it is a \emph{\textbf{bijection}} which is \emph{\textbf{continuous}} and which has \emph{\textbf{a continuous inverse}}. 

If it is \emph{\textbf{norm preserving}}, it is called \underline{\emph{\textbf{an isometric isomorphism}}} (any \emph{norm preserving} map is called an \emph{\textbf{isometry}}).
\end{definition}


\item \begin{definition} (\emph{\textbf{Norm Equivalence}})\\
Two norms, $\norm{\cdot}{1}$ and $\norm{\cdot}{2}$, on a normed linear space $X$ are called \underline{\emph{\textbf{equivalent}}} if there are positive constants $C$ and $C'$ such that, for all $x \in X$,
\begin{align*}
C\norm{x}{2} \le \norm{x}{1} \le C'\norm{x}{2}
\end{align*}
\end{definition}

\item \begin{remark}
This concept is motivated by the following fact. 

\underline{\emph{\textbf{Equivalent norms} on $X$ define \textbf{the same topology} for $X$.}}
\end{remark}

\item \begin{proposition}
The \textbf{completions} of the space in the two norms will be \textbf{isomorphic} if and only if \textbf{the norms are equivalent}.
\end{proposition}



\item \begin{proposition}
Two norms, $\norm{\cdot}{1}$ and $\norm{\cdot}{2}$, on a normed linear space $X$ are \textbf{equivalent} if and only if \textbf{the identity map} is an \textbf{isomorphism}.
\end{proposition}

\item \begin{remark}
An example is provided by \emph{the sequence spaces}. The \emph{\textbf{completion}} of $f$ in the $\norm{\cdot}{\infty}$ norm is $c_{0}$  while the completion in
the $\norm{\cdot}{p}$ norm is $\ell^{p}$.
\end{remark}
\end{itemize}

\subsection{Subspace of a Banach Space}
\begin{itemize}
\item \begin{definition}
A \emph{\textbf{subspace}} $Y$ of a normed space $X$ is a subspace of $X$ considered as \emph{a vector space}, with the \emph{norm} obtained by \emph{\textbf{restricting} the norm on $X$ to the subset $Y$}. This norm on $Y$ is said to be \emph{\textbf{induced} by the norm on $X$}. 

If $Y$ is closed in $X$, then $Y$ is called \emph{\textbf{a closed subspace}} of $X$.
\end{definition}

\item \begin{remark}
A subspace $Y$ of a \emph{\textbf{Banach space}} $X$ is a subspace of $X$ considered as a normed space. Hence \emph{we do not require $Y$ to be complete}. 
\end{remark}

\item \begin{proposition}(\textbf{Subspace of a Banach space}). \citep{kreyszig1989introductory} \\
 A subspace $Y$ of a Banach space $X$ is \textbf{complete} if and only if the set $Y$ is \textbf{closed} in $X$.
\end{proposition}
\end{itemize}

\subsection{Basis and Separability}
\begin{itemize}
\item \begin{definition} (\emph{\textbf{Basis of Normed Space}})\\
If a normed space $X$ contains \emph{a sequence $(e_i)$} with the property that for \emph{every} $x \in X$ there is a \emph{\textbf{unique}} \emph{sequence of scalars} $(u^i)$ such that
\begin{align}
\lim\limits_{n\rightarrow \infty}\norm{x - \sum_{i=1}^{n}u^i\,e_i}{} = 0, \label{eqn: normed_space_schauder_basis}
\end{align}
then $(e_i)$ is called \emph{a \underline{\textbf{Schauder basis (or basis)}} for $X$}. The series $\sum_{i=1}^{\infty}u^i\,e_i$ which has the sum $x$ is then called \emph{the \textbf{expansion} of $x$ with respect to $(e_i)$}, and we write
\begin{align*}
x = \sum_{i=1}^{\infty}u^i\,e_i
\end{align*}
\end{definition}

\item \begin{example}
The (Schauder) basis of $\ell^{p}$ is $(e_n)$ and 
\begin{align*}
e_n := (\delta_{n,i}) = (0 \xdotx{,} 0, 1, 0, \,\ldots)
\end{align*}  where the $i$-th component is $1$ and the others are all zeros.
\end{example}

\item \begin{proposition}
If a normed space $X$ has a Schauder basis, then $X$ is \textbf{separable}.
\end{proposition}

\item \begin{theorem} (\textbf{Completion}). \citep{kreyszig1989introductory} \\
Let $X = (X, \norm{\cdot}{})$ be a normed space. Then there is a Banach space $X$ and an isometry $A$ from $X$ onto a
subspace $W$ of $X$ which is \textbf{dense} in $X$. The space $X$ is \textbf{unique}, except for isometries.
\end{theorem}
\end{itemize}



\subsection{Direct Sum of Banach Spaces}
\begin{itemize}
\item \begin{definition} (\emph{\textbf{Direct Sum of Banach Spaces}})\\
Let $A$ be an index set (not necessarily countable), and suppose that for each $\alpha\in A$, $X_{\alpha}$ is a Banach space. Let 
\begin{align*}
X := \set{(x_\alpha)_{\alpha \in A}: x_{\alpha} \in X_{\alpha}, \;\; \sum_{\alpha \in A}\norm{x_{\alpha}}{X_{\alpha}} < \infty }.
\end{align*} Then $X$ with the norm
\begin{align*}
\norm{(x_\alpha)_{\alpha \in A}}{X} := \sum_{\alpha \in A}\norm{x_{\alpha}}{X_{\alpha}}
\end{align*} is a Banach space. It is called \underline{\emph{\textbf{the direct sum}}} of the spaces $X_{\alpha}$ and is often written as $X = \bigoplus_{\alpha \in A} X_{\alpha}$.
\end{definition}

\item \begin{remark} (\emph{\textbf{Banach Spaces Direct Sum $\neq$  Hilbert Spaces Direct Sum}})\\
Note that \emph{the direct sum of Banach spaces} is \emph{\textbf{not} necesssarily} \emph{the direct sum of Hilbert spaces}. 

For instance, if we take countable numbers of copies of $\bC$, \emph{the Banach space direct sum} is $\ell_1$, while \emph{the Hilbert space direct sum} is $\ell_2$.

However, if only \emph{\textbf{finite number} of Hilbert spaces} are involved, then both \emph{Hilbert space direct sum} and \emph{their Banach space direct sum} are \emph{isomorphic} to each other.
\end{remark}
\end{itemize}

\subsection{Finite Dimensional Case}
\begin{itemize}
\item \begin{remark} (\emph{\textbf{Finite Dimensional Normed Space is Simple}})\\
We summarizes the \emph{\textbf{unique}} simple strcuture of finite dimensional normed space in terms of various concepts we discussed in this chapter:
\begin{enumerate}
\item \underline{\emph{\textbf{Completeness}}}: \emph{Every finite dimensional normed vector space} is \emph{\textbf{complete}} so it is a \emph{\textbf{Banach space}};

\item \underline{\emph{\textbf{Norm Equivalence}}}: \emph{\textbf{All norms} in a finite dimensional normed space are \textbf{equivalent}}; therefore, \emph{\textbf{convergence}} in one norm means convergence in all other norms.

\item \underline{\emph{\textbf{Topological Equivalence}}}: There exists \emph{\textbf{only one distinct norm topology}} in a finite dimensional normed space; 

\item \underline{\emph{\textbf{Compactness}}}: In a finite dimensional normed space, \emph{\textbf{compactness}} is equivalent to \emph{\textbf{closedness}} and \emph{\textbf{boundedness}}.

\item \underline{\emph{\textbf{Bounded Linear Operator}}}:  \emph{\textbf{Every linear operator}} between  finite dimensional normed spaces  is \emph{\textbf{bounded}}. Thus in finite dimensional space, every linear operator  is \emph{\textbf{continuous}}.
\end{enumerate}
\end{remark}

\item \begin{lemma} (\textbf{Linear combinations}). \citep{kreyszig1989introductory} \\
Let $(x_1 \xdotx{,} x_n)$ be a \textbf{linearly independent} set of vectors in a normed space $X$ (of any dimension).
Then there is a number $c > 0$ such that for every choice of scalars $\alpha_1 \xdotx{,} \alpha_n$ we have
\begin{align}
\norm{\sum_{i=1}^{n}\alpha_i\,x_i}{} \ge c\,\sum_{i=1}^{n}\abs{\alpha_i}.\label{eqn: linear_combination_greater_than_coefficient}
\end{align}
\end{lemma}

\item \begin{theorem} (\textbf{Completeness}). \citep{kreyszig1989introductory} \\
Every finite dimensional subspace $Y$ of a normed space $X$ is \textbf{complete}. In particular, \underline{every \textbf{finite dimensional} normed space is \textbf{complete}}.
\end{theorem}

\item \begin{remark}
In other words, \underline{\emph{every finite dimensional normed vector space} is a \emph{\textbf{Banach space}}}.
\end{remark}

\item \begin{proposition} (\textbf{Closedness}).  \citep{kreyszig1989introductory}\\
Every finite dimensional subspace $Y$ of a normed space $X$ is \textbf{closed} in $X$.
\end{proposition}

\item \begin{theorem} (\textbf{Equivalent Norms}). \citep{kreyszig1989introductory}\\
If a vector space $X$ is \underline{\textbf{finite dimensional}}, \textbf{all norms are equivalent}.
\end{theorem}


\item \begin{remark}
This theorem is of considerable practical importance. For instance, it implies that \emph{\textbf{convergence}} or \emph{divergence} of a sequence in \emph{a finite dimensional vector space} \emph{\textbf{does not depend} on the particular \textbf{choice of a norm} on that space}. There is no ambiguity when we say $x_n \rightarrow x$ in \emph{finite dimensional space}.

In fact, \underline{\emph{\textbf{there exists only one distinct norm topology for finite dimensional space}}}. 
\end{remark}

\item \begin{definition} (\emph{\textbf{Compactness}}). \\
A metric space $X$ is said to be \underline{\emph{\textbf{(sequentially) compact}}} if \emph{every sequence} in $X$ has a \emph{\textbf{convergent subsequence}}. A subset $M$ of $X$ is said to be \emph{\textbf{compact}} if $M$ is \emph{compact} considered as a subspace of $X$, that is, if every sequence in $M$ has a convergent subsequence \emph{whose limit is an element of $M$}.
\end{definition}

\item \begin{lemma} (\textbf{Compactness}). \\
A \textbf{compact} subset M of a metric space is \textbf{closed} and \textbf{bounded}.
\end{lemma}

\item \begin{remark}
\emph{The \textbf{converse} of this lemma is in general \textbf{false}.} But for \emph{finite dimensional space}, the converse is true:
\end{remark}

\item \begin{theorem} (\textbf{Compactness}). \citep{kreyszig1989introductory}\\
In a \textbf{finite} dimensional normed space $X$, any subset $M \subseteq X$ is \textbf{compact} \textbf{if and only if} $M$ is \textbf{closed} and \textbf{bounded}.
\end{theorem}

\item \begin{remark}
\emph{In finite dimensional space}, the \emph{compact} subsets are precisely \emph{the closed and bounded} subsets, so that this property (\emph{\textbf{closedness}} and \emph{\textbf{boundedness}}) can be used for \emph{defining \textbf{compactness}}. 

However, \emph{this can no longer be done} in the case of \emph{an \textbf{infinite dimensional normed space}}.
\end{remark}

\item \begin{lemma} (\textbf{F. Riesz's Lemma}). \citep{kreyszig1989introductory}\\
Let $Y$ and $Z$ be \textbf{subspaces} of a normed space $X$ (of any dimension), and suppose that $Y$ is \textbf{closed} and is a \textbf{proper subset} of $Z$. Then for every real number $\theta$ in the interval $(0,1)$ there is a $z \in Z$ such that
\begin{align*}
\norm{z}{} = 1, \quad \norm{z - y}{} \ge \theta,\quad \text{for all }y \in Y.
\end{align*}
\end{lemma}

\item \begin{theorem} (\textbf{Bounded Linear Operator})\\
If a normed space $X$ is finite dimensional, then every linear operator on $X$ is \textbf{bounded}.
\end{theorem}
\end{itemize}

\section{Bounded Linear Operators on Banach Space}
\subsection{Definitions and Examples}
\begin{itemize}
\item \begin{definition} (\emph{\textbf{Bounded Linear Operator}})\\
A \underline{\emph{\textbf{bounded linear transformation}}} (or \emph{\textbf{bounded operator}}) is a mapping $T: (X, \norm{\cdot}{X}) \rightarrow (Y, \norm{\cdot}{Y})$ from a normed linear space $X$ to a normed linear space $Y$ that satisfies 
\begin{enumerate}
\item (\emph{\textbf{Linearity}}) $T(\alpha x + \beta y) = \alpha T(x) + \beta T(y)$ for all $x, y \in X$, $\alpha, \beta \in \bR$ or $\bC$
\item (\emph{\textbf{Boundedness}}) $\norm{Tx}{Y} \le C\,\norm{x}{X}$ for small $C \ge 0$.
\end{enumerate} The smallest such $C$ is called \underline{\emph{the \textbf{norm} of $T$}}, written $\norm{T}{}$ or $\norm{T}{X, Y}$. Thus
\begin{align*}
\norm{T}{} &:= \sup_{\norm{x}{X} = 1 }\norm{Tx}{Y}
\end{align*}
\end{definition}

\item \begin{remark}
\emph{A linear operator} $T$ is a \emph{\textbf{homomorphism} of a vector space (its domain) into another vector space}, that is, \emph{$T$ \textbf{preserves} the \textbf{two operations} of vector space}.
\end{remark}

\item \begin{proposition} \citep{reed1980methods, kreyszig1989introductory}\\
Let $T$ be a linear transformation between two \textbf{normed linear spaces}. The following are \textbf{equivalent}:
\begin{enumerate}
\item $T$ is \textbf{continuous} at \textbf{one} point.
\item $T$ is \textbf{continuous} at \textbf{all} points.
\item $T$ is \textbf{bounded}.
\end{enumerate}
\end{proposition}

\item \begin{definition}(\emph{\textbf{The Bounded Operators}})\\
In above we defined the concept of a \emph{bounded linear transformation} or \emph{bounded operator} from one normed linear space, $X$, to another $Y$; we will denote \emph{the set of all bounded linear operators from $X$ to $Y$ by $\cL(X, Y)$}. We can introduce a \emph{norm} on  $\cL(X, Y)$ by defining
\begin{align*}
\norm{A}{} &:= \sup_{x \neq 0, \, x \in X} \frac{\norm{Ax}{Y}}{\norm{x}{X}}.
\end{align*}
This norm is often called \underline{\emph{\textbf{the operator norm}}}.
\end{definition}

\item We have the following proposition
\begin{proposition} \label{prop: bounded_operator_banach}
If $Y$ is \textbf{complete}, $\cL(X, Y)$ is a \textbf{Banach space}.
\end{proposition}

\item \begin{theorem} (\textbf{The B.L.T. Theorem}) \citep{reed1980methods}\\
Suppose $T$ is a bounded linear transformation from a normed linear space $(V_1, \norm{\cdot}{1})$ to a \textbf{complete} normed linear space $(V_2, \norm{\cdot}{2})$. Then $T$ can be \textbf{uniquely extended} to a bounded linear transformation (with the same bound), $\widetilde{T}$, from the \textbf{completion} of $V_1$ to $(V_2, \norm{\cdot}{2})$.
\end{theorem}
\end{itemize}

\subsection{Dual Space}
\begin{itemize}
\item \begin{definition} (\emph{\textbf{Dual Space}})\\
\emph{The space $\cL(X, \bC)$ of all \underline{\textbf{bounded linear functionals}}} on \emph{a normed linear space} $X$ is called the \underline{\emph{\textbf{dual space} of $X$}}. This space $\cL(X, \bC)$ is denoted as $X^{*}$.

\emph{\underline{\textbf{The dual space $X^{*}$ is a Banach space}} if $X$ is a Banach space} (See Proposition \ref{prop: bounded_operator_banach}). The \emph{\textbf{norm}} of dual space is 
\begin{align*}
\norm{\lambda}{} &:= \sup_{x \neq 0, \, \norm{x}{} \le 1} \abs{\lambda(x)},
\end{align*} for all $\lambda \in X^{*}$.
\end{definition}

\item \begin{remark} (\emph{\textbf{Cauchy-Schwartz inequality}})\\
By definition, we have the \emph{dual norm inequality} 
\begin{align}
\abs{\lambda(x)} &\le \norm{\lambda}{X^{*}}\,\norm{x}{X}. \label{eqn: dual_norm_inequality}
\end{align} In Hilbert space, since $\lambda(x) = \inn{y_{\lambda}}{x}$ for some $y_{\lambda}$, it becomes \emph{\textbf{the Cauchy-Schwartz inequality}}.
\begin{align*}
\abs{\inn{y_{\lambda}}{x}} &\le \norm{y_{\lambda}}{}\norm{x}{}
\end{align*}
\end{remark}

\item \begin{example} (\textbf{\emph{Hilbert Space}})\\
Any \emph{\textbf{Hilbert space}} $\cH$ is \emph{\textbf{isomorphic}} to its \emph{\textbf{dual}} $\cH^{*}$ according to \emph{the Riesz Representation Theorem}. For instance $L^2(X, \mu) = (L^2(X, \mu))^{*}$.
\end{example}

\item  \begin{example} (\emph{\textbf{$L^{p}(X, \mu)$ Spaces, $1 < p < \infty$}})\\
Suppose that \underline{$1 < p < \infty$} and \underline{$p^{-1} + q^{-1} = 1$}. If $f \in L^{p}(X, \mu)$ and $g \in L^{q}(X, \mu)$ then. according to \emph{the H{\"o}lder inequality}, $fg$ is in $L^1(X, \mu)$.  Thus,
\begin{align*}
\int_{X} f(x) \overline{g(x)} d\mu(x) < \infty
\end{align*} makes sense, Let $g \in L^{q}(X, \mu)$ be fixed and define
\begin{align*}
G(f) &:= \int_{X} f \overline{g}  d\mu
\end{align*} for each $f \in L^{p}(X, \mu)$.  The H{\"o}lder inequality shows that \emph{$G(f)$ is a \textbf{bounded linear functional}} on $L^{p}(X, \mu)$ with \emph{norm} less than or equal to $\norm{g}{q}$; actually \emph{\textbf{the norm $\norm{G}{}$ is equal to $\norm{g}{q}$}}. 

The \emph{converse} of this statement is also \emph{true}. That is, \emph{\textbf{every
bounded linear functional on $L^{p}$ is of the form $G(f)$ for some $g \in L^{q}$}}. Furthermore, \emph{different \textbf{functions} in $L^q$ give rise to different \textbf{functionals} on $L^{p}$}. Thus, the mapping 
\begin{align*}
L^q(M, \mu) \rightarrow (L^{p}(X, \mu))^{*}, \quad g \mapsto G_{g}(\cdot)
\end{align*}  is a \underline{\emph{\textbf{(conjugate linear) isometric isomorphism}}}. 

In this sense, \underline{\emph{\textbf{$L^q(M, \mu)$ is the dual of $L^{p}(X, \mu)$}}}.  Since the roles of $p$ and $q$ in the expression $p^{-1} + q^{-1} = 1$ are \emph{symmetric}, it is clear that $L^{p}(X, \mu) = (L^{q}(X, \mu))^{*} =  (L^{p}(X, \mu))^{**}$. That is, the \emph{\textbf{dual}} of the \emph{\textbf{dual}} of $L^{p}(X, \mu)$ is again $L^{p}(X, \mu)$. \qed
\end{example}

\item \begin{remark}
Note that $L^{\infty}(X, \mu)$ space and $L^{1}(X, \mu)$ space are \emph{\textbf{not dual}} spaces to each other. The dual space of $L^{\infty}(X, \mu)$ space is much larger than $L^{1}(X, \mu)$ space. In fact, \emph{\textbf{$L^{1}(X, \mu)$ space is not dual to any Banach space}}.  This is different from $\ell^{\infty}$ and $\ell^1$.
\end{remark}

\item \begin{example} ($\ell^{\infty} = (\ell^1)^{*}$, $\ell^1 = (c_0)^{*}$)\\
Suppose that $(\lambda_k)_{k= 1}^{\infty} \in \ell^1$. Then for each $(a_k)_{k=1}^{\infty} \in c_0$,
\begin{align*}
\Lambda\paren{(a_k)_{k=1}^{\infty}} &= \sum_{k=1}^{\infty}\lambda_k\, a_k
\end{align*} \emph{converges} and $\Lambda(\cdot)$ is a \emph{\textbf{continuous linear functional}} on $c_0$ with \emph{\textbf{norm}} equal
to $\sum_{k=1}^{\infty}\abs{\lambda_k}$. 
\end{example}

\item \begin{remark}
We see that $c_0 \subseteq  (c_0)^{**} = (\ell^1)^{*} = \ell^{\infty}$.
\end{remark}

\item \begin{definition} (\emph{\textbf{Double Dual}})\\
Since the \emph{\textbf{dual}} $X^{*}$ of \emph{a Banach space} is itself \textit{a Banach space}, it also has \emph{a \textbf{dual} space}, denoted by $X^{**}$. $X^{**}$ is called \underline{\emph{\textbf{the second dual}}}, \emph{\textbf{the bidual}}, or \emph{\textbf{\underline{the double dual}} of the space $X$}.
\end{definition}

\item \begin{proposition} \citep{reed1980methods}\\
Let $X$ be a Banach space. For each $x \in X$, let $\widetilde{x}(\cdot)$ be the linear functional on $X^{*}$ which assigns to each $\lambda \in X^{*}$ the number $\lambda(x)$. Then the map $J: x \mapsto \widetilde{x}$ is an \textbf{isometric isomorphism} of $X$ onto a (possibly proper) subspace of $X^{**}$.
\end{proposition}

\item \begin{remark}
From above proposition, we see that there exists an \emph{embedding} from $X$ to a subset of $X^{**}$
\begin{align*}
X \subseteq X^{**}, \quad X \xhookrightarrow{} X^{**}
\end{align*}
\end{remark}

\item \begin{definition}
If the map $J: x \mapsto \widetilde{x}$ is \emph{\textbf{surjective}}, then $X$ is said to be \underline{\emph{\textbf{reflexive}}}. In other word, $X$ is reflective if and only if $X = X^{**}$.
\end{definition}

\item \begin{example}
$L^{p}(X, \mu)$ spaces are \emph{\textbf{reflective}} for $1 < p < \infty$. Note that $L^{p}(X, \mu) = (L^{q}(X, \mu))^{*} =  (L^{p}(X, \mu))^{**}$
\end{example}

\item \begin{example}
All Hilbert spaces $\cH$ are \emph{\textbf{reflective}}.
\end{example}

\item \begin{example}
Since $c_0 \subseteq  (c_0)^{**} = (\ell^1)^{*} = \ell^{\infty}$, $c_0$ is \emph{not reflective}.
\end{example}
\end{itemize}


\subsection{Dual Space of Compact Supported Continuous Functions}
\subsubsection{Radon Measure}
\begin{itemize}
\item \begin{definition} (\emph{\textbf{Outer Regularity}}) \citep{folland2013real} \\
Let $\mu$ be a \textbf{\emph{Borel}} measure on $X$ and $E$ a \emph{Borel subset} of $X$. The measure $\mu$ is called \underline{\textbf{\emph{outer regular}}} on $E$  if
\begin{align*}
\mu(E) &= \inf\set{\mu(U): U \supseteq E, U \text{ is open}}
\end{align*}
\end{definition}

\item \begin{definition} (\emph{\textbf{Inner Regularity}}) \citep{folland2013real} \\
Let $\mu$ be a \textbf{\emph{Borel}} measure on $X$ and $E$ a \emph{Borel subset} of $X$. The measure $\mu$ is called \underline{\textbf{\emph{inner regular}}} on $E$  if
\begin{align*}
\mu(E) &= \sup\set{\mu(C): C \subseteq E, C \text{ is compact}}
\end{align*}
\end{definition}

\item \begin{definition}
If $\mu$ is \emph{outer} and \emph{inner regular} on \emph{all Borel sets}, $\mu$ is called \underline{\textbf{\emph{regular}}}. 
\end{definition}

\item \begin{remark}
\emph{\textbf{Baire measure}} is equivalent to a \emph{\textbf{regular Borel measure (Randon measure)}} in the context of \emph{\textbf{compact space}} $X$.
\end{remark}

\item \begin{definition} (\emph{\textbf{Radon Measure}}) \citep{folland2013real} \\
A \underline{\textbf{\emph{Radon measure}}} $\mu$ on $X$ is a \emph{Borel measure} that is 
\begin{enumerate}
\item \emph{\textbf{finite}} on \emph{all \textbf{compact} sets}; i.e. for any \emph{\textbf{compact subset}} $K \subseteq X$, 
\begin{align*}
\mu(K) < \infty.
\end{align*}
\item \emph{\textbf{outer regular}} on \emph{all Borel sets}; i.e. for any \emph{Borel set} $E$ 
\begin{align*}
\mu(E) &= \inf\set{\mu(U): E \subseteq U, U \text{\emph{ is open}}}.
\end{align*}
\item  \emph{\textbf{inner regular}} on all \emph{open sets}; i.e. for any \emph{open set} $E$
\begin{align*}
\mu(E) &= \sup\set{\mu(C): C \subseteq E, C \text{\emph{ is compact and Borel}}}.
\end{align*}
\end{enumerate}
\end{definition}

\item \begin{remark}
\emph{\textbf{Randon measure}} is called \underline{\emph{\textbf{regular Borel measure}}}.
\end{remark}
\end{itemize}

\subsubsection{The Riesz-Markov Representation Theorem}
\begin{itemize}
\item \begin{definition} (\emph{\textbf{Positive Linear Functional}})\\
Let $\cC(X)$ be \emph{the space of \textbf{continuous} functions} on $X$. A \underline{\textbf{\emph{positive linear functional}}} on $\cC(X)$ is a (not necessarily a 
priori continuous) \emph{linear functiona} $I$ with $I(f) > 0$ for all $f$  with $f(x) \ge 0$ pointwise. 
\end{definition}

\item \begin{lemma} (\textbf{Bounded by Unit Ball in Uniform Metric}) \citep{folland2013real}\\
If $I$ is a positive linear functional on $\cC_{c}(X)$, for each compact $C \subseteq X$ there is a constant $\kappa_{C}$ such that $\abs{I(f)} < \kappa_C\norm{f}{u}$ for all $f \in \cC_{c}(X)$ such that $\text{supp}(f) \subset K$.
\end{lemma}


\item \begin{remark}
If $\mu$ is a \emph{Borel measure} on $X$ such that $\mu(C) < \infty$ for every compact subset $C \subseteq X$, then $\cC_c(X) \subseteq L^1(X, \mu)$. Therefore, $f \mapsto \int f d\mu$ is a \emph{\textbf{positive linear functional}} on $\cC_c(X)$.

The following theorem shows that the \underline{\emph{\textbf{every positive linear functionals}}} on $\cC_c(X)$ can be \emph{\textbf{represented}} as the \emph{integral} \emph{with respect to \underline{\textbf{some Radon measure}}} $\mu$.
\end{remark}

\item \begin{theorem} (\textbf{The Riesz-Markov Representation Theorem}). \citep{folland2013real}\\
Let $X$ be a \textbf{locally compact Hausdorff} space, if $I$ is a \underline{\textbf{positive linear functional}} on \underline{$\cC_{c}(X)$}, there is a \underline{\textbf{unique Radon measure}} $\mu$ on $X$ such that
\begin{align*}
I(f) &= \int f d\mu 
\end{align*}
for all $f \in \cC_{c}(X)$. Moreover, $\mu$ satisfies the following conditions:
\begin{enumerate}
\item for all \textbf{open} sets $U \subseteq X$,
\begin{align*}
\mu(U) &= \sup\set{I(f): f \in \cC_c(X), \text{supp}(f) \subseteq U, \; 0 \le f \le 1}.
\end{align*} 
\item for all \textbf{compact} sets $K \subseteq X$
\begin{align*}
\mu(K) &= \inf\set{I(f): f \in \cC_c(X), f \ge \mathds{1}_{K}}.
\end{align*}
\end{enumerate}
\end{theorem}

\item \begin{remark}
Following \emph{the Riesz-Markov Theorem}
\begin{align*}
\mu(X) &= \sup\set{\int_X f d\mu: f \in \cC_c(X),  \; 0 \le f \le 1}.
\end{align*}
\end{remark}

\item The following theorem is another version of \emph{the Riesz representation theorem}:
 \begin{theorem} (\textbf{The Riesz-Markov Theorem}) \citep{reed1980methods}\\
Let $X$ be a \underline{\textbf{compact Hausdorff}} space. For any \textbf{positive linear functional} $I$ on \underline{$\cC(X)$}, there is a \textbf{unique Baire measure} $\mu$ on $X$ with 
\begin{align*}
I(f) &= \int f d\mu 
\end{align*} 
\end{theorem}

\item \begin{remark} (\emph{\textbf{Radon Measures $\Leftrightarrow$ Positive Linear Functionals on $\cC_c(X)$}})\\
\emph{The Riesz-Markov theorem} relates \emph{\textbf{linear functionals}} on spaces of \emph{\textbf{continuous} functions on a \textbf{locally compact} space} to \emph{\textbf{measures}} in \textbf{\emph{measure theory}}. 
\end{remark}

\item \begin{remark}
\emph{\textbf{Not to be confused}} with \emph{another Riesz representation theorem}, which related \emph{linear functions on Hilbert space} as inner product with some element in Hilbert space
\begin{align*}
I(f) &= \inn{f}{g_{I}}
\end{align*} for some $g_{I} \in \cH$.
\end{remark}

\item \begin{remark} (\emph{\textbf{Duality between $\cC_{0}(X)$ and $\cM(X)$}})\\
\emph{The Riesz representation theorem} establishes the \emph{\textbf{foundation}} of the \underline{\emph{\textbf{the duality}}} between \emph{the space of compactly supported continuous functions} and \emph{the space of all Radon \textbf{measures}} on $X$. 

In particular, for \emph{locally compact Hausdorff} $X$, 
\begin{align*}
\set{\mu: \mu \text{\emph{ is a Radon measure on }}X} \simeq \set{I \in \cC_{0}(X)^{*}: I \text{ is positive}}
\end{align*}
\end{remark}
\end{itemize}

\subsubsection{Dual Space of $\cC_0(X)$}
\begin{itemize}
\item \begin{theorem} (\textbf{Monotone Convergence Theorem for Nets}) \citep{reed1980methods}\\
Let $\mu$ be a \textbf{regular Borel} measure on a \textbf{compact Hausdorff} space $X$. Let $\set{f_{\alpha}}_{\alpha \in J}$ be an \textbf{increasing net} of continuous functions.  Then
\begin{align*}
f_{\alpha} \rightarrow f  \in L^1(X, \mu), \quad a.e.
\end{align*}  \textbf{if and only if} $\sup_{\alpha}\norm{f_{\alpha}}{1} < \infty$ and in that case
\begin{align*}
\norm{f_\alpha - f}{1} \rightarrow 0.
\end{align*}
\end{theorem}

\item \begin{lemma} \citep{reed1980methods}\\
Let $f,g \in \cC(X)$ with $f, g \ge 0$. Suppose $h \in \cC(X)$ and $0 \le h \le f + g$. 
Then, we can write $h = h_1 + h_2$ with $0\le h_1 \le f$, $0 \le h_2 \le g$,  $h_1, h_2 \in \cC(X)$. 
\end{lemma}

\item \begin{theorem}  (\textbf{Decomposition of Real Linear Functional}) \citep{reed1980methods, folland2013real}\\
Let  $X$ be a \textbf{compact} space, $I \in (\cC(X))^{*}$ be any continuous linear functional on $\cC(X)$. Then $I$ can be written 
\begin{align*}
I &= I_{+} - I_{-}
\end{align*}
with $I_{+}$ and $I_{-}$ \textbf{positive linear functionals}. Moreover, 
\begin{align*}
I_{+} + I_{-}= \norm{I}{}
\end{align*}
and this \textbf{uniquely determines} $I_{+}$ and $I_{-}$ . 
\end{theorem}

\item \begin{definition} (\emph{\textbf{Complex Radon Measure}})\\
A \underline{\emph{\textbf{signed Radon measure}}} is a \emph{\textbf{signed Borel measure}} whose \emph{\textbf{positive}} and \emph{\textbf{negative variations}} are \emph{\textbf{Radon}}, and a \underline{\emph{\textbf{complex Radon measure}}} is a \emph{\textbf{complex Borel measure}} whose \emph{real and imaginary parts} are \textit{signed Radon measures}. 
% a \underline{\emph{\textbf{complex Radon Measure}}} $\nu$, if $\nu$ can be represented as \emph{a \textbf{finite linear complex combinations}} of \emph{\textbf{Radon measures}}, i.e. $\nu = \sum_{i=1}^{n}c_i\, \nu_i$ where $\nu_i$ are  \emph{Radon measures} and $\set{c_i}$ are \emph{complex linear coefficients}.
\end{definition}

\item \begin{remark}
In \citep{reed1980methods}, one defines \emph{\textbf{the complex Baire measure}} as a \emph{finite linear complex combination} of \emph{Baire measures}. 
\end{remark}

\item \begin{definition} (\emph{\textbf{Space of Complex Radon Measures}})\\
On \emph{locally compact Hausdorff space} $X$, We denote \emph{the space of complex Radon measures
on} $X$ by $\cM(X)$. For $\mu \in \cM(X)$ we define
\begin{align*}
\norm{\mu}{} &= \abs{\mu}(X),
\end{align*}
where $ \abs{\mu}$ is the \emph{\textbf{total variation}} of $\mu$.  
\end{definition}

\item \begin{proposition} (\textbf{$\cM(X)$ is Normed Linear Space}) \citep{folland2013real}\\
If $\mu$ is a \textbf{complex Borel measure}, then $\mu$ is \textbf{Radon} if and only if $\abs{\mu}$ is \textbf{Radon}.
Moreover, $\cM(X)$ is a vector space and $\mu \mapsto \norm{\mu}{}$ is a \textbf{norm} on it.
\end{proposition}



\item \begin{theorem} (\textbf{The Riesz-Markov Theorem, Locally Compact Version}) \citep{reed1980methods, folland2013real}\\
Let $X$ be a \textbf{locally compact Hausdorff} space. For any continuous linear functional $I$ on $\cC_{0}(X)$, (the space of \emph{continuous functions} on $X$ that vanishes at infinity), there is a \textbf{\underline{unique regular countably additive complex Borel measure}} $\mu$ on $X$ such that
\begin{align*}
I(f) &= \int_{X} f d\mu, \quad \text{ for all } f \in \cC_0(X).
\end{align*} The \underline{\textbf{norm}} of $I$ as a linear functional is \underline{\textbf{the total variation}} of $\mu$, that is
\begin{align*}
\norm{I}{} &= \abs{\mu}(X).
\end{align*}
Finally, $I$ is \textbf{positive} if and only if the measure $\mu$ is \textbf{non-negative}.
\end{theorem}

\item \begin{remark}
In other word, the map $\mu \mapsto I_{\mu}$, is an \emph{\textbf{isometric isomorphism}} from $\cM(X)$  to $(\cC_0(X))^{*}$, or 
\begin{align*}
\cM(X) \simeq (\cC_0(X))^{*}.
\end{align*}
\end{remark}

\item \begin{corollary}  \citep{reed1980methods, folland2013real}\\
Let $X$ be a \textbf{compact Hausdorff} space. Then the \underline{\textbf{dual space} $\cC(X)^{*}$} is \textbf{isometric isomorphism} to $\cM(X)$. 
\end{corollary}

\item \begin{definition}
Given $\cM(X) \simeq (\cC_{0}(X))^{*}$, we define subspaces of $\cM$:
\begin{align*}
\cM_{+}(X) &= \set{I \in  \cM(X): I \text{\emph{ is a positive linear functional}}},\\
\cM_{+,1}(X) &= \set{I \in  \cM(X): \norm{I}{} = 1}.
\end{align*} Thus $\cM_{+}(X)$ is \emph{identified} with \emph{\textbf{the space of all positive Randon measures on $X$}}.
\end{definition}

\item \begin{remark} (\emph{\textbf{Isometric Embedding of $L^1(\mu)$ into $M(X)$}})\\
Let $\mu$ be a fixed \emph{positive Radon measure} on $X$. If $f \in L^1(\mu)$, \emph{the complex measure}
\begin{align*}
d\nu_f = f d\mu 
\end{align*}
is easily seen to be \emph{\textbf{Radon}}, and $\norm{\nu}{} = \int \abs{f}d\mu = \norm{f}{1}$.
Thus $f \mapsto \nu_f$ is an \emph{\textbf{isometric embedding}} of $L^1(\mu)$ into $M(X)$ whose range consists precisely of those $\nu \in \cM(X)$ such that $\nu \ll \mu$. 
\end{remark}

\item \begin{remark} (\emph{\textbf{Two Perspectives of Measures}})\\
For \emph{regular Borel measure} $\mu$ or in general, \emph{Radon measures} on \emph{\textbf{locally compact}} space $X$, there are two perspectives:
\begin{enumerate}
\item \emph{\textbf{Nonegative set function on the $\sigma$-algebra $\srA$}}: as a \emph{\textbf{measure of the volume}} of a \emph{subset} in $X$;
\item \emph{\textbf{Positive linear functional on $\cC_0(X)$}}: as a \emph{\textbf{integral} of compactly supported continuous functions with respect to \textbf{given measure}}.
\end{enumerate}
In some cases, it is important to think of \emph{\textbf{measures}} not merely as individual  objects but instead as \emph{elements of $(\cC_{0}(X))^{*}$}, so that we can employ \emph{geometric} ideas. 
\end{remark}

\item \begin{proposition} (\textbf{Criterion for Weak$^{*}$ (Vague) Convergence on $\cM(X)$}) \citep{folland2013real}\\
Suppose $\mu_1, \mu_2, \ldots \in \cM(\bR)$, and let $F_{n}(x) = \mu_n((-\infty, x])$ and $F(x) = \mu((-\infty, x])$.
\begin{enumerate}
\item  If $\sup_n\norm{\mu_n}{} < \infty$ and $F_n(x) \rightarrow F(x)$ for \textbf{every x} at which $F$ is \textbf{continuous},
then $\mu_n \rightarrow \mu$ \textbf{vaguely}.
\item If $\mu_n \rightarrow \mu$ \textbf{vaguely}, then $\sup_n\norm{\mu_n}{} < \infty$. If, in addition, the $\mu_n$’s are \textbf{positive},
then $F_n(x) \rightarrow F(x)$ at \textbf{every} $x$ at which $F$ is \textbf{continuous}.
\end{enumerate}
\end{proposition}

\item Finally, we tends to the geometrical properties of subspace of $\cM(X)$
\begin{definition} (\emph{\textbf{Convex Cone}})\\
A set $A$ in a vector space $Y$ is called \emph{\textbf{convex}} if $x$ and $y \in A$  and $0 \le t \le 1$ implies $tx + (1 - t)y \in A$. Thus $A$ is \emph{convex} if the \emph{\textbf{line segment}}  between $x$ and $y$ is in $A$ whenever $x$ and $y$ are in $A$. $A$ is called a \emph{\textbf{cone}} if $x \in A$ implies $tx \in A$ for all $t > 0$. If $A$ is \emph{convex} and a \emph{cone}, it is called  a \emph{\textbf{convex cone}}. 
\end{definition}

\item \begin{proposition} (\textbf{Geometry of $\cM_{+}(X)$ and $\cM_{+,1}(X)$}) \citep{reed1980methods}\\
Let $X$ be a \textbf{compact Hausdorff} space. Then $\cM_{+,1}(X)$ is \textbf{convex} and $\cM_{+}(X)$ is a \textbf{convex cone}. 
\end{proposition}
\end{itemize}

\subsection{Adjoints of Bounded Operator}
\begin{itemize}
\item \begin{definition} (\emph{\textbf{Banach Space Adjoint}})\\
Let $X$ and $Y$ be \emph{Banach spaces},  $T$ \emph{a \textbf{bounded linear operator}} from $X$ to $Y$. The \underline{\emph{\textbf{Banach space adjoint of $T$}}}, denoted by $T^{'}$, is \emph{the \textbf{bounded linear operator} from $Y^{*}$ to $X^{*}$} defined by 
\begin{align*}
\paren{T'f}(x) &= f\paren{Tx}
\end{align*} for all $f \in Y^{*}$, $x \in X$. 
\end{definition}

\item \begin{example} (\emph{\textbf{Adjoint of Right Shift Operator}})\\
Let $X = \ell^1 = Y$ and let $Τ$ be \emph{\textbf{the right shift operator}}
\begin{align*}
T\paren{\xi_1, \xi_2, \ldots} &= \paren{0, \xi_1, \xi_2, \ldots}
\end{align*}
Then $T':  \ell^{\infty} \rightarrow \ell^{\infty}$ is \emph{\textbf{the left shift  operator}} 
\begin{align*}
T'\paren{\xi_1, \xi_2, \ldots} &= \paren{\xi_2, \xi_3, \ldots}.
\end{align*}
\end{example}

\item \begin{proposition} (\textbf{Isomorphism between Bounded Operator and its Adjoint}). \citep{reed1980methods}\\
Let $X$ and $Y$ be \textbf{Banach spaces}. The map $T \rightarrow T'$ is an \textbf{isometric isomorphism} of  $\cL(X, Y)$ into $\cL(Y^{*}, X^{*})$. 
\end{proposition}

\item \begin{remark} (\emph{\textbf{Hilbert Space Adjoint}})\\
Let $\cL(\cH):= \cL(\cH, \cH)$ be \emph{the space of bounded linear operators} on $\cH$. \emph{The Banach space adjoint} of $T^{*}$  then a mapping of  $\cH^{*}$ to $\cH^{*}$. Let $C:  \cH \rightarrow  \cH^{*}$ be the map which assigns to each $y \in \cH$, the bounded linear functional $\inn{y}{\cdot}$ in $\cH^{*}$.  $C$ is a \underline{\emph{\textbf{conjugate linear isometry}}} which is \emph{\textbf{surjective}} by \emph{the Riesz Representation theorem} (so it is \emph{\textbf{unitary}}). Now define a map $T^{*}: \cH \rightarrow \cH$ by 
\begin{align*}
T^{*} &= C^{-1}T' C
\end{align*}
Then $T^{*}$ satisfies 
\begin{align*}
\inn{x}{Ty} = (Cx)(Ty) = (T'Cx)(y) = \inn{C^{-1}T'Cx}{y} = \inn{T^{*}x}{y},
\end{align*}
$T^*$ is called  \underline{\emph{\textbf{the Hilbert space adjoint of $T$}}}, but usually we will just call it the  adjoint and let the $T^*$ distinguish it from $T'$. Notice that the map $T \rightarrow T^*$ is \emph{\textbf{conjugate linear}}, that is, $\alpha T  \rightarrow \bar{\alpha} T^*$. This is because $C$ is conjugate linear. 
\end{remark}

\item \begin{proposition} \citep{reed1980methods}\\
The map $T \rightarrow T^*$ is always \textbf{continuous} in the \textbf{weak} and \textbf{uniform operator topologies} but is only continuous in the \textbf{strong operator topology} if $\cH$ is \textbf{finite dimensional}. 
\end{proposition}
\end{itemize}

\section{Compactness in Banach Space}
\begin{remark} (\emph{\textbf{Compactness in Function Space}})\\
The importance of \emph{\textbf{compactness}} in analysis is well-known, and the fact tha \emph{closed bounded sets} are \emph{compact} in \emph{finite dimensional spaces} lies at the heart of much of the analysis on these spaces. \emph{\textbf{Unfortunately}}, as we have seen, this is \emph{not true} in \emph{infinite dimensional spaces}.

There are \emph{\textbf{two main compactness results}} in \emph{function space}:
\begin{enumerate}
\item The \underline{\textbf{\emph{Ascoli's theorem}}}: Let $X$ be a \emph{compact Hausdorff space}; let $d$ denote either the square metric or the euclidean metric on $\bR^n$; give $\cC(X, \bR^n)$ the corresponding \textbf{\emph{uniform topology}}. \emph{A subspace $\srF$ of $\cC(X, \bR^n)$ is \textbf{compact} if and only if it is \underline{\textbf{closed}, \textbf{bounded}} under the \underline{\textbf{sup metric} $\rho$}, and \underline{\textbf{equicontinuous}} under $d$}.

\item The \underline{\textbf{\emph{Banach-Alaoglu theorem}}}: Let $X$ be a \emph{Banach space}. \emph{The \textbf{unit ball} in $X^{*}$, $\{f \in X^{*}: \norm{f}{}\le 1\}$ is \textbf{compact} in the \underline{\textbf{weak$^{*}$ topology}}}.

In this section we will show that a \emph{partial analogue} of this result can be obtained in \emph{\textbf{infinite dimensions}} if we adopt a \emph{weaker definition of the convergence} of a sequence than the usual definition.
\end{enumerate}
\end{remark}
\subsection{Strong and Weak Convergence}
\begin{itemize}
\item \begin{definition} (\emph{\textbf{Strong Convergence}}). \citep{kreyszig1989introductory}\\
A sequence $(x_n)$ in a normed space $X$ is said to be \underline{\emph{\textbf{strongly convergent}}} (or \emph{\textbf{convergent \underline{in the norm}}}) if
there is  an $x \in X$ such that
\begin{align*}
\lim\limits_{n\rightarrow \infty} \norm{x_n - x}{} = 0.
\end{align*}
This is written $\lim\limits_{n\rightarrow \infty}x_n = x$ or simply $x_n \rightarrow x$ is called \emph{the \textbf{strong limit} of $(x_n)$}, and we say that $(x_n)$ \emph{converges \textbf{strongly} to $x$}. 
\end{definition}

\item \begin{definition} (\emph{\textbf{Weak Convergence}}). \citep{kreyszig1989introductory}\\
A sequence $(x_n)$ in a normed space $X$ is said to be \underline{\emph{\textbf{weakly convergent}}} if there is an $x \in X$ such that \emph{for \textbf{every}} $f \in X^{*}$,
\begin{align*}
\lim\limits_{n\rightarrow \infty} f(x_n) = f(x).
\end{align*}
This is written $x_n \stackrel{w}{\rightarrow} x$ or $x_n \rightharpoonup x$. The element $x$ is called \emph{\textbf{the weak limit} of $(x_n)$}, and we say
that \underline{$(x_n)$ \emph{\textbf{converges weakly} to $x$.}}
\end{definition}

\item \begin{remark}
For weak convergence, we see it as convergence of \emph{real numbers} $s_n = f(x_n)$ in $\bR$.
\end{remark}

\item \begin{remark} (\emph{\textbf{Weak Convergence Analysis is Common}})\\
\emph{\textbf{Weak convergence}} has various applications throughout analysis (for instance, in the \emph{calculus of variations}, \emph{the general theory of
differential equations} and \emph{probability theory}). 

The concept illustrates \emph{\textbf{a basic principle of functional analysis}}, namely, the fact that \underline{\textbf{\emph{the investigation of spaces is often related to that of their dual spaces}}}, i.e. \emph{probing a variable by using a test functional}.
\end{remark}

\item \begin{remark}
In \emph{Hilbert space} $\cH$, we say $x_n \stackrel{w}{\rightarrow} x$ if there exists an $x\in \cH$ such that for all $y \in \cH$
\begin{align*}
\lim\limits_{n\rightarrow \infty} \inn{x_n}{y} = \inn{x}{y}.
\end{align*} Note that given a set of orthonormal basis $(e_n)$, we have $f(e_n):= \inn{e_n}{y}$ and from Bessel inequality
\begin{align*}
\sum_{n=1}^{\infty}\abs{\inn{e_n}{y}}^2 \le \norm{y}{}^2 < \infty \\
\Rightarrow \lim\limits_{n\rightarrow \infty}\abs{\inn{e_n}{y}} \rightarrow 0 \\
\Rightarrow e_n \stackrel{w}{\rightarrow} 0.
\end{align*} But $\norm{e_n - e_{m}}{} \not\rightarrow 0$, $(e_n)$ does not converge in norm (strongly).
\end{remark}

\item \begin{lemma}  (\textbf{Weak Convergence}).\\
Let $(x_n)$ be a \textbf{weakly convergent} sequence in a normed space $X$, say, $x_n \stackrel{w}{\rightarrow} x$. Then:
\begin{enumerate}
\item The weak limit $x$ of $(x_n)$ is \textbf{unique}.
\item Every \textbf{subsequence} of $(x_n)$ converges weakly to $x$.
\item The sequence $(\norm{x_n}{})$ is \textbf{bounded}.
\end{enumerate}
\end{lemma}

\item \begin{proposition} (\textbf{Strong and Weak Convergence}).  \citep{kreyszig1989introductory}\\
Let $(x_n)$ be a sequence in a normed space $X$. Then:
\begin{enumerate}
\item \textbf{Strong convergence} implies \textbf{weak convergence} with the same limit.
\item The converse of (1) is \textbf{not} generally true.
\item If $\text{dim }X < \infty$, then \textbf{weak convergence} implies\textbf{ strong convergence}.
\end{enumerate}
\end{proposition}

\item \begin{remark}
From above, we see that in \emph{\textbf{finite dimensional normed spaces}} the distinction between \emph{\textbf{strong}} and \emph{\textbf{weak convergence}} disappears completely. 
\end{remark}

\end{itemize}
\subsection{Weak Topology}
\begin{itemize}
\item \begin{remark}
The weak convergence, $x_n \stackrel{w}{\rightarrow} x$, can be considered as \emph{convergence of net $\set{x_n}_{n=1}^{\infty}$ in the \textbf{weak topology}}.
\end{remark}

\item \begin{definition} (\textbf{\emph{Weak Topology on a Set $S$}})  \citep{reed1980methods}\\
Let $\cF$ be a family of functions from a set $S$ to a topological vector space $(X, \srT)$. The \emph{\textbf{$\cF$-weak} (or simply \textbf{weak}) \textbf{topology}} on $S$ is the weakest topology for which \emph{\textbf{all the functions} $f \in \cF$ are \textbf{continuous}}.
\end{definition}

\item \begin{remark} (\emph{\textbf{Construction of Weak Topology}}) \citep{reed1980methods} \\
To construct a \emph{\textbf{$\cF$-weak topology}} on $S$, we take the family of \emph{all \underline{\textbf{finite intersections}} of sets} of the form $f^{-1}(U)$ where $f \in \cF$ and $U \in \srT$. The collections of these finite intersections of sets \emph{form \textbf{a basis} of \textbf{the $\cF$-weak topology}}.

In other word, \emph{\textbf{the subbasis}} for the \emph{$\cF$-weak topology} on $S$ is of form 
\begin{align*}
\srS = \set{f^{-1}(U):  f \in \srF, \text{ and }  U \in \srT}
\end{align*}
And the basis of $\srT$
\begin{align*}
\srB &= \set{f_1^{-1}(U_1) \xdotx{\cap} f_k^{-1}(U_k): f_1 \xdotx{,} f_k \in \srF, \;\; U_1 \xdotx{,} U_k \in \srT, \;1 \le k < \infty}\\
B \in \srB \Rightarrow B &= \set{x: f_1(x) \in U_1 \xdotx{,} f_k(x) \in U_k},\; 1 \le k < \infty \\
&= \set{x: (f_1(x) \xdotx{,} f_k(x)) \in U}.
\end{align*} The basis element is called a \underline{\emph{\textbf{$k$-dimensional cylinder set}}}.
\end{remark}

\item \begin{remark}
Given a topology  on $Y$ and a family of functions in $Y^X= \set{f: X\rightarrow Y}$, $\srF$-weak topolgy is \emph{\textbf{a natural topology}} on $X$ without additional information. 

A product topology on $Y^{\omega}$ can be seen as a $\srF$-weak topology when $\srF = \{\pi_{\alpha}: \prod_{i}Y_i \rightarrow Y_{\alpha} \}$.
\end{remark}

\item \begin{remark}
A set $S$ equipped with \emph{$\cF$-weak topology} \emph{\textbf{has little knowledge on itself besides the output of functions}} $f \in \cF$ from a family $\cF$. The induced topology through a family of functions thus does not tell much besides  the behavior of its output. 

For instance, $S$ is the space of hidden states, $\cF = \set{f_1 \xdotx{,} f_n} \subset 2^{S}$ is a series of binary statistical tests, the weak topology on $S$ \emph{partition the domain according to the output of each test}. 
\end{remark}

\item \begin{remark}
By construction,  the \emph{\textbf{neighborhood base}} of each point $x \in S$ under the $\cF$-weak topology is contained in the pre-images  $\{f_{n}^{-1}(U_{n})\}$ for \emph{\textbf{finitely many}} of $(f_n) \in \cF$.
\end{remark}


\item \begin{definition} (\textbf{\emph{Weak Topology on Banach Space}})\\
Let $X$ be a \emph{\textbf{Banach space}} with dual space $X^{*}$. The \underline{\emph{\textbf{weak topology}} on $X$} is \emph{the weakest topology} on $X$ so that \emph{$f(x)$ is \textbf{continuous} \textbf{for all} $f \in X^{*}$}.
\end{definition}

\item \begin{remark}
For infinite dimensional Banach spaces, \emph{\textbf{the weak topology does not arise from a metric}}. This is one of the main reasons we have introduced topological spaces.
\end{remark}

\item \begin{remark}
Thus a \emph{\textbf{neighborhood} base at zero} for \emph{\textbf{the weak topology}} is given by the sets
of the form
\begin{align*}
N(f_1 \xdotx{,} f_n; \epsilon) &= \set{x: \abs{f_{j}(x)} <\epsilon; \; j= 1 \xdotx{,} n}
\end{align*}
that is, neighborhoods of zero contain \emph{\textbf{cylinders} with \textbf{finite-dimensional} open bases}. A net $\{x_{\alpha}\}$ converges \emph{weakly} to $x$, written $x_\alpha \stackrel{w}{\rightarrow} x$, if and only if $f(x_{\alpha}) \rightarrow f(x)$ for all $f \in X^{*}$.
\end{remark}



\item \begin{proposition} \citep{reed1980methods}
\begin{enumerate}
\item The weak topology is \textbf{weaker} than \textbf{the norm topology}, that is, every weakly open set is norm open.
\item Every \textbf{weakly convergent} sequence is \textbf{norm bounded}.
\item The weak topology is a \textbf{Hausdorff} topology.
\end{enumerate}
\end{proposition}


\item \begin{proposition} (\textbf{Weak Topology on Hilbert Space}) \citep{reed1980methods}\\
Let $\cH$ be a \textbf{Hilbert space}. Let $\set{\varphi_{\alpha}}_{\alpha \in I}$ be an \textbf{orthonormal basis} for $\cH$. Given a sequence $\psi_{n} \in \cH$, let 
\begin{align*}
\psi_{n}^{(\alpha)}= \inn{\psi_n}{\varphi_{\alpha}}
\end{align*}
be the coordinates of $\psi_n$. Then $\psi_n \rightarrow \psi$  in the \textbf{weak topology} (or  $\psi_n \stackrel{w}{\rightarrow} \psi$) \textbf{if and only if}
\begin{enumerate}
\item $\psi_{n}^{(\alpha)} \rightarrow \psi^{(\alpha)}$ for each $\alpha$; and
\item $\norm{\psi_n}{}$ is \textbf{bounded}.
\end{enumerate}
\end{proposition}
\begin{proof}
Suppose $\psi_n \stackrel{w}{\rightarrow} \psi$; then (1) follows by  definition and (2) comes from the fact that \emph{every weakly convergent sequence is norm bounded}. 

On the other hand, let (1) and (2) hold and let $\cF \subset \cH$ be the subspace of \emph{finite linear combinations} of the $\varphi_{\alpha}$. By (1), 
$\inn{\psi_n}{\varphi_{\alpha}} \rightarrow \inn{\psi}{\varphi_{\alpha}}$ if $\varphi \in \cF$. Using the fact that $\cF$ is dense, (2), and an $\epsilon/3$ argument, the weak convergence follows. \qed
\end{proof}

\item \begin{proposition} (\textbf{Weak Topology of $\cC(X)$ on Compact Hausdorff Space}) \citep{reed1980methods}\\
Let $X$ be a \textbf{compact Hausdorff} space and consider the \textbf{weak topology on} $\cC(X)$ (i.e. $\cC(X, \bR)$). Let $\{f_n\}$ be a sequence in $\cC(X)$. Then $f_n \rightarrow f$  in the \textbf{weak topology} (or  $f_n \stackrel{w}{\rightarrow} f$) \textbf{if and only if}
\begin{enumerate}
\item $f_{n}(x) \rightarrow f(x)$ for each $x \in X$; and
\item $\norm{f_n}{}$ is \textbf{bounded}.
\end{enumerate}
\end{proposition}
\begin{proof}
For if  $f_n \stackrel{w}{\rightarrow} f$, then (1) holds since $f \rightarrow f(x)$ is an element of $\cC(X)^{*}$ and (2) 
comes from  the fact that \emph{every weakly convergent sequence is norm bounded}.  

On the other hand, if (1) and (2) hold,  then 
\begin{align*}
\abs{f_n(x)} \le \sup_{n}\norm{f_n}{\infty}
\end{align*}
which is $L^1$ with respect to any \emph{Baire measure} $\mu$. Thus, by the \emph{dominated convergence theorem}, for any $\mu \in \cM_{+}(X)$, $\int f_n d\mu \rightarrow \int f d\mu$. Since any $\lambda \in \cM(X) = \cC(X)^{*}$ is a \emph{finite linear combination} of measures in $\cM_{+}(X)$, we conclude that $f_n \rightarrow f$ weakly. \qed
\end{proof}


\item \begin{proposition} (\textbf{Banach Space Weak Continuity $=$ Norm Continuity}) \citep{reed1980methods}\\
A linear functional $f$ on a \textbf{Banach space} is \textbf{weakly continuous} \textbf{if and only if} it is \textbf{norm continuous}.
\end{proposition}

\end{itemize}

\subsection{Weak$^{*}$  Topology}
\begin{itemize}
\item \begin{definition}  (\textbf{\emph{Weak$^{*}$ Topology on Banach Space}})\\
Let $X$ be a \emph{normed vector space} and $X^{*}$ be its dual space. The \underline{\emph{\textbf{weak$^{*}$ topology}} on $X^{*}$} is \emph{the weakest topology} on $X^{*}$ so that \emph{$f(x)$ is \textbf{continuous} \textbf{for all} $x \in X$}.
\end{definition}

\item \begin{remark}
\emph{\textbf{The weak$^{*}$ topology}} can be considered as a topology induced by $x \in X$ on dual space $X^{*}$, i.e. a topology on functional space on $X$ induced by point in $X$. 

In fact, \underline{\emph{\textbf{the weak$^{*}$ topology}} is \emph{the \textbf{topology} of \textbf{pointwise convergence}}}:
\begin{align*}
f_{\alpha} \rightarrow  f \quad \Leftrightarrow \quad f_{\alpha}(x) \rightarrow  f(x) \text{ for all }x \in X.
\end{align*} Moreover, \underline{\emph{\textbf{the weak$^{*}$ topology}} is \emph{the \textbf{product topology} on product space $\bR^X$}}.
\end{remark}

\item \begin{definition} (\emph{\textbf{$Y$-Weak Topology  $\sigma(X, Y)$}})\\
Let $X$ be a \emph{vector space} and let $Y$ be a \emph{family of \textbf{linear functionals}} on $X$ which \emph{\textbf{separates points}} of $X$. That is, for any $x_1 \neq x_2$ in $X$, there exists a $f \in Y$ so that $f(x_1) \neq f(x_2)$. Then \underline{\emph{\textbf{the $Y$-weak topology on $X$}}, written $\sigma(X, Y)$}, is \emph{the weakest topology} on $X$ for which all the \emph{functionals} in $Y$ are \emph{continuous}. 
\end{definition}

\item \begin{remark}
\emph{$Y$-weak topology} $\sigma(X, Y)$ is \emph{the $\srF$-weak topology} when \emph{domain} of  $\srF$ is a \emph{vector space} and $\srF$ is a family of \emph{linear functionals}.
\end{remark}

\item \begin{remark}
Because $Y$ is assumed to \emph{separate points}, $\sigma(X, Y)$ is \emph{a \textbf{Hausdorff topology}} on $X$.  Note that
\begin{enumerate}
\item \emph{the weak topology} on $X$ is \emph{the $\sigma(X, X^{*})$ topology}
\item \emph{the weak$^{*}$ topology} on $X^{*}$ is \emph{the $\sigma(X^{*}, X)$ topology}
\end{enumerate}

The \emph{$\sigma(X, Y)$ topology} \emph{depends} only on \emph{\textbf{the vector space generated by }$Y$} so we henceforth suppose  that  \emph{$Y$ is a vector space}. 
\end{remark}


\item \begin{remark}
Notice that \emph{\textbf{the weak$^{*}$ topology} is even \textbf{weaker} than \textbf{the weak topology}}.
\begin{align*}
 \text{\emph{\textbf{the norm topology}}} \; \subset \; \text{\emph{\textbf{the weak topology}}} \; \subseteq \; \text{\emph{\textbf{the weak$^{*}$ topology}}}
\end{align*}
\end{remark}



\item \begin{example} (\emph{\textbf{Weak$^{*}$ Topology on $\cM(X)$}})\\
\emph{The weak$^{*}$ topology} on $\cM(X)$, $X$ a \emph{\textbf{compact Hausdorff}} space, is often called \emph{\textbf{the vague topology}}. Note that $\mu_n \stackrel{w^{*}}{\rightarrow} \mu$ if and only if $\int f d\mu_n \rightarrow \int f d\mu$ for all $f \in \cC_0(X)$.

It can be shown that \emph{the linear combinations of point masses} are \emph{\textbf{weak$^{*}$ dense}} in  $\cM(X)$. That is, for given $\mu \in \cM(X)$,  $f_1 \xdotx{,} f_n \in \cC(X)$ and $\epsilon > 0$, that we can find $\alpha_1 \xdotx{,} \alpha_m \in \bC$ and $x_1 \xdotx{,} x_m \in X$ so that 
\begin{align*}
\abs{\mu(f_i) - \sum_{j=1}^{m}\alpha_j f_{i}(x_j) } < \epsilon, \quad \forall\, i=1 \xdotx{,} n,
\end{align*} i.e. $\sum_{j=1}^{m}\alpha_j \delta_{x_j} \rightarrow \mu$ where $\delta_x(f) = f(x)$ is the \emph{\textbf{evaluation map}} and $\delta_x(\cdot) \mapsto \delta_x$ is identified with the \emph{\textbf{point mass}}.
\end{example}


\item \begin{remark}
As one might expect, \underline{$X$ is \emph{\textbf{reflexive}}} if and only if the \emph{\textbf{weak}} and \emph{\textbf{weak$^{*}$ topologies}} \underline{\emph{\textbf{coincide}}}, and many \emph{characterizations} of \emph{reflexivity} depend on relations involving the \emph{weak} and \emph{weak$^{*}$ topologies}.
\end{remark}

\item \begin{proposition} (\textbf{$\sigma(X, Y)$ Topology $=$ Pointwise Convergence Topology on $X$}) \citep{reed1980methods} \\
The $\sigma(X, Y)$-\textbf{continuous} linear functionals on $X$ are \textbf{precisely} $Y$, in particular the only \textbf{weak$^{*}$ continuous functionals} on $X^{*}$ are  the \textbf{elements} of $X$. 
\end{proposition}

\item \begin{theorem} (\textbf{The Banach-Alaoglu Theorem}) \citep{reed1980methods}\\
 Let $X^{*}$ be the dual of some Banach space, $X$. Then \textbf{the unit ball} in $X^{*}$, $\set{f \in X^{*}: \norm{f}{}\le 1}$ is \underline{\textbf{compact}} in the \underline{\textbf{weak$^{*}$ topology}}.
\end{theorem}

\item \begin{corollary} (\textbf{The Banach-Alaoglu Theorem, Sequential Version}) \citep{rynne2007linear}\\
If $X$ is \textbf{separable} and $\set{f_n}$ is a \textbf{bounded} sequence in $X^{*}$, then  $\set{f_n}$ has a \textbf{\underline{weak$^{*}$ convergent} \underline{subsequence}}.
\end{corollary}

\item \begin{theorem}(\textbf{Kakutani's Theorem}) \citep{rynne2007linear}\\
$X$ is \textbf{reflexive} Banach space \underline{\textbf{if and only if}} \textbf{the unit ball} in $X$, $\set{x \in X: \norm{x}{}\le 1}$ is \underline{\textbf{compact}} in the \underline{\textbf{weak topology}}.
\end{theorem}

\item \begin{corollary} \citep{rynne2007linear}\\
If $X$ is \textbf{reflexive} Banach space and $\{x_n\}$ is a \textbf{bounded} sequence in $X$, then $\{x_n\}$ has a \textbf{\underline{weakly convergent subsequence}}.
\end{corollary}

\item \begin{corollary}\citep{rynne2007linear}\\
If $X$ is \textbf{reflexive} Banach space and $M \subseteq X$ is \underline{\textbf{bounded}, \textbf{closed} and \textbf{convex}}, then any sequence
in $M$ has a \textbf{subsequence} which is \textbf{weakly convergent} to an element of $M$.
\end{corollary}

\item \begin{exercise}\citep{rynne2007linear}\\
Suppose that $X$ is \textbf{reflexive} Banach space, $M$ is a \underline{\textbf{closed}, \textbf{convex} subset} of $X$, and
$y \in X \setminus M$.  Show that there is a point $y_M \in M$ such that
\begin{align*}
y - y_M = \inf\set{y- x: x \in M}.
\end{align*}
Show that this result is \textbf{not true} if the assumption that $M$ is \textbf{convex} is omitted.
\end{exercise}

\item \begin{example} (\emph{\textbf{Convergence in Distribution}})\\
\underline{\emph{\textbf{Convergence in distribution}}} is also called \underline{\emph{\textbf{weak convergence}}} in probability theory \citep{folland2013real}. In functional analysis, however, \emph{\textbf{weak convergence}} is actually reserved for a different mode of convergence, while \emph{\textbf{the convergence in distribution}} is \emph{\textbf{the weak$^*$ convergence}} on $\cM(X)$.

In general, it is actually \emph{\textbf{not} a mode of \textbf{convergence of functions} $f_n$ \textbf{itself}} but instead is \underline{\emph{the \textbf{convergence} of \textbf{bounded linear functionals}}} $\int f d\mu_n$. Equivalently, it is \emph{the \textbf{convergence of measures} $F_n$ on $\srB(\bR)$}.

\begin{align*}
 \text{weak convergence} && \int f_n d\mu \rightarrow \int f d\mu, \quad \forall \mu \in \cM(X), \\
\text{convergence in distribution}  &&  \int f d\mu_n \rightarrow \int f d\mu, \quad \forall f \in \cC_{0}(X)
\end{align*}

\begin{definition} (\emph{\textbf{Cumulative Distribution Function}}) \citep{van2000asymptotic} \\
Let $(\Omega,\srF, \mu)$ be a probability space. Given any real-valued measurable function $\xi : \Omega \rightarrow \bR$, we define the \emph{\textbf{cumulative distribution function}} $F : \bR \rightarrow [0, \infty]$ of $\xi$ to be the function
\begin{align*}
F_{\xi}(\lambda) :=  \mu\paren{\set{x \in  X : \xi(x) \le \lambda}} = \int_{X} \ind{\xi(x) \le \lambda}d\mu(x).
\end{align*}
\end{definition}

\begin{definition}  (\emph{\textbf{Converge in Distribution}}) \citep{van2000asymptotic}\\
Let $\xi_n : \Omega \rightarrow \bR$ be a sequence of real-valued \emph{measurable functions}, and $\xi: \Omega \rightarrow \bR$ be another measurable function. We say that $\xi_n$ \underline{\emph{\textbf{converges in distribution}}} to $\xi$ if \emph{the cumulative distribution function} $F_n(\lambda)$ of $\xi_n$
\underline{\emph{\textbf{converges pointwise}}} to the cumulative distribution function $F(\lambda)$ of $\xi$ at all $\lambda \in  \bR$ for which $F$ is continuous. Denoted as $\xi_{n}\stackrel{F}{\rightarrow} \xi$ or \underline{$\xi_{n}\stackrel{d}{\rightarrow} \xi$} or \underline{$\xi_n \rightsquigarrow \xi$}. 
\begin{align*}
\xi_{n}\stackrel{d}{\rightarrow} \xi \; \Leftrightarrow \; F_n(\lambda) \rightarrow F(\lambda), \text{ for all }\lambda \in \bR
\end{align*}
\end{definition}

\begin{theorem} (\textbf{The Portmanteau Theorem}).  \citep{van2000asymptotic}\\
 The following statements are equivalent.
 \begin{enumerate}
 \item $X_n \rightsquigarrow X$.
 \item $\E{}{h(X_n)} \rightarrow \E{}{h(X)}$ for all \textbf{continuous functions} $h: \bR^d \rightarrow \bR$ that are non-zero only on a \textbf{closed} and \textbf{bounded} set.
 \item $\E{}{h(X_n)} \rightarrow \E{}{h(X)}$ for all \textbf{bounded continuous functions} $h: \bR^d \rightarrow \bR$.
 \item $\E{}{h(X_n)} \rightarrow \E{}{h(X)}$ for all \textbf{bounded measurable functions} $h: \bR^d \rightarrow \bR$ for which $\bP(X \in \{x: h\text{ is continuous at }x\})=1$.
 \end{enumerate}
\end{theorem}

We can reformulate the definition of \emph{convergence in distribution} as below:
\begin{definition} \citep{wellner2013weak}\\
Let $(\cX, d)$ be a \emph{metric space}, and $(\cX, \srB)$ be \emph{a measurable space}, where $\srB$ is \emph{\textbf{the Borel $\sigma$-field on $\cX$}}, the smallest $\sigma$-field containing \emph{all the open balls} (as the basis of \emph{metric topology} on $\cX$). Let $\{\cP_n \}$ and $\cP$ be \emph{\textbf{Borel probability measures}} on $(\cX, \srB)$.

Then the sequence $\cP_n$ \underline{\emph{\textbf{converges in distribution}}} to $\cP$, which we write as $\cP_n \rightsquigarrow \cP$, if and only if
\begin{align*}
\int_{\Omega} f d\cP_n \rightarrow \int_{\Omega} f d\cP, \quad \text{ for all } f \in \cC_{b}(\cX).
\end{align*}
Here $\cC_{b}(\cX)$ denotes the set of \emph{all \textbf{bounded}, \textbf{continuous}, real functions on $\cX$}.
\end{definition} 
We can see that \underline{\emph{\textbf{the convergence in distribution}} is actually \emph{\textbf{a weak$^{*}$ convergence}}}. That is, it is \emph{\textbf{the weak convergence}} of  \emph{\textbf{bounded linear functionals}} $I_{\cP_n} \stackrel{w^{*}}{\rightarrow} I_{\cP}$ on \emph{the space of all probability measures} $\cP(\cX) \simeq (\cC_{b}(\cX))^{*}$ on $(\cX, \srB)$ where 
\begin{align*}
I_{\cP}: f \mapsto \int_{\Omega} f d\cP.
\end{align*} Note that the $I_{\cP_n} \stackrel{w^{*}}{\rightarrow} I_{\cP}$ is equivalent to $I_{\cP_n}(f) \rightarrow I_{\cP}(f)$ \emph{for all $f \in  \cC_{b}(\cX)$}.

%It also defines \emph{\textbf{a weak topology}} on $\cP(\Omega)$, the set of \emph{all probability measures} defined on measurable space $(\Omega, \srF)$. 
%Here $P_n \stackrel{w}{\rightarrow} P$ if and only if the cumulative distribution function $F_n(\lambda) \rightarrow F(\lambda)$ for all points $\lambda \in \bR$ at which $F$ is continuous.
\end{example}
\end{itemize}

\section{Fundamental Theorems}
\subsection{The Hahn-Banach Theorem}
\subsubsection{Extension Form of The Hahn-Banach Theorem}
\begin{itemize}
\item  \begin{remark}
In dealing with Banach spaces, one often needs to \emph{construct \textbf{linear functionals}} with \emph{\textbf{certain properties}}. This is usually done in two steps:
\begin{enumerate}
\item one \emph{\textbf{defines the linear functional}} on \emph{a \textbf{subspace} of the Banach space} where it is easy to verify the desired properties;
\item one appeals to (or proves) a general theorem which says that \emph{any such functional} can be \emph{\textbf{extended to the whole space}} while \emph{\textbf{retaining the desired properties}}.
\end{enumerate}
One of the basic tools of the second step is \emph{the Hahn-Banach theorem}.
\end{remark}

\item \begin{definition} (\emph{\textbf{Sublinear Functional}})\\
If $X$ is a vector space, a \underline{\emph{\textbf{sublinear functional}}} on $X$ is a map $p: X \rightarrow \bR$ such that 
\begin{enumerate}
\item (\emph{\textbf{Homogeneity}}): $p(\lambda x) = \lambda p(x)$ for all $\lambda \ge 0$ and $x \in X$;
\item (\emph{\textbf{Sublinearity}}): $p(x + y) \le p(x) + p(y)$,
\end{enumerate}
\end{definition}

\item \begin{example}
Every \emph{\textbf{semi-norm}} is a \emph{sublinear functional}. If $p$ is a semi-norm, then the condition $f \le p$ is equivalent to $\abs{f} \le p$.
\end{example}

\item \begin{theorem} (\underline{\textbf{The Hahn-Banach Theorem, Extension Form}}) \citep{kreyszig1989introductory, reed1980methods, luenberger1997optimization, folland2013real}\\
Let $X$ be a real normed linear space and $p$ a \underline{\textbf{sublinear functional}} on $X$. Let $f$ be a \textbf{linear functional} defined on a \underline{\textbf{subspace}} $M$ of X satisfying $f(x) \le p(x)$ for all $x \in M$. Then there exists a \textbf{linear functional $F$} on $X$ such that $F(x) \le p(x)$  for all $x \in X$ and $F|_{M} = f$. ($F$ is called an \underline{\textbf{extension}} of $f$.) 
\end{theorem}


\item \begin{theorem} (\textbf{The Complex Hahn-Banach Theorem, Extension Form}) \citep{kreyszig1989introductory, reed1980methods, luenberger1997optimization, folland2013real}\\
Let $X$ be a complex normed linear space and $p$ a \underline{\textbf{semi-norm}} on $X$. Let $f$ be a \textbf{complex linear functional} defined on a \underline{\textbf{subspace}} $M$ of X satisfying $\abs{f(x)} \le \abs{p(x)}$ for all $x \in M$. Then there exists a \textbf{complex linear functional $F$} on $X$ such that $\abs{F(x)} \le \abs{p(x)}$  for all $x \in X$ and $F|_{M} = f$. ($F$ is called an \underline{\textbf{extension}} of $f$.) 
\end{theorem}

\item \begin{corollary} (\underline{\textbf{The Existance of Minimum Norm Extension}})\\
Let $f \in M^{*}$ be a bounded linear functional defined on a \textbf{subspace} $M$ of a real normed vector space $X$. Then there is a bounded linear functional
$F \in X^{*}$ defined on $X$ which is an \textbf{extension} of $f$ satisfying $\norm{F}{X^{*}} = \norm{f}{M^{*}}$.
\end{corollary} 
Note let $p(x) = \norm{f}{M^{*}}\,\norm{x}{}$.

\item \begin{corollary}
Let $y$ be an element of a normed linear space $X$. Then there is a nonzero $F \in X^{*}$ such that $F(y) = \norm{F}{X^{*}}\,\norm{y}{X}$.
\end{corollary}

\item \begin{corollary}(\underline{\textbf{The Existance of Distance Functional}})\\
Let $Z$ be a subspace of a normed linear space $X$ and suppose that $y$ is an element of $X$ whose \textbf{distance} from $Z$ is $d = \inf_{z \in Z}\norm{y -z}{}$. Then there
exists a $F \in X^{*}$ so that $\norm{F}{} \le 1$, $F(y) = d$, and $F(z) = 0$ \textbf{for all $z$ in $Z$}.
\end{corollary}

\item \begin{remark}
\emph{The Hahn-Banach theorem}, particularly \emph{Corollary 3.3}, is perhaps most profitably viewed as \emph{\textbf{an existence theorem for a minimization problem}}. Given an $f$ on a subspace $M$ of a normed space, it is not difficult to extend $f$ to the whole space. \emph{\textbf{An arbitrary extension}}, however, will in general be \emph{\textbf{unbounded}} or \emph{have norm greater than the norm of $f$ on $M$}. We therefore pose the problem of \emph{selecting the extension of minimum norm}. The Hahn-Banach theorem both guarantees \emph{\textbf{the existence of a minimum norm extension}} and tells us \emph{\textbf{the norm of the best extension}}.
\end{remark}

\item \begin{proposition}
Let $X$ be a Banach space. If $X^{*}$ is \textbf{separable}, then $X$ is \textbf{separable}.
\end{proposition}
\end{itemize}

\subsubsection{Geometric Form of The Hahn-Banach Theorem}
\begin{itemize}
\item \begin{definition}
The \emph{\textbf{translation}} of a subspace is called \emph{a \underline{\textbf{linear variety}}}. It is written as $x + M$ where $x\in X$ is a fixed point and $M\subseteq X$ is a subspace of $X$.
\end{definition}

\item \begin{remark}
\emph{A linear variety} is also called \emph{an \textbf{affine subspace}}.
\end{remark}

\item \begin{definition}
A \underline{\emph{\textbf{hyperplane}}} $H$ in a linear vector space $X$ is a \emph{\textbf{maximal} proper linear variety}, that is, a linear variety $H$ such that $H \neq X$, and if $V$ is any linear variety containing $H$, then either $V = X$ or $V = H$.
\end{definition}

\item \begin{remark}
A \emph{hyperplane} $H = x + M$ where $M$ has \emph{\textbf{codimension}} $1$ in $X$, i.e. 
\begin{align*}
X = \text{span}\{x, \text{basis of }M\}.
\end{align*}
\end{remark}

\item \begin{proposition} \citep{luenberger1997optimization}\\
Let $H$ be a \textbf{hyperplane} in a linear vector space $X$. Then there is a \textbf{linear functional} $f$ on $X$ and a constant $c$ such that $H =\{x: f(x) =c\}$.
\textbf{Conversely}, $f$ is a nonzero linear functional on $X$, the set $\{x: f(x) =c\}$ is a hyperplane in $X$.
\end{proposition}

\item There exists an \emph{\textbf{one-to-one correspondence}} between linear functional and hyperplane that does not passes the origin.
\begin{proposition} (\textbf{Unique Linear Functional for Hyperplane}) \citep{luenberger1997optimization}\\
Let $H$ be a hyperplane in a linear vector space $X$. If $H$ \textbf{does not contain the origin}, there is a \textbf{unique} linear functional $f$ on $X$ such that
$H =\{x: f(x) =1 \}$.
\end{proposition}

\item \begin{proposition}
Let $f$ be a nonzero linear functional on a normed space $X$. Then the hyperplane $H =\{x: f(x) =c\}$ is \textbf{closed} for every $c$ if and only if
$f$ is \textbf{continuous}.
\end{proposition}

\item \begin{remark}
If $f$ is a nonzero linear functional on a linear vector space $X$, we associate with the hyperplane $H =\{x: f(x) =c\}$  \emph{the four sets}
\begin{align*}
\{x: f(x) \le c\}, \quad \{x: f(x) < c\}, \quad \{x: f(x) \ge c\}, \quad \{x: f(x) > c\}
\end{align*}
called \underline{\emph{\textbf{half-spaces determined by $H$}}}. The first two of these are referred to as \emph{\textbf{negative half-spaces} determined by $f$} and the second two as \emph{\textbf{positive half-spaces}}.

If $f$ is \emph{continuous},\emph{ the first and the third} half-spaces are \emph{\textbf{closed}} and \emph{the second and fourth} are \emph{\textbf{open}}.
\end{remark}

\item \begin{definition} (\emph{\textbf{The Minkowski Functional}}) \citep{luenberger1997optimization}\\
Let $K$ be \emph{\textbf{a convex set}} in a \emph{normed linear vector space} $X$ and suppose $0$ is \emph{an \textbf{interior point} of $K$}. Then \underline{\emph{\textbf{the Minkowski functional (or gauge)}}} $p$ of $K$ is defined on $X$ by
\begin{align*}
p(x) := \inf\set{r: \frac{x}{r} \in K, r > 0} = [\sup\set{t: t\,x \in X, t > 0}]^{-1}.
\end{align*}
We note that for \emph{$K$ equal to the unit sphere in $X$}, \emph{the Minkowski functional is $\norm{x}{}$}. In the general case, $p(x)$ defines a kind of \emph{\textbf{distance}} from \emph{the origin to $x$} measured \emph{with respect to $K$}; it is the \emph{factor} by which $K$ must be expanded so as to \emph{include} $x$.
\end{definition}

\item \begin{lemma}
Let $K$ be a convex set containing $0$ as an interior point. Then \textbf{the Minkowski functional} $p$ of $K$ satisfies:
\begin{enumerate}
\item $0 \le p(x) < \infty$ for all $x \in X$;
\item (\textbf{Homogeneity}): $p(\lambda x) = \lambda p(x)$ for all $\lambda \ge 0$ and $x \in X$;
\item (\textbf{Sublinearity}): $p(x + y) \le p(x) + p(y)$,
\item $p$ is \textbf{continuous};
\item $\overline{K} = \{x: p(x) \le 1\}$ and $\mathring{K} = \{x : p(x) < 1\}$.
\end{enumerate}
\end{lemma}
That is, \emph{the Minkowski functional} is \emph{a sublinear functional}.

\item \begin{theorem} (\textbf{Mazur's Theorem, \underline{Geometric Hahn-Banach Theorem}}) \citep{luenberger1997optimization}\\
Let $K$ be a \underline{\textbf{convex set} having a \textbf{nonempty interior}} in a real normed linear vector space $X$. Suppose $V$ is a \textbf{linear variety} in $X$ \underline{containing no interior points} of $K$. Then there is a \underline{\textbf{closed hyperplane}} in $X$ \underline{containing $V$ but \textbf{containing no interior points} of $K$}; i.e., there is an element $f \in X^{*}$ and a constant $c$ such that $f(v) = c$ for all $v \in V$ and $f(k) < c$ for all $k \in K$.
\end{theorem}


\item \begin{remark} (\emph{\textbf{Geometric Interpretation of the Hahn-Banach theorem}})\\
\emph{The \textbf{geometric form} of the \textbf{Hahn-Banach theorem}}, in simplest form, says that given \emph{a \textbf{convex set}} $K$ containing \emph{\textbf{an interior point}}, and given a point $x_0$ \emph{not in} $\mathring{K}$, there is a \emph{\textbf{closed hyperplane}} \emph{\textbf{containing}} $x_0$ but \emph{\textbf{disjoint}} from $\mathring{K}$.
\end{remark}

\item \begin{definition} (\emph{\textbf{Supporting Hyperplane}})\\
A \emph{\textbf{closed} hyperplane} $H$ in a normed space $X$ is said to be \underline{\emph{\textbf{a supporting hyperplane}}} (or a  \emph{\textbf{support}}) for the \emph{\textbf{convex set}} $K$ if $K$ is \emph{contained} in \emph{one of the \textbf{closed half-spaces}} determined by $H$ and $H$ contains a point of $\overline{K}$.
\end{definition}

\item \begin{remark}
Suppose $K \subseteq \bR^n$, and $x_0$ is a point in its boundary $\partial K$, i.e.,
\begin{align*}
x_0 \in \partial K = \overline{K} \setminus \mathring{K}.
\end{align*}
If $a \neq 0$ satisfies $\inn{a}{x} \le  \inn{a}{x_0}$ for all $x \in K$, then the hyperplane $\{x: \inn{a}{x} = \inn{a}{x_0}\}$
is called \emph{\textbf{a supporting hyperplane}} to $K$ at the point $x_0$.
\end{remark}

\item \begin{theorem} (\underline{\textbf{Supporting Hyperplane Theorem}}) \citep{luenberger1997optimization, rockafellar1970convex} \\
If $x$ is \textbf{not an interior point} of a convex set $K$ which contains interior points, there is a \textbf{closed hyperplane} $H$ containing $x$ such that $K$ lies on one side of $H$.
\end{theorem}

\item As a consequence of the above theorem, it follows that, for a convex set $K$ with interior points, \emph{\textbf{a supporting hyperplane} can be constructed
containing \textbf{any boundary point of $\overline{K}$}}.
\begin{theorem} (\underline{\textbf{Eidelheit's Separation Theorem}})  \citep{luenberger1997optimization, rockafellar1970convex} \\
Let $K_1$ and $K_2$ be \textbf{convex sets} in $X$ such that $K_1$ has interior points and $K_2$ \textbf{contains no interior point of $K_1$}.
Then there is a \textbf{closed hyperplane} $H$ \textbf{separating} $K_1$ and $K_2$; i.e., there exists $f \in X^{*}$ such that 
\begin{align}
\sup_{x \in K_1}f(x) &\le \inf_{x \in K_2}f(x) \label{eqn: convex_duality_weak}
\end{align}
In other words, $K_1$ and $K_2$ lie in \textbf{opposite} \textbf{half-spaces} determined by $H$.
\end{theorem}
\begin{proof}
Let $K = K_1 - K_2 = \set{x_1 - x_2: x_1 \in K_1, x_2 \in K_2}$; then $K$ contains an interior point and $0$ not one of them. Also $K$ is a convex set. By \emph{The Supporting Hyperplane Theorem},  there is an $f \in X^{*}$, $f \neq 0$, such that $f(x) \le 0$ for $x \in K$. Thus for $x_1 \in K_l$, $x_2 \in K_2$, $f(x_1) \le f(x_2)$. Consequently, there is a real number $c$ such that $\sup_{x \in K_1}f(x) \le c \le \inf_{x \in K_2}f(x)$. The desired hyperplane is $H = \{x:  f(x) = c\}$.  \qed
\end{proof}

\item \begin{corollary}
If $K$ is a \textbf{closed convex} set and $x \not\in K$, there is a \textbf{closed halfspace} that contains $K$ but does not contain $x$.
\end{corollary}

\item \begin{theorem} (\textbf{Dual Representation of Convex Set})\citep{luenberger1997optimization, rockafellar1970convex} \\
If $K$ is a \textbf{closed convex} set in a normed space, then $K$ is equal to the \textbf{intersection} of all the \textbf{closed half-spaces} that contain it. 
\end{theorem}

\item \begin{remark} (\emph{\textbf{Duality for Convex Set}})\\
Theorem above is often regarded as \emph{\textbf{the geometric foundation} of \textbf{duality theory} for \textbf{convex sets}}. By associating \emph{closed hyperplanes} (or \textit{half-spaces}) with elements of $X^{*}$, the theorem expresses \emph{\textbf{a convex set in $X$ as a collection of elements in $X^{*}$}}. See more in \citep{rockafellar1970convex}.
\end{remark}

\item \begin{definition}
Let $K$ be a convex set in a real normed vector space $X$. The functional
\begin{align*}
h(f) &:= \sup_{x \in K}f(x)
\end{align*} defined on $X^{*}$ is called \underline{\emph{\textbf{the support functional}}} of $K$. $h \in X^{**}$.
\end{definition}

\item \begin{remark}
The \emph{\textbf{support functional}} of a \emph{convex set $K$} \emph{completely specifies the set (to within \textbf{closure})}
\begin{align*}
\overline{K} &= \bigcap_{f \in X^{*}}\set{x: f(x) \le h(f)}.
\end{align*}
\end{remark}
\end{itemize}


\subsection{Baire Category Theorem}
\begin{itemize}
\item \begin{remark} (\emph{\textbf{Empty Interior $=$ Complement is Dense}}) \\
Recall that if $A$ is a subset of a space $X$, the \emph{\textbf{interior}} of $A$ is defined as \emph{the union of all open sets of $X$ that are contained in $A$}. 

To say that $A$ has \underline{\emph{\textbf{empty interior}}} is to say then that \emph{\textbf{$A$ \underline{contains no open set} of $X$} other than the empty set}. \emph{\textbf{Equivalently}}, $A$ has \emph{\textbf{empty interior}} if every point of $A$ is \emph{a \textbf{limit point} of the \textbf{complement} of $A$}, that is, if \underline{\emph{the \textbf{complement} of $A$ is \textbf{dense} in $X$}}.
\begin{align*}
\mathring{A} = \emptyset \;\; \Leftrightarrow \;\; A^{c}\text{ is dense in }X
\end{align*} In \citep{reed1980methods}, if a subset $\overline{A}$ of $X$ has \emph{empty interior}, $A$ is said to be \underline{\emph{\textbf{nowhere dense}}} in $X$.
\end{remark}

\item \begin{example} 
Some  examples:
\begin{enumerate}
\item The set $\bQ$ of \emph{rationals} has \emph{\textbf{empty interior}} as a subset of $\bR$
\item The \emph{interval} $[0, 1]$ has \emph{\textbf{nonempty interior}}. 
\item The \emph{interval} $[0, 1] \times 0$ has \emph{\textbf{empty interior}} as a \emph{subset of the plane} $\bR^2$, and so does the \emph{subset} $\bQ \times \bR$.
\end{enumerate}
\end{example}

\item \begin{definition} (\emph{\textbf{Baire Space}})\\
A space $X$ is said to be a \underline{\emph{\textbf{Baire space}}} if the following condition holds:  Given  \emph{\textbf{any countable}} collection $\set{A_n}$ of \emph{\textbf{closed}} sets of $X$ each of which has \emph{\textbf{empty interior}} in $X$, their \emph{\textbf{union}}  $\bigcup_{n=1}^{\infty} A_n$ also has \emph{\textbf{empty interior}} in $X$.
\end{definition}

\item \begin{example} 
Some  examples:
\begin{enumerate}
\item The space $\bQ$ of \emph{rationals} is \emph{\textbf{not} a \textbf{Baire space}}. For each one-point set in $\bQ$ is \emph{closed and has empty interior \textbf{in $\bQ$}}; and $\bQ$ is \emph{the countable union of its one-point subsets}.
\item The space $\bZ_{+}$, on the other hand, does form \emph{a \textbf{Baire space}}. Every subset of $\bZ_{+}$ is \emph{open}, so that there exist \emph{no subsets} of $\bZ_{+}$ having \emph{empty interior}, except for the empty set. Therefore, $\bZ_{+}$ satisfies the Baire condition vacuously.
\item The \emph{interval} $[0, 1] \times 0$ has \emph{\textbf{empty interior}} as a \emph{subset of the plane} $\bR^2$, and so does the \emph{subset} $\bQ \times \bR$.
\end{enumerate}
\end{example}

\item \begin{definition}  (\emph{\textbf{Baire Category}})\\
A subset $A$ of a space $X$ was said to be of \underline{\emph{\textbf{the first category in $X$}}} if it \emph{\textbf{was contained} in the \textbf{union} of a \textbf{countable} collection of \textbf{closed} sets of $X$ having \textbf{empty interiors} in $X$}; \emph{\textbf{otherwise}}, it was said to be of \underline{\emph{\textbf{the second category in $X$}}}. 
\end{definition}

\item \begin{remark}
\emph{A space $X$ is a \textbf{Baire space} if and only if every \textbf{nonempty open} set in $X$ is of \textbf{the second category}}.
\end{remark}

\item \begin{lemma}(\textbf{Open Set Definition of Baire Space}) \citep{munkres2000topology} \\
$X$ is a \textbf{Baire space} \textbf{if and only if} given any \textbf{countable} collection $\set{U_n}$ of \textbf{open} sets in $X$, each of which is \textbf{dense} in $X$, their \textbf{intersection} $\bigcap_{n=1}^{\infty}U_n$ is also \textbf{dense} in $X$.
\end{lemma}

\item \begin{theorem} (\textbf{Baire Category Theorem}).  \citep{munkres2000topology} \\
If $X$ is a \textbf{compact Hausdorff} space or a \textbf{complete metric space}, then $X$ is a \textbf{Baire space}.
\end{theorem}

\item \begin{remark}
In other word,  neither \textbf{\emph{compact Hausdorff}} space or a \textbf{\emph{complete metric space}} is a \emph{countable union of closed subsets with empty interior (that are nowhere dense)}.
\end{remark}

\item \begin{lemma}\citep{munkres2000topology} \\
Let $C_1 \supset C_2 \supset \ldots$ be a \textbf{nested} sequence of \textbf{nonempty closed sets} in the \textbf{complete metric space} $X$. If $\text{diam }C_n \rightarrow 0$, then $\bigcap_{n}C_n  = \emptyset$.
\end{lemma}

\item \begin{lemma} \citep{munkres2000topology} \\
Any \textbf{open} subspace $Y$ of a Baire space $X$ is itself a Baire space.
\end{lemma}

\item \begin{theorem} (\textbf{Discontinuity Point of Pointwise Convergence Function}) \citep{munkres2000topology} \\
Let $X$ be a space; let $(Y, d)$ be a metric space. Let $f_n : X \rightarrow Y$ be a sequence of continuous functions such that $f_n(x) \rightarrow f(x)$ for all $x \in X$, where $f : X \rightarrow Y$. If $X$ is a \textbf{Baire space}, the set of points at which $f$ is \textbf{continuous} is \textbf{dense} in $X$.
\end{theorem}

\item \begin{remark} (\textbf{\emph{Use Baire Category Theorem as Proof by Contradition}})\\
\emph{\textbf{The Baire category theorem}} is used to prove a certain subset $C$ is \emph{\textbf{dense}} in $X$ by stating that $X$ is a Baire space and $C$ is countable intersection of dense open subsets in $X$ (\emph{$C$ is a $G_{\delta}$ sets}). 

On the other hand, if $M =  \bigcup_{n=1}^{\infty}A_n$ has \emph{\textbf{nonempty interior}}, then \emph{\textbf{some}} of the  sets $\bar{A}_n$ \emph{\textbf{must have nonempty interior}}. Otherwise, it contradicts with the Baire space definition.
\end{remark}
\end{itemize}

\subsection{Uniform Boundedness Theorem}
\begin{itemize}
\item \begin{proposition}\label{prop: bounded_non_empty_int} \citep{reed1980methods}\\
Let $X$ and $Y$ be normed linear spaces. Then a linear map  $Τ: X\rightarrow Y$ is \textbf{bounded} if and only if 
\begin{align*}
T^{-1}\set{y: \norm{y}{Y} \le 1}
\end{align*} has a \textbf{nonempty interior}. 
\end{proposition}


\item \begin{theorem} (\textbf{The Uniform Boundedness Theorem}). \citep{reed1980methods} \\
Let $X$ be a \textbf{Banach space}. Let $\srF$ be a family of \textbf{bounded} linear transformations from $X$ to some \textbf{normed linear space} $Y$. Suppose that for each $x \in X$, $\set{\norm{Tx}{Y}:  T \in \srF}$ is  \textbf{bounded}, i.e.
\begin{align*}
\sup_{T \in \srF}\norm{Tx}{Y} < \infty.
\end{align*} Then $\set{\norm{T}{}: T \in \srF}$ is \textbf{bounded}, i.e.
\begin{align*}
\sup_{T \in \srF}\norm{T}{} < \infty.
\end{align*}
\end{theorem}

\item \begin{corollary} (\textbf{Separately Continuity of Bilinear Form on Banach Space $=$ Joint Continuity}) \citep{reed1980methods}\\
Let $X$ and $Y$ be Banach spaces and let $B(\cdot,  \cdot)$ be a \textbf{separately continuous bilinear mapping} from $X \times Y$ to $\bC$, that is, it is a \textbf{bounded} linear transformation if one of the two arguments is fixed. Then $B(\cdot,  \cdot)$ is \textbf{jointly continuous}, that is, if $x_n \rightarrow 0$ and  $y_n \rightarrow 0$ then $B(x_n, y_n) \rightarrow 0$. 
\end{corollary}
\end{itemize}

\subsection{Open Mapping Theorem}
\begin{itemize}
\item 
\begin{theorem} (\textbf{Open Mapping Theorem}) \citep{reed1980methods} \\
Let $T: X \rightarrow Y$ be a \textbf{\underline{surjective} bounded linear transformation} of one \textbf{Banach} space \underline{\textbf{onto}} another \textbf{Banach} space $Y$.  Then if $M$ is an \textbf{open} set in $X$, $T(M)$ is \textbf{open} in $Y$. 
\end{theorem}

\item \begin{corollary} (\textbf{Inverse Mapping Theorem}) \citep{reed1980methods} \\
A \textbf{continuous} \textbf{bijection} of one Banach space onto another has a \textbf{continuous} \textbf{inverse}. 
\end{corollary}

\item \begin{remark}
Note $T$ is an open map and $A = T^{-1}(T(A))$ for surjective map, then $T^{-1}$ is \emph{continuous}.
\end{remark}

\item \begin{theorem} (\textbf{Banach-Schauder Theorem}) \citep{reed1980methods}\\
Let $T$ be a \textbf{continuous} linear map, $T: E \rightarrow F$,  where $E$ and $F$ are Banach spaces. Then either $T(A)$ is \textbf{open} in $T(E)$  for \textbf{each open} $A \subseteq E$,  or $T(E)$  is of \textbf{first category} in $\overline{T(E)}$.
\end{theorem}
\end{itemize}

\subsection{Closed Graph Theorem}
\begin{itemize}
\item \begin{definition} (\emph{\textbf{Graph of Function}})\\
Let $Τ$ be a mapping of a normed linear space X into a normed  linear space Y. The \underline{\emph{\textbf{graph of $T$}}}, denoted by $\Gamma(T)$, is defined as 
\begin{align*}
\Gamma(T) := \set{(x,y) \in X \times Y: y = Tx}.
\end{align*}
\end{definition}


\item \begin{theorem} (\textbf{Closed Graph Theorem})\citep{reed1980methods} \\
Let $X$ and $Y$ be Banach spaces  and $T$ a linear map of $X$ into $Y$. Then $T$ is \textbf{bounded} if and only if the \textbf{graph} of 
$Τ$ is \textbf{closed}. 
\end{theorem}

\item \begin{remark}
To avoid future confusion, we emphasize that the $T$ in this theorem is implicitly assumed to be \emph{\textbf{defined on all of $X$}}. 
\end{remark}

\item \begin{remark}
Consider the following statements: 
\begin{enumerate}
\item $x_n$ converges to some element $x$;
\item $T x_n$ converges to some element $y$; 
\item $T x_n = y$.
\end{enumerate} 
Usually to prove $T$ is continuous, one need to show that given statement $1$, the statement $2$ and $3$ are true. That is, we need to \emph{\textbf{prove convergence}} of $T x_n$ and need to show \emph{\textbf{identification}} of $T x$ and the limit of $T x_n$.

With \emph{\textbf{close graph theorem}}, we just need to show that given statement $1$ \emph{\textbf{and}} $2$, statement $3$ is true; that is, we just need to prove the identification part.
\end{remark}

\item \begin{corollary}(\textbf{The Hellinger-Toeplitz Theorem}) \citep{reed1980methods} \\
Let $A$ be an \textbf{everywhere defined} linear operator on a \textbf{Hilbert space} $\cH$ with
\begin{align*}
\inn{x}{Ay} &= \inn{Ax}{y}
\end{align*} for all $x, y \in \cH$; that is $A$ is \textbf{self-adjoint}. Then $A$ is \textbf{bounded}. 
\end{corollary}
\end{itemize}

\section{Spectrum of Bounded Linear Operator in Banach Space}
\begin{itemize}
\item 

\item \begin{definition} (\emph{\textbf{Spectral Radius of Linear Operator}})\\
Let 
\begin{align*}
r(T) &= \sup_{\lambda \in \sigma(T)} \abs{\lambda}
\end{align*}
$r(T)$ is called  \underline{\textbf{\emph{the spectral radius of $T$}}}. 
\end{definition}

\item \begin{proposition} (\textbf{Spectral Radius Calculation}) \citep{reed1980methods}\\
Let $X$ be a \textbf{Banach space}, $T \in \cL(X)$. Then 
\begin{align*}
\lim\limits_{n\rightarrow \infty}\norm{T^n}{}^{1/n}
\end{align*}
exists and is equal to $r(T)$. 
\end{proposition}

\item \begin{theorem} (\textbf{Spectrum and Resolvent of Adjoint}) (\textbf{Phillips}) \citep{reed1980methods}\\ 
Let $X$ be a \textbf{Banach space},  $T \in \cL(X)$. Then 
\begin{align*}
\sigma(T) = \sigma(T')\; \text{ and }\; R_{\lambda}(T') = (R_{\lambda}(T))'.
\end{align*}
\end{theorem}

\item \begin{proposition}  (\textbf{Spectrum of Adjoint}) \citep{reed1980methods}\\ 
Let $X$ be a Banach space and $T \in \cL(X)$. Then, 
\begin{enumerate}
\item If $\lambda$ is in the \textbf{residual spectrum} of $T$, then $\lambda$ is in the \textbf{point spectrum} of $T'$. 
\item If  $\lambda$ is in the \textbf{point spectrum} of $T$, then $\lambda$ is in \textbf{either} the \textbf{point} or the \textbf{residual spectrum} of $T'$. 
\end{enumerate}
\end{proposition}
\end{itemize}

\newpage
\bibliographystyle{plainnat}
\bibliography{reference.bib}
\end{document}