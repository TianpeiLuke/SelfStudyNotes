\documentclass[11pt]{article}
\usepackage[scaled=0.92]{helvet}
\usepackage{geometry}
\geometry{letterpaper,tmargin=1in,bmargin=1in,lmargin=1in,rmargin=1in}
\usepackage[parfill]{parskip} % Activate to begin paragraphs with an empty line rather than an indent %\usepackage{graphicx}
\usepackage{amsmath,amssymb, mathrsfs,  mathtools, dsfont}
\usepackage{tabularx}
\usepackage{tikz-cd}
\usepackage[font=footnotesize,labelfont=bf]{caption}
\usepackage{graphicx}
\usepackage{xcolor}
%\usepackage[linkbordercolor ={1 1 1} ]{hyperref}
%\usepackage[sf]{titlesec}
\usepackage{natbib}
\usepackage{../../Tianpei_Report}

%\usepackage{appendix}
%\usepackage{algorithm}
%\usepackage{algorithmic}

%\renewcommand{\algorithmicrequire}{\textbf{Input:}}
%\renewcommand{\algorithmicensure}{\textbf{Output:}}



\begin{document}
\title{Lecture 0: Summary (Part 1)}
\author{ Tianpei Xie}
\date{ Dec. 15th., 2022 }
\maketitle
\tableofcontents
\newpage
\section{Topology Basis}
\subsection{Topology, Basis and Subbasis}
\begin{itemize}
\item 
\begin{definition} 
Let $X$ be a set. \underline{\emph{A \textbf{topology}}} on $X$ is \emph{a collection} $\mathscr{T}$ of \emph{subsets} of X, called \emph{\textbf{open subsets}}, satisfying
\begin{enumerate}
\item $X$ and $\emptyset$ are \emph{open}.
\item The \emph{\textbf{union}} of \emph{\textbf{any family}} of open subsets is open.
\item The \emph{\textbf{intersection}} of \emph{any \textbf{finite} family} of open subsets is open.
\end{enumerate}
A pair $(X, \mathscr{T})$ consisting of a set $X$ together with a topology $\mathscr{T}$ on $X$ is called \emph{\textbf{a topological space}}.
\end{definition}

\item \begin{definition}
A map $F: X \rightarrow Y$ is said to be \underline{\emph{\textbf{continuous}}} if for every open subset $U \subseteq Y$, the \emph{\textbf{preimage}} $F^{-1}(U)$ is \emph{\textbf{open}} in $X$.
\end{definition}

\item \begin{definition}
A \emph{\textbf{continuous bijective}} map $F: X \rightarrow Y$ with \emph{\textbf{continuous inverse}} is called a \underline{\emph{\textbf{homeomorphism}}}. If there exists a \emph{homeomorphism} from $X$ to $Y$, we say that X and Y are \emph{\textbf{homeomorphic}}.
\end{definition}

\item \begin{definition}
Suppose $X$ is a topological space. A collection $\mathscr{B}$ of open subsets of $X$ is said to be \emph{\textbf{a basis}} for \emph{the topology of $X$} (plural: \emph{\textbf{bases}}) if every open subset of $X$ is the \emph{union of some collection of elements} of $\mathscr{B}$.

More generally, suppose $X$ is merely a set, and $\mathscr{B}$ is a collection of \emph{subsets} of $X$ satisfying the following conditions:
\begin{enumerate}
\item $X = \bigcup_{B \in \mathscr{B}}B$.
\item If $B_1, B_2 \in \mathscr{B}$ and $x \in B_1 \cap B_2$, then there exists $B_3 \in \mathscr{B}$ such that $x \in B_3 \subseteq B_1 \cap B2$.
\end{enumerate}
Then \emph{the collection of \textbf{all unions} of elements of $\mathscr{B}$} is a \emph{topology} on X, called \emph{\textbf{the topology generated by $\mathscr{B}$}}, and $\mathscr{B}$ is a \underline{\emph{\textbf{basis}} for this \emph{topology}}.
\end{definition}

\item \begin{lemma} (\textbf{Obtaining Basis from Given Topology}). \citep{munkres2000topology}\\
Let $X$ be a topological space. Suppose that $\srC$ is a collection of open sets of $X$ such that for each open set $U$ of $X$ and each $x$ in $U$, there is an element $C$ of $\srC$ such that $x \in C \subset U$. Then $C$ is a basis for the topology of $X$.
\end{lemma}

\item \begin{lemma} (\textbf{Topology Comparison via Bases}). \citep{munkres2000topology}\\
Let $\srB$ and $\srB'$ be bases for the topologies $\srT$ and $\srT'$, respectively, on $X$. Then the following are equivalent:
\begin{enumerate}
\item  $\srT'$ is \textbf{finer} than $\srT$.
\item For each $x \in X$ and each basis element $B \in \srB$ containing $x$, there is a basis element $B' \in \srB'$ such that $x \in B' \subset B$.
\end{enumerate}
\end{lemma}

\item \begin{definition}(\emph{\textbf{Subbasis}})\\
\underline{\emph{\textbf{A subbasis}} $\srS$ for a \emph{topology} on $X$} is a collection of subsets of $X$ whose union equals $X$. The topology generated by the \emph{subbasis} $\srS$ is defined to be the  collection $\srT$ of \emph{\textbf{all unions} of \textbf{finite intersections} of elements of $\srS$}.
\end{definition}

\item \begin{remark}(\textbf{\emph{Basis from Subbasis}})\\
For a \emph{subbasis} $\srS$, the collection $\srB$ of \emph{\textbf{all finite intersections}} of elements of $\srS$ is a \emph{\textbf{basis}},
\end{remark}
\end{itemize}

\subsection{Limit Point and Closure}
\begin{itemize}
\item \begin{definition}
A subset $A$ of a topological space $X$ is said to be \emph{\textbf{closed}} if the set $X \setminus A$ is \emph{open}.
\end{definition}

\item \begin{definition}
Given a subset $A$ of a topological space $X$, \underline{\emph{\textbf{the interior of $A$}}} is defined as \emph{the union of all open sets} \emph{\textbf{contained}} in $A$, and \underline{\emph{\textbf{the closure of $A$}}} is defined as \emph{the intersection of all closed sets} \emph{\textbf{containing}} $A$.

\emph{\textbf{The interior of $A$}} is denoted by $\text{Int }A$ or by $\mathring{A}$ and \emph{\textbf{the closure of $A$}} is denoted by $\text{CI }A$ or
by $\bar{A}$. Obviously $\mathring{A}$ is \emph{an open set} and $\bar{A}$ is \emph{a closed set}; furthermore,
\begin{align*}
\mathring{A} \subseteq A \subseteq \bar{A}.
\end{align*}
If $A$ is \emph{\textbf{open}}, $A = \mathring{A}$; while if $A$ is \emph{\textbf{closed}}, $A = \bar{A}$.
\end{definition}

\item \begin{proposition} (\textbf{Characterization of Closure in terms of Basis}) \citep{munkres2000topology} \\
Let $A$ be a subset of the topological space $X$.
\begin{enumerate}
\item Then $x \in \bar{A}$ if and only if every \textbf{open} set $U$ \textbf{containing} $x$ \textbf{intersects} $A$.
\item Supposing the topology of $X$ is given by a \textbf{basis}, then $x \in \bar{A}$ if and only if every basis element $B$ \textbf{containing} $x$ \textbf{intersects} $A$.
\end{enumerate}
\end{proposition}

\item \begin{remark}
We can say ``\emph{$U$ is a \textbf{neighborhood} of $x$}'' if ``\emph{$U$ is an open set containing $x$}".
\end{remark}

\item \begin{definition} (\emph{\textbf{Limit Point}})\\
If $A$ is a subset of the topological space $X$ and if $x$ is a point of $X$, we say that $x$ is a \underline{\emph{\textbf{limit point}}} (or ``\emph{\textbf{cluster point}}," or ``\emph{\textbf{point of accumulation}}") of $A$ if \emph{\textbf{every neighborhood} of $x$ \textbf{intersects} $A$} \emph{in some point \textbf{other than} $x$ itself}. 

Said differently, $x$ is \emph{\textbf{a limit point}} of $A$ if it belongs to \emph{\textbf{the closure of $A \setminus \{x\}$}}. The point $x$ may lie in $A$ or not; for this definition it does not matter.
\end{definition}

\item \begin{theorem} (\textbf{Decomposition of Closure})\\
Let $A$ be a subset of the topological space $X$; let $A'$ be the set of \textbf{all limit points} of $A$. Then
\begin{align*}
\bar{A} &= A \cup A'.
\end{align*}
\end{theorem}

\item \begin{corollary}
A subset of a topological space is \textbf{closed} if and only if it contains all its \textbf{limit points}.
\end{corollary}
\end{itemize}

\subsection{Subspace, Product and Quotient Topologies}
\subsubsection{Subspace Topology}
\begin{itemize}
\item \begin{definition}
If $X$ is a topological space and $S \subseteq X$ is an arbitrary subset, we define \emph{\textbf{the subspace topology}} on $S$ (sometimes called \emph{the \textbf{relative topology}}) by declaring a subset $U \subseteq S$ to be \emph{open} in $S$ \emph{if and only} if there exists an open subset $V \subseteq X$ such that $U = V \cap S$. 

Any subset of $X$ endowed with the subspace topology is said to be \emph{\textbf{a subspace of $X$}}.
\end{definition}

\item \begin{lemma} (\textbf{Basis of Subspace Topology})\\
If $\srB$ is a basis for the topology of $X$ then the collection
\begin{align*}
\srB_{S} = \set{B \cap S:  B \in \srB}
\end{align*}
is a \textbf{basis}  for the subspace topology on $S \subset X$.
\end{lemma}

\item \begin{proposition}
Let $Y$ be a subspace of $X$. If $A$ is closed in $Y$ and $Y$ is closed in $X$, then $A$ is closed in $X$.
\end{proposition}

\item \begin{proposition} (\textbf{Closure in Subspace Topology})\\
Let $Y$ be a subspace of $X$; let $A$ be a subset of $Y$; let $\bar{A}$ denote the closure of $A$ in $X$. Then the closure of $A$ in $Y$ equals $\bar{A} \cap Y$.
\end{proposition}
\end{itemize}
\subsubsection{Product Topology}
\begin{itemize}
\item \begin{definition} (\emph{\textbf{$J$-tuples}})\\
Let $J$ be an index set. Given a set $X$, we define a \emph{\underline{\textbf{$J$-tuple}} of elements} of $X$ to be a function $x : J \rightarrow X$. If $\alpha$ is an element of $J$, we often denote \emph{\textbf{the value of $X$ at $\alpha$}} by $X_{\alpha}$ rather than $x(\alpha)$; we call it \underline{\emph{\textbf{the $\alpha$-th coordinate}}} of $x$. And we often \emph{denote the function $x$ itself} by the symbol
\begin{align*}
(x_{\alpha})_{\alpha \in J}
\end{align*}
which is as close as we can come to a ``\emph{tuple notation}" for an arbitrary index set $J$. We denote \emph{\textbf{the set of all $J$-tuples}} of elements of $X$ by $X^{J}$.
\end{definition}

\item \begin{definition} (\emph{\textbf{Arbitrary Cartestian Products}})\\
Let $\{A_{\alpha}\}_{\alpha \in J}$ be an \emph{indexed} family of sets; let $X = \bigcup_{\alpha \in J}A_{\alpha}$. \emph{The \textbf{cartesian product} of this indexed family}, denoted by
\begin{align*}
\prod_{\alpha \in J} A_{\alpha}
\end{align*}
is defined to be the set of all $J$-tuples $(x_{\alpha})_{\alpha \in J}$ of elements of $X$ such that $x_{\alpha} \in A_{\alpha}$ for each $\alpha \in J$. That is, it is the set of all functions
\begin{align*}
x: J \rightarrow \bigcup_{\alpha \in J}A_{\alpha}
\end{align*}
such that $x(\alpha) \in A_{\alpha}$ for each $\alpha \in J$.
\end{definition}

\item \begin{definition} (\emph{\textbf{Projection Mapping or Coordinate Projection}})\\
Let
\begin{align*}
\pi_{\beta}: \prod_{\alpha \in J} X_{\alpha} \rightarrow X_{\beta}
\end{align*}
be the function assigning to each element of the product space its $\beta$-th coordinate,
\begin{align*}
\pi_{\beta}((x_{\alpha})_{\alpha \in J}) = x_{\beta};
\end{align*}
it is called \emph{\underline{\textbf{the projection mapping}} associated with the index $\beta$}.
\end{definition}

\item \begin{definition} (\emph{\textbf{Product Topology}})\\
Let $\srS_\beta$ denote the collection
\begin{align*}
\srS_\beta = \set{ \pi_\beta^{-1}(U_\beta): U_{\beta}\text{ open in }X_\beta},
\end{align*}
and let $\srS$ denote \emph{the union of these collections},
\begin{align*}
\srS &= \bigcup_{\beta \in J} \srS_\beta.
\end{align*}
The topology generated by \emph{the \textbf{subbasis} S} is called \underline{\emph{\textbf{the product topology}}}. In this 
topology $\prod_{\alpha \in J} X_{\alpha}$ is called \emph{\textbf{a product space}}.
\end{definition}

\item \begin{remark} (\emph{\textbf{Product Topology $=$ Weak Topology by Coordinate Projections}})\\
\emph{The product topology} on $\prod_{\alpha \in J} X_{\alpha}$ is \emph{\textbf{the weak topology}} generated by \emph{a family of projection mappings} $(\pi_{\beta})_{\beta \in J}$. It is \underline{\emph{\textbf{the coarest (weakest) topology} such that $(\pi_{\beta})_{\beta \in J}$ are \textbf{continuous}}}.

\emph{\textbf{A typical element of the basis}} from \emph{the product topology} is \emph{\textbf{the finite intersection} of subbasis} where the \emph{index is different}:
\begin{align*}
\pi_{\beta_1}^{-1}(V_{\beta_1}) \xdotx{\cap} \pi_{\beta_n}^{-1}(V_{\beta_n})
\end{align*} Thus a \emph{\textbf{neighborhood}} of $x$ in \emph{\textbf{the product topology}} is \
\begin{align*}
N(x) &= \set{(x_{\alpha})_{\alpha \in J}:  x_{\beta_1} \in V_{\beta_1} \xdotx{,} x_{\beta_n} \in V_{\beta_n}}
\end{align*} where there is \emph{\textbf{no restriction}} for $\alpha \in \set{\beta_1 \xdotx{,} \beta_n}$.

Note that for \emph{\textbf{the box topology}}, a \emph{neighborhood} of $x$ is
\begin{align*}
N_b(x) &= \set{(x_{\alpha})_{\alpha \in J}:  x_{\alpha} \in U_{\alpha}, \; \forall \alpha \in J} \subset N(x)
\end{align*} Thus \underline{\emph{\textbf{the box topology}} is \emph{\textbf{finer}} than \emph{\textbf{the product topology}}}. Moreover, \emph{for \textbf{finite product}}  $\prod_{\alpha =1}^{n} X_{\alpha}$, the box topology and the product topology is the \emph{\textbf{same}}.
\end{remark}


\item \begin{definition}
If $X$ and $Y$ are topological spaces, a continuous injective map $F: X \rightarrow Y$ is called a \underline{\emph{\textbf{topological embedding}}} if it is a \emph{\textbf{homeomorphism}} onto its image $F(X) \subseteq Y$ in the subspace topology.
\end{definition}
\end{itemize}

\subsubsection{Quotient Topology}
\begin{itemize}
\item \begin{definition} (\emph{\textbf{Quotient Map}})\\
Let $X$ and $Y$ be topological spaces; let $\pi : X \rightarrow Y$ be a \emph{\textbf{surjective map}}. The map $\pi$ is said to be \underline{\emph{\textbf{a quotient map}}} provided a subset $U$ of $Y$ is \emph{\textbf{open}} in $Y$ \underline{\emph{\textbf{if and only if}}} $\pi^{-1}(U)$ is \emph{\textbf{open}} in $X$.
\end{definition}

\item \begin{remark}(\emph{\textbf{Quotient Map $=$ Strong Continuity}})\\
The condition of quotient map is \emph{\textbf{stronger}} than continuity (it is called \underline{\emph{\textbf{strong continuity}}} in some literature). 
\begin{align*}
\text{continuity}: \quad U \text{ is open in }Y & \Rightarrow \pi^{-1}(U) \text{ is open in }X\\
\text{open map}: \quad \pi(V) \text{ is open in }Y & \Leftarrow V \text{ is open in }X\\
\text{quotient map}: \quad U \text{ is open in }Y & \Leftrightarrow \pi^{-1}(U) \text{ is open in }X
\end{align*}
\emph{An equivalent condition} is to require that a subset $A$ of K be \emph{\textbf{closed}} in $Y$ \emph{if and only if} $\pi^{-1}(A)$ is \emph{\textbf{closed}} in $X$. Equivalence of the two conditions follows from equation
\begin{align*}
\pi^{-1}(Y \setminus B) = X \setminus \pi^{-1}(B).
\end{align*}
\end{remark}

\item \begin{definition} (\emph{\textbf{Saturated Set and Fiber}})\\
If $\pi: X \rightarrow Y$ is a \emph{\textbf{surjective} map}, a subset $U \subseteq X$ is said to be \underline{\emph{\textbf{saturated}}} with respect to $\pi$ if $U$ contains every set $\pi^{-1}(\{y\})$ that it \emph{\textbf{intersects}}. Thus $U$ is \emph{\textbf{saturated}} if it equals to the \textbf{\emph{entire preimage}} of its \emph{\textbf{image}}: $U =\pi^{-1}(\pi(U))$. 

Given $y \in Y$, the \underline{\emph{\textbf{fiber}}} of $\pi$ over $y$ is the set $\pi^{-1}(\{y\})$. 
\end{definition}

\item \begin{definition} (\emph{\textbf{Quotient Map via Saturated Set}})\\
A surjective map $\pi : X \rightarrow Y$ is a \underline{\emph{\textbf{quotient map}}} if $\pi$ is \emph{\textbf{continuous}} and $\pi$ maps \emph{\textbf{saturated open sets}} of $X$ to \emph{\textbf{open sets}} of $Y$ (or \emph{saturated closed sets} of $X$ to \emph{closed sets} of $Y$).
\end{definition}


\item \begin{definition} (\emph{\textbf{Open Map} and \textbf{Closed Map}})\\
A map $f: X \rightarrow Y$ (continuous or not) is said to be an \underline{\emph{\textbf{open map}}} if for every \emph{open} subset $U \subseteq X$, the image set $f(U)$ is \emph{open} in $Y$, and a  \underline{\emph{\textbf{closed map}}} if for every \emph{closed} subset $K \subseteq X$, the image $f(K)$ is \emph{closed} in Y . 
\end{definition}


\item \begin{definition} (\emph{\textbf{Quotient Topology}})\\
If $X$ is a space and $A$ is a set and if $\pi: X \rightarrow A$ is a \textbf{\emph{surjective}} map, then there exists \textbf{exactly one topology} $\srT$ on $A$ relative to which $\pi$ is a quotient map; it is called \underline{\emph{\textbf{the quotient topology}} induced by $\pi$}.
\end{definition}

\item \begin{definition}(\emph{\textbf{Quotient Space}})\\
Suppose $X$ is a topological space and $\sim$ is \emph{an equivalence relation} on $X$. Let $X/\sim$ denote \emph{\textbf{the set of equivalence classes}} in $X$, and let $\pi: X \rightarrow X/\sim$ be the \emph{\textbf{natural projection}} sending each \emph{point} to its \emph{equivalence class}. Endowed with \emph{\textbf{the quotient topology}} determined by $\pi$, the space $X/\sim$ is called \underline{\emph{\textbf{the quotient space}}} (or \emph{identification space}) of $X$ determined by $\pi$.
\end{definition}
\end{itemize}

\subsection{Continuous Function}
\subsubsection{Definitions}
\begin{itemize}
\item \begin{definition}
A map $F: X \rightarrow Y$ is said to be \underline{\emph{\textbf{continuous}}} if for every open subset $U \subseteq Y$, the \emph{\textbf{preimage}} $F^{-1}(U)$ is \emph{\textbf{open}} in $X$.
\end{definition}

\item \begin{remark}
\underline{\emph{\textbf{Continuity of a function}}} depends \emph{not only upon \underline{\textbf{the function $f$ itself}}}, but also \underline{\emph{on the \textbf{topologies} specified for \textbf{its domain} and \textbf{range}}}. If we wish to emphasize this fact, we can say that \emph{$f$ is \textbf{continuous relative to} specific topologies on $X$ and $Y$}.
\end{remark}

\item \begin{remark} (\emph{\textbf{Prove Continuity via Basis}})\\
If the topology of \emph{\textbf{the range space}} $Y$ is given by a \emph{\textbf{basis}} $\srB$, then to prove \emph{\textbf{continuity of $f$}} it suffices to show that \emph{the \textbf{inverse image} of every \textbf{basis element} is \textbf{open}}: The arbitrary open set $V$ of $Y$ can be written as \emph{a union of basis elements}
\begin{align*}
V &= \bigcup_{\alpha \in J}B_{\alpha}\\
\Rightarrow f^{-1}(V) &= \bigcup_{\alpha \in J}f^{-1}(B_{\alpha})
\end{align*}
\end{remark}

\item \begin{remark} (\emph{\textbf{Prove Continuity via Subbasis}})\\
If the topology on $Y$ is given by \emph{\textbf{a subbasis $\srS$}}, to prove continuity of $f$ it will even suffice to show that \emph{\textbf{the inverse image} of each \textbf{subbasis} element is \textbf{open}}: The arbitrary basis element $B$ for $Y$ can be written as \emph{\textbf{a finite intersection}} $S_1 \xdotx{\cap} S_{n}$ of subbasis elements; it follows from the equation
\begin{align*}
f^{-1}(B) &= f^{-1}(S_1) \xdotx{\cap} f^{-1}(S_n)
\end{align*}
that the inverse image of every basis element is \emph{open}.
\end{remark}


\item \begin{example} (\emph{\textbf{$\srF$-Weak Topology using Continuity Only}})\\
One can \emph{\textbf{define a topology}} \emph{\textbf{just}} based on \emph{\textbf{the notion of continuity}} from a family of functions.  Let $\srF$ be a family of functions from a set $S$ to a topological  space $(X, \srT)$. The \emph{\textbf{$\srF$-weak} (or simply \textbf{weak}) \textbf{topology}} on $S$ is \emph{the \textbf{coarest topology}} for which \emph{\textbf{all the functions} $f \in \srF$ are \textbf{continuous}}.   

The \emph{\textbf{$\srF$-weak} topology $\srT$} is generated by \emph{\textbf{subbasis $\srS$}} of the preimage sets $S = f^{-1}(U)$ where $f \in \srF$ and $U \in \srT$. And the basis of $\srT$ is \emph{the collection} of \emph{\textbf{all finite intersections}} of preimages $f^{-1}(U)$ for $f \in \srF$ and $U \in \srT$. 
\end{example}

\item \begin{proposition} (\textbf{Equivalent Definition of Continuity}) \citep{munkres2000topology} \\
Let $X$ and $Y$ be topological spaces; let $f: X \rightarrow Y$. Then the following are equivalent:
\begin{enumerate}
\item $f$ is \textbf{continuous}.
\item For every subset $A$ of $X$, one has $f(\bar{A}) \subseteq \overline{f(A)}$.
\item For every \textbf{closed} set $B$ of $Y$, the set $f^{-1}(B)$ is \textbf{closed} in $X$.
\item For \textbf{each} $x \in X$ and each \textbf{neighborhood} $V$ of $f(x)$, there is a \textbf{neighborhood} $U$ of $X$ such that $f(U) \subseteq V$.
\end{enumerate}
If the condition in (4) holds for the point $x$ of $X$, we say that \underline{\textbf{$f$ is continuous at the point $x$}}.
\end{proposition}
\end{itemize}
\subsubsection{Homemorphism}
\begin{itemize}
\item \begin{definition} (\textbf{\emph{Homemorphism}})\\
A \emph{\textbf{continuous bijective}} map $f: X \rightarrow Y$ with \emph{\textbf{continuous inverse}} 
\begin{align*}
f^{-1}: Y \rightarrow X
\end{align*}
is called a \underline{\emph{\textbf{homeomorphism}}}. If there exists a \emph{homeomorphism} from $X$ to $Y$, we say that $X$ and $Y$ are \emph{\textbf{homeomorphic}}.
\end{definition}


\item \begin{definition} (\emph{\textbf{Topological Embedding}})\\
If $X$ and $Y$ are topological spaces, a \emph{\textbf{continuous injective}} map $f: X \rightarrow Y$ is called a \underline{\emph{\textbf{topological embedding}}} if it is a \emph{\textbf{homeomorphism}} \emph{onto} its image $f(X) \subseteq Y$ \emph{in the subspace topology} (i.e. $f^{-1}|_{f(X)}: f(X) \rightarrow X$ is \emph{continuous in $Y$}).
\end{definition}

\item \begin{remark}(\emph{\textbf{Smooth Embedding}})\\
If $X$ and $Y$ are smooth manifoolds, \emph{\textbf{a smooth embedding}} $f: X \rightarrow Y$ when it is a \textbf{\emph{topological embedding}}, and it is \emph{smooth map} with \emph{injective differential} $df_{x}$ for all $x \in X$ (called a \emph{\textbf{smooth immersion}}).
\end{remark}
\end{itemize}

\subsubsection{Constructing Continuous Functions}
\begin{itemize}
\item \begin{proposition}(\textbf{Rules for Constructing Continuous Functions}). \citep{munkres2000topology}\\
Let $X$, $Y$, and $Z$ be topological spaces.
\begin{enumerate}
\item (\textbf{Constant Function}) If $f : X \rightarrow Y$ maps all of $X$ into the \textbf{single point} $y_0$ of Y, then $f$ is \textbf{continuous}.
\item (\textbf{Inclusion}) If $A$ is a subspace of $X$, the \textbf{inclusion function} $\iota : A \xhookrightarrow X$ is  \textbf{continuous}.
\item (\textbf{Composites}) If $f : X \rightarrow Y$ and $g : Y \rightarrow Z$ are continuous, then the map $g \circ f : X \rightarrow Z$ is continuous
\item (\textbf{Restricting the Domain}) If $f : X \rightarrow Y$ is \textbf{continuous}, and if $A$ is a subspace of $X$, then \textbf{the restricted function} $f|_{A}: A \rightarrow Y$ is continuous.
\item (\textbf{Restricting or Expanding the Range}) Let $f : X \rightarrow Y$ be \textbf{continuous}. If $Z$ is a \textbf{subspace} of $Y$ containing the \textbf{image} set $f(X)$, then the function $g : X \rightarrow Z$ obtained by \textbf{restricting the range} of $f$ is \textbf{continuous}.  If $Z$ is a space having $Y$ as a \textbf{subspace}, then the function $h : X \rightarrow Z$ obtained by \textbf{expanding the range} of $f$ is \textbf{continuous}.
\item (\textbf{Local Formulation of Continuity}) The map  $f : X \rightarrow Y$ is \textbf{continuous} if $X$ can be written as the \textbf{union of open sets} $U_{\alpha}$ such that $f|_{U_{\alpha}}$ is \textbf{continuous} for each $\alpha$.
\end{enumerate}
\end{proposition}

\item \begin{theorem} (\textbf{The Pasting Lemma / Gluing Lemma}). \citep{munkres2000topology} \\ 
Let $X = A \cup B$, where $A$ and $B$ are \textbf{closed} in $X$. Let $f : A \rightarrow Y$ and $g : B \rightarrow Y$ be \textbf{continuous}. If $f(x) = g(x)$ for \textbf{every} $x \in A \cap B$, then $f$ and $g$ combine to give a \textbf{continuous function} $h : X \rightarrow Y$, defined by setting $h|_{A} = f$, and $h|_{B} = g$.
\end{theorem}

\item \begin{remark}
The set $A$ and $B$ can be open sets, and the gluing lemma comes ``\emph{\textbf{Local Formulation of Continuity}}".
\end{remark}

\item \begin{remark}
Notice the condition for \emph{the gluing lemma}:
\begin{enumerate}
\item The domain $X$ is a union of two \emph{\textbf{closed sets (or open sets)}} $A$ and $B$
\item The two functions $f$ and $g$ are \emph{\textbf{continuous}} each of closed domain sets, respectively
\item $f$ and $g$ \emph{\textbf{agree on the intersection}} of two sets $A \cap B$.
\end{enumerate}
\end{remark}

\item \begin{theorem} (\textbf{Maps into Products}). \citep{munkres2000topology}\\
Let $f : A \rightarrow X \times Y$ be given by the equation
\begin{align*}
f(a)  &= (f_1(a), f_2(a)).
\end{align*} Then $f$ is \textbf{continuous} if and only if the functions
\begin{align*}
f_1: A \rightarrow X\quad \text{ and }\quad f_2: A \rightarrow Y
\end{align*} 
are \textbf{continuous}. The maps $f_1$ and $f_2$ are called \underline{\textbf{the coordinate functions}} of $f$.
\end{theorem}
\end{itemize}

\subsection{Metric Topology}
\begin{itemize}
\item \begin{definition} (\emph{\textbf{Metric Space}})\\
A \emph{\textbf{metric space}} is a set $M$ and a real-valued function $d(\cdot , \cdot): M \times M \rightarrow \bR$  which satisfies:
\begin{enumerate}
\item (\emph{\textbf{Non-Negativity}}) $d(x, y) \ge 0$
\item (\emph{\textbf{Definiteness}}) $d(x, y) = 0$ if and only if $x = y$
\item (\emph{\textbf{Symmetric}}) $d(x, y) = d(y, x)$
\item (\emph{\textbf{Triangle Inequality}}) $d(x, z) \le d(x, y) + d(y, z)$
\end{enumerate} The function $d$ is called a \underline{\emph{\textbf{metric}}} on $M$. The metric space $M$ equipped with metric $d$ is denoted as $(M, d)$.
\end{definition}

\item \begin{definition} (\emph{\textbf{$\epsilon$-Ball}})\\
Given a metric $d$ on $X$, the number $d(x, y)$ is often called \emph{the \textbf{distance} between $x$ and $y$ in the metric $d$}. Given $\epsilon > 0$, consider the set
\begin{align*}
B_{d}(x, \epsilon) &= \set{y: d(x,y) < \epsilon}
\end{align*}
of all points $y$ whose distance from $x$ is less than $\epsilon$. It is called \underline{\emph{\textbf{the $\epsilon$-ball centered at $x$}}}. Sometimes we omit the metric $d$ from the notation and write this ball simply as $B(x, \epsilon)$, when no confusion will arise.
\end{definition}

\item \begin{definition} (\emph{\textbf{Metric Topology}})\\
If $d$ is a \emph{metric} on the set $X$, then \emph{the collection of all $\epsilon$-balls $B_{d}(x, \epsilon)$}, for $x \in X$ and $\epsilon > 0$, is a \emph{\textbf{basis}} for a \emph{topology} on $X$, called \emph{\underline{\textbf{the metric topology}} induced by $d$}.
\end{definition}

\item \begin{definition} (\textbf{\emph{Metrizability}})\\
If $X$ is a topological space, $X$ is said to be \underline{\emph{\textbf{metrizable}}} if \emph{there exists a metric} $d$ on the set $X$ that \emph{induces the topology} of $X$. \underline{\emph{\textbf{A metric space}}} is \emph{a metrizable space} $X$ together with a specific metric $d$ that \emph{gives the topology of $X$}.
\end{definition}


\item \begin{theorem} (\textbf{$\epsilon$-$\delta$ Definition of Continuous Function in Metric Space}). \citep{munkres2000topology} \\
Lei $f: X \rightarrow Y$; let $X$ and $Y$ be \textbf{metrizable} with metrics $d_x$ and $d_y$, respectively. Then \textbf{continuity} of $f$ is \textbf{equivalent} to the requirement that given $x \in X$ and given $\epsilon > 0$, there exists $\delta > 0$ such that
\begin{align*}
d_x(x, y) < \delta \Rightarrow d_{y}(f(x), f(y)) < \epsilon.
\end{align*}
\end{theorem}

\item \begin{remark}
To use $\epsilon$-$\delta$ definition, \emph{both \textbf{domain} and \textbf{codomain}} need to be \emph{\textbf{metrizable}}.
\end{remark}

\item \begin{lemma}(\textbf{The Sequence Lemma}). \citep{munkres2000topology}\\
Let $X$ be a topological space; let $A \subseteq X$. If there is a sequence of points of $A$ \textbf{converging} to $x$, then $x \in \bar{A}$; the \textbf{converse} holds if $X$ is \textbf{metrizable}.
\end{lemma}

\item \begin{proposition}
Let $f: X \rightarrow Y$. If the function $f$ is \textbf{continuous}, then for every \textbf{convergent} sequence $x_n \rightarrow x$ in $X$, the sequence $f(x_n)$ \textbf{converges} to $f(x)$. The \textbf{converse} holds if $X$ is \textbf{metrizable}.
\end{proposition}

\item \begin{remark}
To show the converse part, i.e. ``\emph{if $x_n \rightarrow x \Rightarrow f(x_n) \rightarrow f(x)$ then $f$ is continuous}", we just need the space $X$ to be \emph{\textbf{first countable}}. That is, at each point $x$, there is \emph{\textbf{a countable collection}} $(U_{n})_{n \in \bZ_{+}}$ of \emph{\textbf{neighborhoods}} of $x$ such that any neighborhood $U$ of $x$ \emph{contains} at least one of the sets $U_n$.
\end{remark}

\item \begin{proposition} (\textbf{Arithmetic Operations of Continuous Functions}).\\
If $X$ is a topological space, and if $f, g : X \rightarrow Y$ are continuous functions, then $f + g$,  $f - g$, and $f \cdot g$ are continuous. If $g(x) \neq 0$ for all $x$, then $f/g$ is continuous.
\end{proposition}

\item \begin{definition} (\emph{\textbf{Uniform Convergence}})\\
Let $f_n : X \rightarrow Y$ be a sequence of functions from the \textbf{\emph{set}} $X$ to \emph{\textbf{the metric space}} $Y$. Let $d$ be the metric for $Y$. We say that the sequence $(f_n)$ \underline{\emph{\textbf{converges uniformly}}} to the function $f: X \rightarrow Y$ if given $\epsilon > 0$, there exists an integer $N$ such that
\begin{align*}
d(f_n(x), f(x)) < \epsilon
\end{align*}
for all $n > N$ and \textbf{\emph{all $x$ in $X$}}.
\end{definition}

\item \begin{theorem} (\textbf{Uniform Limit Theorem}). \citep{munkres2000topology}\\
Let $f_n : X \rightarrow Y$ be a sequence of  \textbf{continuous} functions from the \textbf{topological} space $X$ to the \textbf{metric space} $Y$. If $(f_n)$ converges
\textbf{uniformly} to $f$, then $f$ is \textbf{continuous}.
\end{theorem}
\end{itemize}

\subsection{Connectedness and Local Connectedness}
\begin{itemize}
\item  \begin{remark}
\emph{\textbf{Connectedness}} and \emph{\textbf{compactness}} are basic \emph{\textbf{topological properties}}. Both of them are defined based on a collection of open subsets. 
\begin{enumerate}
\item \emph{\textbf{Connectedness}} is a \emph{\textbf{global topological property}}: a topological space is \emph{connected} if it cannot be partitioned by two \emph{disjoint nonempty open} subsets. \emph{Connectedness} reveals the information of \emph{\textbf{entire space}} \emph{not just within a neighborhood}.  Connectedness is \emph{\textbf{compatible}} with the \emph{\textbf{continuity}} of functions as it implies \emph{\textbf{the intermediate value theorem}}, which in turn, can be used to construct \emph{inverse function}.  Moreover, \emph{connectedness} defines \emph{\textbf{an equivalence relationship}} which allows a \emph{\textbf{partition}} of the space into \emph{\textbf{components}}. 
\item \emph{\textbf{Connectedness}} is a \emph{\textbf{local-to-global topological property}}: a topological space is \emph{compact} if every open cover have a finite sub-cover. Using \emph{\textbf{finite sub-cover}}, \emph{\textbf{local properties}} defined \emph{within each neighborhood} can be \emph{\textbf{generalized globally}}  to entire space. Concept of functions that are closely related to compactness is \emph{\textbf{the uniformly continuity}} and \emph{\textbf{the maximum value theorem}}. The compactness allows us to drop dependency on each individual point $x$.
\end{enumerate}
Compared to \emph{connectedness}, \emph{\textbf{compactness}} is usually a \emph{\textbf{strong condition}} on the topological space.
\end{remark}




\item \begin{definition}(\emph{\textbf{Separation} and \textbf{Connectedness}})\\
Let $X$ be a topological space. A \emph{\textbf{separation}} of $X$ is a pair $U$, $V$ of \emph{\textbf{disjoint} \textbf{nonempty} \textbf{open} subset}s of $X$ whose union is $X$. 

The space $X$ is said to be \underline{\emph{\textbf{connected}}} if there \emph{does not exist a separation} of $X$.
\end{definition}

\item  \begin{definition} 
Equivalently, $X$ is \emph{\textbf{connected}} if and only if the only subsets of $X$ that are \emph{\textbf{both open and closed}} are $\emptyset$
and $X$ itself.
\end{definition}

\item \begin{definition}
Recall that a topological space $X$ is
\begin{itemize}
\item \underline{\emph{\textbf{connected}}} if there do not exist two \emph{disjoint}, \emph{nonempty}, \emph{open} subsets of $X$ whose union is $X$;
\item \underline{\emph{\textbf{path-connected}}} if every pair of points in $X$ can be \emph{\textbf{joined by a path}} in $X$, and
\item \emph{\textbf{locally path-connected}} if $X$ has a \emph{\textbf{basis}} of \emph{path-connected open subsets}.
\end{itemize}
\end{definition}

\item \begin{theorem} (\textbf{Intermediate Value Theorem}).  \citep{munkres2000topology}\\
Let $f : X \rightarrow Y$ be a \textbf{continuous} map, where $X$ is a \textbf{connected} space and $Y$ is an ordered set in the \textbf{order topology}. If $a$ and $b$ are two points of $X$ and if $r$ is a point of $Y$ lying between $f(a)$ and $f(b)$, then there \textbf{exists} a point $c$ of X such that $f(c) = r$.
\end{theorem}

\item  \emph{\textbf{Concepts Related to Connectedness}}
\[
  \begin{tikzcd}
 \text{\emph{\textbf{path-connected} space}}  \arrow{d}{\text{only one}}   \arrow{r}{} &  \text{\emph{\textbf{connected} space}}  \arrow{d}{\text{only one}} \\
   \text{\emph{\textbf{path components}}}   \arrow{r}{\subseteq }  &\arrow[l, swap, bend left, dashed, "\text{if locally path connected}"]  \text{\emph{\textbf{components}}}\\
   & & \\
   \text{\emph{\textbf{path components} of every open subset}}   \arrow{r}{\subseteq } \arrow{uu}{\subseteq } & \text{\emph{\textbf{components}  of every open subset}} \arrow{uu}{\subseteq }\\
  \text{\emph{\textbf{locally path connected} space}} \arrow[u, "\text{open}"]  & \text{\emph{\textbf{locally connected} space}} \arrow[u, swap, "\text{open}"].
  \end{tikzcd}
\] 
\end{itemize}

\subsection{Compactness and Local Compactness}
\begin{itemize}
\item  \emph{\textbf{Concepts Related to Compactness}}
\[
  \begin{tikzcd}
   &  \arrow[dl, swap, bend right, dashed, "\text{metrizable}"]   \text{\emph{\textbf{limit point compact}}} \arrow{r}{} & \arrow[l, bend right, dashed, swap,  "\text{metrizable}"]  \text{\emph{\textbf{sequential compact}}}\\
  \text{\emph{\textbf{compact}}}  \arrow{ur}{}  \arrow{dr}{} \arrow[ddr, bend right] \arrow[dddr, bend right] & &\\
  & \arrow[bend right, swap]{ul}{\text{closure}}  \text{\emph{\textbf{precompact}}}    & \arrow{l}{\text{precompact basis}}   \text{\emph{\textbf{locally compact}}}  \arrow[ull,  dashed,  swap, "X \bigcup \set{\infty} \simeq C"]     \arrow[ddl,   bend left,  "\text{second-countable $+$ Hausdorff}"]  \\
 & \text{\emph{\textbf{Lindel{\"o}f}}} \arrow{d}{\text{regular}} & &\\
 & \text{\emph{\textbf{paracompact}}} &  &
  \end{tikzcd}
\] 

\item \begin{definition} (\emph{\textbf{Covering of Set} and \textbf{Open Covering of Topological Set}})\\
\emph{A collection $\srA$ of subsets} of a space $X$ is said to \underline{\emph{\textbf{cover}} $X$,} or to be \emph{a \underline{\textbf{covering}} of $X$}, if the union of the elements of $\srA$ is equal to $X$. 

It is called an \underline{\emph{\textbf{open covering of $X$}}} if its elements are \emph{open subsets} of $X$.
\end{definition}

\item \begin{definition} (\emph{\textbf{Compactness}})\\
A topological space $X$ is said to be \underline{\emph{\textbf{compact}}} if \emph{every open covering} $\srA$ of $X$ contains a \emph{\textbf{finite} subcollection} that also \emph{covers} $X$.
\end{definition}

\item To prove \emph{compactness}, the following property is useful:
\begin{definition} (\emph{\textbf{Finite Intersection Property}})\\
\emph{A collection $\srC$ of subsets} of $X$ is said to have \underline{\emph{\textbf{the finite intersection property}}} if for \emph{every finite subcollection}
\begin{align*}
\{C_1 \xdotx{,} C_n\}
\end{align*}
 of $\srC$, the \emph{\textbf{intersection}} $C_1 \xdotx{\cap} C_n$ is \emph{\textbf{nonempty}}.
\end{definition}

\item \begin{proposition} (\textbf{Equivalent Definition of Compactness}) \citep{munkres2000topology} \\
Let $X$ be a topological space. Then $X$ is \textbf{compact} \textbf{if and only if} for every collection $\srC$ of \textbf{closed} sets in $X$ having \textbf{the finite intersection property}, the intersection $\bigcap_{C\in \srC}C$ of all the elements of $\srC$ is \textbf{nonempty}.
\end{proposition}

\item \begin{definition}
If $X$ and $Y$ are topological spaces, a map $F: X \rightarrow Y$ (continuous or not) is said to be \underline{\emph{\textbf{proper}}} if for every \textbf{\emph{compact}} set $K \subseteq Y$, the \emph{\textbf{preimage}} $F^{-1}(K)$ is \emph{\textbf{compact}}.
\end{definition}

\item \begin{corollary} (\textbf{Closed Interval in Real Line is Compact})\citep{munkres2000topology}\\
Every \textbf{closed interval} in $\bR$ is \textbf{compact}.
\end{corollary}

\item \begin{proposition}  (\textbf{Closed and Bounded in Euclidean Metric $=$ Compact})\citep{munkres2000topology}\\
A subspace $A$ of $\bR^n$ is \textbf{compact} if and only if it is \underline{\textbf{closed} and is \textbf{bounded}} in the \textbf{\underline{euclidean} metric} $d$ or the \textbf{square metric} $\rho$
\end{proposition}

\item \begin{theorem} (\textbf{Extreme Value Theorem}). \citep{munkres2000topology} \\
Let $f : X \rightarrow Y$ be \textbf{continuous}, where $Y$ is an \textbf{ordered set} in the order topology. If $X$ is \textbf{compact}, then there exist points $c$ and $d$ in $X$ such that $f(c) \le f(x) \le f (d)$ for every $x \in X$.
\end{theorem}


\item \begin{definition} (\emph{\textbf{Uniform Continuity}})\\
A function $f: (X, d_X) \rightarrow (Y, d_Y)$ is said to be \underline{\emph{\textbf{uniformly continuous}}} if given $\epsilon > 0$, there is a $\delta > 0$ such that for every pair of points $x_0$, $x_1$ of $X$,
\begin{align*}
d_X(x_0, x_1) < \delta \quad \Rightarrow \quad d_Y(f(x_0), f(x_1)) < \epsilon. 
\end{align*}
\end{definition}

\item \begin{theorem} (\textbf{Uniform Continuity Theorem}). \citep{munkres2000topology} \\
 Let $f: X \rightarrow Y$ be a \textbf{continuous} map of the \textbf{compact} metric space $(X, d_X) $ to the metric space $(Y, d_Y)$. Then $f$ is \textbf{uniformly continuous}.
\end{theorem}


\item \begin{definition} (\emph{\textbf{Limit Point Compactness}})\\
A space $X$ is said to be \underline{\emph{\textbf{limit point compact}}} if \emph\textbf{{every infinite subset}} of $X$ has a \emph{\textbf{limit point}}.
\end{definition}

\item \begin{proposition}(\textbf{Compactness $\Rightarrow$ Limit Point Compactness}) \citep{munkres2000topology}\\
\textbf{Compactness} implies \textbf{limit point compactness}, but \textbf{not conversely}.
\end{proposition}

\item \begin{example}(\textbf{Limit Point Compactness $\not\Rightarrow$ Compactness}) \\
Let $Y$ consist of \emph{\textbf{two points}}; give $Y$ the topology consisting of $Y$ and the empty set. Then the space $X = \bZ_{+} \times Y$ is \textbf{\emph{limit point compact}}, for \emph{every nonempty subset of $X$ has a \textbf{limit point}}. It is \emph{\textbf{not compact}}, for the \emph{covering} of $X$ by the open sets $U_n = \set{n} \times Y$ has \emph{no finite subcollection covering} $X$. \qed
\end{example}

\item \begin{definition} (\emph{\textbf{Sequential Compactness}})\\
Let $X$ be a topological space. If $(x_n)$ is a \emph{sequence} of points of $X$, and if
\begin{align*}
n_1 < n_2 < \ldots < n_i < \ldots
\end{align*}
is an increasing sequence of positive integers, then the sequence $(y_i)$ defined by setting $y_i = x_{n_i}$ is called a \emph{\textbf{subsequence}} of the sequence $(x_n)$. 

The space $X$ is said to be \underline{\emph{\textbf{sequentially compact}}} if \emph{every sequence of points of $X$ has a \textbf{convergent subsequence}}.
\end{definition}

\item \begin{theorem} (\textbf{Equivalent Definitions of Compactness in Metric Space}) \citep{munkres2000topology}\\
Let $X$ be a \textbf{metrizable space}. Then the following are \textbf{equivalent}:
\begin{enumerate}
\item $X$ is \textbf{compact}.
\item $X$ is \textbf{limit point compact}.
\item $X$ is \textbf{sequentially compact}.
\end{enumerate}
\end{theorem}

\item \begin{definition}
A topological space $X$ is said to be \underline{\emph{\textbf{locally compact}}} if every point has a \emph{\textbf{neighborhood}} contained in a \emph{\textbf{compact subset}} of $X$. 

A subset of $X$ is said to be \emph{\textbf{precompact}} in $X$ if its \emph{\textbf{closure}} in $X$ is \emph{compact}.
\end{definition}



\item If $X$ is \emph{not a compact Hausdorff space}, then \emph{under what conditions} is $X$ \emph{homeomorphic} with \emph{a \textbf{subspace} of a compact Hausdorff space} ?

\begin{theorem} (\textbf{Unique One-Point Compactification}) \citep{munkres2000topology}\\
Let $X$ be a space. Then $X$ is \underline{\textbf{locally compact Hausdorff}} if and only if there exists a space $Y$ satisfying the following conditions:
\begin{enumerate}
\item $X$ is a subspace of $Y$.
\item The set $Y \setminus X$ consists of \textbf{a single point} (which is the limit point of $X$).
\item $Y$ is a \textbf{compact Hausdorff} space.
\end{enumerate} 
If $Y$ and $Y'$ are two spaces satisfying these conditions, then there is a \textbf{homeomorphism} of $Y$ with $Y'$ that equals \textbf{the identity map} on $X$.
\end{theorem}

\item \begin{definition} (\emph{\textbf{One-Point Compactification}})\\
If $Y$ is a \emph{\textbf{compact Hausdorff}} space and $X$ is a proper \emph{subspace} of $Y$ whose \emph{\textbf{closure}} equals $Y$, then $Y$ is said to be a \underline{\textbf{\emph{compactification}}} of $X$. 

If $Y\setminus X$ equals \emph{a single point}, then $Y$ is called \underline{\textbf{\emph{the one-point compactification}}} of $X$.
\end{definition}

\item \begin{proposition} (\textbf{Locally Compact Hausdorff $=$ Precompact Basis}) \citep{munkres2000topology} \\
Let $X$ be a \textbf{Hausdorff} space. Then $X$ is \textbf{locally compact} \textbf{if and only if} given $x$ in $X$, and given a neighborhood $U$ of $x$, there is a neighborhood $V$ of $x$ such that $\bar{V}$ is \textbf{compact} and $\bar{V} \subseteq U$.
\end{proposition}

\item \begin{corollary} (\textbf{Closed or Open Subspace}) \citep{munkres2000topology} \\
Let $X$ be locally compact Hausdorff; let $A$ be a subspace of $X$. If $A$ is \textbf{closed} in $X$ or \textbf{open} in $X$, then $A$ is locally compact.
\end{corollary}

\item \begin{corollary}  \citep{munkres2000topology} \\
A space $X$ is \textbf{homeomorphic} to an \textbf{open} subspace of a \textbf{compact Hausdorff} space \textbf{if and only if} $X$ is \textbf{locally compact Hausdorff}.
\end{corollary}

\item For a \emph{\textbf{Hausdorff space}} $X$,  the following are equivalent:
\begin{enumerate}
\item $X$ is \emph{\textbf{locally compact}}.
\item Each point of $X$ has a \emph{\textbf{precompact}} neighborhood. 
\item $X$ has a basis of \emph{\textbf{precompact}} open subsets.
\end{enumerate}

\item \begin{theorem} (\textbf{Tychonoff Theorem}). \citep{munkres2000topology} \\
 An \underline{\textbf{arbitrary product}} of \textbf{compact} spaces is \textbf{compact} in the \underline{\textbf{product topology}}.
\end{theorem}
\end{itemize}

\subsection{Countability and Separability}
\subsubsection{Countability Axioms}
\begin{itemize}
\item \emph{\textbf{Concepts Related to Countablity Axioms}}
\[
  \begin{tikzcd}
     &  \text{\emph{\textbf{first-countable} space}} &\arrow{l}{} \text{\emph{\textbf{metrizable} space}}\\
   \text{\emph{\textbf{second-countable} space}}  \arrow[ur, bend left] \arrow{r}{} \arrow[dr]  & \arrow[l, bend right, dashed, "metrizable"] \text{\emph{\textbf{separable} space}} &\\
     &  \arrow[ul, dashed,  bend left,  "metrizable"] \text{\emph{\textbf{Lindel{\"o}f} space}}&.
  \end{tikzcd}
\] 

\item \begin{definition} (\emph{\textbf{Countability}})\\
A topological space $X$ is said to be 
\begin{enumerate}
\item \emph{\textbf{first-countable}} if there is a \emph{\textbf{countable neighborhood basis}} at each point, 
\item \underline{\emph{\textbf{second-countable}}} if there is \emph{\textbf{a countable basis}} for its topology.
\end{enumerate}
\end{definition}

\item \begin{proposition} (\textbf{Limit Point Detected by Convergent Sequence}) \citep{munkres2000topology}\\
Let $X$ be a topological space.
\begin{enumerate}
\item Let $A$ be a subset of $X$. If there is a sequence of points of $A$ converging to $x$, then $x \in \bar{A}$; the \textbf{converse} holds if $X$ is \textbf{first-countable}.
\item Let $f : X \rightarrow Y$. If $f$ is continuous, then for every convergent sequence $x_n \rightarrow x$ in $X$, the sequence $f(x_n)$ converges to $f(x)$. The \textbf{converse} holds if X is \textbf{first-countable}.
\end{enumerate}
\end{proposition}

\item \begin{definition} (\emph{\textbf{Dense Subset}})\\
A subset $A$ of a space $X$ is said to be \underline{\emph{\textbf{dense}}} in $X$ if $\bar{A}=X$. (That is, \emph{every point in $X$ is a limit point of $A$.})
\end{definition}

\item \begin{definition} (\emph{\textbf{Separability}})\\
A topological space $X$ is called \underline{\emph{\textbf{separable}}} if and only if it has a \emph{\textbf{countable dense set}}.
\end{definition}

\item \begin{definition} (\emph{\textbf{Lindel{\"o}f Space}})\\
A space for which \emph{every open covering} contains \emph{a \textbf{countable} subcovering} is called a \underline{\emph{\textbf{Lindel{\"o}f space}}}. 
\end{definition}

\item \begin{proposition} (\textbf{Properties of Second-Countability}) \citep{munkres2000topology}\\
Suppose that $X$ has a \textbf{countable basis}. Then:
\begin{enumerate}
\item Every \textbf{open covering} of $X$ contains a \textbf{countable} subcollection covering $X$. ($X$ is \textbf{Lindel{\"o}f space})
\item There exists a \textbf{countable} subset of $X$ that is \textbf{dense} in $X$. ($X$ is \textbf{separable})
\end{enumerate}
\end{proposition}

\item \begin{proposition} (\textbf{Metric Space Countablility and Separablility})
\begin{enumerate}
\item Every \textbf{metric space} is \textbf{first countable}.
\item A metric space is \textbf{second countable} if and only if it is \textbf{separable}.
\item Any \textbf{second countable} topological space is \textbf{separable}.
\end{enumerate}
\end{proposition}
\end{itemize}

\subsubsection{Separability Axioms}
\begin{itemize}
\item \emph{\textbf{Concepts Related to Separation Axioms}}
\[
  \begin{tikzcd}
   \text{\emph{\textbf{$T_1$} space}}  & &\arrow{lld}{} \text{\emph{\textbf{metrizable} space}} \arrow{d}{} \arrow[dddd, shift left = 9ex, bend left, ""]  \\
   \arrow{u}{} \text{\emph{\textbf{Hausdorff} space}} &\arrow{l}{} \text{\emph{\textbf{regular} space}} \arrow[ur, leftrightarrow, bend left, "\text{ $+$ countable locally finite basis}"] \arrow[dddr,  bend right,"\text{Lindel\"of}"] &\arrow{l}{} \arrow{dl}{\text{Urysohn lemma}} \text{\emph{\textbf{normal} space}}  \\
  &  \arrow{u}{} \text{\emph{\textbf{completely regular} space}}   &  \\
   & &  \text{\emph{\textbf{compact Hausdorff} space}} \arrow{uu}{} \arrow{d}{} \\
   & \text{\emph{\textbf{locally compact Hausdorff} space}} \arrow{uu}{}  \arrow[dashed, swap]{ur}{\text{\emph{compactification}}} & \text{\emph{\textbf{paracompact Hausdorff} space}} \arrow[bend right, shift right = 8ex]{uuu}{}
  \end{tikzcd}
\] 

\item \begin{definition} (\textbf{\emph{Separation Axioms}})
\begin{enumerate}
\item A topological space is called a \underline{\emph{\textbf{$T_1$ space}}} if and only if for all $x$ and $y$, $x\neq y$, there is an \emph{\textbf{open set}} $U$ with $y \in U$, $x \not\in U$. 

Equivalently, a space is $T_1$ \emph{if and only if} $\{x\}$ is \emph{\textbf{closed}} for each $x$.

\item A topological space is called \underline{\emph{\textbf{Hausdorff}} (or $T_2$)} if and only if for all all $x$ and $y$, $x\neq y$, there are \emph{\textbf{open sets}}  $U$,  $V$ such that $x \in U$, $y \in V$, and $U \cap V = \emptyset$.

\item A topological space is called \underline{\emph{\textbf{regular}} (or $T_3$)} if and only if it is $T_1$ and for all $x$ and $C$, \emph{\textbf{closed}}, with $x \not\in C$, there are \emph{\textbf{open sets}} $U$, $V$ such that $x \in U$, $C \subset V$, and $U \cap V = \emptyset$. 

Equivalently, a space is $T_3$ if \emph{the \textbf{closed neighborhoods} of any point are a \textbf{neighborhood base}}.

\item A topological space is called \underline{\emph{\textbf{normal}} (or $T_4$)} if and only if it is $T_1$ and for all $C_1$, $C_2$, \emph{\textbf{closed}}, with $C_1 \cap C_2 = \emptyset$, there are \emph{\textbf{open sets}} $U$, $V$ with $C_1 \subset U$,  $C_2 \subset V$, and $U \cap V = \emptyset$.
\end{enumerate}
\end{definition}

\item \begin{proposition}
\begin{align*}
T_4 \Rightarrow T_3 \Rightarrow T_2 \Rightarrow T_1
\end{align*}
\end{proposition}

\item \begin{proposition} (\underline{\textbf{Limit Point in $T_1$ Axiom}}). \citep{munkres2000topology} \\
Let $X$ be a space satisfying the $T_1$ axiom; let $A$ be a subset of $X$. Then the point $x$ is \textbf{a limit point} of $A$ if and only if every \textbf{neighborhood} of $x$ contains \textbf{infinitely many points} of $A$.
\end{proposition}

\item \begin{proposition} (\underline{\textbf{Limit Point is Unique in Hausdorff Space}}). \citep{munkres2000topology} \\
If $X$ is a \textbf{Hausdorff space}, then a sequence of points of $X$ \textbf{converges to at most one point} of $X$.
\end{proposition}

\item \begin{lemma}
Let $X$ be a topological space. Let one-point sets in $X$ be closed.
\begin{enumerate}
\item $X$ is \textbf{regular} if and only if given a point $x$ of $X$ and a neighborhood $U$ of $x$,
there is a \textbf{neighborhood} $V$ of $x$ such that $\bar{V} \subseteq U$.
\item $X$ is \textbf{normal} if and only if given a \textbf{closed} set $A$ and an open set $U$ containing $A$,
there is an \textbf{open set} $V$ containing $A$ such that $\bar{V}\subseteq U$.
\end{enumerate}
\end{lemma}


\item \begin{proposition}  \citep{munkres2000topology} \\
Every \underline{\textbf{locally compact Hausdorff}} space is \textbf{regular}.
\end{proposition}
\end{itemize}

\subsection{Important Results and Theorems on Normal Space}
\begin{itemize}
\item \begin{theorem} (\textbf{Regular $+$ Second-Countable $\Rightarrow$ Normal})\citep{munkres2000topology}\\
Every \underline{\textbf{regular} space with a \textbf{countable basis}} is \textbf{normal}.
\end{theorem}


\item \begin{theorem} \citep{munkres2000topology}\\
Every \underline{\textbf{metrizable}} space is \textbf{normal}.
\end{theorem}


\item \begin{theorem} \citep{munkres2000topology, reed1980methods}\\
Every \underline{\textbf{compact Hausdorff}} space $X$ is \textbf{normal}.
\end{theorem}


\item \begin{theorem} (\textbf{Urysohn Lemma}). \citep{munkres2000topology}\\
Let $X$ be a \textbf{normal} space; let $A$ and $B$ be \textbf{disjoint closed subsets} of $X$. Let $[a, b]$ be a \textbf{closed interval} in the real line. Then there exists a \textbf{continuous} map
\begin{align*}
f : X \rightarrow [a, b]
\end{align*}
such that $f(x) = a$ for \textbf{every} $x$ in $A$, and $f(x) = b$ for \textbf{every} $x$ in $B$.
\end{theorem}

\item \begin{theorem}  (\textbf{Urysohn Lemma, Locally Compact Version}). \citep{folland2013real}\\ 
Let $X$ be a \textbf{locally compact Hausdorff} space and $K \subseteq U \subseteq X$ where $K$ is \textbf{compact} and $U$ is \textbf{open}.  Then there exists a \textbf{continuous} map
\begin{align*}
f : X \rightarrow [0, 1]
\end{align*}
such that $f(x) = 1$ for \textbf{every} $x \in K$, and $f(x) = 0$ for $x$ outside a \textbf{compact subset} of $U$.
\end{theorem}


\item \begin{theorem} (\textbf{Tietze Extension Theorem}) \citep{munkres2000topology, reed1980methods}\\
Let $X$ be a \textbf{normal space}; let $A$ be a \textbf{closed subspace} of $X$.
\begin{enumerate}
\item Any \textbf{continuous} map of $A$ into the \textbf{closed interval} $[a, b]$ of $\bR$ may be \textbf{extended}
to a \textbf{continuous} map of \textbf{all of $X$} into $[a, b]$.
\item Any \textbf{continuous} map of $A$ into $\bR$ may be \textbf{extended} to a \textbf{continuous} map of \textbf{all of $X$} into $\bR$.
\end{enumerate}
\end{theorem}


\item \begin{theorem} (\textbf{Tietze Extension Theorem, Locally Compact Version}) \citep{folland2013real}\\
Let $X$ be a \textbf{locally compact Hausdorff space}; let $K$ be a \textbf{compact subspace} of $X$. If $f \in \cC(K)$ is a \textbf{continuous} map of $K$ into $\bR$,   there exists a \textbf{continuous} extension $F \in \cC(X)$ of \textbf{all of $X$} into $\bR$ such that $F|_{K} = f$. Moreover, $F$ may be taken to \textbf{vanish}\textbf{ outside a compact set}.
\end{theorem} 

\item \begin{theorem} (\textbf{The Urysohn Metrization Theorem}). \citep{munkres1975topology, folland2013real}\\
Every \textbf{second countable} \textbf{normal} space is \textbf{metrizable}.
\end{theorem}
\end{itemize}

\subsection{Nets}
\begin{itemize}
\item \begin{definition} (\emph{\textbf{Directed System of Index Set}})\\
A \underline{\emph{\textbf{directed system}}} is \emph{an index set} $I$ together with an \emph{\textbf{ordering}} $\prec$ which satisfies:
\begin{enumerate}
\item If $\alpha, \beta \in l$ then there exists $\gamma \in I$ so that $\gamma \succ \alpha$ and $\gamma \succ \beta$.
\item $\prec$  is a \textbf{\emph{partial ordering}}.
\end{enumerate}
\end{definition}

\item \begin{definition} (\textbf{\emph{Net}})\\
A \underline{\emph{\textbf{net}}} in a topological space $X$ is a mapping from a \emph{directed system} $I$ to $X$; we denote it by $\set{x_\alpha}_{\alpha \in I}$
\end{definition}

\item \begin{remark} (\emph{Net vs. Sequence})\\
\emph{\textbf{Net}} is a generalization and abstraction of \emph{\textbf{sequence}}. The directed system $I$ is \emph{\textbf{not necessarily countable}}. So $\set{x_\alpha}_{\alpha \in I}$ may not be a countable sequence. \emph{A sequence is a net with countable index set $I \subseteq \bN$}. The directed system can be any set e.g. a graph.
\end{remark}

\item \begin{definition}
If $P(\alpha)$ is a \emph{\textbf{proposition}} depending on an \emph{\textbf{index}} $\alpha$ in a \emph{directed set} $I$ we say \underline{\emph{\textbf{$P(\alpha)$ is eventually true}}} if there is a $\beta$ in $I$ with $P(\alpha)$ \emph{true} if \emph{for all} $\alpha \succ \beta$. 

We say \underline{\emph{\textbf{$P(\alpha)$ is frequently true}}} if it is \emph{\textbf{not eventually false}}, that is, if for any $\beta$ \emph{there exists} an $\alpha \succ \beta$ with $P(\alpha)$ \emph{true}.
\end{definition}

\item \begin{definition} (\emph{\textbf{Convergence}})\\
A \emph{\textbf{net}} $\set{x_\alpha}_{\alpha \in I}$  in a topological space $X$ is said to \underline{\emph{\textbf{converge}}} to a point $x \in X$ (written $x_{\alpha} \rightarrow x$) if for \textbf{\emph{any neighborhood}} $N$ of $x$, \emph{\textbf{there exists}} a $\beta \in l$ so that $x_{\alpha} \in N$ if $\alpha \succ \beta$. The point $x$ that being converged to is called \underline{\emph{\textbf{the limit point}}} of  $x_{\alpha}$.

Note that if $x_\alpha \rightarrow x$, then $x_{\alpha}$ is \emph{\textbf{\underline{eventually} in all neighborhoods of}} $x$. If $x_{\alpha}$ is \emph{\textbf{\underline{frequently} in any neighborhood of}} $x$, we say that $x$ is a \underline{\emph{\textbf{cluster point}}} of $x_{\alpha}$. 
\end{definition}

\item \begin{proposition} \citep{reed1980methods}\\
Let $A$ be a set in a topological space $X$. Then, a point $x$ is in the \textbf{closure} of $A$ if and only if there is a net $\set{x_\alpha}_{\alpha \in I}$ with $x_{\alpha} \in A$, So that $x_{\alpha} \rightarrow x$.
\end{proposition}

\item \begin{proposition} \citep{reed1980methods}
\begin{enumerate}
\item (\textbf{Continuous Function}): A function $f$ from a topological space $X$ to a topological space $Y$ is \textbf{continuous} if and only if for \textbf{every convergent net} $\set{x_\alpha}_{\alpha \in I}$ \textbf{in $X$}, with $x_{\alpha} \rightarrow x$, the net $\{f(x_{\alpha})\}_{\alpha \in I}$ \textbf{converges in $Y$} to $f(x)$.
\item (\textbf{Uniqueness of Limit Point for Hausdorff Space}): Let $X$ be a \textbf{Hausdorff space}. Then a net $\set{x_\alpha}_{\alpha \in I}$ in $X$ can have \textbf{at most one limit}; that is, if $x_{\alpha} \rightarrow x$ and $x_{\alpha} \rightarrow y$, then $x = y$.
\end{enumerate}
\end{proposition}

\item \begin{definition}
A net  $\set{x_\alpha}_{\alpha \in I}$ is a \underline{\emph{\textbf{subnet}}} of a net  $\set{y_\beta}_{\beta \in J}$ if and only if there is
a function $F: I \rightarrow J$ such that
\begin{enumerate}
\item $x_\alpha = y_{F(\alpha)}$ for each $\alpha \in I$.
\item For all $\beta' \in J$, there is an $\alpha' \in I$ such that $\alpha \succ \alpha'$ implies $F(\alpha) \succ \beta'$ (that is,
$F(\alpha)$ is \emph{\textbf{eventually} \textbf{larger} than any fixed} $\beta \in J$).
\end{enumerate}
\end{definition}

\item \begin{proposition}
A point $x$ in a topological space $X$ is a \textbf{cluster point} of a \textbf{net} $\set{x_\alpha}_{\alpha \in I}$ if and only if \textbf{some subnet} of $\set{x_\alpha}_{\alpha \in I}$ \textbf{converges} to $x$.
\end{proposition}

\item \begin{theorem} (\textbf{The Bolzano-Weierstrass Theorem}) \citep{reed1980methods} \\
A space $X$ is \textbf{compact} if and only if \textbf{every net} in $X$ \textbf{has a convergent subnet}.
\end{theorem}
\end{itemize}


\section{Topology in Function Space}
\subsection{Complete Metric Space}
\begin{itemize}
\item \begin{definition} (\emph{\textbf{Cauchy Net in Topological Vector Space}})\\
A \emph{net}  $\set{x_\alpha}_{\alpha \in I}$ in \emph{\textbf{toplogocial vector space}} $X$ is called \underline{\emph{\textbf{Cauchy}}} if the net $\set{x_{\alpha} - x_{\beta}}_{(\alpha, \beta) \in I \times I}$
\emph{\textbf{converges} to zero}. (Here $I \times I$ is \emph{\textbf{directed}} in the usual way: $(\alpha, \beta) \prec (\alpha', \beta')$ if and only if $\alpha \prec \alpha'$ and $\beta \prec \beta'$.) 
\end{definition}

\item \begin{definition} (\emph{\textbf{Completeness}})\\
A toplogocial vector space $X$ is \emph{\textbf{complete}} if every Cauchy net converges.
\end{definition}

\item \begin{proposition} (\textbf{Complete First Countable Topological Vector Space})\\
If $X$ is a \textbf{first-countable topological vector space} and every \textbf{Cauchy sequence} in $X$ converges, then every \textbf{Cauchy net} in $X$ converges.
\end{proposition}

\item \begin{proposition} (\textbf{Completeness of Euclidean Space}) \citep{munkres2000topology} \\
Euclidean space $\bR^k$ is \textbf{complete} in either of its usual \textbf{metrics}, the \textbf{euclidean metric} $d$ or the \textbf{square metric} $\rho$.
\end{proposition}

\item \begin{lemma} (\textbf{Convergence in Product Space is Weak Convergence}) \citep{munkres2000topology} \\
Let $X$ be the product space $X = \prod_{\alpha}X_{\alpha}$; let $x_n$ be a sequence of points of $X$. Then $x_n \rightarrow x$ if and only if $\pi_{\alpha}(x_n) \rightarrow  \pi_{\alpha}(x)$ for each $\alpha$.
\end{lemma}

\item \begin{proposition} (\textbf{Completeness of Countable Product Space}) \citep{munkres2000topology} \\
There is a metric for the product space $\bR^{\omega}$ relative to which $\bR^{\omega}$ is \textbf{complete}.
\end{proposition}

\item \begin{definition} (\emph{\textbf{Uniform Metric in Function Space}})\\
Let $(Y, d)$ be a metric space; let $\bar{d}(a, b) = \min\{d(a, b), 1\}$ be the \emph{\textbf{standard bounded metric}} on $Y$ derived from $d$. If $x = (x_{\alpha})_{\alpha \in J}$ and  $y = (y_{\alpha})_{\alpha \in J}$ are points of the cartesian product $Y^J$, let
\begin{align*}
\bar{\rho}(x, y) = \sup\set{\bar{d}(x_{\alpha}, y_{\alpha}): \alpha \in J}.
\end{align*}
It is easy to check that $\bar{\rho}$ is a metric; it is called \underline{\emph{\textbf{the uniform metric}}} on $Y^J$ corresponding to the metric $d$ on $Y$.

Note that \emph{\textbf{the space of all functions} $f: J \rightarrow Y$}, \emph{\textbf{denoted}} as $Y^{J}$, is a subset of the product space $J \times Y$. We can define uniform metric in the function space: if $f$, $g : J \rightarrow Y$, then
\begin{align*}
\bar{\rho}(f, g) = \sup\set{\bar{d}(f(\alpha), g(\alpha)): \alpha \in J}.
\end{align*}
\end{definition}

\item \begin{proposition} (\textbf{Completeness of Function Space  Under Uniform Metric}) \citep{munkres2000topology} \\
If the space $Y$ is \textbf{complete} in the metric $d$, then the space $Y^J$ is \textbf{complete} in the \textbf{uniform metric} $\bar{\rho}$ corresponding to $d$.
\end{proposition}

\item \begin{definition} (\emph{\textbf{Space of Continuous Functions and Bounded Functions}})\\
Let $Y^{X}$ be  \emph{the space of all functions} $f: X \rightarrow Y$, where $X$ is a \emph{topological space} and $Y$ is a \emph{metric space with metric $d$}. Denote the \emph{\textbf{subspace}}  of $Y^X$ consisting of all \emph{\textbf{continuous functions} $f$} as $\cC(X, Y)$. 

Also denote \emph{the set  of all \textbf{bounded functions}} $f: X \rightarrow Y$ as $\cB(X, Y)$. (A function $f$ is said to be \emph{\textbf{bounded}} if its image $f(X)$ is a \emph{\textbf{bounded subset}} of \emph{the metric space $(Y, d)$}.) 
\end{definition}

\item \begin{proposition}  (\textbf{Completeness of $\cC(X, Y)$ and   $\cB(X, Y)$  Under Uniform Metric}) \citep{munkres2000topology} \\
Let $X$ be a topological space and let $(Y, d)$ be a metric space. The set $\cC(X, Y)$ of \textbf{continuous} functions is \textbf{closed} in $Y^X$ under the \textbf{uniform metric}. So is the set $\cB(X, Y)$ of \textbf{bounded functions}. Therefore, if $Y$ is \textbf{complete}, these spaces are \textbf{complete} in the \textbf{uniform metric}.
\end{proposition}

\item \begin{definition} (\emph{\textbf{Sup Metric on Bounded Functions}})\\
If $(Y, d)$ is a metric space, one can define another metric \emph{on the set $\cB(X, Y)$ of \textbf{bounded functions}} from $X$ to $Y$ by the equation
\begin{align*}
\rho(x, y) = \sup\set{d(f(x), g(x)): x \in X}.
\end{align*}
It is easy to see that $\rho$ is well-defined, for the set $f(X) \cup g(X)$ is \emph{\textbf{bounded}} if both $f(X)$ and $g(X)$ are. The metric $\rho$ is called \underline{\emph{\textbf{the sup metric}}}.
\end{definition}

\item \begin{theorem} (\textbf{Existence of Completion}) \citep{munkres2000topology}\\
Let $(X, d)$ be a metric space. There is an \textbf{isometric embedding} of $X$ into a \textbf{complete} metric space.
\end{theorem}

\item \begin{definition} (\emph{\textbf{Completion}})\\
Let $X$ be a \emph{metric space}. If $h : X \rightarrow Y$ is an \textbf{\emph{isometric embedding}} of $X$ into a \emph{\textbf{complete} metric space} $Y$, then the \emph{\textbf{subspace}} $h(X)$ of $Y$ is a \emph{complete metric space}. It is called \underline{\emph{\textbf{the completion of $X$}}}.
\end{definition}

\item \begin{definition} (\emph{\textbf{Topological Complete}})\\
A space $X$ is said to be \underline{\emph{\textbf{topologically complete}}} if there \emph{exists} a metric for the \emph{topology} of $X$ relative to which $X$ is \emph{complete}.
\end{definition}

\item \begin{proposition} (\textbf{Properties of Topological Complete}) \citep{munkres2000topology}\\
The followings are properties of topological completeness:
\begin{enumerate}
\item A \textbf{closed} subspace of a topologically complete space is topologically complete.
\item A \textbf{countable product} of topologically complete spaces is topologically complete (in the \textbf{product topology}).
\item An \textbf{open} subspace of a topologically complete space is topologically complete.
\item A \textbf{$G_{\delta}$ set} in a topologically complete space is topologically complete. 
\end{enumerate}
\end{proposition}
\end{itemize}

\subsection{Compactness in Metric Spaces}
\begin{itemize}
\item \begin{remark} (\emph{\textbf{Compactness} and \textbf{Completeness}})\\
How is \emph{\textbf{compactness}} of a metric space $X$ related to \emph{\textbf{completeness}} of $X$? 

The followings is from \emph{the sequential compactness} and definition of \emph{completeness}:
\begin{proposition}
Every \textbf{compact} metric space is \textbf{complete}.
\end{proposition}
The \emph{converse} does not hold -- \emph{\textbf{a complete metric space need not be compact}}. It is reasonable to ask what \emph{\textbf{extra condition}} one needs to impose on a complete space to be assured of its compactness.
Such a condition is the one called \emph{total boundedness}.
\end{remark}

\item \begin{definition} (\emph{\textbf{Total Boundedness}})\\
A metric space $(X, d)$ is said to be \underline{\emph{\textbf{totally bounded}}} if for every $\epsilon > 0$, there is a \emph{\textbf{finite covering} of $X$ by \textbf{$\epsilon$-balls}}.
\end{definition}

\item \begin{theorem} \citep{munkres2000topology}\\
A metric space $(X, d)$ is \textbf{compact} \underline{if and only if} it is \textbf{complete} and \textbf{totally bounded}.
\end{theorem}

\item \begin{remark}
We now apply this result to find \emph{\textbf{the compact subspaces}} of the space $\cC(X, \bR^n)$, \emph{in the \textbf{uniform topology}}. We know that a subspace of $\bR^n$ is compact if and only if it is \emph{\textbf{closed}} and \emph{\textbf{bounded}}. 

One might hope that an analogous result holds for $\cC(X, \bR^n)$. \emph{\textbf{But}} it does not, even if $X$ is \emph{compact}. One needs to assume that the subspace of $\cC(X, \bR^n)$ satisfies \emph{an \textbf{additional condition}}, called \emph{\textbf{equicontinuity}}. 
\end{remark}

\item \begin{definition}  (\emph{\textbf{Equicontinuity}}) \citep{reed1980methods, munkres2000topology} \\
Let $(Y, d)$ be a \emph{metric space}. Let $\srF$ be a \emph{subset} of the function space $\cC(X, Y)$ (i.e. $f \in \srF$ is continuous). If $x_0 \in X$, \emph{the set $\srF$ of functions} is said to be \underline{\emph{\textbf{equicontinuous at $x_0$}}} if given $\epsilon >0$, there is a neighborhood $U$ of $x_0$ such that \emph{for all $x \in U$} and \underline{\emph{\textbf{all $f \in \srF$}}},
\begin{align*}
d(f(x), f(x_0)) < \epsilon.
\end{align*}
If the set $\srF$ is \emph{equicontinuous} at $x_0$ for each $x_0 \in X$, it is said simply to be \underline{\emph{\textbf{equicontinuous}}} or $\srF$ is an \underline{\emph{\textbf{equicontinuous family}}}.

We say $\srF$ is a \underline{\emph{\textbf{uniformly equicontinuous family}}} if and only if for all $\epsilon >0$, there exists $\delta > 0$ such that $d(f(x), f(x')) < \epsilon$ whenever $p(x, x') < \delta$ for all $x, x' \in X$ and \emph{\textbf{every $f \in \srF$}}.
\end{definition}

\item \begin{remark}
An \emph{equicontinuous family} of functions is \emph{a family of continuous functions}.
\end{remark}

\item \begin{remark}
\emph{\textbf{Continuity}} of the function $f$ at $x_0$ means that \emph{\textbf{given} $f$} and given $\epsilon >0$, there exists a neighborhood $U$ of $x_0$ such that $d(f(x), f(x_0)) < \epsilon$ for $x \in U$. 
\textbf{
\emph{\textbf{Equicontinuity}} of $\srF$ means that \emph{\textbf{a single neighborhood}} $U$ can be chosen that will \emph{}work for all the functions} $f$ in the collection $\srF$.
\end{remark}

\item \begin{lemma} (\textbf{Total Boundedness $\Rightarrow$ Equicontinuous}) \citep{munkres2000topology}\\ 
Let $X$ be a \textbf{space}; let $(Y, d)$ be a \textbf{metric} space. If the subset $\srF$ of $\cC(X, Y)$ is \textbf{totally bounded} under the \textbf{uniform metric} corresponding to $d$, then $\srF$ is \textbf{equicontinuous} under $d$.
\end{lemma}

\item \begin{lemma} (\textbf{Equicontinuous $+$ Compactness  $\Rightarrow$ Total Boundedness})  \citep{munkres2000topology}\\ 
Let $X$ be a space; let $(Y, d)$ be a metric space; assume $X$ and $Y$ are \textbf{compact}. If the subset $\srF$ of $\cC(X, Y)$ is \textbf{equicontinuous} under $d$, then $\srF$ is \textbf{totally bounded} under the \textbf{uniform} and \textbf{sup} metrics corresponding to $d$.
\end{lemma}


%\item \begin{proposition}
%Let $f_n$ be a sequence of functions from one metric space to another with the property that the family $\{f_n\}$ is \textbf{equicontinuous}. Suppose
%that $f_n(x) \rightarrow f(x)$ \textbf{pointwise} for each $x$. Then $f$ is \textbf{continuous}.
%\end{proposition}
%
%\item We see that \emph{\textbf{pointwise convergence}} on a \emph{\textbf{dense set}} combined with \emph{\textbf{equicontinuity}} implies \emph{\textbf{pointwise convergence everywhere}}.
%\begin{proposition} \citep{reed1980methods}\\
%Let $\{f_n\}$ be an \textbf{equicontinuous family} of functions from one metric space $(X, p)$ to another $(Y, d)$ with $Y$ complete. Suppose that for a \textbf{dense} set $D \subseteq X$, we know $f_{n}(x)$ converges for all $x \in D$. Then $f_{n}(x)$ converges for all $x \in X$.
%\end{proposition}
%
%\item The following shows that uniformly equicontinuous combined with pointwise convergence implies uniform convergence.
%\begin{proposition} \citep{reed1980methods}\\
%Let $\{f_n\}$ be a \textbf{uniformly equicontinuous family} of functions on $[0, 1]$. Suppose that $f_n(x) \rightarrow f(x)$ for each $x$ in $[0, 1]$. Then $f_n(x) \rightarrow f(x)$  \textbf{uniformly} in $x$.
%\end{proposition}
%
%\item \begin{remark}
%For functions on $[0, 1]$, \emph{every \textbf{equicontinuous family}} is \emph{\textbf{uniformly equicontinuous}}.
%\end{remark}

\item \begin{definition} (\emph{\textbf{Pointwise Bounded}})\\
If $(Y, d)$ is a \emph{metric space}, a \emph{subset} $\srF$ of $\cC(X, Y)$ is said to be \underline{\emph{\textbf{pointwise bounded}}} under $d$ if for each $x \in X$, the subset
\begin{align*}
F_{a} &= \set{f(a): f\in \srF}
\end{align*}
of $Y$ is \emph{\textbf{bounded}} under $d$.
\end{definition}

\item \begin{theorem} (\textbf{Ascoli’s Theorem, Classical Version}). \citep{munkres2000topology}\\
Let $X$ be a \textbf{compact} space; let $(\bR^n , d)$ denote euclidean space in either the square metric or the euclidean metric; give $\cC(X, \bR^n)$ the corresponding \textbf{uniform topology}. A subspace $\srF$ of $\cC(X, \bR^n)$ has \underline{\textbf{compact closure}} \textbf{if and only if} $\srF$ is \underline{\textbf{equicontinuous}} and \underline{\textbf{pointwise bounded}} under $d$.
\end{theorem}

\item \begin{corollary}
Let $X$  be  \textbf{compact}; let $d$ denote either the square metric or the euclidean metric on $\bR^n$; give $\cC(X, \bR^n)$ the corresponding \textbf{uniform topology}. A subspace $\srF$ of $\cC(X, \bR^n)$ is \underline{\textbf{compact}} \textbf{if and only if} it is \underline{\textbf{closed}, \textbf{bounded}} under the \underline{\textbf{sup metric} $\rho$}, and \textbf{equicontinuous} under $d$.
\end{corollary}

\item \begin{remark} (\textbf{\emph{Ascoli's Theorem, Sequence Version}}) \citep{reed1980methods}\\
\emph{Let $\{f_n\}$ be a family of \textbf{uniformly bounded equicontinuous functions} on $[0, 1]$. Then \textbf{some subsequence} $\{f_{n,m}\}$ converges \textbf{uniformly} on $[0, 1]$.}
\end{remark}

\item \begin{definition} (\emph{\textbf{Continuous Functions that Vanish At Infinity $\cC_0(X, \bR)$}}) \\
Let $X$ be a space. A subset $\cF$ of $\cC(X, \bR)$ is said to \underline{\emph{\textbf{vanish uniformly at infinity}}} if given $\epsilon > 0$, there is a \emph{\textbf{compact subspace}} $C$ of $X$ such that $\abs{f(x)} < \epsilon$  for $x \in X \setminus C$ and $f \in \cF$. 

If $\cF$ consists of a single function $f$, we say simply that \underline{\emph{\textbf{$f$ vanishes at infinity}}}. Let $\cC_0(X, \bR)$ denote \emph{the set of continuous functions $f : X \rightarrow \bR$ that \textbf{vanish at infinity}}.
\end{definition}

\item \begin{corollary} \citep{munkres2000topology}\\
Let $X$ be \underline{\textbf{locally compact Hausdorff}}; give $\cC_0(X, \bR)$ the uniform topology. A subset $\cF$ of $\cC_0(X, \bR)$ has \underline{\textbf{compact closure}}  \textbf{if and only if} it is \underline{\textbf{pointwise bounded}}, \underline{\textbf{equicontinuous}}, and \underline{\textbf{vanishes uniformly at infinity}}.
\end{corollary}
\end{itemize}



\subsection{Pointwise and Compact Convergence}
\begin{itemize}
\item \begin{remark} (\emph{\textbf{Useful Topologies on $Y^X$}})
\begin{enumerate}
\item \underline{\emph{\textbf{Uniform Topology}}}: generated by the \emph{\textbf{basis}}
\begin{align*}
B_{U}(f, \epsilon) &= \set{g \in Y^X: \sup_{x\in X}\bar{d}(f(x), g(x)) < \epsilon }
\end{align*} It corresponds to \emph{\textbf{the uniform convergence}} of $f_n$ to $f$ in $Y^X$. $\cC(X, Y)$ is \emph{\textbf{closed}} in $Y^X$ under the \emph{uniform topology}, following \emph{the Uniform Limit Theorem}.

\item  \underline{\emph{\textbf{Topology of Pointwise Convergence}}}: generated by the \emph{\textbf{basis}}
\begin{align*}
B_{U_1 \xdotx{,} U_n}(x_1 \xdotx{,} x_n, \epsilon)  &= \bigcap_{i=1}^{n}S(x_i, U_i) \\
&= \set{f \in Y^X: f(x_1) \in U_1 \xdotx{,} f(x_n) \in U_n}, \quad   1 \le n < \infty.
\end{align*} It corresponds to \emph{\textbf{the pointwise convergence}} of $f_n$ to $f$ in $Y^X$. $\cC(X, Y)$ is \emph{\textbf{not closed}} in $Y^X$ under the \emph{topology of pointwise convergence}. Note that \emph{the topology of poinwise convergence} is \emph{the \textbf{product topology} of $Y^X$}.

\item \underline{\emph{\textbf{Topology of Compact Convergence}}}: generated by the \emph{\textbf{basis}}
\begin{align*}
B_{C}(f, \epsilon) &= \set{g \in Y^X: \sup_{x \in C}d(f (x), g(x)) < \epsilon },\; C \text{ is compact set}.
\end{align*} It corresponds to \emph{\textbf{the uniform convergence}} of $f_n$ to $f$ in $Y^X$ for $x \in C$. $\cC(X, Y)$ is \emph{\textbf{closed}} in $Y^X$ under the \emph{topology of compact convergence} \emph{\textbf{if $X$ is compactly generated}}.

On $\cC(X)$, the topology of compact convergence is equal to the compact-open topology: 
 \begin{definition} (\emph{\textbf{Compact-Open Topology on Continuous Function Space}})\\
Let $X$ and $Y$ be topological spaces. If $C$ is a \emph{\textbf{compact subspace}} of $X$ and $U$ is an \emph{open} subset of $Y$, define
\begin{align*}
S(C,U) = \set{ f \in \cC(X, Y): f(C) \subseteq U}.
\end{align*}
The sets $S(C, U)$ form a \emph{\textbf{subbasis}} for a \emph{topology} on $\cC(X, Y)$ that is called \underline{\emph{\textbf{the compact-open}}} \underline{\emph{\textbf{topology}}}.
\end{definition}
\end{enumerate} 
We see that the \emph{uniform topology} is the \emph{finest} among them all and the \emph{topology of pointwise convergence} is the \emph{coarest}.
\begin{align*}
\text{\textbf{\emph{(uniform)}}} \supseteq \text{\textbf{\emph{(compact convergence)}}} \supseteq \text{\emph{\textbf{(pointwise convergence)}}}.
\end{align*}
\end{remark}

\item \begin{proposition} (\textbf{Topology of Uniform Convergence in Compact Sets}) \citep{munkres2000topology}\\
A sequence $f_n : X \rightarrow Y$ of functions converges to the function $f$ in the \textbf{topology of compact convergence} if and only if for \textbf{each compact subspace} $C$ of $X$, the sequence $f_n|_{C}$ converges \textbf{uniformly} to $f|_C$.
\end{proposition}

\item \begin{definition} (\emph{\textbf{Compactly Generated Space}})\\
A space $X$ is said to be \underline{\emph{\textbf{compactly generated}}} if it satisfies the following condition: A set $A$ is \emph{\textbf{open (or closed)}} in $X$ if $A \cap C$ is \emph{\textbf{open (or closed)}} in $C$ for each \textbf{\emph{compact subspace}} $C$ of $X$.
\end{definition}

\item \begin{lemma} \citep{munkres2000topology}\\
If $X$ is \textbf{locally compact}, or if $X$ satisfies \textbf{the first countability axiom}, then $X$ is \textbf{compactly generated}.
\end{lemma}



\item \begin{proposition}
Let $X$ and $Y$ be spaces; give $\cC(X, Y)$ the \textbf{compact-open topology}. If $f: X \times Z \rightarrow Y$ is \textbf{continuous}, then \textbf{so is} the induced function $F : Z \rightarrow \cC(X, Y)$. The \textbf{converse} holds if $X$ is \textbf{locally compact Hausdorff}.
\end{proposition}

\item \begin{theorem} (\textbf{Ascoli's Theorem, General Version}). \citep{munkres2000topology} \\
Let $X$ be a space and let $(Y, d)$ be a \underline{\textbf{metric}} space. Give $\cC(X, Y)$ the \underline{\textbf{topology of compact}} \underline{\textbf{convergence}}; let $\cF$ be a subset of $\cC(X, Y)$.
\begin{enumerate}
\item If $\cF$ is \underline{\textbf{equicontinuous}} under $d$ and the set
\begin{align*}
F_{a} &= \set{f(a): f \in \cF}
\end{align*}
has \underline{\textbf{compact closure}} for each $a \in X$, then $\cF$ is \underline{\textbf{contained} in a \textbf{compact subspace}} of $\cC(X, Y)$.
\item  The \textbf{converse} holds if $X$ is \underline{\textbf{locally compact Hausdorff}}.
\end{enumerate}
\end{theorem}


\item \begin{proposition} (\textbf{Equicontinuity $+$ Pointwise Convergence $\Rightarrow$ Compact Convergence}) \citep{munkres2000topology}\\
Let $(Y, d)$ be a metric space; let $f_n : X \rightarrow Y$ be a sequence of \textbf{continuous} functions; let $f : X \rightarrow Y$ be a function (not necessarily continuous). Suppose $f_n$ converges to $f$ in the \textbf{topology of pointwise convergence}. If $\{f_n\}$ is \textbf{equicontinuous}, then $f$ is \textbf{continuous} and $f_n$ converges to $f$ in the \textbf{topology of compact convergence}.
\end{proposition}
\end{itemize}

\subsection{Subspaces of Continuous Functions}
\begin{itemize}
\item \begin{definition} (\emph{\textbf{Subspace of Continuous Functions}})\\
Let $\cC(X) := \cC(X, \bR)$ be \emph{the space of \textbf{continuous} real-valued functions} on topological space $X$ and $\cB(X): = \cB(X, \bR)$ be \emph{the space of \textbf{bounded} real-valued functions} on $X$.
\begin{enumerate}
\item The intersection of  $\cB(X)$ and $\cC(X)$ is the space of all \emph{\underline{\textbf{bounded continuous}} functions}
\begin{align*}
\cB\cC(X) := \mathcal{BC}(X, \bR) &= \cB(X, \bR) \cap \cC(X, \bR)
\end{align*} Note that $\mathcal{BC}(X) \subseteq \cB(X)$ is a \emph{\textbf{closed subspace}}. 

\item Define the \emph{\textbf{support}} of a function $f$, $\text{supp}(f)$ as  \emph{the \textbf{smallest closed set} outside of which  $f$ vanishes}. The subset $\cC_{c}(X) \subseteq \cC(X)$ is the space of all \emph{continuous functions} with \underline{\emph{\textbf{compact support}}}
\begin{align*}
\cC_{c}(X) &= \set{f \in \cC(X, \bR): \text{supp }(f) \text{\emph{ is compact}}}.
\end{align*} Note that by \emph{Tietze Extension Theorem}, \emph{the locally compact Hausdorff space} $X$ has a rich supply of continuous functions that vanishes outside a compact set.

\item Recall also that $\cC_{0}(X)$ is the space of \emph{continuous functions} on $X$ that \underline{\emph{\textbf{vanishes at infinity}}}, i.e. for all $\epsilon >0$, $\abs{f(x)} < \epsilon$ if $x \in X\setminus C$ for some \emph{\textbf{compact subset}} $C \subseteq X$.
\begin{align*}
\cC_{0}(X) &= \set{f \in \cC(X, \bR): f \text{\emph{ vanishes at infinity}}}.
\end{align*} 
\end{enumerate}
Note that 
\begin{align*}
\cC_{c}(X)  \subseteq \cC_{0}(X) \subseteq \mathcal{BC}(X) \subseteq \cC(X)
\end{align*}
\end{definition}


\item The crucial fact about compactly generated spaces is the following:
\begin{lemma} (\textbf{Continuous Extension on Compact Generated Space})  \citep{munkres2000topology}\\
If $X$ is compactly generated, then a function $f : X \rightarrow Y$ is \textbf{continuous} if for each \textbf{compact subspace} $C$ of X, the restricted function $f |_{C}$ is \textbf{continuous}.
\end{lemma}

\item \begin{theorem} (\textbf{$\cC(X, Y)$ on Compact Generated Space})  \citep{munkres2000topology}\\
Let $X$ be a \underline{\textbf{compactly generated space}}: let $(Y, d)$ be a metric space. Then $\cC(X, Y)$ is \underline{\textbf{closed}} in $Y^X$ in the \underline{\textbf{topology of compact convergence}}.
\end{theorem}

\item Recall that 
\begin{proposition}
If $X$ is a \textbf{locally compact Hausdorf} space, $\cC(X)$ is a \textbf{closed} subspace of $\bR^{X}$ in the \textbf{topology of compact convergence}.
\end{proposition}

\item \begin{proposition} \citep{folland2013real}\\
If $X$ is a topological space, $\mathcal{BC}(X)$ is a \textbf{closed} subspace of $\cB(X)$ in the \textbf{uniform metric}; in particular, $\mathcal{BC}(X)$ is \textbf{complete}.
\end{proposition}

\item \begin{proposition}\citep{folland2013real}\\
If $X$ is a \textbf{locally compact Hausdorf} space, $\cC_{0}(X)$ is a \textbf{closure} of $\cC_{c}(X)$ in the \textbf{uniform metric}.
\end{proposition}

\item \begin{remark}
Note that $\cC_0(X) = \overline{\cC_c(X)}$ is the \emph{\textbf{completion}} of $\cC_c(X)$ \emph{under uniform metrc}.
\end{remark}
\end{itemize}

\subsection{Baire Category}
\begin{itemize}
\item \begin{remark} (\emph{\textbf{Empty Interior $=$ Complement is Dense}}) \\
Recall that if $A$ is a subset of a space $X$, the \emph{\textbf{interior}} of $A$ is defined as \emph{the union of all open sets of $X$ that are contained in $A$}. 

To say that $A$ has \underline{\emph{\textbf{empty interior}}} is to say then that \emph{\textbf{$A$ \underline{contains no open set} of $X$} other than the empty set}. \emph{\textbf{Equivalently}}, $A$ has \emph{\textbf{empty interior}} if every point of $A$ is \emph{a \textbf{limit point} of the \textbf{complement} of $A$}, that is, if \underline{\emph{the \textbf{complement} of $A$ is \textbf{dense} in $X$}}.
\begin{align*}
\mathring{A} = \emptyset \;\; \Leftrightarrow \;\; A^{c}\text{ is dense in }X
\end{align*} In \citep{reed1980methods}, if a subset $\overline{A}$ of $X$ has \emph{empty interior}, $A$ is said to be \underline{\emph{\textbf{nowhere dense}}} in $X$.
\end{remark}

\item \begin{example} 
Some  examples:
\begin{enumerate}
\item The set $\bQ$ of \emph{rationals} has \emph{\textbf{empty interior}} as a subset of $\bR$
\item The \emph{interval} $[0, 1]$ has \emph{\textbf{nonempty interior}}. 
\item The \emph{interval} $[0, 1] \times 0$ has \emph{\textbf{empty interior}} as a \emph{subset of the plane} $\bR^2$, and so does the \emph{subset} $\bQ \times \bR$.
\end{enumerate}
\end{example}

\item \begin{definition} (\emph{\textbf{Baire Space}})\\
A space $X$ is said to be a \underline{\emph{\textbf{Baire space}}} if the following condition holds:  Given  \emph{\textbf{any countable}} collection $\set{A_n}$ of \emph{\textbf{closed}} sets of $X$ each of which has \emph{\textbf{empty interior}} in $X$, their \emph{\textbf{union}}  $\bigcup_{n=1}^{\infty} A_n$ also has \emph{\textbf{empty interior}} in $X$.
\end{definition}

\item \begin{example} 
Some  examples:
\begin{enumerate}
\item The space $\bQ$ of \emph{rationals} is \emph{\textbf{not} a \textbf{Baire space}}. For each one-point set in $\bQ$ is \emph{closed and has empty interior \textbf{in $\bQ$}}; and $\bQ$ is \emph{the countable union of its one-point subsets}.
\item The space $\bZ_{+}$, on the other hand, does form \emph{a \textbf{Baire space}}. Every subset of $\bZ_{+}$ is \emph{open}, so that there exist \emph{no subsets} of $\bZ_{+}$ having \emph{empty interior}, except for the empty set. Therefore, $\bZ_{+}$ satisfies the Baire condition vacuously.
\item The \emph{interval} $[0, 1] \times 0$ has \emph{\textbf{empty interior}} as a \emph{subset of the plane} $\bR^2$, and so does the \emph{subset} $\bQ \times \bR$.
\end{enumerate}
\end{example}

\item \begin{definition}  (\emph{\textbf{Baire Category}})\\
A subset $A$ of a space $X$ was said to be of \underline{\emph{\textbf{the first category in $X$}}} if it \emph{\textbf{was contained} in the \textbf{union} of a \textbf{countable} collection of \textbf{closed} sets of $X$ having \textbf{empty interiors} in $X$}; \emph{\textbf{otherwise}}, it was said to be of \underline{\emph{\textbf{the second category in $X$}}}. 
\end{definition}

\item \begin{remark}
\emph{A space $X$ is a \textbf{Baire space} if and only if every \textbf{nonempty open} set in $X$ is of \textbf{the second category}}.
\end{remark}

\item \begin{lemma}(\textbf{Open Set Definition of Baire Space}) \citep{munkres2000topology} \\
$X$ is a \textbf{Baire space} \textbf{if and only if} given any \textbf{countable} collection $\set{U_n}$ of \textbf{open} sets in $X$, each of which is \textbf{dense} in $X$, their \textbf{intersection} $\bigcap_{n=1}^{\infty}U_n$ is also \textbf{dense} in $X$.
\end{lemma}

\item \begin{theorem} (\textbf{Baire Category Theorem}).  \citep{munkres2000topology} \\
If $X$ is a \textbf{compact Hausdorff} space or a \textbf{complete metric space}, then $X$ is a \textbf{Baire space}.
\end{theorem}

\item \begin{remark}
In other word,  neither \textbf{\emph{compact Hausdorff}} space or a \textbf{\emph{complete metric space}} is a \emph{countable union of closed subsets with empty interior (that are nowhere dense)}.
\end{remark}

\item \begin{lemma}\citep{munkres2000topology} \\
Let $C_1 \supset C_2 \supset \ldots$ be a \textbf{nested} sequence of \textbf{nonempty closed sets} in the \textbf{complete metric space} $X$. If $\text{diam }C_n \rightarrow 0$, then $\bigcap_{n}C_n  = \emptyset$.
\end{lemma}

\item \begin{lemma} \citep{munkres2000topology} \\
Any \textbf{open} subspace $Y$ of a Baire space $X$ is itself a Baire space.
\end{lemma}
\end{itemize}


\section{Locally Convex Topological Space}
\subsection{Topological Vector Space}
\begin{itemize}
\item \begin{definition} (\emph{\textbf{Topological Vector Space}})\\
A vector space $X$ endowed with a topology $\srT$ is called a \underline{\emph{\textbf{topological vector space}}}, denoted as $(X, \srT)$, if the addition $+: X\times X \rightarrow X$ and scale multiplication $\cdot: \bR\times X \rightarrow X$ are \emph{\textbf{continuous}}. 
\end{definition}

 \item \begin{theorem} \citep{treves2016topological}\\
 Every \textbf{locally compact Hausdorff} topological vector space is \textbf{finite-dimensional}.
 \end{theorem}
\end{itemize}
\subsection{Locally Convex Topological Vector Space}
\begin{itemize}
\item \begin{definition} (\emph{\textbf{Locally Convex Space}})\\
\emph{A topological vector space} $X$ is a \underline{\emph{\textbf{locally convex topological vector space}}} (or just \emph{l\textbf{ocally convex space}}), if $V$ is open and $x \in V$, then one can find a \emph{convex} \emph{open} set $U\subset X$ such that $x \in U\subset V$. That is, there exists \emph{a \textbf{base of convex sets} $\srB$ that \textbf{generates the topology} $\srT$}. 
\end{definition}

\item \begin{remark}
The most common way of defining locally convex topologies on vector spaces is in terms of \emph{semi-norms}. 
\end{remark}

\item  \begin{definition} (\emph{\textbf{Semi-Norm}})\\
A \emph{\textbf{semi-norm}} on a vector space $X$ is a mapping $q: X\rightarrow \bR_{+}$ satisfying the following conditions: 
\begin{enumerate}
\item \emph{homogeneity}: $q(\gamma \mb{x}) = \abs{\gamma}q(\mb{x})$;
\item the \emph{triangle inequality}: $q(\mb{x}+\mb{y})\le q(\mb{x})+ q(\mb{y})$.
\end{enumerate}
 If furthermore $q(\mb{x})=0 \Rightarrow \mb{x}=0$, then $q$ is a \emph{\textbf{norm}}.
\end{definition}

\item \begin{remark}
 A \emph{\textbf{metric}} $d: X\times X \rightarrow \bR_{+}$ that \emph{\textbf{induced}} from a norm is given by $d_{\theta}(\mb{x}, \mb{y})= q_{\theta}(\mb{y}-\mb{x}),\; \forall \mb{x,y}\in X $.
 \end{remark}

\item \begin{proposition}
A normed space $(X, \srT)$ induced by $\set{q_{\theta},\theta\in \Theta}$ is \emph{Hausdorff} if and only if for any $\mb{x}\neq 0, \mb{x}\in X$, $\exists \theta\in \Theta$, such that $q_{\theta}(\mb{x})>0$.
\end{proposition}

\item \begin{definition} (\emph{\textbf{Locally Convex Space} generated by \textbf{Semi-Norms}})\\
The \emph{\textbf{smallest topology}} $\srT$ induced by the set of \emph{\textbf{semi-norms}} $\set{q_{\theta},\theta\in \Theta}$ is generated by \emph{\textbf{the convex basis}} $U_{\mb{x},r,\theta} = \set{\mb{y}\in X\,|\, q_{\theta}(\mb{y}-\mb{x}) \le r }\in \srB, \mb{x}\in X, r>0$. The topological vector space $(X, \srT)$ is thus \underline{\emph{\textbf{locally convex space}}}. 

If $\set{q_{\theta},\theta\in \Theta}$ is a set of \emph{\textbf{norms}}, then $(X, \srT)$ is a \emph{\textbf{normed space}}. 
\end{definition}



\item \begin{remark}
The most commonly seen \emph{topological vector space} are \emph{\textbf{the normed linear space}}. It is a vector space $X$ equipped with norm $\norm{\cdot}{}$ and the topology generated by the norm induced metric $d$. It is denoted as $(X, \norm{\cdot}{})$. 

The \emph{\textbf{locally convex space}} is seen as a generalization of \emph{normed vector space}.
\end{remark}

\item \begin{proposition} (\textbf{Continuous Linear Operator}) \citep{folland2013real}\\
Suppose $X$ and $Y$ are vector spaces with topologies defined, respectively, by the families $\set{p_{\alpha}}_{\alpha\in A}$ and $\set{q_{\beta}}_{\beta \in B}$ of \textbf{semi-norms}, and $T: X \rightarrow Y$ is a linear map. Then $T$ is \textbf{continuous} \textbf{if and only if} for each $\beta \in B$, there exists $\alpha_1 \xdotx{,} \alpha_k \in A$ and $C > 0$ such that $q_{\beta}(T x) \le C\, \sum_{i=1}^{k}p_{\alpha_i}(x)$.
\end{proposition}

\item \begin{remark}
If the semi-norms are \emph{norms}, then the condition above is \emph{the bounded condition} for continuous linear operator.
\end{remark}

\item \begin{proposition}\citep{folland2013real}\\
Let $X$ be a vector space equipped with the topology defined by a family $\set{p_{\alpha}}_{\alpha\in A}$ of \textbf{seminorms}.
\begin{enumerate}
\item $X$ is \textbf{Hausdorff} if and only if for each $x \neq 0$ there exists $\alpha \in A$ such that $p_{\alpha}(x) \neq 0$.
\item If $X$ is \textbf{Hausdorff} and $A$ is \textbf{countable}, then $X$ is \textbf{metrizable} with a \textbf{translation invariant metric} (i.e., $d(x, y) = d(x +z, y+ z)$ for all $x, y, z \in X$).
\end{enumerate}
\end{proposition}

\item \begin{definition} (\emph{\textbf{Fr{\'e}chet Space}})\\
A \emph{\textbf{\underline{complete Hausdorff} topological vector space}} $X$ whose topology is defined by a \textbf{\emph{countable}} family of \emph{seminorms} $\set{q_{\theta},\theta\in \Theta}$ is called a \underline{\emph{\textbf{Fr{\'e}chet space}}}.
\end{definition}

\item \begin{example}
\begin{enumerate}
\item \emph{A \textbf{Fr{\'e}chet space}} is a \emph{\textbf{complete locally convex space}}. 

\item \emph{A \textbf{Banach space} is a \textbf{Fr{\'e}chet space}}.
\end{enumerate}
\end{example}

\item \begin{example} (\emph{\textbf{Locally Integrable Functions $L_{loc}^{1}(X, \mu)$}})\\
\emph{The space of all \textbf{locally integrable functions} on $\bR$, $L_{loc}^{1}(\bR)$, is a \textbf{Fr{\'e}chet space}} with the topology defined by the \emph{\textbf{semi-norms}}
\begin{align*}
 p_{k}(f) = \int_{\abs{x} \le k} \abs{f(x)} dx.
\end{align*}
Completeness follows easily from the completeness of $L^1$.  An obvious \emph{generalization} of this construction yields a \emph{\textbf{locally convex topological vector space}}  $L_{loc}^{1}(X, \mu)$ where $X$ is any \emph{locally convex Hausdorff (LCH) space} and $\mu$ is a \emph{Borel measure} on $X$ that is \emph{finite} on \emph{compact sets}.
\end{example}
\end{itemize}



\newpage
\bibliographystyle{plainnat}
\bibliography{reference.bib}
\end{document}